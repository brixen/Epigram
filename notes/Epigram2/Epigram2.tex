\documentclass{article}
\usepackage{palatino}
\usepackage{a4}

\begin{document}

\title{Epigram 2\\
       more design thoughts}
\author{Conor McBride}
\maketitle

\section{Introduction}

With the underlying type theory beginning to settle, it's time to
think a bit more about how the high-level Epigram language might
work. This document contains some ideas, and delineates some gaps.

I shall write mainly in the present tense, intending that the
assertions I make become true at some point.

\section{The Coauthor Transducer Model}

As an experience, Epigram programming is interactive. As a document,
an Epigram program is the `minute' of an interaction. To load and
check an Epigram program is to replay the minuted interaction,
checking its plausibility.

Epigram operates as a \textbf{transducer}, consuming a source
file, and producing a modified source file, marked up with error
reports and responses to requests for help. The transducer is
idempotent. Epigram thus acts like a mechanical coauthor, offering
opinions and responding to requests for contribution.

Epigram has a high-level \textbf{source} language, in which
\texttt{.epi} documents are written, and a low-level \textbf{evidence}
language, into which documents are \textbf{elaborated}. The system
thus works by constructing a sequence of definitions in the evidence
language elaborating the definitions found in its input. The
elaboration process is, by its nature, incremental. The system
maintains the correspondence between regions of the source document
and definitions at the evidence layer: the output file may thus be
generated from the final \textbf{proof state} of the elaboration
process.

In addition to generating the output file, the system will (on
request) dump the final proof state in a textual format---a
\texttt{.gram} file. From a \texttt{.gram} file, both input and output
\texttt{.epi} files are recoverable, hence the patch from one to the
other is computable. Given a \texttt{.gram} file and
a patch to its output \texttt{.epi}, a new input \texttt{.epi} can be
constructed, which can be processed in turn, resulting in a new output
\texttt{.epi}, and hence a further patch. Editor integration is thus
unnecessary, but in any case easy.

A separate (or at least separable) component, based on Edwin Brady's
\textsc{epic} compiler, can generate a \texttt{.gram} file from an
executable. The latter, if executed, offers a straightforward
read-eval-print loop for first-order expressions constructed over the
signature of definitions provided by the original source. Incomplete
definitions in the source may result in run-time errors: these are the
only run-time errors.  Such an executable, invoked with an argument,
treats that argument as a single command session.

Markdown is king.

In the event that Epigram programs ever become large enough to
necessitate modular development, \texttt{.gram} files will be generated
on a per-module basis. We may thus become interested in patches
between states of these \texttt{.gram} files as a means to transmit
changes downstream.


\section{Interactive Problem Solving}

Epigram documents show the declaration and partial solution of problems.
Problem solving is mediated as an interactive process. Every problem
in the source language can be encoded as a type in the evidence
language. Solving a problem in the source language amounts to finding
an inhabitant of the type which encodes it, by a process of
hierarchical refinement.

A \textbf{document} is a series of \textbf{developments}. A
development comprises a \textbf{declaration} introducing a problem,
and a \textbf{refinement} reducing the problem to a hopefully
empty collection of subproblems. Too much abstraction: example, please!
\newcommand{\capbox}[2]{\raisebox{0.12in}{\begin{array}[t]{|l|}\hline
      #1 \vspace*{-0.05in}\\
    \hfill\mbox{\scriptsize{#2}} \\ \hline\end{array}}}
\newcommand{\Ts}[1]{\texttt{#1}\;}
\newcommand{\co}{\;:\;}
\newcommand{\sco}{;\;\;}
\newcommand{\cm}{,\;}
\[
\capbox{
  \Ts{let}
    \capbox{
      \capbox{\Ts{x} \cm \Ts{y}\co \Ts{Nat}}{declaration}\\
      \Ts{--------------}\\
      \capbox{\Ts{x} \Ts{+} \Ts{y}}{template} \co \capbox{\Ts{Nat}}{type}
      }
      {declaration}\\ \\
  \capbox{
     \capbox{\Ts{x}\Ts{+}\Ts{y}}{problem}
       \capbox{\Ts{<=}\capbox{\Ts{induction}\Ts{x}}{eliminator}}{tactic}\\
     \capbox{
      \{\;\capbox{\capbox{\Ts{'zero}\Ts{+}\Ts{y}}{problem}
                  \capbox{\Ts{=} \capbox{\Ts{y}}{term}}{tactic}
                  \capbox{}{block}
          }{refinement}\\
      ;\;\capbox{\capbox{\Ts{'suc}\Ts{x}\Ts{+}\Ts{y}}{problem}
                 \capbox{\Ts{=} \capbox{\Ts{'suc}
                   \capbox{\{!\;!\}}{shed}}{term}}{tactic}
                  \capbox{}{block}
         }{refinement}\\
      \}
      }{block}
    }
    {refinement}}
    {development}
\]


\end{document}