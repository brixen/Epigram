\documentclass{article}
\usepackage{palatino}
\usepackage{a4}
\usepackage{alltt}
\usepackage{xspace}
\usepackage{pig}

%%%%%%%%%%%%%%%%%%%%%%%%%%%%%%%%%%%%%%%%%%%%%%%%%%%%%%%%%%%%%%%%%%%%%%%%%%%%%
%%% Inference Rules (some ancient macros by Conor)                        %%%
%%%%%%%%%%%%%%%%%%%%%%%%%%%%%%%%%%%%%%%%%%%%%%%%%%%%%%%%%%%%%%%%%%%%%%%%%%%%%

\newlength{\rulevgap}
\setlength{\rulevgap}{0.05in}
\newlength{\ruleheight}
\newlength{\ruledepth}
\newsavebox{\rulebox}
\newlength{\GapLength}
\newcommand{\gap}[1]{\settowidth{\GapLength}{#1} \hspace*{\GapLength}}
\newcommand{\dotstep}[2]{\begin{tabular}[b]{@{}c@{}}
                         #1\\$\vdots$\\#2
                         \end{tabular}}
\newlength{\fracwid}
\newcommand{\dotfrac}[2]{\settowidth{\fracwid}{$\frac{#1}{#2}$}
                         \addtolength{\fracwid}{0.1in}
                         \begin{tabular}[b]{@{}c@{}}
                         $#1$\\
                         \parbox[c][0.02in][t]{\fracwid}{\dotfill} \\
                         $#2$\\
                         \end{tabular}}
\newcommand{\Rule}[2]{\savebox{\rulebox}[\width][b]                         %
                              {\( \frac{\raisebox{0in} {\( #1 \)}}       %
                                       {\raisebox{-0.03in}{\( #2 \)}} \)}   %
                      \settoheight{\ruleheight}{\usebox{\rulebox}}          %
                      \addtolength{\ruleheight}{\rulevgap}                  %
                      \settodepth{\ruledepth}{\usebox{\rulebox}}            %
                      \addtolength{\ruledepth}{\rulevgap}                   %
                      \raisebox{0in}[\ruleheight][\ruledepth]               %
                               {\usebox{\rulebox}}}
\newcommand{\Case}[2]{\savebox{\rulebox}[\width][b]                         %
                              {\( \dotfrac{\raisebox{0in} {\( #1 \)}}       %
                                       {\raisebox{-0.03in}{\( #2 \)}} \)}   %
                      \settoheight{\ruleheight}{\usebox{\rulebox}}          %
                      \addtolength{\ruleheight}{\rulevgap}                  %
                      \settodepth{\ruledepth}{\usebox{\rulebox}}            %
                      \addtolength{\ruledepth}{\rulevgap}                   %
                      \raisebox{0in}[\ruleheight][\ruledepth]               %
                               {\usebox{\rulebox}}}
\newcommand{\Axiom}[1]{\savebox{\rulebox}[\width][b]                        %
                               {$\frac{}{\raisebox{-0.03in}{$#1$}}$}        %
                      \settoheight{\ruleheight}{\usebox{\rulebox}}          %
                      \addtolength{\ruleheight}{\rulevgap}                  %
                      \settodepth{\ruledepth}{\usebox{\rulebox}}            %
                      \addtolength{\ruledepth}{\rulevgap}                   %
                      \raisebox{0in}[\ruleheight][\ruledepth]               %
                               {\usebox{\rulebox}}}
\newcommand{\RuleSide}[3]{\gap{\mbox{$#2$}} \hspace*{0.1in}               %
                            \Rule{#1}{#3}                          %
                          \hspace*{0.1in}\mbox{$#2$}}
\newcommand{\AxiomSide}[2]{\gap{\mbox{$#1$}} \hspace*{0.1in}              %
                             \AxiomB{#2}                            %
                           \hspace*{0.1in}\mbox{$#1$}}
\newcommand{\RULE}[1]{\textbf{#1}}
\newcommand{\hg}{\hspace{0.2in}}


\ColourEpigram

\begin{document}


\section*{Notion of \emph{equivalence relation}:}

\[
\stkl{
\SYMBequivrel : \PI{\V{A}}{\Set} 
                (\V{A} \To \V{A} \To \Prop) \To
                \Prop \\
\begin{array}{@{}lcl}
\equivrel{\V{A}}{\V{R}} \mapsto &        & \eforall{\V{x}}
                                              {\V{A}}
                                              {\V{x} \mathop{\V{R}} \V{x}} \\
                        & \SYMBeand & \eforall{\V{x}\, \V{y}}
                                              {\V{A}}
                                              {\V{x} \mathop{\V{R}} \V{y} 
                                               \Imp \V{y} \mathop{\V{R}} \V{x}} \\
                        & \SYMBeand & \eforall{\V{x}\, \V{y}\, \V{z}}
                                              {\V{A}}
                                              {\V{x} \mathop{\V{R}} \V{y} 
                                               \Imp \V{y} \mathop{\V{R}} \V{z}
                                               \Imp \V{x} \mathop{\V{R}} \V{z}} 
\end{array}
}
\]

\section*{Type formers:}

Given a set \(\V{A}\) and an equivalence relation \(\V{R}\) over
\(\V{A}\), the quotient of \(\V{A}\) by \(\V{R}\) is denoted
\(\Quotient{\V{A}}{\V{R}}{\V{p}}\), with \(\V{p}\) the proof that
\(\V{R}\) is indeed an equivalence relation.


\[
\stkl{
\Rule{\stkc{\Bhab{\V{A}}{\Set} \quad
            \Bhab{\V{R}}{\V{A} \To \V{A} \To \Prop} \\
            \Bhab{\V{p}}{\prf{(\equivrel{\V{A}}{\V{R}})}}}}
     {\Bhab{\Quotient{\V{A}}{\V{R}}{\V{p}}}{\Set}}
\qquad

\Rule{\Bhab{\V{x}}{A}}
     {\Bhab{\class{\V{x}}}{\Quotient{A}{R}{p}}}
}
\]

\section*{Elimination rule:}

\[
\stkl{
\begin{array}{@{}ll}
\SYMBelimQuotient : & \PITEL{\V{A}}{\Set}
                      \PITEL{\V{R}}{\V{A} \To \V{A} \To \Prop}
                      \PITEL{\V{p}}{\prf{(\equivrel{\V{A}}{\V{R}})}} \\
                    & \PITEL{\V{z}}{\Quotient{\V{A}}{\V{R}}{\V{p}}}
                      \PITEL{\V{P}}{\Quotient{\V{A}}{\V{R}}{\V{p}} \To \Set} \\
                    & \PITEL{\V{m}}{\PI{\V{x}}{\V{A}}{\V{P}\:(\class{\V{x}})}}
                      \PI{\V{h}}{\prf{(\eforall{\V{x}\, \V{y}}{\V{A}}
                                       \V{x} \mathop{\V{R}} \V{y}
                                       \Imp (\Bhab{\V{m}\: \V{x}}{\V{P}\: \V{x}}) 
                                              \PropEq 
                                            (\Bhab{\V{m}\: \V{y}}{\V{P}\: \V{y}}))}} 
                      \V{P}\: \V{z}
\end{array}\\
\elimQuotient{\V{A}}{\V{R}}{\V{p}}{(\class{\V{x}})}{\V{P}}{\V{m}}{\V{h}} \mapsto \V{m}\: \V{z}
}
\]


\section*{Propositional equality:}

\[
\Rule{\eforall{\V{b}}{\V{B}}
      (\Bhab{\V{x}}{\V{A}}) \PropEq (\Bhab{\V{b}}{\V{B}})
       \Imp \V{b} \mathop{S} \V{y}}
     {(\Bhab{\class{\V{x}}}{\Quotient{A}{R}{p}})
        \PropEq
      (\Bhab{\class{\V{y}}}{\Quotient{B}{S}{q}})}
\]


\section*{Note:}

The notion of propositional equality in use here comes from the
``Observational Equality, Now!'' paper. In particular, the \(\PropEq\)
equality is heterogeneous (hence the type ascriptions). The \(\prf{}\)
constructor embeds objects of \(\Prop\) into \(\Set\).

\end{document}
