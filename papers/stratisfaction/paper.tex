%% * Header 

\documentclass[preprint
              , authoryear
              , onecolumn
              ]{sigplanconf}

%% ** Packages

\usepackage{amsmath}
\usepackage{amssymb}
\usepackage{amsthm}
\usepackage{verbatim}
\usepackage{url}
\usepackage{xspace}
\usepackage{subfigure}

\usepackage{pig}

%%%%%%%%%%%%%%%%%%%%%%%%%%%%%%%%%%%%%%%%%%%%%%%%%%%%%%%%%%%%%%%%
%% Comments
%%%%%%%%%%%%%%%%%%%%%%%%%%%%%%%%%%%%%%%%%%%%%%%%%%%%%%%%%%%%%%%%


\newcommand{\question}[1]{\textcolor{red}{question: #1}}
\newcommand{\pierre}[1]{\textcolor{red}{Pierre:} #1}
\newcommand{\conor}[1]{\textcolor{red}{Conor:} #1}

\newenvironment{danger}[1][]{\begin{proof}[\textlhdbend #1]}{\end{proof}}

\newcommand{\Set}{\mbox{Set}}

%%%%%%%%%%%%%%%%%%%%%%%%%%%%%%%%%%%%%%%%%%%%%%%%%%%%%%%%%%%%%%%%%%%%%%% 
%%%%%%
%%% Inference Rules (some ancient macros by  Conor)                        %%%
%%%%%%%%%%%%%%%%%%%%%%%%%%%%%%%%%%%%%%%%%%%%%%%%%%%%%%%%%%%%%%%%%%%%%%% 
%%%%%%


\newlength{\rulevgap}
\setlength{\rulevgap}{0.05in}
\newlength{\ruleheight}
\newlength{\ruledepth}
\newsavebox{\rulebox}
\newlength{\GapLength}
\newcommand{\gap}[1]{\settowidth{\GapLength}{#1} \hspace*{\GapLength}}
\newcommand{\dotstep}[2]{\begin{tabular}[b]{@{}c@{}}
                          #1\\$\vdots$\\#2
                          \end{tabular}}
\newlength{\fracwid}
\newcommand{\dotfrac}[2]{\settowidth{\fracwid}{$\frac{#1}{#2}$}
                          \addtolength{\fracwid}{0.1in}
                          \begin{tabular}[b]{@{}c@{}}
                          $#1$\\
                          \parbox[c][0.02in][t]{\fracwid}{\dotfill} \\
                          $#2$\\
                          \end{tabular}}
\newcommand{\Rule}[2]{\savebox{\rulebox}[\width] 
[b]                         %
                               {\( \frac{\raisebox{0in} {\( #1  
\)}}       %
                                        {\raisebox{-0.03in}{\( #2 \)}}  
\)}   %
                       \settoheight{\ruleheight} 
{\usebox{\rulebox}}          %
                       \addtolength{\ruleheight} 
{\rulevgap}                  %
                       \settodepth{\ruledepth} 
{\usebox{\rulebox}}            %
                       \addtolength{\ruledepth} 
{\rulevgap}                   %
                       \raisebox{0in}[\ruleheight] 
[\ruledepth]               %
                                {\usebox{\rulebox}}}
\newcommand{\Case}[2]{\savebox{\rulebox}[\width] 
[b]                         %
                               {\( \dotfrac{\raisebox{0in} {\( #1  
\)}}       %
                                        {\raisebox{-0.03in}{\( #2 \)}}  
\)}   %
                       \settoheight{\ruleheight} 
{\usebox{\rulebox}}          %
                       \addtolength{\ruleheight} 
{\rulevgap}                  %
                       \settodepth{\ruledepth} 
{\usebox{\rulebox}}            %
                       \addtolength{\ruledepth} 
{\rulevgap}                   %
                       \raisebox{0in}[\ruleheight] 
[\ruledepth]               %
                                {\usebox{\rulebox}}}
\newcommand{\CAxiom}[1]{\savebox{\rulebox}[\width] 
[b]                        %
                                {$\frac{}{\raisebox{-0.03in}{$#1$}} 
$}        %
                       \settoheight{\ruleheight} 
{\usebox{\rulebox}}          %
                       \addtolength{\ruleheight} 
{\rulevgap}                  %
                       \settodepth{\ruledepth} 
{\usebox{\rulebox}}            %
                       \addtolength{\ruledepth} 
{\rulevgap}                   %
                       \raisebox{0in}[\ruleheight] 
[\ruledepth]               %
                                {\usebox{\rulebox}}}
\newcommand{\RuleSide}[3]{\gap{\mbox{$#2$}}  
\hspace*{0.1in}               %
                             \Rule{#1}{#3}                          %
                           \hspace*{0.1in}\mbox{$#2$}}
\newcommand{\AxiomSide}[2]{\gap{\mbox{$#1$}}  
\hspace*{0.1in}              %
                              \CAxiom{#2}                            %
                            \hspace*{0.1in}\mbox{$#1$}}
\newcommand{\RULE}[1]{\textbf{#1}}
\newcommand{\hg}{\hspace{0.192in}}



%% ** Commands

\newtheorem{definition}{Definition}
\newtheorem{theorem}{Theorem}
\newtheorem{lemma}{Lemma}

\newcommand{\note}[1]{\textcolor{red}{#1}}
\newcommand{\ie}{\emph{ie.}\xspace}

\newcommand{\redb}[1]{\fcolorbox{red}{white}{$#1$}}
\newcommand{\blueb}[1]{\fcolorbox{blue}{white}{$#1$}}
\newcommand{\greenb}[3]{\fcolorbox{green}{white}{$#1$}_{\textcolor{green}{#2}}^{\textcolor{green}{#3}}}
\newcommand{\blackb}[1]{\fcolorbox{black}{white}{$#1$}}

\newcommand{\ruleName}[2]{(\mathrm{\textsc{#1}}^{#2})}
\newcommand{\ruleSet}[1]{\ruleName{Set}{#1}}
\newcommand{\rulePi}[1]{\ruleName{Pi}{#1}}
\newcommand{\ruleLam}[1]{\ruleName{Lam}{#1}}
\newcommand{\ruleConv}[1]{\ruleName{Conv}{#1}}
\newcommand{\ruleVar}[1]{\ruleName{Var}{#1}}
\newcommand{\ruleAsc}[1]{\ruleName{Asc}{#1}}
\newcommand{\ruleApp}[1]{\ruleName{App}{#1}}
\newcommand{\ruleSetEq}[1]{\ruleName{Set-Eq}{#1}}
\newcommand{\ruleVarEq}[1]{\ruleName{Var-Eq}{#1}}
\newcommand{\ruleSetKEq}[1]{\ruleName{Set-K-Eq}{#1}}
\newcommand{\ruleVarKEq}[1]{\ruleName{Var-K-Eq}{#1}}
\newcommand{\ruleSetVEq}[1]{\ruleName{Set-Var-Eq}{#1}}
\newcommand{\ruleVarVEq}[1]{\ruleName{Var-Var-Eq}{#1}}
\newcommand{\rulePiEq}[1]{\ruleName{Pi-Eq}{#1}}
\newcommand{\ruleLamEq}[1]{\ruleName{Lam-Eq}{#1}}
\newcommand{\ruleAppEq}[1]{\ruleName{App-Eq}{#1}}

\newcommand{\N}{\mathbb{N}}

\newcommand{\sublift}[3]{#2 \sqsubseteq_{#1} #3}

\begin{document}

%% ** Conference info

%\conferenceinfo{WXYZ '05}{date, City.} 
%\copyrightyear{2005} 
%\copyrightdata{[to be supplied]} 

%\titlebanner{banner above paper title}        % These are ignored unless
%\preprintfooter{short description of paper}   % 'preprint' option specified.


%% * Title


\title{Stratisfaction}


%% Alphabetical ordering.
\authorinfo{Pierre-\'{E}variste Dagand}
           {University of Strathclyde}
           {dagand@cis.strath.ac.uk}
\authorinfo{Conor McBride}
           {University of Strathclyde}
           {conor@cis.strath.ac.uk}

\maketitle

\ColourEpigram
%\MonochromeEpigram


%% * Abstract

\begin{abstract}
\end{abstract}

%\category{CR-number}{subcategory}{third-level}

%\terms
%term1, term2

%\keywords
%keyword1, keyword2


%% * Introduction
\section{Introduction}

%% ** Problem
%% *** <- Often, definitions are set polymorphic
%% **** <- Nat
%% **** <- Sigma type
%% **** <- Others?
%% *** -> Leads to code duplication
%% **** <- Define these objects at each level

%% ** Related work
%% *** <- Harper-Pollack: constraints too difficult
%% **** -> Lego, Coq, Agda experiments
%% ***** -> Leads to more or less powerful systems
%% **** -> Coq partial implementation, currently extended
%% **** -> Agda set polymorphism

%% *** -> Agda standard libray
%% *** -> Formalizations of CT, Voevoedsky

%% * Stratified type-theory
\section{A Universe Polymorphic Type Theory}

%% ** <- Presentation

\newcommand{\ECC}{ECCE\xspace}

%% *** <- Similar to ECC (FIND REF)
%% *** <- With CONV and extended CUML rule
%% **** -> Luo: subtyping: x :<:(N) A -> B ==> x :<:(O) A -> C with B =< C
%% **** -> Us: polymorphism: x :<:(N) A -> B ==> x :<:(O) Up A -> Up B

%% ***** Lifting operator

\begin{definition}[Lifting operator]

The lifting operator, denoted $\SYMBUp$, is defined by:
%%
$$
\begin{array}{l@{\:\:\mapsto\:\:}l}
\Up{k}{\Set{i}}                & \Set{i+k} \\
\Up{k}{\PI{\V{x}}{S}{T}}       & \PI{\V{x}}{\Up{k}{S}}{\Up{k}{T}} \\
\Up{k}{x_i}                    & x_{i+k} \\
\Up{k}{\LAM{x}{b}}             & \LAM{x}{\Up{k}{b}} \\
\Up{k}{f\: s}                  & (\Up{k}{f})\: s
\end{array}
$$

\note{$\Up{k}{(\Bhab{t}{T})}$ is embarrassing. In practice, lifts are
      applied on values, therefore ascriptions have disappeared.}

\end{definition}

%% ***** Lemma: lifting and substitution

\begin{lemma}[Lifting and substitution]
\label{lemma:up-bound-var}

Lifting commutes with substitution:
%%
$$
\Up{}{(T[s/x])} = (\Up{}{T})[s/x]
$$
%%
And with computation: for a term $a$ and $n$ its $\beta$-normal form,
\ie $a \leadsto n$, we have:
%%
$$
\Up{}{a} \leadsto \Up{}{n}
$$


\end{lemma}

Hence, we do not need to precise parenthesing: $\Up{}{(T[s/x])} =
(\Up{}{T})[s/x] = \Up{}{T}[s/x]$.

%% ****** Proof

\begin{proof}

The proof can be decomposed in two cases. First, if $x$ does not
appear in $T$, $T[s/x] = T$ and the proof is trivial. Second, without
loss of generality, if $x$ appears once in $T$. Being a variable, it
carries a number $j$ of pending liftings. There are two subsequent
cases: $x$ is at the head of a function application (possibly with an
empty spine), or it is in a spine of argument. In the latter case, by
definition of $\SYMBUp$, the lifting has no effect on $x$ therefore
both terms are equal. 

In the former, we have to analyse the effect of lifting and
substitution on both side. On the right side, $x_j$ is updated to
$x_{j+k}$ when first computing $\Up{k}{T}$. When substitution happens,
the liftings are discharged on $s$: the variable $x$ is replaced by
$\Up{j+k}{s}$. We obtain the term $\Up{k}{T}$ where $x$ is replaced by
$\Up{j+k}{s}$. On the left side, we first compute $T[s/x]$: $x$ is
replaced by $\Up{j}{s}$ in $T$. Then, the lifting occurs on the very
same term at the exception of $\Up{j}{s}$ replacing $x_j$. Hence,
$\Up{k}{\cdotp}$ happens on $\Up{j}{s}$. We obtain the term
$\Up{k}{T}$ where $x$ is replaced by $\Up{j+k}{s}$, concluding the
proof.

\note{Do we consider that computation happens right after (during)
      substitution? If so, we are in trouble.}

\end{proof}

%% ***** Context lifting

\begin{definition}[Context lifting]

Let $\Gamma$ a valid context.

We define valid \emph{context liftings} by the following sequents:
%%
$$
\CAxiom{\sublift{i}{\epsilon}{\epsilon}}
\qquad
\RuleSide{\sublift{i}{\Gamma}{\Delta}}
         {j \leq i}
         {\sublift{i}{\Gamma, \Bhab{x}{A}}{\Delta, \Bhab{x}{\Up{j}{A}}}}
$$

\end{definition}



\note{There is an elephant in the room here: what kind of result do we
  have about this polymorphic type-system? Indeed, it's similar to
  Luo's subtyping presentation of ECC. But it cannot be translated to
  it (how do you translate our cumulativity rule?). Doing the full
  normalization proof (Tait and co.) is not super exciting.}

%% *** <- Figure: type theory

\begin{figure}[htb]

\begin{center}
$$
\stkc{
%% Form
\boxed{\Gamma \vdash \Bhab{\CN{term}}{\CN{type}}}
\\
\\
%% Set
\AxiomSide{\ruleName{Set}{0}}{\Bhab{\Set{i}}{\Set{i+1}}}
\qquad
%% Arrow
\RuleSide{\Bhab{S}{\Set{i}} \quad
          \Bhab{x}{S} \vdash \Bhab{T}{\Set{i}}}
         {\ruleName{Pi}{0}}
         {\Bhab{\PI{\V{x}}{S}{T}}{\Set{i}}}
\\
%% Inhabitant of Arrow
\RuleSide{\Bhab{x}{S} \vdash \Bhab{b}{T} }
         {\ruleName{Lam}{0}}
         {\Bhab{\LAM{\V{x}_S}{b}}{\PI{\V{x}}{S}{T}}}
\\
%% Variable extraction
\AxiomSide{\ruleName{Var}{0}}{\Bhab{x}{A} \vdash \Bhab{x}{A}}
\qquad
%% Type ascription
\RuleSide{\Bhab{T}{\SetTop} \quad
          \Bhab{t}{T}}
         {\ruleName{Asc}{0}}
         {\Bhab{(\Bhab{t}{T})}{T}}
\qquad
%% Function application
\RuleSide{\Bhab{f}{\PI{\V{x}}{S}{T}} \quad
          \Bhab{s}{A}}
         {\ruleName{App}{0}}
         {\Bhab{f\: s}{T[s/x]}}
\\
%% Conversion
\RuleSide{\Bhab{M}{A} \quad
          \Bhab{A \Eq B}{\Set{i}}}
         {\ruleName{Conv}{0}}
         {\Bhab{M}{B}}
\\
%% Set equality
\AxiomSide{\ruleName{Set-Eq}{0}}{\Bhab{\Set{i} \Eq \Set{i}}{\SetTop}}
\\
%% Variable equality
\AxiomSide{\ruleName{Var-Eq}{0}}{\Bhab{x}{S} \vdash \Bhab{x \Eq x}{S}}
\\
%% Pi equality
\RuleSide{\Bhab{S_1 \Eq S_2}{\SetTop} \quad
          \Bhab{x}{S_2} \vdash \Bhab{T_1 \Eq T_2}{\SetTop}}
         {\ruleName{Pi-Eq}{0}}
         {\Bhab{\PI{\V{x}}{S_1}{T_1} \Eq 
                \PI{\V{x}}{S_2}{T_2}}{\SetTop}}
\\
%% Function equality
\RuleSide{\Bhab{x}{S} \vdash \Bhab{f\: x \Eq g\: x}{T}}
         {\ruleName{Lam-Eq}{0}}
         {\Bhab{f \Eq g}{\PI{\V{x}}{S}{T}}}
\\
%% Application equality
\RuleSide{\Bhab{f \Eq g}{\PI{\V{x}}{S}{T}} \quad
          \Bhab{a \Eq b}{S}}
         {\ruleName{App-Eq}{0}}
         {\Bhab{f\: a \Eq g\: b}{T[b/x]}}
\\
%% Cumulativity
\RuleSide{\Bhab{M}{A}}
         {\ruleName{Cuml}{0}}
         {\Bhab{M}{\Up{}{A}}}
}
$$

\end{center}

\caption{\ECC (Specification)}

\end{figure}


%% ** <- Full-reflection
%% *** <- Reflect equality as well as objects
%% **** <- Stratif. should be as unobtrusive as possible
%% ***** -> Dealt as much as possible definitionally
%% *** -> Early Martin-Lof papers
%% *** -> Palmgren
%% ** <- Meta-theoretical properties
%% *** ???
%% ** /> Just a spec
%% *** <- Flow of information unclear
%% *** <- CUML non algorithmic


%% * Bidirectionalized
\section{Bidirectional Presentation}

%% ** <- Push/pull distinction
\subsection{Introduction to bidirectionality}

%% *** -> Explicit flow of information
%% *** -> Algorithmic type-checking/synthesis
%% *** <- Push
%% **** -> Calculus of constants
%% *** <- Pull
%% **** -> Calculus of functions
%% *** <- Conversion
%% **** -> Calculus of conversion
%% *** <- Figure: bidirectional presentation (excepted CUML)

\begin{figure}[p]

\begin{center}
\subfigure[Algorithm: push-side]{
$$
\stkc{
%% Form
\boxed{\Gamma \vdash \pushN{\CN{term}}{1}{\CN{type}}}
\\
\\
%% Set
\AxiomSide{\ruleName{Set}{1}}{\pushN{\Set{i}}{1}{\Set{i+1}}}
\qquad
%% Arrow
\RuleSide{\pushN{S}{1}{\Set{i}} \quad
          \Bhab{x}{S} \vdash \pushN{T}{1}{\Set{i}}}
         {\ruleName{Pi}{1}}
         {\pushN{\PI{\V{x}}{S}{T}}{1}{\Set{i}}}
\\
%% Inhabitant of Arrow
\RuleSide{\Bhab{x}{S} \vdash \pushN{b}{1}{T} }
         {\ruleName{Lam}{1}}
         {\pushN{\LAM{\V{x}}{b}}{1}{\PI{\V{x}}{S}{T}}} 
}
\label{fig:ccomega-algo-push}
} \quad
\subfigure[Algorithm: pull-side]{
$$
\stkc{
%% Form
\boxed{\Gamma \vdash \pullN{\CN{term}}{1}{\CN{type}}}
\\
\\
%% Variable extraction
\AxiomSide{\ruleName{Var}{1}}{\Bhab{x}{A} \vdash \pullN{x}{1}{A}}
\qquad
%% Type ascription
\RuleSide{\pushN{T}{1}{\SetTop} \quad
          \pushN{t}{1}{T}}
         {\ruleName{Asc}{1}}
         {\pullN{(\Bhab{t}{T})}{1}{T}}
\\
%% Function application
\RuleSide{\pullN{f}{1}{\PI{\V{x}}{S}{T}} \quad
          \pushN{s}{1}{A}}
         {\ruleName{App}{1}}
         {\pullN{f\: s}{1}{T[s/x]}}
}
$$
\label{fig:ccomega-algo-pull}
}
\subfigure[Cumulativity]{
$$
\stkc{
%% Cumulativity (push)
\RuleSide{\pushN{M}{1}{A}}
         {\ruleName{Cuml-push}{1}}
         {\pushN{M}{1}{\Up{}{A}}}
\\
%% Cumulativity (pull)
\RuleSide{\pullN{M}{1}{A}}
         {\ruleName{Cuml-pull}{1}}
         {\pullN{M}{1}{\Up{}{A}}}
}
$$
\label{fig:ccomega-algo2-cuml}
}



\subfigure[Algorithm: conversion]{
$$
\stkc{
%% Conversion rule
\RuleSide{\pullN{e}{1}{S} \quad
          \equalN{S}{T}{1}{\SetTop}}
         {\ruleName{Conv}{1}}
         {\pushN{e}{1}{T}}
\\
\\
%% Form
\boxed{\Gamma \vdash \equalN{\CN{term}}{\CN{term}}{1}{\CN{type}}}
\\
\\
%% Set equality
\AxiomSide{\ruleName{Set-Eq}{1}}{\equalN{\Set{i}}{\Set{i}}{1}{\SetTop}}
\\
%% Variable equality
\AxiomSide{\ruleName{Var-Eq}{1}}{\Bhab{x}{S} \vdash \equalN{x}{x}{1}{S}}
\\
%% Pi equality
\RuleSide{\equalN{S_1}{S_2}{1}{\SetTop} \quad
          \Bhab{x}{S_2} \vdash \equalN{T_1}{T_2}{1}{\SetTop}}
         {\ruleName{Pi-Eq}{1}}
         {\equal{\PI{\V{x}}{S_1}{T_1}} 
                {\PI{\V{x}}{S_2}{T_2}}{1}{\SetTop}}
\\
%% Function equality
\RuleSide{\Bhab{x}{S} \vdash \equalN{f\: x}{g\: x}{1}{T}}
         {\ruleName{Lam-Eq}{1}}
         {\equalN{f}{g}{1}{\PI{\V{x}}{S}{T}}}
\\
%% Application equality
\RuleSide{\equalN{f}{g}{1}{\PI{\V{x}}{S}{T}} \quad
          \equalN{a}{b}{1}{S}}
         {\ruleName{App-Eq}{1}}
         {\equalN{f\: a}{g\: b}{1}{T[b/x]}}
}
$$
\label{fig:ccomega-algo-cuml}
}

\end{center}

\caption{Bidirectional presentation (excepted CUML)}

\end{figure}

%% *** <- Proof
%% **** Equational theory remains unchanged
%% ***** <- Same rules
%% **** Sound
%% ***** [OK] \Gamma \vdash A \ni^1 a \Rightarrow \Gamma \vdash a : A
%% ***** [OK] \Gamma \vdash a \in^1 A \Rightarrow \Gamma \vdash a : A
%% ***** Theorem

\begin{theorem}[Soundness]

Let $\Gamma$ a valid context.

\begin{itemize}
\item If \xspace $\Gamma \vdash \pushN{a}{1}{A}$, then $\Gamma \vdash \Bhab{a}{A}$
\item If \xspace $\Gamma \vdash \pullN{a}{1}{A}$, then $\Gamma \vdash \Bhab{a}{A}$
\end{itemize}

\end{theorem}

%% ***** Proof

\begin{proof}

The proof is by induction on the length of a derivation. The base
cases are $\ruleSet{1}$ and $\ruleVar{1}$ that translates (resp.) to
$\ruleSet{0}$ and $\ruleVar{0}$. We then proceed structurally for the
other rules: we simply forget the flow of information, turning $\in$
and $\ni$ into $:$. Because the equational theory is the same, the
$\ruleConv{1}$ rule translates directly.

\end{proof}

%% **** Complete
%% ***** [OK] \Gamma \vdash a : A \Rightarrow \Gamma \vdash A \ni^1 a
%% ***** Theorem

\begin{theorem}[Completeness]

Let $\Gamma$ a valid context. If \xspace$\Gamma \vdash \Bhab{a}{A}$, then
$\Gamma \vdash \push{a}{A}$.

\end{theorem}

%% ****** Proof

\begin{proof}

The proof is by induction on the length of a derivation. The base
cases are $\ruleSet{0}$ and $\ruleVar{0}$. The rule $\ruleSet{0}$
translates directly to its counterpart $\ruleSet{1}$. In the case of
the $\ruleVar{}$ rule, we use $\ruleVar{1}$ to get $\pull{x}{A}$,
followed by the conversion rule $\ruleConv{1}$ for $T = A$, asking for
$\equalN{A}{A}{1}{\SetTop}$ to hold (trivial).

The induction step is again direct for rules that have a counterpart
on the push-side. For the one on the pull-side, we proceed as for
$\ruleVar{0}$-$\ruleVar{1}$: we obtain a pull-oriented rule that we
trivially convert to its push-side equivalent.

\end{proof}

%% ** <- Fusing CUML push-side
\subsection{Fusing cumulativity on the push-side}

%% *** <- Types pushed in
%% **** -> No guess work needed
%% *** <- Figure: CUML fused on push-side

\begin{figure}

\begin{center}
\subfigure[Algorithm: cumulative push-side]{
$$
\stkc{
%% Form
\boxed{\Gamma \vdash \pushN{\CN{term}}{2}{\CN{type}}}
\\
\\
%% Set
\AxiomSide{i<j \quad \ruleName{Set}{2}}{\pushN{\Set{i}}{2}{\Set{j}}}
\qquad
%% Arrow
\RuleSide{\pushN{S}{2}{\Set{i}} \quad
          \Bhab{x}{S} \vdash \pushN{T}{2}{\Set{i}}}
         {\ruleName{Pi}{2}}
         {\pushN{\PI{\V{x}}{S}{T}}{2}{\Set{i}}}
\\
%% Inhabitant of Arrow
\RuleSide{\Bhab{x}{S} \vdash \pushN{b}{2}{T} }
         {\ruleName{Lam}{2}}                         
         {\pushN{\LAM{\V{x}}{b}}{2}{\PI{\V{x}}{S}{T}}} 
}
\label{fig:ccomega-algo2-push}
}

\subfigure[Cumulativity]{
$$
\stkc{
%% Cumulativity (pull)
\RuleSide{\pullN{M}{2}{A}}
         {\ruleName{Cuml-pull}{2}}
         {\pullN{M}{2}{\Up{1}{A}}}
}
$$
\label{fig:ccomega-algo2-cuml}
}


\subfigure[Algorithm: pull-side]{
$$
\stkc{
%% Form
\boxed{\Gamma \vdash \pullN{\CN{term}}{2}{\CN{type}}}
\\
\\
%% Variable extraction
\AxiomSide{\ruleName{Var}{2}}{\Bhab{x}{A} \vdash \pullN{x}{2}{A}}
\qquad
%% Type ascription
\RuleSide{\pushN{T}{2}{\SetTop} \quad
          \pushN{t}{2}{T}}
         {\ruleName{Asc}{2}}
         {\pullN{(\Bhab{t}{T})}{2}{T}}
\\
%% Function application
\RuleSide{\pullN{f}{2}{\PI{\V{x}}{S}{T}} \quad
          \pushN{s}{2}{A}}
         {\ruleName{App}{2}}
         {\pullN{f\: s}{2}{T[s/x]}}
}
$$
\label{fig:ccomega-algo-pull}
}
\subfigure[Algorithm: conversion]{
$$
\stkc{
%% Conversion rule
\RuleSide{\pullN{e}{2}{S} \quad
          \equalN{S}{T}{2}{\SetTop}}
         {\ruleName{Conv}{2}}
         {\pushN{e}{2}{T}}
\\
\\
%% Form
\boxed{\Gamma \vdash \equalN{\CN{term}}{\CN{term}}{2}{\CN{type}}}
\\
\\
%% Set equality
\AxiomSide{\ruleName{Set-Eq}{2}}{\equalN{\Set{i}}{\Set{i}}{2}{\SetTop}}
\\
%% Variable equality
\AxiomSide{\ruleName{Var-Eq}{2}}{\Bhab{x}{S} \vdash \equalN{x}{x}{2}{S}}
\\
%% Pi equality
\RuleSide{\equalN{S_1}{S_2}{2}{\SetTop} \quad
          \Bhab{x}{S_2} \vdash \equalN{T_1}{T_2}{2}{\SetTop}}
         {\ruleName{Pi-Eq}{2}}
         {\equalN{\PI{\V{x}}{S_1}{T_1}} 
                 {\PI{\V{x}}{S_2}{T_2}}{2}{\SetTop}}
\\
%% Function equality
\RuleSide{\Bhab{x}{S} \vdash \equalN{f\: x}{g\: x}{2}{T}}
         {\ruleName{Lam-Eq}{2}}
         {\equalN{f}{g}{2}{\PI{\V{x}}{S}{T}}}
\\
%% Application equality
\RuleSide{\equalN{f}{g}{2}{\PI{\V{x}}{S}{T}} \quad
          \equalN{a}{b}{2}{S}}
         {\ruleName{App-Eq}{2}}
         {\equalN{f\: a}{g\: b}{2}{T[b/x]}}
}
$$
\label{fig:ccomega-algo-conv}
}

\end{center}

\caption{Cumulativity fused push-side}

\end{figure}


%% ** <- Proof
%% *** Equational theory unchanged
%% *** Sound
%% **** \Gamma \vdash A \ni^2 a \Rightarrow \Gamma \vdash A \ni^1 a
%% **** \Gamma \vdash a \ni^2 A \Rightarrow \Gamma \vdash a \in^1 A
%% **** Theorem 

\begin{theorem}[Soundness]

Let $\Gamma$ a valid context.

\begin{itemize}
\item If \xspace$\Gamma \vdash \pushN{a}{2}{A}$, then $\Gamma \vdash \pushN{a}{1}{A}$
\item If \xspace$\Gamma \vdash \pullN{a}{2}{A}$, then $\Gamma \vdash \pullN{a}{1}{A}$
\end{itemize}

\end{theorem}

%% ***** Proof

\begin{proof}

The proof is by induction on the height of a derivation. The base
cases are $\ruleSet{2}$ and $\ruleVar{2}$. The translation of
$\ruleVar{2}$ is direct. Concerning $\ruleSet{2}$, we have the
following judgment:
%%
$$
\AxiomSide{i < j}{\pushN{\Set{i}}{2}{\Set{j}}}
$$

This translates to the following derivation:
%%
$$
\stkc{
\AxiomSide{\ruleSet{1}}{\pushN{\Set{i}}{1}{\Set{i+1}}} \\
\AxiomSide{\ruleName{Cuml-pull}{1}}{\pushN{\Set{i}}{1}{\Set{i+2}}} \\
\ldots \\
\CAxiom{\pushN{\Set{i}}{1}{\Set{j}}}
}
$$

The induction step is trivial on the pull-side: the rules being the
same, they map directly to their counterpart. Similarly on the
push-side, the only rule that changed is $\ruleSet{2}$ that we have
already treated in the base case. Hence, all rules translates
directly.

\end{proof}

%% *** Lemma: lifting preserves equality

\begin{lemma}[Lifting preserves equality]
\label{lemma:lifting-equality}

Let $\Gamma$ a valid context. Let $k \in N$.

If $\equalN{S}{T}{2}{A}$, then $\equalN{\Up{k}{S}}{\Up{k}{T}}{2}{A}$.

\end{lemma}

%% **** Proof

\begin{proof}

The proof is by induction on the length of an equality judgement. The
base cases are $\ruleSetEq{2}$ and $\ruleVarEq{2}$. In the
$\ruleSetEq{2}$ rule, our goal is to prove that
$\equalN{\Set{i+k}}{\Set{i+k}}{2}{\SetTop}$ that is trivially
verified. Similarly, in the $\ruleVarEq{2}$ case, we have to verify
that $\equalN{x_{i+k}}{x_{i+k}}{2}{\SetTop}$, which is indeed true.

The induction step follows naturally, by the structure of the
judgements and the definition of $\SYMBUp$. 

\end{proof}


%% *** Lemma: cumulativity fused push-side

\begin{lemma}[$\ruleName{Cuml-push}{1}$ fused]
\label{lemma:cuml-push-fused}

Let $\Gamma$ a valid context.

\begin{itemize}
\item If \xspace$\Gamma \vdash \pushN{M}{1}{A}$, then for all $k \geq 0$ and
  context $\Delta$ such that $\sublift{k}{\Gamma}{\Delta}$, we have
  $\Delta \vdash \pushN{M}{2}{\Up{k}{A}}$.
\item If \xspace$\Gamma \vdash \pullN{A}{1}{\Set{i}}$, then $\Gamma \vdash
  \pullN{A}{2}{\Set{i}}$
\end{itemize}

\end{lemma}

%% **** Proof

\begin{proof}

The proof is by induction on the height of a derivation. We focus on
the push-side rules, as the pull-side remains un-changed between both
systems. The base case of interest is $\ruleSet{1}$. In this
case, we have indeed for any $k \geq 0$:
%%
$$
\AxiomSide{i+k+1 > i}{\pushN{\Set{i}}{2}{\Set{i+k+1}}}
$$


The induction step is purely structural in the case of
$\rulePi{1}$. The interesting cases are $\ruleName{Cuml-push}{1}$,
$\ruleConv{1}$, and $\ruleLam{1}$. In the $\ruleName{Cuml-push}{1}$
case, the last rule is the following:
%%
$$
\Rule{\pushN{M}{1}{A}}{\pushN{M}{i}{\Up{}{A}}}
$$
%%
By induction hypothesis on the premise with $\Delta = \Gamma$, we have
that:
%%
$$
\forall k \geq 0, \Gamma \vdash \pushN{M}{2}{\Up{k}{A}}
$$
%%
From which we can easily conclude the proof of this case:
%%
$$
\forall k \geq 0, \Gamma \vdash \pushN{M}{2}{\Up{k+1}{A}}
$$

As for the $\ruleConv{1}$ case, the last derivation is the following:
%%
$$
\Rule{\pullN{e}{1}{S} \quad
      \equalN{S}{T}{1}{\SetTop}}
     {\pushN{e}{1}{T}}
$$
%%

The equational theories of both systems being the same, we can
transport the equality from the first to the second. By
Lemma~\ref{lemma:lifting-equality}, we deduce that $\Up{k}{S} \Eq
\Up{k}{T}$. Moreover, by induction hypothesis, we have that
$\pullN{e}{2}{S}$. Applying the $\ruleName{Cuml-pull}{2}$ rule
$k$-times, we obtain $\pullN{e}{2}{\Up{k}{S}}$.  Applying the
$\ruleConv{2}$ rule, we get the desired conclusion:
$\pushN{e}{2}{\Up{k}{T}}$.

Finally, the $\ruleLam{1}$ rule corresponds to the following
derivation:
%%
$$
\Rule{\Bhab{x}{S} \vdash \pushN{b}{1}{T}}
     {\pushN{\LAM{x}{b}}{1}{\PI{\V{x}}{S}{T}}}
$$
%%
Applying the induction hypothesis on the premises, with a carefully
chosen $\Delta$, we obtain:
%%
$$
\Bhab{x}{\Up{k}{S}} \vdash \pushN{b}{1}{\Up{k}{T}}
$$
%%
From $\ruleLam{2}$, we then deduce the desired conclusion.

\end{proof}


%% *** Complete
%% **** \Gamma \vdash A \ni^1 a \Rightarrow \Gamma \vdash A \ni^2 a
%% **** \Gamma \vdash a \ni^1 A \Rightarrow \Gamma \vdash a \in^2 A
%% **** Theorem

\begin{theorem}[Completeness]

Let $\Gamma$ a valid context.

\begin{itemize}
\item If \xspace$\Gamma \vdash \pushN{a}{1}{A}$, then $\Gamma \vdash \pushN{a}{2}{A}$
\item If \xspace$\Gamma \vdash \pullN{a}{1}{A}$, then $\Gamma \vdash \pullN{a}{2}{A}$
\end{itemize}

\end{theorem}

%% ***** Proof

\begin{proof}

The proof is by induction on the length of a derivation. The base
cases are $\ruleSet{1}$ and $\ruleVar{1}$. They both translate
directly to the other system. 

The induction is trivial for all rules --- by existence of an
identical rule in the other system --- at the exception of the
$\ruleName{Cuml-push}{1}$ rule. In this case, the last derivation is
the following:
%%
$$
\Rule{\pushN{M}{1}{A}}
     {\pushN{M}{1}{\Up{}{A}}}
$$
%%
Applying Lemma \ref{lemma:cuml-push-fused} on the premise, we obtain:
%%
$$
\forall k \geq 0, \Gamma \vdash \pushN{M}{2}{\Up{k}{A}}
$$
From which we can conclude, by taking $k$ to be 1.

\end{proof}

%% * Annotated pull-side
\section{Manually Annotated Stratisfactor}

%% ** <- Generalizing CUML to Up
%% *** <- x :<: S for S = Set_i
%% *** <- x :<: A -> B ?
%% *** -> FTP/GFTP
%% *** -> x : Set(i) means x : Set(i+)
%% ** -> Annotating by the lifting required
%% *** -> Guess work in the hand of the annotator
%% *** -> Figure: annotated type system
%% ** <- Figure: Annotated pull-side

\begin{figure}

\begin{center}
\subfigure[Algorithm: push-side]{
$$
\stkc{
%% Form
\boxed{\Gamma \vdash \pushN{\CN{term}}{3}{\CN{type}}}
\\
\\
%% Set
\AxiomSide{i < j}{\pushN{\Set{i}}{3}{\Set{j}}}
\qquad
%% Arrow
\Rule{\pushN{S}{3}{\Set{i}} \quad
      \Bhab{x}{S} \vdash \pushN{T}{3}{\Set{i}}}
     {\pushN{\PI{\V{x}}{S}{T}}{3}{\Set{i}}}
\\
%% Inhabitant of Arrow
\Rule{\Bhab{x}{S} \vdash \pushN{b}{3}{T} }
     {\pushN{\LAM{\V{x}}{b}}{3}{\PI{\V{x}}{S}{T}}} 
}
$$
\label{fig:ccomega-algo3-push}
} \quad
\subfigure[Algorithm: pull-side]{
$$
\stkc{
%% Form
\boxed{\Gamma \vdash \pullN{\CN{term}}{3}{\CN{type}}}
\\
\\
%% Variable extraction
\CAxiom{\Bhab{x}{A} \vdash \pullN{x^{(j)}}{3}{\Up{j}{A}}}
\qquad
%% Type ascription
\Rule{\pushN{T}{3}{\SetTop} \quad
      \pushN{t}{3}{T}}
     {\pullN{(\Bhab{t}{^{(j)}T})}{3}{\Up{j}{T}}}
\\
%% Function application
\Rule{\pullN{f^{(j)}}{3}{\PI{\V{x}}{\Up{j}{S}}{\Up{j}{T}}} \quad
      \pushN{s}{3}{\Up{j}{S}}}
     {\pullN{{f\: s}^{(j)}}{3}{\Up{j}{T}[s/x]}}
}
$$
\label{fig:ccomega-algo3-pull}
}

\subfigure[Algorithm: conversion]{
$$
\stkc{
%% Conversion rule
\Rule{\pullN{e^{(j)}}{3}{\Up{j}{S}} \quad
      \equalN{\Up{j}{S}}{T}{3}{\SetTop}}
     {\pushN{e}{3}{T}}
\\
\\
%% Form
\boxed{\Gamma \vdash \equalN{\CN{term}}{\CN{term}}{3}{\CN{type}}}
\\
\\
%% Set equality
\CAxiom{\equalN{\Set{i}}{\Set{i}}{3}{\SetTop}}
\\
%% Variable equality
\CAxiom{\Bhab{x}{S} \vdash \equalN{x_i}{x_i}{3}{S}}
\\
%% Pi equality
\Rule{\equalN{S_1}{S_2}{3}{\SetTop} \quad
      \Bhab{x}{S_2} \vdash \equalN{T_1}{T_2}{3}{\SetTop}}
     {\equalN{\PI{\V{x}}{S_1}{T_1}} 
             {\PI{\V{x}}{S_2}{T_2}}{3}{\SetTop}}
\\
%% Function equality
\Rule{\Bhab{x}{S} \vdash \equalN{f\: x}{g\: x}{3}{T}}
     {\equalN{f}{g}{3}{\PI{\V{x}}{S}{T}}}
\\
%% Application equality
\Rule{\equalN{f}{g}{3}{\PI{\V{x}}{S}{T}} \quad
      \equalN{a}{b}{3}{S}}
     {\equalN{f\: a}{g\: b}{3}{T[b/x]}}
}
$$
\label{fig:ccomega-algo3-conv}
}

\end{center}

\caption{Annotated pull-side}

\end{figure}


%% ** <- Proof
%% *** GFTP
%% **** Theorem

\begin{theorem}[GFTP]
\label{thm:gftp}

Let $\Gamma$ a valid context. 

If \xspace$\Gamma \vdash \pullN{a^{(j)}}{3}{\Up{j}{A}}$, then for any $i \geq j$ and
for all $\Delta$ such that $\sublift{i}{\Gamma}{\Delta}$, we have:
%%
$$
\RuleSide{\Gamma \vdash \pullN{a^{(j)}}{3}{\Up{j}{A}}}
         {k \in \N}
         {\Delta \vdash \pullN{a^{(k)}}{3}{\Up{i}{A}}}
$$

And similarly on the push-side, for if \xspace
$\Gamma \vdash \pushN{a}{3}{A}$.

\end{theorem}

%% ***** Proof

\begin{proof}

The proof is by induction on the length of derivations. The base cases
are $\ruleSet{3}$ and $\ruleVar{3}$. It is easily verified for the
$\ruleSet{3}$ rule, the interesting case being $\ruleVar{3}$. We have
that:
%%
$$
\CAxiom{\Bhab{x}{A} \vdash \pullN{x^{(l)}}{3}{\Up{l}{A}}}
$$
%%
We have to find the right annotation $k$ on $x$ so that
$\pullN{x^{(k)}}{\Up{i}{A}}$ in a context $\Delta$ containing
$\Bhab{x}{\Up{j}{A}}$ with $j \leq i$. Taking $l$ to be 
\xspace$i-j \geq 0$, we have the following valid derivation:
%%
$$
\CAxiom{\Bhab{x}{\Up{j}{A}} \vdash \pullN{x^{(i-j)}}{3}{\Up{i-j}{(\Up{j}{A})}}}
$$
%%
Which concludes this case.

The induction steps can be handled easily for all rules pushing a
$\Set{l}$ into a term. First, the push side is identical to the
previous one. Hence, Lemma \ref{lemma:cuml-push-fused} still applies
and gives:
%%
$$
\forall k \geq 0, \Gamma \vdash \pushN{A}{3}{\Set{l+k}}
$$
%%

By definiton of $\SYMBUp$, this is equivalent to our goal:
$\forall i \geq 0, \Gamma \vdash \pushN{a}{3}{\Up{i}{\Set{l}}}$. 

There are two remaining rules on the push side: $\ruleConv{3}$ and
$\ruleLam{3}$. The premises of the $\ruleConv{3}$ rule tell us that
$\pullN{e^{(j)}}{3}{\Up{j}{S}}$ and
$\equalN{\Up{j}{S}}{T}{3}{\SetTop}$. By induction hypothesis, we
deduce that $\pullN{e^{(k)}}{3}{\Up{i+j}{S}}$ with $k \geq 0$. By
Lemma \ref{lemma:lifting-equality}, we also deduce that
$\equalN{\Up{i}{\Up{j}{S}}}{\Up{i}{T}}{3}{\SetTop}$. Hence, by $\ruleConv{3}$,
we conclude the desired judgement, \ie $\pushN{e}{3}{\Up{i}{T}}$.

The premise of the $\ruleLam{3}$ rule gives $\Bhab{x}{S} \vdash
\pushN{b}{3}{T}$. By applying the induction hypothesis with a suitably
lifted context, we get that: $\Bhab{x}{\Up{i}{S}} \vdash
\pushN{b}{3}{T}$. Applying rule $\ruleLam{3}$, we conclude that
$\pushN{\LAM{x}{b}}{3}{\Up{i}{\PI{\V{x}}{S}{T}}}$, by definition of
$\SYMBUp$ on a $\Pi$-type.

Finally, we handle the two remaining rules: $\ruleAsc{3}$ and
$\ruleApp{3}$. The ascription rule is easy as its type is governed by
the annotation we set: hence, to obtain an element in $\Up{i}{T}$, we
simply annotate $\Bhab{t}{T}$ by $(i)$. The premises of the
$\ruleApp{3}$ rule inform us that
$\pullN{f^{(j)}}{3}{\PI{\V{x}}{\Up{j}{S}}{\Up{j}{T}}}$ and
$\pushN{s}{3}{\Up{j}{S}}$. Applying the induction hypothesis on both,
we get that $\pullN{f^{(k)}}{3}{\PI{\V{x}}{\Up{i}{S}}{\Up{i}{T}}}$ and
$\pushN{s}{3}{\Up{i}{S}}$. We can apply $\ruleApp{3}$ straightaway and
conclude that $\pullN{{f\: s}^{(k)}}{3}{\Up{i}{T}[s/x]}$. 

\end{proof}

%% **** FTP

\begin{theorem}[FTP]

In a context $\Gamma$ valid, if $\Gamma \vdash \pullN{a}{3}{A}$, then
the term $a$ checks all polymorphic instances of $A$, \ie $\forall i
\geq 0, \Gamma \vdash \pullN{a}{3}{\Up{i}{A}}$

\end{theorem}

%% ***** Proof

\begin{proof}

Corollary of Theorem~\ref{thm:gftp}, taking $\Delta$ to be $\Gamma$.

\end{proof}

%% *** Sound
%% **** \Gamma \vdash A \ni^3 a \Rightarrow \forall j \geq 0, \Gamma \vdash \Up{j}{A} \ni^2 a
%% **** \Gamma \vdash a \in^3 A \Rightarrow \forall j \geq 0, \Gamma \vdash a \in^2 \Up{j}{A}
%% **** Theorem

\begin{theorem}[Soundness]

Let $\Gamma$ a valid context.

\begin{itemize}
\item If \xspace$\Gamma \vdash \pullN{a^{(j)}}{3}{\Up{j}{A}}$, then for all 
      $k \in \N$, we have $\Gamma \vdash \pullN{a}{2}{\Up{k}{A}}$
\item If \xspace$\Gamma \vdash \pushN{a}{3}{A}$, then 
      $\Gamma \vdash \pushN{a}{2}{A}$
\end{itemize}

\end{theorem}

%% ***** Proof

\begin{proof}

Let $k \in \N$.

First, the push-side rules of both systems being identical, the
translation from one system to the other is trivial. Therefore, we
will focus on the pull-side.

The proof is by induction on the height of derivations. The base case
of interest here is $\ruleVar{3}$. We have the following:
%%
$$
\CAxiom{\Bhab{x}{A} \vdash \pullN{x^{(j)}}{3}{\Up{j}{A}}}
$$

This translates to an application of the $\ruleVar{2}$ rule followed
by as many application of the $\ruleName{Cuml-pull}{2}$ necessary to
reach $k$:
%%
$$
\stkc{
\CAxiom{\Bhab{x}{A} \vdash \pullN{x}{2}{A}} \\
\CAxiom{\Bhab{x}{A} \vdash \pullN{x}{2}{\Up{}{A}}} \\
\ldots \\
\CAxiom{\Bhab{x}{A} \vdash \pullN{x}{2}{\Up{k}{A}}}
}
$$

The induction step consists in treating the $\ruleAsc{3}$ and
$\ruleApp{3}$ rules. The $\ruleAsc{3}$ requires exactly the same
treatment than the $\ruleVar{3}$ rule: by induction hypothesis on the
premises, we can apply the corresponding rule in the previous
system. We then use $\ruleName{Cuml-pull}{2}$ to complete the missing
liftings to reach $\Up{k}{T}$.

The premises of the $\ruleApp{3}$ rule give us that
$\pullN{f^{(j)}}{3}{\PI{\V{x}}{\Up{j}{S}}{\Up{j}{T}}}$ and
$\pushN{s}{3}{\Up{j}{S}}$. By induction hypothesis, we get that
$\pullN{f}{2}{\PI{\V{x}}{\Up{k}{S}}{\Up{k}{T}}}$ and
$\pushN{s}{2}{S}$. Applying Lemma~\ref{lemma:cuml-push-fused} for
$\Delta = \Gamma$, we obtain that
$\pushN{s}{2}{\Up{k}{S}}$. Therefore, by application of $\ruleApp{2}$,
we deduce that $\pushN{f\: s}{2}{\Up{k}{T}[s/x]}$. 

\end{proof}

%% *** Complete
%% **** \Gamma \vdash A \ni^2 a \Rightarrow \Gamma \vdash A \ni^3 a
%% **** \Gamma \vdash a \in^2 A \Rightarrow \Gamma \vdash a \in^3 A
%% **** Theorem

\begin{theorem}[Completeness]

Let $\Gamma$ be a valid context.

\begin{itemize}
\item If \xspace$\Gamma \vdash \pullN{a}{2}{A}$, then for any 
      $k \in \N$, there exists an annotation $j \in \N$ such that we have 
      $\Gamma \vdash \pullN{a^{(j)}}{3}{\Up{k}{A}}$
\item If \xspace$\Gamma \vdash \pushN{a}{2}{A}$, then $\Gamma \vdash
      \pushN{a}{3}{A}$
\end{itemize}

\end{theorem}

%% ***** Proof

\begin{proof}

Let $k \in \N$

First, the push-side is rather unexciting, as it is the same in both
systems. Hence, it translates directly from one to another. Therefore,
we will focus on the pull-side.

The proof is by induction on the height of derivations. The base case
of interest is $\ruleVar{2}$ here. Hence, we have $\Bhab{x}{A}$ in the
context. By direct application of $\ruleVar{3}$, we obtain the desired
result:
%%
$$
\CAxiom{\Bhab{x}{A} \vdash \pullN{x^{(k)}}{3}{\Up{k}{A}}}
$$

The induction steps consists of the rules $\ruleAsc{2}$,
$\ruleApp{2}$, and $\ruleName{Cuml-pull}{2}$. The $\ruleAsc{2}$ rule
translates the same way than the $\ruleVar{2}$ rule. The premises of
the $\ruleApp{2}$ give us that $\pullN{f}{2}{\PI{\V{x}}{S}{T}}$ and
$\pushN{s}{2}{S}$. By induction hypothesis, we deduce that
$\pullN{f^{(k)}}{3}{\Up{k}{(\PI{\V{x}}{S}{T})}}$ and
$\pushN{s}{3}{S}$. By application of the FTP, we get that
$\pushN{s}{3}{\Up{k}{S}}$. Applying the $\ruleApp{3}$ rule, we
conclude that $\pullN{{f\: s}^{(k)}}{3}{\Up{k}{T}[s/x]}$.

We conclude with the $\ruleName{Cuml-pull}{2}$ rule. The derivation
ends with the following judgment:
%%
$$
\Rule{\pullN{M}{2}{A}}
     {\pullN{M}{2}{\Up{}{A}}}
$$
%%
By induction hypothesis on the premise, we deduce that $\forall l \in
\N, \pullN{M^{(l)}}{3}{\Up{l}{A}}$. In particular, we have that
$\pullN{M^{(k+1)}}{3}{\Up{k}{(\Up{}{A})}}$, which is the expected
conclusion.

\end{proof}

%% * Semi-automatic
\section{Semi-automatic Stratisfactor}


%% ** <- State properties
%% *** /> Prove them a posteriori
%% ** <- Shape classification
%% *** <- Calculus of constants
%% **** <- They automatically lift
%% **** -> No need to adjust them
%% *** <- Calculus of *polymorphic* functions
%% **** <- We have to specify where they exist
%% ***** /> Cannot predict at which level on use-site
%% ***** -> Introduce variable lifting

The type $\greenb{\Up{j}{A}}{j}{k}$ is an object lifted by an unknown
amount $j \geq k$.

%% **** <- Use-site: check that a coherent assignment can be made
%% **** /> App-rule: green on red side
%% ***** -> Find the minimal assignment k such that A_j=k :>: s

\begin{definition}[Minimal assignment]

Let $k \in \N$.

We say that \emph{$k$ is a minimal assignement} in 
%%
$$
\Rule{\pi}
     {\pushN{s}{4}{\redb{\greenb{\Up{j}{S}}{j=k}{0}}}}
$$
%%
If, for all uses of the $\ruleSet{4}$ rule in $\pi$, the minimal
difference of index to a $j$-indexed set is 1.

Put otherwise, the bound on $j$ is tight: decrementing $k$ would lead
to a size issue.

\end{definition}

%% ***** /> Rely on annotation for now
%% *** <- Calculus of conversions
%% **** <- Conversion rule: use-site of a polymorphic object
%% ***** -> Set the variable
%% **** <- Push-side orders a given type
%% ***** -> Can we find an assignment to the variable on S?
%% ***** -> Conversion rule: S == T
%% ****** -> Solve variable
%% ****** -> Check for coherence
%% ** Figure: semi-automatic type system

\begin{figure}

\begin{center}
\subfigure[Push-side]{
\stkc{
%% Form
\boxed{\Gamma \vdash \pushN{\CN{term}}{4}{\redb{\CN{type}}}}
\\
\\
%% Set
\AxiomSide{i < j \quad \ruleSet{4}}{\pushN{\Set{i}}{4}{\redb{\Set{j}}}}
\qquad
%% Arrow
\RuleSide{\pushN{S}{4}{\redb{\Set{i}}} \quad
          \Bhab{x}{\redb{S}} \vdash \pushN{T}{4}{\redb{\Set{i}}}}
         {\rulePi{4}}
         {\pushN{\PI{\V{x}}{S}{T}}{4}{\redb{\Set{i}}}}
\\
%% Inhabitant of Arrow
\RuleSide{\Bhab{x}{\redb{S}} \vdash \pushN{b}{4}{\redb{T}} }
         {\ruleLam{4}}
         {\pushN{\LAM{\V{x}}{b}}{4}{\redb{\PI{\V{x}}{S}{T}}}} 
}
} \qquad
\subfigure[Pull-side]{
\stkc{
%% Form
\boxed{\Gamma \vdash \pullN{\CN{term}}{4}{\greenb{\CN{type}}{j}{k}}}
\\
\\
%% Variable extraction
\AxiomSide{\ruleVar{4}}
          {\Bhab{x}{\redb{A}} \vdash \pullN{x}{4}{\greenb{\Up{j}{\redb{A}}}{j}{0}}}
\qquad
%% Type ascription
\RuleSide{\pushN{T}{4}{\blueb{\SetTop}} \quad
          \pushN{t}{4}{\redb{T}}}
         {\ruleAsc{4}}
         {\pullN{(\Bhab{t}{\redb{T}})}{4}{\greenb{\Up{j}{T}}{j}{0}}}
\\
%% Function application
\RuleSide{\pullN{f}{4}{\greenb{\PI{\V{x}}{\greenb{\Up{j}{S}}{j}{k}}{\greenb{\Up{T}{j}}{j}{k}}}{j}{k}} \quad
          \pushN{s}{4}{\redb{\greenb{\Up{j}{S}}{j=l}{k}}}}
         {k \in \mathbb{N} \quad \ruleApp{4}}
         {\pullN{f\: s}{4}{\greenb{\Up{j}{T}[s/x]}{j}{k+l}}}
}
}

\subfigure[Conversion]{
$$
\stkc{
%% Conversion rule
\RuleSide{\pullN{e}{4}{\greenb{\Up{j}{S}}{j}{k}} \quad
          \equalN{\greenb{\Up{j}{S}}{j}{k}}
                 {\redb{T}}
                 {4}
                 {\blueb{\SetTop}}, j = k_1}
         {\ruleConv{4}}
         {\pushN{e}{4}{\redb{T}}}
\\
\\
%% Form
\boxed{\Gamma \vdash \equalN{\greenb{\CN{term}}{j}{k}}{\redb{\CN{term}}}{4}
                            {\blueb{\CN{type}}}, j = l }
\\
\\
%% Set equality
\AxiomSide{k_1 = l \quad 
           \ruleSetKEq{4}}
          {\equalN{\greenb{\Set{k_1}}{j}{k}}{\redb{\Set{l}}}{4}{\blueb{\SetTop}}, .}
\\
\AxiomSide{l - k_1 \geq 0 \quad
           \ruleSetVEq{4}}
          {\equalN{\greenb{\Set{j+k_1}}{j}{k}}{\redb{\Set{l}}}{4}{\blueb{\SetTop}}, j = l - k_1}
\\
%% Variable equality
\AxiomSide{k_1 = l \quad \ruleVarKEq{4}}
          {\Bhab{x}{\blueb{S}} \vdash \equalN{\greenb{x_{k_1}}{j}{k}}
                                             {\redb{x_l}}
                                             {4}
                                             {\blueb{S}}, .}
\\
\AxiomSide{l - k_1 \geq 0 \quad \ruleSetVEq{4}}
          {\Bhab{x}{\blueb{S}} \vdash \equalN{\greenb{x_{j+k_1}}{j}{k}}
                                             {\redb{x_l}}
                                             {4}
                                             {\blueb{S}}, j = l - k_1}
\\
%% Pi equality
\RuleSide{\equalN{\greenb{S_1}{j}{k}}
                 {\redb{S_2}}
                 {4}
                 {\blueb{\SetTop}}, j = k_1 \quad
          \Bhab{x}{\redb{S_2}} \vdash \equalN{\greenb{T_1}{j}{k}}
                                             {\redb{T_2}}
                                             {4}
                                             {\blueb{\SetTop}}, j = k_1}
         {\rulePiEq{4}}
         {\equalN{\greenb{\PI{\V{x}}{\greenb{S_1}{j}{k}}{\greenb{T_1}{j}{k}}}{j}{k}} 
                 {\redb{\PI{\V{x}}{\redb{S_2}}{\redb{T_2}}}}
                 {4}
                 {\blueb{\SetTop}}, j = k_1}
\\
%% Function equality
\RuleSide{\Bhab{x}{\blueb{S}} \vdash \equalN{\greenb{f}{j}{k}\: x}
                                            {\redb{g}\: x}
                                            {4}
                                            {\blueb{T}}, j = k_1 }
         {\ruleLamEq{4}}
         {\equalN{\greenb{f}{j}{k}}
                 {\redb{g}}
                 {4}
                 {\blueb{\PI{\V{x}}{S}{T}}}, j = k_1}
\\
%% Application equality
\RuleSide{\equalN{\greenb{f}{j}{k}}
                 {\redb{g}}
                 {4}
                 {\blueb{\PI{\V{x}}{S}{T}}}, j = k_1 \quad
          \equalN{\greenb{a}{j}{k}}
                 {\redb{b}}
                 {4}
                 {\blueb{S}}, j = k_1}
         {\ruleAppEq{4}}
         {\equalN{\greenb{\greenb{f}{j}{k}\: \greenb{a}{j}{k}}{j}{k}}
                 {\redb{\redb{g}\: \redb{b}}}
                 {4}
                 {\blueb{T [\redb{b}/x]}}, j = k_1}
}
}
\end{center}

\caption{Semi-automatic stratisfactor}
\label{fig:semi-automatic}

\end{figure}



%% ** <- Proofs
%% *** <- Color respected - only one live variable at a time
%% ***** <- Greens under control

\begin{lemma}[Color respect and variable uniqueness]

The following scheme is preserved by the judgments presented in
Figure~\ref{fig:semi-automatic}:
\begin{itemize}
\item A push rule is colored as:
      $\pushN{\CN{term}}{4}{\redb{\CN{type}}}$,
\item A pull rule is colored as:
      $\pullN{\CN{term}}{4}{\greenb{\CN{type}}{j}{k}}$,
\item An equality judgement is colored as:
      $\equalN{\greenb{\CN{term}}{j}{k}}{\redb{\CN{term}}}{4}{\blueb{\CN{term}}}$,
      and
\item Terms (left of the $\in$ symbol, right of the $\ni$ symbol) are
      red boxed.
\end{itemize}

And at any point in time, there exists only one live green variable.

\end{lemma}

%% ***** Proof

\begin{proof}

The proof is by induction on the height of a derivation. The base
cases are $\ruleSet{4}$, $\ruleVar{4}$, $\ruleSetEq{4}$, and
$\ruleVarEq{4}$. We simply check that the color of these rules follows
the color scheme. Note that in $\ruleVar{4}$, uniqueness of variable
comes from the fact that $A$ is red (containing no variable) from the
context. Hence, creating a new variable $j$ is safe.

The induction step then consists in considering each rule in turn:
assuming that the premises are well-colored, we have to check that the
conclusion is well-colored as well. This is trivial for most
rules. The only interesting case is $\ruleApp{4}$. Applying the
induction hypothesis on the premise, we have that
$\pullN{f}{4}{\greenb{\PI{\V{x}}{\greenb{S}{j}{k}}{\greenb{T}{j}{k}}}{j}{k}}$
with $f$ red. By setting $j$ to some $l$, we turn the green $S$ into a
the red term $\redb{\greenb{S}{j=l}{k}}$ hence being coherent with the
induction hypothesis that also tells us that $s$ is red. Hence, we
deduce that $f\: s$ is red, inhabiting
$\greenb{T[s/x]}{j}{k+l}$. Substituting $s$ in $T$ preserves
uniqueness of variable, as $s$ is red, hence containing no
variable. Uniqueness of variables also require some care in the
$\ruleAsc{4}$ rule where we introduce a fresh variable: this is again
justified by the fact that the type $T$ is red, hence containing no
variable.

\end{proof}

%% *** <- Variable assigmnent equivalent to lifting on base
%% **** \Up{l}{\greenb{A}{j=0}{k}} == \greenb{A}{j=l}{k}
%% **** Lemma

\begin{lemma}[Equivalence of variable assignment and lifting]
\label{lemma:lifting-var-assgnmt}

Let $l \geq k$. Let $A$ be a green type of variable $j$:
$\greenb{A}{j}{k}$. The following equality is always verified:
%%
$$
\greenb{\Up{j}{A}}{j=l}{k} = \Up{l}{\greenb{\Up{j}{A}}{j=0}{k}}
$$

\end{lemma}

%% ***** Proof

\begin{proof}

By manipulating the definitions:
%%
$$
\greenb{\Up{j}{A}}{j=l}{k} = \Up{l}{A} = \Up{l}{\greenb{\Up{j}{A}}{j=0}{k}}
$$

\end{proof}

%% *** <- No change to the definitional equality

\begin{lemma}[Equivalence of definitional equalities]
\label{lemma:equiv-def-eq-3-4}

Let $k \in \N$.

\begin{itemize}
\item If \xspace$\equalN{\Up{l}{S}}{T}{3}{A}$, then
      $\equalN{\greenb{\Up{j}{S}}{j}{k}}{\redb{T}}{4}{\blueb{A}},j=l$
\item If \xspace$\equalN{\greenb{\Up{j}{S}}{j}{k}}{\redb{T}}{4}{\blueb{A}},j=k_1$,
      then $\equalN{\Up{k_1}{S}}{T}{3}{A}$
\end{itemize}

\end{lemma}

%% **** Proof

\begin{proof}

Both proofs are by induction on the height of a derivation. The
induction step is purely structural in both cases. The only
interesting cases are the base cases, $\ruleSetEq{}$ and
$\ruleVarEq{}$. This is simply a matter of book-keeping.

\end{proof}

%% *** <- Sound
%% **** \Gamma \vdash A \ni^4 a \Rightarrow \Gamma \vdash A \ni^3 a
%% **** \Gamma \vdash a \in^4 \greenb{A}{j}{k} \Rightarrow \forall j \leq k, \Gamma \vdash a^{(j)} \in^3 A
%% **** Theorem

\begin{theorem}[Soundness]

Let $\Gamma$ a valid context. Let $k \in \N$.

\begin{itemize}
\item If \xspace$\Gamma \vdash \pushN{a}{4}{\redb{A}}$, then 
      $\Gamma \vdash \pushN{a}{3}{A}$
\item If \xspace$\Gamma \vdash \pullN{a}{4}{\greenb{\Up{j}{A}}{j}{k}}$, then for all
      $m \geq k$, we have that $\Gamma \vdash \pullN{a^{(m)}}{4}{\Up{m}{A}}$
\end{itemize}

\end{theorem}

%% ***** Proof

\begin{proof}

First, apart from the conversion rule, the push-side has not changed
between the previous system and the new one. Thefore, we will not
treat thoses cases, the translation being the obvious one.

The proof is by induction on the height of a derivation. The base case
of interest is $\ruleVar{4}$, which gives us that $\Bhab{x}{A}$
inhabits the context. By annotation $x$ by some $m \geq 0$, we obtain
the desired result:
%%
$$
\CAxiom{\Bhab{x}{A} \vdash \pullN{x^{(m)}}{3}{\Up{m}{A}}}
$$

The induction step consists in studying $\ruleAsc{4}$, $\ruleApp{4}$,
and $\ruleConv{4}$. The translation of the $\ruleAsc{4}$ rule follows
the pattern introduced in the proof of the $\ruleVar{4}$ rule.

The $\ruleApp{4}$ rule is the following:
%%
$$
\Rule{\pullN{f}{4}{\PI{\V{x}}{\greenb{\Up{j}{S}}{j}{k}}{\greenb{\Up{j}{T}}{j}{k}}} \quad
      \pushN{s}{4}{\redb{\greenb{\Up{j}{S}}{j=l}{k}}}}
     {\pullN{f\: s}{4}{\greenb{\Up{j}{T}}{j}{k+l}}}
$$
%%
By induction hypothesis on the premises, we obtain that $\forall j
\geq k, \pullN{f^{(j)}}{3}{\PI{\V{x}}{\Up{j}{S}}{\Up{j}{T}}}$ and
$\Gamma \vdash \pushN{s}{3}{\Up{l}{S}}$. For any $m \geq k+l \geq
k$, we therefore have that
$\pullN{f^{(j)}}{3}{\PI{\V{x}}{\Up{j}{S}}{\Up{j}{T}}}$. Moreover, by
the FTP in (3), we deduce that $\Gamma \vdash \pushN{s}{3}{\Up{m}{S}}$
holds, $m$ being greater or equal to $l$. Applying $\ruleApp{3}$
on these premises, we get the desired conclusion.

Finally, let us consider the conversion rule:
%%
$$
\RuleSide{\pullN{e}{4}{\greenb{\Up{j}{S}}{j}{k}} \quad
          \equalN{\greenb{\Up{j}{S}}{j}{k}}
                 {\redb{T}}
                 {4}
                 {\blueb{\SetTop}}, j = k_1}
         {\ruleConv{4}}
         {\pushN{e}{4}{\redb{T}}}
$$
%%
By induction hypothesis on the premise, we get that $\forall j \geq k,
\pullN{e}{3}{\Up{j}{S}}$. In particular, it also holds for $j = k_1$
($k_1 \geq k$, by construction). By Lemma~\ref{lemma:equiv-def-eq-3-4},
we also deduce that $\equalN{\Up{k_1}{S}}{T}{3}{\SetTop}$. Therefore, by
application of $\ruleConv{3}$, we conclude that $\pushN{e}{3}{T}$.

\end{proof}

%% *** <- Complete
%% **** \Gamma \vdash A \ni^3 a \Rightarrow \Gamma \vdash A \ni^4 a
%% **** \Gamma \vdash a^{(l)} \in^3 A \Rightarrow \Gamma \vdash a \in^4 \greenb{A}{j=l}{0}
%% **** Theorem

\begin{theorem}[Completeness]

Let $\Gamma$ a valid context. Let $k \in \N$.

\begin{itemize}
\item If \xspace$\Gamma \vdash \pushN{a}{3}{A}$, then 
      $\Gamma \vdash \pushN{a}{4}{A}$
\item If for any $l \geq k$, we have 
      \xspace$\Gamma \vdash \pullN{a^{(l)}}{3}{\Up{l}{A}}$, then 
      $\Gamma \vdash \pullN{a}{4}{\greenb{\Up{j}{A}}{j}{k}}$
\end{itemize}

\end{theorem}

%% ***** Proof

\begin{proof}

Again, the push-side is left unchanged between both systems, at the
exception of the conversion rule. Therefore, we will not explicitly
treat those cases, which translation is the obvious one.

The proof is by induction on the height of a derivation. The
interesting base case is $\ruleVar{3}$, which gives us that
$\Bhab{x}{A}$ is in the context. We simply apply $\ruleVar{4}$ and
deduce the desired consequence:
%%
$$
\CAxiom{\Bhab{x}{A} \vdash \pullN{x}{4}{\greenb{\Up{j}{A}}{j}{0}}}
$$

The induction step is reduced to the $\ruleAsc{3}$, $\ruleApp{3}$, and
$\ruleConv{3}$ rules. The translation of $\ruleAsc{3}$ follows the one
of $\ruleVar{3}$. The translation of $\ruleApp{3}$ is slightly more
involved. The premise tells us that $\forall l \geq k,
\pullN{f^{(l)}}{3}{\PI{\V{x}}{\Up{l}{S}}{\Up{l}{T}}}$ and
$\pushN{s}{3}{\Up{l}{S}}$. By induction hypothesis, we deduce that
$\pullN{f}{4}{\greenb{\PI{\V{x}}{\Up{j}{S}}{\Up{j}{T}}}{j}{k}}$ and
$\pushN{s}{4}{\redb{S}}$. The latter is the same as
$\pushN{s}{4}{\redb{\greenb{\Up{j}{S}}{j=0}{k}}}$. Hence, by
application of $\ruleApp{4}$, we deduce the desired conclusion:
$\pullN{f\: s}{4}{\greenb{\Up{j}{T}[s/x]}{j}{k}}$.

Finally, we treat the conversion rule:
%%
$$
\Rule{\pullN{e^{(l)}}{3}{\Up{l}{S}} \quad
      \equalN{\Up{l}{S}}{T}{3}{\SetTop}}
     {\pushN{e}{3}{T}}
$$

We cannot directly apply the induction hypothesis on
$\pullN{e^{(l)}}{3}{\Up{l}{S}}$, as it is true for a specific $l$
here. However, by the FTP, we obtain that $\forall j \geq l,
\pullN{e^{(j)}}{3}{\Up{j}{S}}$. Here induction hypothesis applies and
we get that $\pullN{e}{4}{\greenb{\Up{j}{S}}{j}{l}}$. By
Lemma~\ref{lemma:equiv-def-eq-3-4}, we also have that there exists
$k_1 \leq l$ such that
$\equalN{\greenb{S}{j}{l}}{\redb{T}}{4}{\blueb{A}},j=k_1$. Hence, by
application of $\ruleConv{4}$, we obtain that $\pushN{e}{4}{\redb{T}}$

\end{proof}

%% ** /> How to find k?
%% * Automated
\section{Automatic Stratisfactor}

%% ** <- Finding k
%% *** -> Distinction between type error and size issue
%% **** <- Type error: \forall k. \Up{k}{\greenb{A}{j=0}} \not \ni s
%% **** <- Size issue: \exists k. \Up{k}{\greenb{A}{j=0}} \ni s
%% ***** <- More precisely: \exists k. \forall k' \geq k. \Up{k'}{\greenb{A}{j=0}} \ni s
%% *** -> Modify push-side to return the differential
%% **** -> A \ni^5 a, k \Leftrightarrow \forall k' \geq k, \Up{k'}{A} \ni^4 a, \Up{k-1} \not \ni^4 a (or k = 0)
%% **** -> Proof a posteriori
%% ** <- Figure: automated type system

\begin{figure}

\begin{center}
\subfigure[Push-side]{
%%
\stkc{
%% Form
\boxed{\Gamma \vdash \pushN{\CN{term}}{5}{\redb{\CN{type}}}, k}
\\
\\
%% Set
\CAxiom{\pushN{\Set{i}}{5}{\redb{\Set{j}}}, j - i - 1}
\\
%% Pi
\Rule{\pushN{S}{5}{\redb{\Set{i}}}, k_1 \quad
      \Bhab{x}{\redb{S}} \vdash \pushN{T}{5}{\redb{\Set{i}}}, k_2}
     {\pushN{\PI{\V{x}}{S}{T}}{5}{\redb{\Set{i}}}, \min k_1 k_2}
\\
%% Inhabitant of Pi
\Rule{\Bhab{x}{\redb{S}} \vdash \pushN{b}{5}{\redb{T}}, k }
     {\pushN{\LAM{\V{x}}{b}}{5}{\redb{\PI{\V{x}}{S}{T}}}, k} 
\\
%% Conversion rule
\Rule{\pullN{e}{5}{\greenb{\Up{j}{S}}{j}{k}} \quad
      \equalN{\greenb{S}{j}{k}}
             {\redb{T}}
             {5}
             {\blueb{\SetTop}}, j = k_1}
     {\pushN{e}{5}{\redb{T}}, 0}
}
$$
}
\subfigure[Pull-side]{
$$
\stkc{
%% Form
\boxed{\Gamma \vdash \pullN{\CN{term}}{5}{\greenb{\CN{type}}{j}{k}}}
\\
\\
%% Var
\CAxiom{\Bhab{x}{\redb{T}} \vdash \pullN{x}{5}{\greenb{\Up{j}{\redb{T}}}{j}{k}}} 
\\
%% Type ascription
\RuleSide{\pushN{T}{5}{\blueb{\SetTop}},k_1 \quad
          \pushN{t}{5}{\redb{T}},k_2}
         {k_1 \geq 0, k_2 \geq 0}
         {\pullN{(\Bhab{t}{\redb{T}})}{5}{\greenb{\Up{j}{T}}{j}{k}}}
\\
%% Function application
\RuleSide{\pullN{f}{5}{\greenb{\PI{\V{x}}{\greenb{\Up{j}{S}}{j}{k}}{\greenb{\Up{j}{T}}{j}{k}}}{j}{k}} \quad
          \pushN{s}{5}{\redb{\greenb{\Up{j}{S}}{j=0}{k}}}, l}
         {l' = \min l\: 0}
         {\pullN{f\: s}{5}{\greenb{\Up{j}{T}[\blackb{s}/x]}{j}{k-l'}}}
}
$$
}

\subfigure[Conversion rules]{
$$
\stkc{
%% Form
\boxed{\Gamma \vdash \equalN{\greenb{\CN{term}}{l}{k}}
                            {\redb{\CN{term}}}
                            {5}
                            {\blueb{\CN{type}}}, j = k_1}
\\
\\
%% Set equality
\stkc{
\AxiomSide{k_1 = l}
          {\equalN{\greenb{\Set{k_1}}{j}{k}}
                  {\redb{\Set{l}}}
                  {5}
                  {\blueb{\SetTop}}, .}
\qquad
\AxiomSide{l - k_1 \geq 0}
          {\equalN{\greenb{\Set{j+k_1}}{j}{k}}
                  {\redb{\Set{l}}}
                  {5}
                  {\blueb{\SetTop}}, j = l - k_1}
}
\\
%% Variable equality
\stkc{
\AxiomSide{k_1 = l}
          {\Bhab{x}{\blueb{S}} \vdash \equalN{\greenb{x_{k_1}}{j}{l}}
                                             {\redb{x_l}}
                                             {5}
                                             {\blueb{S}}, .}
\quad
\AxiomSide{l - k_1 \geq 0}
          {\Bhab{x}{\blueb{S}} \vdash \equalN{\greenb{x_{j+k_1}}{j}{k}}
                                             {\redb{x_l}}
                                             {5}
                                             {\blueb{S}}, j = l - k_1}
\\
%% Pi equality
\Rule{\equalN{\greenb{S_1}{j}{k}}
             {\redb{S_2}}
             {5}
             {\blueb{\SetTop}}, j = k_1 \quad
      \Bhab{x}{\redb{S_2}} \vdash \equalN{\greenb{T_1}{j}{k}}
                                         {\redb{T_2}}
                                         {5}
                                         {\blueb{\SetTop}}, j = k_1}
     {\equalN{\greenb{\PI{\V{x}}{\greenb{S_1}{j}{k}}{\greenb{T_1}{j}{k}}}{j}{k}}
             {\redb{\PI{\V{x}}{\redb{S_2}}{\redb{T_2}}}}
             {5}
             {\blueb{\SetTop}}, j = k_1}
\\
%% Function equality
\Rule{\Bhab{x}{\blueb{S}} \vdash \equalN{\greenb{f}{j}{k}\: x}
                                        {\redb{g}\: x}
                                        {5}
                                        {\blueb{T}}, j = k_1 }
     {\equalN{\greenb{f}{j}{k}}
             {\redb{g}}
             {5}
             {\blueb{\PI{\V{x}}{S}{T}}}, j = k_1}
\\
%% Application equality
\Rule{\equalN{\greenb{f}{j}{k}}
             {\redb{g}}
             {5}
             {\blueb{\PI{\V{x}}{S}{T}}}, j = k_1 \quad
      \equalN{\greenb{a}{j}{k}}
             {\redb{b}}
             {5}
             {\blueb{S}}, j = k_1}
     {\equalN{\greenb{\greenb{f}{j}{k}\: \greenb{a}{j}{k}}{j}{k}}
             {\redb{\redb{g}\: \redb{b}}}
             {5}
             {\blueb{T[\redb{b}/x]}}, j = k_1}
}
}
$$
}

\end{center}

\caption{Set polymorphic type theory}

\end{figure}



%% ** <- Proofs
%% *** <- Same equational theory
%% *** <- Push-side gives differential
%% **** A \ni^5 a, k \Rightarrow \forall k' \geq -\min\: k\: 0, \Up{k'}{A} \ni^4 a



\begin{lemma}[Push-side is not worst than the optimal bound]
\label{lemma:push-k-good-enough}

Let $\Gamma$ a valid context.

If \xspace$\Gamma \vdash \pushN{a}{5}{A}, k$, then forall $k'$ such
that $k' \geq -min(k,0)$ and $\Delta$ such that
$\sublift{\Gamma}{k'}{\Delta}$, we have $\Gamma \vdash
\pushN{a}{4}{\Up{k'}{A}}$. For if $\Gamma \vdash \pullN{a}{5}{A}$,
then $\Gamma \vdash \pullN{a}{4}{A}$.

\end{lemma}

%% ***** Proof

\begin{proof}

First, we do not cover the pull-side in depth: at the exception of the
$\ruleApp{5}$ rule, the judgements are identical in both systems,
hence they trivially translate from the one to the other.

We proceed by induction on the height of derivations. The base case is
the $\ruleSet{5}$ rule:
%%
$$
\CAxiom{\pushN{\Set{i}}{5}{\redb{\Set{j}}}, j - i - 1}
$$
%%
There are two cases to consider: first, if $j-i-1 \leq 0$; second, if
$j-i-1 > 0$. In the first case, we have to prove that $\forall k' \geq
i+1-j, \pushN{\Set{i}}{4}{\Up{k'}{\Set{j}}}$. This indeed the case as
$j + k' \geq j - (j - i - 1) \geq i + 1 > i$, hence satisfying the
rule $\ruleSet{4}$. In the second case, we have to prove that $\forall
k' \geq 0, \pushN{Set{i}}{4}{\Up{k'}{\Set{j}}}$. This is the case as
well, as $j + k' \geq k + i + 1 \geq i + 1 > i$.

The induction steps consists in studying the $\ruleLam{5}$,
$\ruleConv{5}$, and $\rulePi{5}$. The $\ruleLam{5}$ follows naturally
from the induction hypothesis on the premise, with $\Delta$ carefully
chosen to lift the type of $x$: $\Bhab{x}{\Up{k'}{A}}$. Applying
$\ruleLam{4}$, we get the desired outcome.

The conversion rule requires more work. We have:
%%
$$
\Rule{\pullN{e}{5}{\greenb{\Up{j}{S}}{j}{k}} \quad
      \equalN{\greenb{S}{j}{k}}{\redb{T}}{5}{\blueb{\SetTop}}, j = k_1}
     {\pushN{e}{5}{\redb{T}}, 0}
$$
%%
First, by induction hypothesis on the premise, we have that
$\pullN{e}{5}{\greenb{\Up{j}{S}}{j}{k}}$ for some $k \in \N$ and a
variable $j$. Moreover, both equational theories are the same, so we
also have that
$\equalN{\greenb{S}{j}{k}}{\redb{T}}{4}{\blueb{\SetTop}}, j = k_1$.
Applying $\ruleConv{4}$, we can readily conclude that
$\pushN{e}{4}{T}$. The FTP then gives us the expected result: 
$\forall k' \geq 0, \pushN{e}{4}{\Up{k'}{T}}$.

The last push-side case is $\rulePi{5}$. We have:
%%
$$
\Rule{\pushN{S}{5}{\redb{\Set{i}}}, k_1 \quad
      \Bhab{x}{\redb{S}} \vdash \pushN{T}{5}{\redb{\Set{i}}}, k_2}
     {\pushN{\PI{\V{x}}{S}{T}}{5}{\redb{\Set{i}}}, \min k_1 k_2}
$$
%%
By induction hypothesis on the premises, we obtain:
%%
$$
\begin{array}{l}
\forall k'_1 \geq -\min(k_1, 0), \Gamma \vdash \pushN{S}{4}{\Up{k'_1}{\Set{i}}} \\
\forall k'_2 \geq -\min(k_2, 0), \Gamma, \Bhab{x}{S} \vdash \pushN{T}{4}{\Up{k'_2}{\Set{i}}}
\end{array}
$$

Finally, we shall treat the only interesting pull-side rule:
$\ruleApp{5}$. We have that:
%%
$$
\RuleSide{\pullN{f}{5}{\greenb{\PI{\V{x}}{\greenb{\Up{j}{S}}{j}{k}}{\greenb{\Up{j}{T}}{j}{k}}}{j}{k}} \quad
          \pushN{s}{5}{\redb{\greenb{\Up{j}{S}}{j=0}{k}}}, l}
         {l' = \min l\: 0}
         {\pullN{f\: s}{5}{\greenb{\Up{j}{T}[\blackb{s}/x]}{j}{k-l'}}}
$$
%% 

By induction hypothesis on the premises, we have that
$\pullN{f}{4}{\greenb{\PI{\V{x}}{\greenb{\Up{j}{S}}{j}{k}}{\greenb{\Up{j}{T}}{j}{k}}}{j}{k}}$
and $\pushN{s}{4}{\Up{-l'}{\redb{\greenb{\Up{j}{S}}{j=0}{k}}}}$. By
Lemma~\ref{lemma:lifting-var-assgnmt}, we get that
$\pushN{s}{4}{\redb{\greenb{\Up{j}{S}}{j=-l'}{k}}}$. Therefore, we can
apply $\ruleApp{4}$ and deduce the expected conclusion: 
$\pullN{f\: s}{4}{\greenb{\Up{j}{T}[\blackb{s}/x]}{j}{k-l'}}$.

\end{proof}

%% *** <- Tight bound gives difference of zero
%% **** \greenb{S}{j=k}{0} \ni^4 s \Rightarrow \greenb{S}{j=k}{0} \ni^5 s, 0

\begin{lemma}[Push-side captures the tight bound]
\label{lemma:push-k-tight}


Let $\Gamma$ a valid context. 

If \xspace$\pushN{s}{4}{\redb{\greenb{\Up{j}{S}}{j=l}{k}}}$ where $l
\in \N$ is the minimal assignment, then
$\pushN{s}{5}{\redb{\greenb{\Up{j}{S}}{j=0}{k}}}, -l$. For if
$\pullN{s}{4}{S}$, then $\pullN{s}{5}{S}$.

\end{lemma}

%% ***** Proof

\begin{proof}

Again, at the exception of the $\ruleApp{4}$ rule, the pull-side is
identical in both systems, hence we take the trivial translation. The
proof is by induction on the height of derivations. The base case is
$\ruleSet{4}$. By definition of $l$, the derivation is the following:
%%
$$
\CAxiom{\pushN{\Set{i}}{4}{\Set{i+1}}}
$$
%%
Which translates to the desired judgement in (5):
%%
$$
\CAxiom{\pushN{\Set{i}}{5}{\Set{i+1}}, 0}
$$

The induction step consists in treating the cases $\ruleLam{4}$,
$\rulePi{4}$, and $\ruleConv{4}$ in turn. The treatment of the
$\ruleLam{4}$ and $\rulePi{4}$ rules is purely structural. Similarly,
the conversion rule follows by converting the pull-premise to itself
in (5) while the equality judgement carry over to the new system
unchanged. We then conclude by applying $\ruleConv{5}$ that directly
indicate a differential of zero. The $\ruleApp{4}$ rule follows
naturally, as the induction hypothesis on the premises give directly
the premises of the $\ruleApp{5}$ rule.

\end{proof}

%% *** <- Sound
%% **** \Gamma \vdash A \ni^5 a, k \wedge k \geq 0 \Rightarrow \Gamma \vdash A \ni^4 a
%% **** \Gamma \vdash a \in^5 A \Rightarrow \Gamma \vdash a \in^4 A
%% **** Theorem

\begin{theorem}[Soundness]

Let $\Gamma$ be a valid context.

\begin{itemize}
\item If \xspace$\Gamma \vdash \pushN{a}{5}{A}, k$ and $k \geq 0$, then $\Gamma \vdash \pushN{a}{4}{A}$
\item If \xspace$\Gamma \vdash \pullN{a}{5}{A}$, then $\Gamma \vdash \pullN{a}{4}{A}$
\end{itemize}

\end{theorem}

%% ***** Proof

\begin{proof}

The proof is by induction on the length of a derivation. The base
cases are $\ruleVar{5}$ and $\ruleSet{5}$. The $\ruleVar{}$ rule
translates directly to its counterpart. In the $\ruleSet{5}$ case, we
know that the differential $k$ is greater or equal to $0$. Therefore,
we can safely deduce:
%%
$$
\CAxiom{\pushN{\Set{i}}{4}{\Set{j}}}
$$

The induction step is trivial for most pull-side rules (they have
their equivalent in the target system), at the exception of the
$\ruleApp{5}$ rule. On the push-side, we have to treat the
$\ruleLam{5}$, $\ruleConv{5}$, and $\rulePi{5}$. The $\ruleLam{5}$ and
$\ruleConv{5}$ cases follow structurally. In the $\rulePi{5}$ case,
the hypothesis gives us that $\min(k_1, k_2) \geq 0$. We thus deduce
that $k_1 \geq 0$ and $k_2 \geq 0$ and can apply the induction
hypothesis on the premises and structurally obtain the desired
conclusion. 

The application rule calls for some work. We have:
%%
$$
\RuleSide{\pullN{f}{5}{\greenb{\PI{\V{x}}{\greenb{\Up{j}{S}}{j}{k}}{\greenb{\Up{j}{T}}{j}{k}}}{j}{k}} \quad
          \pushN{s}{5}{\redb{\greenb{\Up{j}{S}}{j=0}{k}}}, l}
         {l' = \min(l,0)}
         {\pullN{f\: s}{5}{\greenb{\Up{j}{T}[\blackb{s}/x]}{j}{k-l'}}}
$$
%%
By induction hypothesis, we deduce that
$\pullN{f}{4}{\greenb{\PI{\V{x}}{\greenb{\Up{j}{S}}{j}{k}}{\greenb{\Up{j}{T}}{j}{k}}}{j}{k}}$. By
Lemma~\ref{lemma:push-k-good-enough}, we also have that
$\pushN{s}{5}{\Up{-l'}{\redb{\greenb{\Up{j}{S}}{j=0}{k}}}}$. By
Lemma~\ref{lemma:lifting-var-assgnmt}, we deduce that
$\pushN{s}{5}{\redb{\greenb{\Up{j}{S}}{j=-l'}{k}}}$. All conditions
are met to apply the $\ruleApp{4}$ rule.

\end{proof}


%% *** <- Complete
%% **** \Gamma \vdash A \ni^4 a \Rightarrow \Gamma \vdash A \ni^5 a, k \wedge k \geq 0
%% **** \Gamma \vdash a \in^4 A \Rightarrow \Gamma \vdash a \in^5 A
%% **** Theorem

\begin{theorem}[Completeness]

Let $\Gamma$ be a valid context.

\begin{itemize}
\item If \xspace$\Gamma \vdash \pushN{a}{4}{A}$, then $\Gamma \vdash \pushN{a}{5}{A}, k$ with $k \geq 0$
\item If \xspace$\Gamma \vdash \pullN{a}{4}{A}$, then $\Gamma \vdash \pullN{a}{5}{A}$
\end{itemize}

\end{theorem}

%% ***** Proof

\begin{proof}

Most pull-side rules are identical on the source and target side, at
the exception of the application rule. We will therefore ignore them,
assuming the naive translation.

The proof is by induction on the height of derivations. The base case
of interest is $\ruleSet{4}$. The conclusion is easily verified from
the requirement on levels.

The induction step is unsurprisingly structural for the $\rulePi{4}$,
$\ruleLam{4}$, and $\ruleConv{4}$ rules. The $\ruleApp{4}$ rule tells
us that:
%%
$$
\RuleSide{\pullN{f}{4}{\greenb{\PI{\V{x}}{\greenb{\Up{j}{S}}{j}{k}}{\greenb{\Up{T}{j}}{j}{k}}}{j}{k}} \quad
          \pushN{s}{4}{\redb{\greenb{\Up{j}{S}}{j=l}{k}}}}
         {k \in \mathbb{N} \quad \ruleApp{4}}
         {\pullN{f\: s}{4}{\greenb{\Up{j}{T}[s/x]}{j}{k+l}}}
$$
%%
By induction hypothesis, we can readily conclude that
$\pullN{f}{5}{\greenb{\PI{\V{x}}{\greenb{\Up{j}{S}}{j}{k}}{\greenb{\Up{T}{j}}{j}{k}}}{j}{k}}$. By
Lemma~\ref{lemma:push-k-tight}, we deduce that
$\pushN{s}{5}{\redb{\greenb{\Up{j}{S}}{j=0}{k}}}, -l$. Therefore, we
can apply $\ruleApp{5}$ and obtain the expected conclusion:
$\pullN{f\: s}{5}{\greenb{\Up{j}{T}[s/x]}{j}{k+l}}$.

\end{proof}

%% * Polymorphic definitions
%% ** <- Definitions: references
%% *** <- fully lambda-lifted representation
%% **** <- ???
%% *** -> Used through the Var rule
%% **** -> Automatically polymorphic
%% ** <- Examples
%% *** id
%% *** typeof
%% *** typeof/id (Eelis)
%% *** subtyping
%% ** <- Counter-examples (leading to paradoxes)
%% *** Define set : Set, typecheck set : set
%% * With data-types
%% ** <- Based on Levitation
%% *** <- (Brief sum up of levitation)
%% *** /> Work just as well with external presentation
%% **** /> More challenging in our case
%% ** <- Assign levels to things of Desc
%% *** <- data IDesc (I : Set1) : Set1
%% *** <- desc : {I : Set1} -> IDesc I -> (I -> Set0) -> Set0
%% *** <- data IMu {I : Set1}(R : I -> IDesc I)(i : I) : Set0
%% *** <- box : {I : Set1}(D : IDesc I)(P : I -> Set0) -> desc D P -> IDesc (Sigma I P)
%% *** <- induction : (...)
%% *** -> Remark that Mu decreases
%% ** <- Example: Nat : Set0
%% *** <- Canonical object
%% *** -> Get automatically Nat at all levels
%% ** <- Example: List : Set0 -> Set0
%% *** <- Parameterized object
%% *** -> Get List translating with its argument
%% *** -> [ Set0, Set1, Set2 ] : List Set3
%% ** <- Example: Vector : (n : N) -> Set0 -> Set0
%% *** <- Indexed object
%% *** -> Index, canonical, moves as well
%% ** <- Example: IDesc in IDesc
%% *** <- Levitation trick
%% *** -> Fall back to principal type!

%% * Related Work
%% ** Palmgren
%% ** Harper-Pollack
%% ** Courant
%% ** Agda experimental universe polymorphism
%% * Conclusion
\section{Conclusion}

%% ** We did
%% ** Future work

%% * Bibliography

\bibliography{stratisfaction}
\bibliographystyle{abbrvnat}

% The bibliography should be embedded for final submission.
%\begin{thebibliography}{}
%\softraggedright
%\end{thebibliography}

%% * Outro

\end{document}



%% Local Variables:
%% mode: outline-minor
%% outline-regexp: "%% [*\f]+"
%% outline-level: outline-level
%% End:

