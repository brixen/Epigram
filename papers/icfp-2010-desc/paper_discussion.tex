\section{Discussion}
\label{sec:discussion}

\subsection{Universe stratification}

\begin{wstructure}
!!! Need Help !!!
<- Universe stratification
    <- Stratified agda model
        <- Fully stratified
        <- Proof of iso between host and embedding
    ???
\end{wstructure}

As presented, our type theory suffers from a major weakness. Indeed,
we are subject to Girard's paradox, as we assume that $\Set$ lives in
$\Set$. We made that choice for presentational convenience, as
universe stratification is orthogonal to our work. Nonetheless, our
universes of description rather naturally leads itself to
stratification. Unsurprisingly, $\IDesc{\!}$ at level $l$ is of type
$\Set^{\blue{l+1}}$. Similarly, the interpretation of $\IDesc{\!}$ at
level $l$ is an object of type $\Set^{\blue{l}}$:

\[\stk{
\data \IDesc{\!}^{\blue{l}} (\Bhab{\V{I}}{\Set^{\blue{l+1}}}) : \Set^{\blue{l+1}} \where \\
\;\;\ldots \\
\\
\idescop{\_\:}{}{}^{\blue{l}} : \PI{\V{I}}{\Set^{\blue{l+1}}} \IDesc{{\!}^{\blue{l}}\V{I}} \To (\V{I} \To \Set^{\blue{l}}) \To \Set^{\blue{l}}    \\
\ldots
}\]

Consequently, we can implement the operations and examples developed
above. We refer the reader to our Agda implementation, which take
advantage of set polymorphism to implement the universe of indexed
descriptions at any level. Further, we have coded $\IDesc{\!}$ in
itself and have proved the isomorphism between the host and the
embedded universes.

\subsection{Related Work}

\begin{structure}
!!! Need Help !!!
<- Comparison with Induction Recursion
    ???
\end{structure}


\begin{wstructure}
!!! Need Help !!!
<- Related Work
    <- Generic in simply-typed functional languages
        <- PolyP \cite{jansson:polyp}
        <- Generic Haskell \cite{hinze:generic-haskell}
        <- Scratch your boilerplate \cite{spj:syb}
\end{wstructure}

Generic programming is a vast topic. We refer our reader to Garcia et
al.~\cite{garcia:generic-comparative-study} for a broad overview of
generic support in various languages. In the sole context of Haskell,
there is a myriad of proposals. These approaches are presented and
compared in Hinze et al.~\cite{hinze:generic-approach-comparative} and
Rodriguez et al.~\cite{rodriguez:generic-libs-comparative}.

In particular, our approach is similar in spirit with polytypic
programming, as initiated by PolyP~\cite{jansson:polyp}. Indeed,
generic functions, in our system, are built by induction on the
pattern functor. Unlike PolyP, we do not have to resort to
pre-processing code: our data-types are, natively, nothing but codes.

Our approach also support the Generic
Haskell~\cite{hinze:generic-haskell} model. This model, based on
type-indexed data types, corresponds to computing new data-types from
others. This is natural in our system, as data-types descriptions are
first-class.

Interestingly, our generic induction principle allows a programming
style reminiscent to the Scrap Your Boilerplate~\cite{spj:syb} (SYB)
approach. Like SYB, we provide a generic operator to manipulate and
carry operations over data-types. However, unlike SYB, we do not need
support from the compiler, as introduced by the $\CN{Typeable}$
class. Indeed, in our setting, data-types are reflected in the
language, hence our ability to manipulate them directly.

\begin{wstructure}
    <- Generic in dependent types
        <- Norell \cite{norell:msc-thesis}
        <- Polytypic prog in Coq \cite{verbruggen:polytype-coq}
        <- Universes for generic prog \cite{benke:universe-generic-prog}
\end{wstructure}

The interest in generic programming for dependent types is not new
either. Norell~\cite{norell:msc-thesis} have shown the benefit of
polytypic programming in the setting of Alfa, a precursor of
Agda. Similarly, Verbruggen et al.~\cite{verbruggen:polytype-prog-coq,
  verbruggen:polytype-coq} have developed a framework for polytypic
programming in the Coq theorem prover. However, these works aim at
\emph{modelling} Generic Haskell in a dependently-typed setting, for
the purpose of proving correctness properties of Haskell code. Our
approach is different in that we aim at building a foundation for
data-types in a dependently typed theory.

Closer to us is the work by Benke et
al.~\cite{benke:universe-generic-prog}. This seminal work introduced
the usage of universes for developing generic programs. Moreover, our
own universes are rather similar to theirs: our universe of
descriptions is similar to their universe of iterated induction, and
our universe of indexed descriptions is isomorphic to their universe of
finitary indexed induction. This is not surprising, as we share the
same source of inspiration, namely induction recursion.

However, we differ in several ways. First, they adopt a generative
perspective: each universe extends the base type theory with both type
formers and elimination rules. Thanks to the levitation, we only rely
on a generic induction and a specialised
$\switchD{\!}{\!}{\!}$. Second, the authors develop several universes,
each with its own strengths and weaknesses, but lack a formal,
unifying universe. With indexed descriptions, we subsume simple
descriptions, hence avoiding the move to and from several
codings. Finally, the authors often resort to an extensional notion of
equality, while we have given an equality-agnostic
presentation. Moreover, our presentation is arranged so as to use
definitional equality as much as possible. Hence, in practice, the
developer is relieved from many trivial proof obligations.
