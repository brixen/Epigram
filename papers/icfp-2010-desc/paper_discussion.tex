\section{Discussion}
\label{sec:discussion}

\subsection{Universe stratification}

\begin{wstructure}
!!! Need Help !!!
<- Universe stratification
    <- Stratified agda model
        <- Fully stratified
        <- Proof of iso between host and embedding
    ???
\end{wstructure}

As such, our type theory suffers from an inconsistency. Indeed, the
typing rule $\Bhab{\Set}{\Set}$ leads to Girard's paradox. We made
that choice for presentational convenience, as universe stratification
is orthogonal to our work. Nonetheless, our universe of description
stratifies naturally. $\SYMBIDesc$ is self-encoding only in a
level-polymorphic sense.  Unsurprisingly, $\SYMBIDesc$ at level $l$ is
of type $\Set^{\blue{l+1}}$. Similarly, the interpretation of
$\SYMBIDesc$ at level $l$ is an object of type $\Set^{\blue{l}}$:
%
\[\stk{\begin{array}{@{}ll}
\SYMBIDesc^{\blue{l}} (\Bhab{\M{I}}{\Set^{\blue{l+1}}}) & :\Set^{\blue{l+1}} \\
\DVar{(\Bhab{\M{i}}{\M{I}})} &: \SYMBIDesc^{\blue{l}}{\M{I}} \\
\DConst{(\Bhab{\M{A}}{\Set^{\blue{l}}})} &:\SYMBIDesc^{\blue{l}}{\M{I}}       \\
\DProd{(\Bhab{\M{D}}{\SYMBIDesc^{\blue{l}}{\M{I}}})}{(\Bhab{\M{D}}{\SYMBIDesc^{\blue{l}}{\M{I}}})}
  & :\SYMBIDesc^{\blue{l}}{\M{I}}       \\
\DSigma{(\Bhab{\M{S}}{\Set^{\blue{l}}})}{(\Bhab{\M{D}}{\M{S}\To\SYMBIDesc^{\blue{l}}{\M{I}}})}
& : \SYMBIDesc^{\blue{l}}{\M{I}}  \\
\DPi{(\Bhab{\M{S}}{\Set^{\blue{l}}})}{(\Bhab{\M{D}}{\M{S}\To\SYMBIDesc^{\blue{l}}{\M{I}}})}
& : \SYMBIDesc^{\blue{l}}{\M{I}}  \\
\end{array}\smallskip \\
\idescop{\_\:}{}{}^{\blue{l}} : _{\PI{\V{I}}{\Set^{\blue{l+1}}}} \SYMBIDesc^{\blue{l}}\:\V{I} \To (\V{I} \To \Set^{\blue{l}}) \To \Set^{\blue{l}}    \\
\ldots
}
\]
Crucially, the types of data stored in an
\(\SYMBIDesc^{\blue{l}}{\M{I}}\) all live no higher---we may store an
\(I\) and a \(\Set^{\blue{l}}\) in a \(\Set^{\blue{l+1}}\).  The code
for \(\SYMBIDesc^{\blue{l}}{\M{I}}\) is an element of
\(\SYMBIDesc^{\blue{l+1}}{\Unit}\), so there is a spiral, not a cycle.
We have checked the construction using Agda's universe polymorphism,
coding $\SYMBIDesc$ in itself and have proving the isomorphism
between the host and the embedded universes.

\subsection{Related work}

\begin{wstructure}
!!! Need Help !!!
<- Comparison with Induction Recursion
    ???
\end{wstructure}


\begin{wstructure}
!!! Need Help !!!
<- Related Work
    <- Generic in simply-typed functional languages
        <- PolyP \cite{jansson:polyp}
        <- Generic Haskell \cite{hinze:generic-haskell}
        <- Scratch your boilerplate \cite{spj:syb}
\end{wstructure}

Generic programming is a vast topic. We refer our reader to
\citet{garcia:generic-comparative-study} for a broad overview of
generic programming in various languages. In the sole context of
Haskell, there is a myriad of proposals. These approaches are compared
in \citet{hinze:generic-approach-comparative} and
\citet{rodriguez:generic-libs-comparative}.

Our approach is follow the polytypic programming style, as initiated
by PolyP~\cite{jansson:polyp}. Indeed, we build generic functions by
induction on pattern functors. Unlike PolyP, we do not have to resort
to preprocessing: our datatypes are, natively, nothing but codes.

We share with Generic Haskell the \emph{type-indexed datatype}
approach~\cite{hinze:generic-haskell}, as exemplified by the free
monad construction: from datatype, we can compute new datatypes and
equip them with their structure. Generic Haskell also features
\emph{generic views}~\cite{holdermans:generic-view}, transparently
transforming the structure of datatype definitions. An example is the
tagged descriptions, presenting datatypes under a sum-of-sigmas
angle. Unlike Generic Haskell, we do not have to modify the compiler
to obtain views on datatypes: we can massage descriptions from inside
our language.

Unlike Generic Haskell, we do not support polykinded
programming~\cite{hinze:polytypic-polykinded}. Our descriptions are
limited to endofunctors on $\Set$ and $\Set^I$. While we could
\emph{encode} higher-kinded datatypes, we do not plan to adopt this
strategy. As future work, we plan to extend our universe to capture
higher-kinded definitions and generic functions over them. For the
same reason, arity-generic programming~\cite{weirich:arity-generic} is
out of reach of our current presentation.

Another generic programming paradigm is Scrap Your
Boilerplate~\cite{spj:syb} (SYB). Our proposal is different in various
ways. The corner stone of SYB is the \emph{spine} view of datatype
constructors. A piece of data is a spine composed by a constructor
applied to some arguments. SYB provides a combinator library to write
generic functions over spines. This relies on a $\CN{Typeable}$
type-class, allowing dynamic dispatch to datatype-specific
operations. As a result, SYB is not reflexive: it is restricted to
datatypes instanciating $\CN{Typeable}$. Moreover, it is limited to
building generic functions, hence type-indexed datatypes cannot be
implemented in this framework.


\begin{wstructure}
    <- Generic in dependent types
        <- Norell \cite{norell:msc-thesis}
        <- Polytypic prog in Coq \cite{verbruggen:polytype-coq}
        <- Universes for generic prog \cite{benke:universe-generic-prog}
\end{wstructure}

Generic programming in dependent types is not new
either. \citet{norell:msc-thesis} has given a formalization of
polytypic programming in Alfa, a precursor of Agda. Similarly,
\citet{verbruggen:polytype-prog-coq, verbruggen:polytype-coq} have
developed a framework for polytypic programming in the Coq theorem
prover. However, these works aim at \emph{modelling} PolyP or Generic
Haskell in a dependently-typed setting for the purpose of proving
correctness properties of Haskell code. Our approach is different in
that we aim at building a foundation for datatypes, in a
dependently-typed system, for a dependently-typed system.

Closer to us is the work of \citet{benke:universe-generic-prog}. This
seminal work introduced the usage of universes for developing generic
programs. Our universes share similarities to theirs: our universe of
descriptions is similar to their universe of iterated induction, and
our universe of indexed descriptions is equivalent to their universe
of finitary indexed induction. This is not surprising, as we share the
same source of inspiration, namely induction-recursion.

However, we differ in several ways. First, there approach is
generative: each universe extends the base type theory with both type
formers and elimination rules. Thanks to levitation, we only rely on a
generic induction and a specialised $\SYMBswitchD$. Second,
the authors do not tackle the issue of \emph{programming} with
codes. We have shown how to abstract away codes and give a convenient
presentation to the developer. The authors often resort to an
extensional equality, while we have given an equality-agnostic
presentation. Beside, our universes are arranged so as to use
definitional equality as much as possible. Hence, in practice, the
developer is relieved from many proof obligations.
