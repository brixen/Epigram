\section{Discussion}
\label{sec:discussion}

\subsection{Universe stratification}

\begin{wstructure}
!!! Need Help !!!
<- Universe stratification
    <- Stratified agda model
        <- Fully stratified
        <- Proof of iso between host and embedding
    ???
\end{wstructure}

As presented, our type theory suffers from a major weakness. Indeed,
we are subject to Girard's paradox, as we assume that $\Set$ lives in
$\Set$. We made that choice for presentational convenience, as
universe stratification is orthogonal to our work. Nonetheless, our
universes of description rather naturally leads itself to
stratification. Unsurprisingly, $\IDesc{}$ at level $l$ is of type
$\Set^{\blue{l+1}}$. Similarly, the interpretation of $\IDesc{}$ at
level $l$ is an object of type $\Set^{\blue{l}}$:

\[\stk{
\data \IDesc{\!}^{\blue{l}} (\Bhab{\V{I}}{\Set^{\blue{l+1}}}) : \Set^{\blue{l+1}} \where \\
\;\;\ldots \\
\\
\idescop{\_\:}{}{}^{\blue{l}} : \PI{\V{I}}{\Set^{\blue{l+1}}} \IDesc{{\!}^{\blue{l}}I} \To (\V{I} \To \Set^{\blue{l}}) \To \Set^{\blue{l}}    \\
\ldots
}\]

Consequently, we can implement the operations and examples developed
above. We refer the reader to our Agda implementation, which take
advantage of set polymorphism to implement the universe of indexed
descriptions at any level. Further, we have coded $\IDesc{}$ in itself
and have proved the isomorphism between the host and the embedded
universes.

\subsection{Related Work}

\begin{structure}
!!! Need Help !!!
<- Comparison with Induction Recursion
    ???
\end{structure}


\begin{wstructure}
!!! Need Help !!!
<- Related Work
    Who? What?
\end{wstructure}

