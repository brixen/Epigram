\section{Discussion}
\label{sec:discussion}

\subsection{Universe stratification}

\begin{wstructure}
!!! Need Help !!!
<- Universe stratification
    <- Stratified agda model
        <- Fully stratified
        <- Proof of iso between host and embedding
    ???
\end{wstructure}

As presented, our type theory suffers from a major weakness. Indeed,
we are subject to Girard's paradox, as we assume that $\Set$ lives in
$\Set$. We made that choice for presentational convenience, as
universe stratification is orthogonal to our work. Nonetheless, our
universes of description rather naturally leads itself to
stratification. Unsurprisingly, $\IDesc{\!}$ at level $l$ is of type
$\Set^{\blue{l+1}}$. Similarly, the interpretation of $\IDesc{\!}$ at
level $l$ is an object of type $\Set^{\blue{l}}$:

\[\stk{
\data \IDesc{\!}^{\blue{l}} (\Bhab{\V{I}}{\Set^{\blue{l+1}}}) : \Set^{\blue{l+1}} \where \\
\;\;\ldots \\
\\
\idescop{\_\:}{}{}^{\blue{l}} : \PI{\V{I}}{\Set^{\blue{l+1}}} \IDesc{{\!}^{\blue{l}}\V{I}} \To (\V{I} \To \Set^{\blue{l}}) \To \Set^{\blue{l}}    \\
\ldots
}\]

The operations and examples developed in this paper stratify just as
well. We refer the reader to our Agda model, which takes advantage of
set polymorphism to implement the universe of indexed descriptions at
any level. Further, we have coded $\IDesc{\!}$ in itself and have
proved the isomorphism between the host and the embedded universes.

\subsection{Related Work}

\begin{structure}
!!! Need Help !!!
<- Comparison with Induction Recursion
    ???
\end{structure}


\begin{wstructure}
!!! Need Help !!!
<- Related Work
    <- Generic in simply-typed functional languages
        <- PolyP \cite{jansson:polyp}
        <- Generic Haskell \cite{hinze:generic-haskell}
        <- Scratch your boilerplate \cite{spj:syb}
\end{wstructure}

Generic programming is a vast topic. We refer our reader to
\citet{garcia:generic-comparative-study} for a broad overview of
generic support in various languages. In the sole context of Haskell,
there is a myriad of proposals. These approaches are presented and
compared in \citet{hinze:generic-approach-comparative} and
\citet{rodriguez:generic-libs-comparative}.

In particular, our approach is similar in spirit with polytypic
programming, as initiated by PolyP~\cite{jansson:polyp}. Indeed,
generic functions, in our system, are built by induction on the
pattern functor. Unlike PolyP, we do not have to resort to pre-process
data-type definitions: our data-types are, natively, nothing but
codes.

We share with Generic Haskell the \emph{type-indexed data type}
approach~\cite{hinze:generic-haskell}, as exemplified by the free
monad construction: from data-type, we can compute new data-types and
equip them with their structure. Generic Haskell also features
\emph{generic views}~\cite{holdermans:generic-view}, transparently
transforming the structure of data-type definitions, to ease the
implementation of generic functions over them. Views in
dependently-typed system are widely
used~\cite{mcbride.mckinna:view-from-the-left}. Unlike in Generic
Haskell, we do not have to modify the compiler to develop views on
data-types: we can massage description codes from inside the language
and present them as a standard view. An example of such operation is
the tagged descriptions, presenting data-types under a particular
angle.

Unlike Generic Haskell, we do not support polykinded
programming~\cite{hinze:polytypic-polykinded}. Our descriptions are
limited to endofunctor on $\Set$ and $\Set^I$. While it is possible to
\emph{encode} higher-kinded data-types in our setting, we do not plan
to adopt this strategy. Rather, as future work, we plan to extend our
universe to capture higher-kinded definitions and generic functions
over them. For the same reason, arity-generic
programming~\cite{weirich:arity-generic} is out of reach of our
current presentation.

Another generic programming framework is Scrap Your
Boilerplate~\cite{spj:syb} (SYB). Our proposal is different in various
ways. The corner stone of SYB is the \emph{spine} view of data-type
constructors. A data-type is a spine composed by a constructor applied
to some arguments. Further, this spine is equipped with some
combinators including, primarily, an iterator. In this setting,
generic programs are written by composing these combinators. This
relies on a $\CN{Typeable}$ type-class, allowing dynamic dispatch to
data-type specific operations. As a result, SYB is not reflexive: it
is restricted to data-types instanciating $\CN{Typeable}$. Moreover,
it forbids type-indexed data types: it can only define generic
functions.


\begin{wstructure}
    <- Generic in dependent types
        <- Norell \cite{norell:msc-thesis}
        <- Polytypic prog in Coq \cite{verbruggen:polytype-coq}
        <- Universes for generic prog \cite{benke:universe-generic-prog}
\end{wstructure}

The interest in generic programming for dependent types is not new
either. Norell~\cite{norell:msc-thesis} have given a formalization of
polytypic programming in the setting of Alfa, a precursor of
Agda. Similarly, \citet{verbruggen:polytype-prog-coq,
  verbruggen:polytype-coq} have developed a framework for polytypic
programming in the Coq theorem prover. However, these works aim at
\emph{modelling} PolyP or Generic Haskell in a dependently-typed
setting, for the purpose of proving correctness properties of Haskell
code. Our approach is different in that we aim at building a
foundation for data-types, in a dependently-typed theory, for a
dependently-typed system.

Closer to us is the work by \citet{benke:universe-generic-prog}. This
seminal work introduced the usage of universes for developing generic
programs. Our own universes are rather similar to theirs: our universe
of descriptions is similar to their universe of iterated induction,
and our universe of indexed descriptions is isomorphic to their
universe of finitary indexed induction. This is not surprising, as we
share the same source of inspiration, namely induction recursion.

However, we differ in several ways. First, they adopt a generative
perspective: each universe extends the base type theory with both type
formers and elimination rules. Thanks to the levitation, we only rely
on a generic induction and a specialised
$\switchD{\!}{\!}{\!}$. Second, the authors do not tackle the issue of
\emph{programming} with such codes: it is not obvious, to us, how one
can define, manipulate, and program over these coded data-types. With
descriptions, we have shown how to abstract away codes and present a
convenient and familiar presentation to the developer. Finally, the
authors often resort to an extensional notion of equality, while we
have given an equality-agnostic presentation. Beside, our presentation
is arranged so as to use definitional equality as much as
possible. Hence, in practice, the developer is relieved from many
proof obligations.
