\section{Levitating the Universe of Descriptions}
\label{sec:desc-levitate}

In this section, we will fulfil our promises and show how we implement
the signatures, first for the enumerations, and then for the codes of
the $\Desc^n$ universe.  Persuading these programs to perform was a perilous
pedagogical peregrination for the protagonist.  Our method was indeed to
hardwire constants implementing the signatures specified above, in the
first instance, but then attempt to replace them, step by step, with
\emph{definitions}: ``Is \(2+2\) still \(4\)?'', ``No, it's a loop!''.
But we did find a way, so now we hope to convey to you the
dizzy feeling of levitation, without the falling.



\subsection{Implementing finite enumerations}

\begin{wstructure}
<- Recall typing rules of 1st section
    -> Make clear they were just promises
    -> Can be implemented now
        <- Simply List UId
\end{wstructure}

In Section~\ref{sec:finite-sets}, we specified the
finite sets of tags. We are going to implement (at every universe level)
the $\EnumU$ type former and its constructors. Recall:
%
\[\Type{\EnumU}^n\qquad\Bhab{\NilE}{\EnumU}\qquad
\Bhab{\ConsE{(\Bhab{\M{t}}{\UId})}{(\Bhab{\M{E}}{\EnumU})}}{\EnumU}
\]
%
The $\NilE$ and $\SYMBConsE$ constructors are just the `nil' and
`cons' or ordinary lists, with elements from \(\UId\).
Therefore, we
implement:
%
\[
\EnumU \mapsto \Mu{(\ListD\: \UId)}
\qquad
\NilE \mapsto \ListNil
\qquad
\ConsE{\M{t}}{\M{E}} \mapsto \ListCons{\M{t}}{\M{E}}
\]


\begin{wstructure}
<- Consequences
    -> Type theory doesn't need to be extended with EnumU, NilE, and ConsE
        <- EnumU == Mu EnumUD
        <- NilE, ConsE are just tags
    -> Do not need a specific \spi eliminator
        <- \spi is an instance of the generic eliminator
            <- Code?
    -> Anything else remains the same (switch, EnumT, 0, 1+)
\end{wstructure}

Let us consider the consequences. We find that the type
theory does not need a special type former $\EnumU$,
or special constructors $\NilE$ and $\SYMBConsE$. 
Moreover, the $\spi{\M{E}}{\M{P}}$ operator, computing tuple types
of \(\M{P}\)s by recursion on \(\M{E}\) need not be hardwired: we can
just use the generic $\F{ind}$ operator, as we would for any ordinary
program.

Note, however, that the universe decoder \(\EnumT{\M{E}}\) \emph{is}
hardwired, as are the primitive \(\Ze\) and \(\Su\) that
we use for low-level values, and indeed the \(\SYMBswitch\) operator.
We cannot dispose of data altogether! We have, however, gained
the ordinariness of the enumeration \emph{codes}, and hence of generic
programs which manipulate them. Our next step is similar: we are going to
condense the entire naming scheme of datatypes \emph{into itself}.

%% Apart from these changes, we are left with implementing the other
%% components of finite sets. This consists of $\EnumT$, $\Ze$, $\Su$, as
%% well as the $\F{switch}$ operator. Note that the actual implementation
%% of $\EnumU$ does not influence our implementation of $\EnumT$: be they
%% hard-coded or codes in the $\Desc$ universe, the $\EnumU$ objects
%% behave the same. We have witnessed this effect when carrying the
%% operation in Epigram, moving from an hard-coded presentation to a
%% self-hosted one. Absolutely no change to the $\EnumT$ objects was
%% required.

\begin{wstructure}
<- Summary of the operation
    <- The content of the type theory is exactly the same
        <- before == after
    /> type naming scheme condenses
        <- Replace named constructors by codes in the universe of datatypes
    -> Our next step is a similar move (in essence)
        /> Condenses the entire naming scheme of datatypes
\end{wstructure}

%In this section, we have replaced a low-level presentation of finite
%sets by a self-hosted one, expressed in the universe of
%descriptions. However, formally, the content of the type theory
%remains unchanged: objects that were present before the modification
%are still there. Conversely, we have not introduced any spurious
%object.

%If not on the content, this modification had an effect on the
%names. The type naming scheme of the type theory has condensed: named
%type formers ($\EnumU$) and constructors ($\NilE$ and $\ConsE$) are
%now replaced by codes and their fixpoint in the universe of
%descriptions. In essence,

\subsection{Implementing descriptions}

\begin{wstructure}
<- Realising our promises
    <- We are going to implement Desc
    /> Desc is itself a datatype
        <- Same situation as EnumU
            <- We want to benefit from generic operations
        -> It ought to be encoded in itself
\end{wstructure}

We shall now fulfil our implementation promises, encoding the universe
of descriptions. $\Desc$ is already some sort of datatype, but we are
in the same situation as with $\EnumU$: we ought to be able to
describe it, coding of $\Desc^n$ in $\Desc^{n+1}$, spiralling
upwards. Hence, this code would be a first-class citizen, born with
the generic equipment of datatypes.

\subsubsection{First attempt}

\begin{wstructure}
<- A partial implementation
    <- '1 and 'indx are easy
    <- 'sigma is partially doable
        /> lack the ability to do an higher-order inductive call
    -> Show partial code [figure]
\end{wstructure}

Our first attempt gets stuck quite quickly:
%
\[\stk{
\DescD^n : \Desc^{n+1} \\
\DescD^n \mapsto
  \tada{\stkm{\DUnit\\ \SYMBDSigma\\ \SYMBDIndx}}
       {\stkm{ \DUnit                                \\
         \DSigma{\Set^n}{\LAM{\V{S}} \SHED}      \\
         \DIndx{\DUnit}    }                    } \\
%\begin{array}{@{}ll}
%\DescD \mapsto \DSigma{\!}{\!} & (\EnumT{[ \DUnit, \DSigma{\!}{\!}, \DIndx{\!} ]})  \\
%                               & \left[\begin{array}{l}
%                                   \DUnit                                \\
%                                   \DSigma{\Set}{\LAM{\V{S}} \SHED}      \\
%                                   \DIndx{\DUnit}                        \\
%                                 \end{array}
%                                 \right]
%\end{array}
}\]
%
Let us explain where we stand. Much as
we have done so far, we first offer a constructor choice from
$\DUnit$, $\SYMBDSigma$, and $\SYMBDIndx$. You may notice
that the `tagged' notation we have used for the $\Desc^n$ constructors now
fits the facts: these were actually the tags we are defining.
For $\DUnit$, we immediately reach the end of the description. For
$\SYMBDIndx$, there is a single recursive argument. Describing
$\SYMBDSigma$ is problematic. Recall the specification of
$\SYMBDSigma$:
%
\[
\DSigma{(\Bhab{\V{S}}{\Set^n})}{(\Bhab{\V{D}}{\V{S} \To \Desc^n})} : \Desc^n\\
\]
%
So, we first pack a  $\Set^n$, $\V{S}$, as well we might when working in
$\Desc^{n+1}$. We should then like
a recursive argument \emph{indexed} by $\V{S}$, but
that is an \emph{exponential}, and our presentation so far delivers only
sums-of-products. To code our universe, we must first enlarge it!


\subsubsection{Second attempt}

\begin{wstructure}
<- Extending the universe of description
    -> With higher-order induction
    <- Intuition: index elements in X by H, and go on reading
        -> indx is isomorph to hindx for H = 1
    /> Keep indx
        <- First order!
        -> Extensionally equal to hindx 1
        /> Practically, definitional equality on Sigma/Pi cannot cope with it
    -> Show DescD code
\end{wstructure}

In order to capture a notion of higher-order induction, we add a code
$\SYMBDHindx$ that takes an indexing set $\V{H}$. This amounts to give
a recursive subobject for each element of $H$.
%
\[\stk{
    \DHindx{(\Bhab{\M{H}}{\Set^n})}{(\Bhab{\M{D}}{\Desc^n})} : \Desc^n
    \smallskip \\
    \descop{\DHindx{\V{H}}{\V{D}}}{\V{X}}     
        \mapsto 
            \TIMES{(\V{H} \To \V{X})}{\descop{\V{D}}{\V{X}}} \\
}\]


Note that up to isomorphism, $\SYMBDIndx$ is subsumed by
$\DHindx{\Unit}{\!}$. However, the apparent duplication has some
value.  Unlike its counterpart, $\SYMBDIndx$ is first-order: we prefer
not to demand dummy functions from \(\Unit\) in ordinary data,
e.g. \(\NatSuc{(\LAM{\_}n)}\). It is na{\"\i}ve to imagine that up to
isomorphism, any representation of data will do.  First-order
representations are finitary by construction, and thus admit a richer,
componentwise decidable equality than functions may in
general possess.\footnote{E.g., extensionally, there is one inhabitant of
\(\EnumT{\void}\To\Nat\); intensionally, there is a countable infinitude
which it is not safe to collapse.}

We are now able to describe our universe of datatypes:
%
\[\stk{
\DescD^n : \Desc^{n+1} \\
\DescD^n \mapsto\tada
{\stkm{ \DUnit\\ \SYMBDSigma\\ \SYMBDIndx\\ \SYMBDHindx}}
{\stkm{
\DUnit                                            \\
                                   \DSigma{\Set^n}{\LAM{\V{S}} \DHindx{\V{S}}{\DUnit}}   \\
                                   \DIndx{\DUnit}                                    \\
                                   \DSigma{\Set^n}{\LAM{\_} \DIndx{\DUnit}}
}}
}\]
%
The $\DUnit$ and $\SYMBDIndx$ cases remain unchanged, as expected. We
successfully describe the $\SYMBDSigma$ case via
the higher-order induction, branching on $\V{S}$. The $\SYMBDHindx$ case
just packs a $\Set^n$ with a recursive argument.

At a first glance, we have achieved our goal. We have described the
codes of the universe of descriptions. The fixpoint of \(\descop{\DescD^n}\)
is a datatype just like $\Desc^n$, in $\Set^{n+1}$. Might we be
so bold as to take \(\Desc^n \mapsto \Mu{\DescD^n}\) as the levitating
definition? If we do, we shall come down with a bump! 
To complete our levitation, just as in the magic trick, requires
hidden assistance. Let us explain the problem and reveal the `invisible
cable' which fixes it.


\subsubsection{Final move}

\begin{wstructure}
<- Subtlety: translation of [ ... ]
    -> Let us do it manually
        -> Code with problem for the motive of switch
\end{wstructure}

The definition \(\Desc^n \mapsto \Mu{\DescD^n}\) is circular,
but the offensive recursion is concealed by a prestidigitation.
Expanding \(\toDesc{-}\) and propagating types as in
Figure~\ref{fig:type-checking} reveals the
awful truth:
\[
\Desc^n\mapsto
\Mu{\stk{(\DSigma{\EnumT{\sqr{ \DUnit\:\: \SYMBDSigma\:\: \SYMBDIndx\:\: \SYMBDHindx}}\\}
    {\;\SYMBswitch\:\sqr{ \DUnit\:\: \SYMBDSigma\:\: \SYMBDIndx\:\: \SYMBDHindx}\:(\LAM{\_}\Desc^{n+1})\\
\;\sqr{\stkm{
\DUnit                                            \\
                                   \DSigma{\Set^n}{\LAM{\V{S}} \DHindx{\V{S}}{\DUnit}}   \\
                                   \DIndx{\DUnit}                                    \\
                                   \DSigma{\Set^n}{\LAM{\_} \DIndx{\DUnit}}
}}})}}
\]
The recursion shows up only because we must specify the return type
of the general-purpose \(\F{switch}\), and it is computing a \(\Desc^{n+1}\)!
Although type propagation
allows us to hide this detail \emph{when defining a function}, we cannot
readily suppress this information and check types
when \(\F{switch}\) is fully applied.

\begin{wstructure}
<- The magician trick
    <- Our problem is to give a motive for switch
        /> We perfectly know what it ought to be: \_ -> DescD
    -> Solution: extend the type theory with a special purpose switchD
        -> Only extension required to the type theory!
        -> Hidden away to the user by the syntactic sugar
            -> Sufficient to ensure unavailability as a raw operator
            <- Another instance of type propagation
\end{wstructure}

%What happens if we unfold the definition? We ought to build the following term:
%
%\[
%\PLAM{\V{x}}{(\EnumT{\V{E}})} \switch{\V{E}}{(\LAM{\_} %\Mu{\DescD})}{\V{\pi^f}}{\V{x}}
%\]
%
%But this is quite problematic. We are still in the process of
%constructing $\DescD$, and the motive of $\F{switch}$ is abruptly
%begging for this very same $\DescD$. Despite our willingness, we
%cannot materialise such motive. However, we perfectly know what the
%motive is.

We are too close to give up now. If only we did not need to
supply that return type, especially when we know what it must be!
We eliminate the recursion by \emph{specialising} \(\F{switch}\):
%
\[
\F{switchD} : \PITEL{\V{E}}{\EnumU}   \To
                (\spi{\V{E}}{\LAM{\_} \Desc^m}) \To
                \EnumT{\V{E}} \To \Desc^m
\]
%
The magician's art rests here, in this extension. We conceal it
behind a type propagation rule for \(\F{switchD}\) which we apply
with higher priority than for \(\F{switch}\) in general.
%
\[
\Rule{\Gamma \Vdash
  \propag{\push{\sqr{\vec{t}}}{\spi{\M{E}}{\PLAM{x}{\EnumT{E}}\Desc^m}}}
                           {\M{t'}}}
     {\Gamma \Vdash
\propag{\push{\sqr{\vec{t}}}{\EnumT{\M{E}}\To\Desc^m}}
 {\switchD{\M{E}}{\M{t'}}}}
\]
As a consequence, our definition above now propagates without
introducing recursion. Of course, by pasting together the declaration
of \(\Desc^n\) and its internal copy, we have made it appear in its own
type. Hardwired as a \emph{fait accompli}, this creates no regress,
although one must assume the definition to recheck it.

Our Agda model does not formalise the $\SYMBswitchD$
construction. Instead, we exhibit the isomorphism between declared
and encoded descriptions. Here, $\SYMBswitchD$ lets us collapse
this isomorphism, operationally identifying defined and coded
descriptions.

There are other ways to achieve a sufficient specialisation to avoid a
recursive code, e.g., extending $\Desc^n$ with specialised codes for
\emph{finite} sums and products, pushing the $\SYMBswitch$ into the
interpretation of codes, rather than the code itself. Here, we prefer
not to add codes to $\Desc^n$ which are otherwise unmotivated.
%%
%\[\stk{
%    \Dsigma{(\Bhab{\V{E}}{\EnumU})}
%           {(\Bhab{\V{B}}{\spi{\V{E}}{\LAM{\_}{\Desc^n}}})} : \Desc^n
%    \smallskip \\
%    \descop{\Dsigma{\V{E}}{\V{B}}}{\V{X}}     
%        \mapsto 
%            \SIGMAS{\Bhab{\V{x}}{\EnumT{\V{E}}}}
%                   %{\descop{\switch{\V{E}}{(\LAM{\_}{\Desc})}{\V{B}}{\V{x}}}{\V{X}}}    
%}\]
%%
%Using $\SYMBDsigma$ to switch over the $\Desc$ constructors, we avoid
%mention of $\Desc$ while levitating it. The $\SYMBswitch$ operator is
%pushed into the interpretation function, at which point $\Desc$ indeed
%exists.

%% * There exists alternative
%% **   <- All we want is to avoid mention of Desc
%% **   -> Add a code for finite sigma : $\sigma : (e : \EnumU) -> \pi e \Desc -> \Desc$
%% **       <- Take a finite set in argument
%% **       <- Interpreted as a switch
%% **       -> No more need to mention Desc during its construction
%% **           <- Pushed into intrepretation, at which time it exists

\begin{wstructure}
<- Generic programming now!
    <- Desc is just data
        -> Can be manipulated
    <- Free induction scheme on Desc
        -> Ability to inspect datatypes
        -> Ability to program on datatypes
\end{wstructure}


We have levitated \(\Desc\) at every level. Beyond its pedagogical
value, this exercise has several practical outcomes. First, it
confirms that each $\Desc$ universe is just plain data. As any piece of
data, it can therefore be inspected and manipulated. Moreover, it is
expressed in a $\Desc$ universe. As a consequence, it is equipped,
for free, with an induction principle. So, our ability to inspect and
program with $\Desc$ is not restricted to a meta-language: we have
the necessary equipment to program with \emph{data}, so we can program over
datatypes. \emph{Generic programming is just programming}.


\subsection{The generic catamorphism}

\begin{wstructure}
<- Making cata
    <- Present the type signature
    <- Starts with a call to generic induction
        <- induction on Desc!
        /> Show types at hand
        -> Explain how to use inductive hypothesis
    <- Implement the 'replace' function
    -> Dependent-typeless catamorphism 
\end{wstructure}

In Section~\ref{sec:desc-fix-point}, we hardwired a dependent
$\F{ind}$uction principle, but sometimes,
iteration suffices. Let us construct the catamorphism.

\newcommand{\cata}{\F{cata}}

We proceed by induction on the \emph{data} in \(\Mu{\V{D}}\):
the non-dependent return type $\V{T}$ is readily propagated.
Given a node $\V{xs}$
and the induction hypotheses, the method ought to build an element of
$\V{T}$. Provided that we know how to make an element of
$\descop{\V{D}}{\V{T}}$, this step will be performed by the algebra
$\V{f}$. Let us take a look at this jigsaw:
%
\[\stk{
\cata : \PITEL{\V{D}}{\Desc}
           \PI{\V{T}}{\Set}
           (\descop{\V{D}}{\V{T}} \To \V{T}) \To 
           \Mu{\V{D}} \To \V{T} \\
\cata\: \V{D}\: \V{T}\: \V{f} \mapsto
  \sind \LAM{\V{xs}}\LAM{\V{hs}} \V{f}\: \SHED
}\]
% 
The hole remains: we have
\(\Bhab{\V{xs}}{\descop{\V{D}}{\Mu{\V{D}}}}\) and
\(\Bhab{\V{hs}}{\All{\V{D}}{\Mu{\V{D}}}{(\LAM{\_} \V{T})}{\V{xs}}}\)
to hand, and we need a \(\descop{\V{D}}{\V{T}}\).
Now, $\V{xs}$ has the right shape, but its
components have the wrong type. However, for each such
component, $\V{hs}$ holds the corresponding value
in $\V{T}$.  We need a function to \(\F{replace}\) the former with the latter:
this pattern
matching sketch yields an induction on \(\V{D}\). We fill the hole
with \(\F{replace}\: \V{D}\:(\Mu{\V{D}})\: \V{T}\: \V{xs}\: \V{hs}\).
%
\[\stk{
\F{replace} : \stk{\PITEL{\V{D}}{\Desc}
                   \PITEL{\V{X},\V{Y}}{\Set}\\
                   \PI{\V{xs}}{\descop{\V{D}}{\V{X}}} 
                   \All{\V{D}}{\V{X}}{(\LAM{\_}\V{Y})}{\V{xs}} \To
                   \descop{\V{D}}{\V{Y}}} \\
\begin{array}{@{}l@{\:}l@{\:}l@{\:\:\mapsto\:\:}l}
\F{replace}\: \DUnit &          \V{X}\: \V{Y}\: \Void&          \Void   &
    \Void  \\
\F{replace}\: (\DSigma{\V{S}}{\V{D}})& \V{X}\: \V{Y}\: \pair{\V{s}}{\V{d}}{}& \V{d'}  &
    \pair{\V{s}}{\F{replace}\: (\V{D}\: \V{s})\: \V{X}\: \V{Y}\: \V{d}\: \V{d'}}{} \\
\F{replace}\: (\DIndx{\V{D}})&     \V{X}\: \V{Y}\: \pair{\V{x}}{\V{d}}{}& \pair{\V{y}}{\V{d'}}{} &
    \pair{\V{y}}{\F{replace}\: \V{D}\: \V{X}\: \V{Y}\: \V{d}\: \V{d'}}{} \\
\F{replace}\: (\DHindx{\V{H}}{\V{D}})& \V{X}\: \V{Y}\: \pair{\V{f}}{\V{d}}{}& \pair{\V{g}}{\V{d'}}{} &
 \pair{\V{g}}{\F{replace}\: \V{D}\: \V{X}\: \V{Y}\: \V{d}\: \V{d'}}{}
\end{array}
}\]
%

%% The astute reader will have been struck by the type of $\F{replace}$:
%% it is \emph{almost} the morphism part -- sometimes called \emph{map}
%% -- of the functor $\V{D}$ from $\V{X}$ to $\V{Y}$ in $\Set$. Just as
%% the $\F{induction}$ is the dependent version of $\cata$, $\F{replace}$
%% is the dependent version of the map, which uses the induction
%% hypotheses. For space reason, we will not present the non-dependent
%% map. It can be found in the Agda model.

\begin{wstructure}
<- Deriving generic functions
    <- Taking a Desc and computing a function
        <- Desc comes equipped with an induction principle
        -> Ability to compute more functions from it
            -> More generic functions
    <- Inspecting datatypes
        <- All described byu a Desc code
        -> Ability to explore the code
            <- Desc equipped with an induction principle
            -> Build new objects based on that structure
\end{wstructure}

We have shown how to derive a generic
operation, \(\cata\), from a pre-existing generic operation,
\(\F{ind}\), by
manipulating descriptions as data: the catamorphism is just a
function taking each $\Desc$ value to a datatype specific
operation. This is polytypic programming, as in
PolyP~\cite{jansson:polyp}, made ordinary.

%% Moreover, the $\F{replace}$ function demonstrates the benefit of an
%% approach based on universes. The datatypes living in the universe of
%% descriptions, we are able to \emph{inspect} them. As shown by
%% $\F{replace}$, it is easy to explore these structures, as well as
%% building new ones.

\subsection{The generic free monad}
\label{sec:desc-free-monad}

\begin{wstructure}
<- A generic program: the free monad construction
    <- Recall free monad construction in Haskell
        -> Based on a functor F
    <- Note that the free monad construction is itself defined by a functor
        -> Extract it
\end{wstructure}

In this section, we try a more ambitious generic operation. Given a
functor---a signature of operations represented as a tagged
description---we build its free monad, extending the signature with
variables and substitution.
%
\newcommand{\FMFreeMonad}{\D{FreeMonad}}
\newcommand{\FMFreeMonadD}{\D{FreeMonadD}}
\newcommand{\FMVar}{\C{Var}}
\newcommand{\FMComposite}{\C{Composite}}
%
Let us recall this construction in, say, Haskell. Given a
functor \texttt{f}, the free monad over \texttt{f} is given
thus:
%
\[
\texttt{data FreeMonad f x = Var x | Op (f (FreeMonad f x))}
\]
%
Provided \texttt{f} is an instance of \texttt{Functor}, we may
take \texttt{Var} for \texttt{return} and use \texttt{f}'s
\texttt{fmap} to define \texttt{>>=} as substitution.

Being an inductive type, $\FMFreeMonad$ arises
by a pattern functor:
%
\[
\FMFreeMonadD\: \V{F}\: \V{X}\: \V{Z} \mapsto \V{X} \mathop{\D{+}} \V{F}\:\V{Z}
\]

\begin{wstructure}
    <- Encode it in the Desc world [equation]
        <- F is the Desc we start with
        <- The free monad functor is what we have just defined
        <- [\_]* : Desc -> Set -> Desc
           [\_]* D X = 'cons ['var ('sigma X (\_ -> '1))] D
        -> Mu does the fixpoint
\end{wstructure}

Our construction takes the functor as a
tagged description, and given a set $\V{X}$ of variables, computes the
tagged description of the free monad pattern functor.
%
\[\stk{
\FreeMonad{\_} : \TagDesc \To \Set \To \TagDesc \\
\FreeMonad{\pair{\V{E}}{\V{D}}{}}\:\V{X} \mapsto
    \pair{\pair{\DVar{}}{\V{E}}{}}
         {\pair{\DSigma{\V{X}}{\DUnit}}{\V{D}}{}}{}
}\]
%
We simply add a constructor, $\SYMBDVar$, making its arguments
$\DSigma{\V{X}}{\DUnit}$---just an element of $\V{X}$. 
$\V{E}$ and $\V{D}$ stay put, leaving the other constructors
unchanged. Unfolding the interpretation of this definition, we
find an extended sum, corresponding to the \(\V{X}\D{+}\) in
$\FMFreeMonadD$. Taking the fixpoint ties the knot and we have our
data.

\begin{wstructure}
<- A generic program: monadic substitution [equation]
    <- subst : \forall T X Y. mu ([T]* X) -> (X -> mu ([T]* Y)) -> mu ([T]* Y)
        -> Using Fold
\end{wstructure}

\newcommand{\subst}{\F{subst}}
\newcommand{\apply}{\F{apply}}

Now we need the operations. As expected, \(\LAM{\x}\DVar{\x}\)
plays the r\^ole of \return, making variables terms. Meanwhile,
\bind is indeed \emph{substitution}, which we now
implement generically, making use of $\cata$. Let us write the type,
and start filling in the blanks:
%
\[\stk{
\begin{array}{@{}ll}
\subst : & \PITEL{\V{D}}{\TagDesc}
           \PI{\V{X}, \V{Y}}{\Set} 
           (\V{X} \To \mude{(\FreeMonad{\V{D}}{\V{Y}})}) \To \\
         & \mude{(\FreeMonad{\V{D}}{\V{X}})} \To
           \mude{(\FreeMonad{\V{D}}{\V{Y}})} 
\end{array} \\
\subst\: \V{D}\: \V{X}\: \V{Y}\: \V{\sigma} \mapsto
  \cata\: (\toDesc{(\FreeMonad{\V{D}}{\V{X}})})\: 
          (\mude{(\FreeMonad{\V{D}}{\V{Y}})})\: 
          \SHED
}\]
%
We are left with implementing the algebra of the catamorphism. Its
role is to catch appearances of $\DVar{\V{x}}$ and replace them by
$\V{\sigma}\: \V{x}$. This corresponds to the following definition:
%
\[\stk{
\begin{array}{@{}ll}
\apply : & \PITEL{\V{D}}{\TagDesc} 
           \PI{\V{X}, \V{Y}}{\Set} 
           (\V{X} \To \mude{(\FreeMonad{\V{D}}{\V{Y}})}) \To \\
         & \descop{\toDesc{(\FreeMonad{\V{D}}{\V{X}})}}{(\mude{(\FreeMonad{\V{D}}{\V{Y}})})}
           \To  \mude{(\FreeMonad{\V{D}}{\V{Y}})}
\\
\end{array} \\
\begin{array}{@{}l@{\:\mapsto\:\:}l}
\apply\: \V{D}\: \V{X}\: \V{Y}\: \V{\sigma}\: \pair{\SYMBDVar}{\V{x}}{}   & \V{\sigma}\: \V{x}                   \\
\apply\: \V{D}\: \V{X}\: \V{Y}\: \V{\sigma}\: \pair{\V{c}}{\V{xs}}{} & \Con{\pair{\V{c}}{\V{xs}}{}}
\end{array}
}\]

\begin{wstructure}
    -> Consequences
        <- We have free monad datatype
            <- Term + variables
        <- We have monad operations
            <- Return / var
            <- Substitution / bind
\end{wstructure}

We complete the hole with $\apply\: \V{D}\: \V{X}\: \V{Y}\:
\V{\sigma}$. Every tagged
description can be seen as a signature of operations: we can uniformly
add a notion of variable, building a new type from an old one, then
providing the substitution structure.

\begin{wstructure}
<- Deriving new data-structure and functions on them
    <- Computing the Free Monad of a datatype
        <- Derive new data-structure from previous one
            <- It is just code
        /> New data-structure comes with some equipment
    <- Computing new functions on computed datatypes
        <- If data comes with structure, we ought to be able to capture it
            <- Induction on Desc
            -> Ability to compute over data
\end{wstructure}

%% Candidate for removal:
%% With the free monad construction, we have seen two kinds of generic
%% operations. Firstly, we have derived a new data-structure from another
%% one: we make the free monad from its underlying functor. To do so, we
%% crucially rely on the fact that datatypes are nothing but codes. We
%% are therefore entitled to modify this code and, in this case, extend
%% it. Extending a datatype might give rise to a more structured object,
%% as was the case here.  So, secondly, we have equipped this new
%% datatype with its inherent structure: the \bind\ and
%% \return\ operations. We have been able to build them as generic
%% functions.

\subsection{Skyhooks all the way up?}

%% * Desc in Desc
%% ** /> Blur the lines between implementation and reflection
%% ** /> Abusing the paradoxical nature of Set : Set

In this section, we have seen how to \emph{levitate}
descriptions. Although our theory, as presented here, takes
$\Bhab{\Set}{\Set}$, our annotations indicate how a stratified theory
could code each level from above.  We do not rely on the paradoxical
nature of $\Bhab{\Set}{\Set}$ to flatten the hierarchy of descriptions
and fit large inside small.  We shall now be more precise about what
we have done.

%% * Summary 
%% ** <- What is the equipment for making data-types
%% ** <- What is reflected, what is implemented
%% ** <- (table 1)

Let us first clarify the status of the implementation. The kit for
making datatypes is presented in Table~\ref{tab:sumup-operators}. For
each operation, we describe its role and its status, making clear
which components are self-described and which ones are actually
implemented.

%% * Paradox
%% ** <- How do we bottom out in a stratified setting?
%% ** <- Spiral
%% ***    <- Encoding of DescD_n : Desc_{n+1}
%% ***    <- Desc_n = Mu DescD_n : Set_{n+1}
%% ** <- Agda model: Desc_42
%% ***    <- No dependent pattern-matching
%% ***    <- No IR
%% ***    <- Normal universes
%% ***    -> Straight Agda (UTT)
%% ** <- ``Self-encoding'' only in a level polymorphic sense
%% ***    -> Agda model: set poly

In a stratified system, the `self-encoded' nature of $\Desc$ appears
only in a set polymorphic sense: the principal type of the encoded
description generalises to the type of $\Desc$ itself. We encode this
much in our set polymorphic model in Agda and in our Coq model,
crucially relying on typical ambiguity~\cite{harper:implicit-universe}.
We step outside current technology only to replace the declared $\Desc$
with its encoding.

Even this last step, we can approximate within a standard predicative
hierarchy. Fix a top level, perhaps \(42\). We may start by declaring
$\Desc^{42}:\Set^{43}$. We can then construct $\DescD^{41}:\Desc^{42}$
and thus acquire an encoded $\Desc^{41}$. Although $\Desc^{41}$ is
encoded, not declared, it includes the relevant descriptions,
including $\DescD^{40}$. We can thus build the tower of descriptions
down to $\Desc^0$, encoding every level below the top.  Description of
descriptions forms a `spiral', rather than a circle. We have modelled
this process exactly in Agda, without any appeal to dependent
pattern-matching, induction-recursion, or set polymorphism. All it
takes to build such a sawn-off model of encodings is inductive
definition and a cumulative predicative hierarchy of set levels.

\begin{table}

{
\begin{center}
\begin{tabular}{|c|c|c|c|}
\hline
Object                & Role                        & Status \\
\hline
\hline
$\EnumU$              & Build finite sets           & Levitated \\
\hline
$\Desc$               & Describe pattern functors   & Levitated \\
\hline
$\SYMBdescop{\_}$     & Interpret descriptions      & Hardwired \\
\hline
$\SYMBMu$, $\SYMBCon$ & Define, inhabit fixpoints   & Hardwired \\
\hline
$\SYMBind$, $\SYMBAll$, $\SYMBall$  
                      & Induction principle         & Hardwired \\
\hline
\end{tabular}
\end{center}
}

\caption{Summary of constructions on Descriptions}
\label{tab:sumup-operators}

\end{table}
