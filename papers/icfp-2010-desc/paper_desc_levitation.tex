\section{Levitating the universe of descriptions}

\subsection{Implementing finite sets}

\begin{structure}
<- Recall typing rules of 1st section
    -> Make clear they were just promises
    -> Can be implemented now
        <- Simply List UId
\end{structure}

\[\stk{
\EnumU : \Set \\
\EnumU \mapsto \Mu{(\List~\UId)}
}\]


\begin{structure}
<- Consequences
    -> Type theory doesn't need to be extended with EnumU, NilE, and ConsE
        <- EnumU == Mu EnumUD
        <- NilE, ConsE are just tags
    -> Do not need a specific \spi eliminator
        <- \spi is an instance of the generic eliminator
            <- Code?
    -> Anything else remains the same (switch, EnumT, 0, 1+)
\end{structure}

\begin{structure}
<- Summary of the operation
    <- The content of the type theory is exactly the same
        <- before == after
    /> type naming scheme condenses
        <- Replace named constructors by codes in the universe of data-types
    -> Our next step is a similar move (in essence)
        /> Condenses the entire naming scheme of data-types
\end{structure}

\subsection{Implementing descriptions}

\begin{structure}
<- Realizing our promises
    <- We are going to implement Desc
    /> Desc is itself a data-type
        <- Same situation as EnumU
            <- We want to benefit from generic operations
        -> It ought to be encoded in itself
\end{structure}

\subsubsection{First try}

\begin{structure}
<- A partial implementation
    <- '1 and 'indx are easy
    <- 'sigma is partially doable
        /> lack the ability to do an higher-order inductive call
    -> Show partial code [figure]
\end{structure}

\[\stk{
\DescD : \Desc \\
\begin{array}{@{}ll}
\DescD \mapsto \DSigma{}{} & (\EnumT [ \DUnit, \DSigma{}{}, \DIndx{} ]) \\
                           & \left[\begin{array}{l}
                                   \DUnit                                \\
                                   \DSigma{\Set}{(\LAM{\V{X}} ???)} \\
                                   \DIndx{\DUnit}                                  \\
                                   \end{array}
                             \right]
\end{array}
}\]

\subsubsection{Second try}

\begin{structure}
<- Extending the universe of description
    -> With higher-order induction
    <- Intuition: index elements in X by H, and go on reading
        -> indx is isomorph to hindx for H = 1
    /> Keep indx
        <- First order!
        -> Extensionally equal to hindx 1
        /> Practically, definitional equality on Sigma/Pi cannot cope with it
    -> Show DescD code
\end{structure}


\[
\stk{
\data \Desc : \Set \where \\
\;\;\begin{array}{@{}l@{\::\:\:}l@{\quad}l}
    \ldots          & \:\:\ldots \\
    \DHindx         & \PI{H}{Set} \Desc \To \Desc
\end{array}
}
\]

\[\stk{
\descop{\_\:}{} : \Desc \To \Set \To \Set \\
\begin{array}{@{}l@{\:=\:\:}ll}
\ldots                        &  \ldots \\
\descop{\DHindx{H}{D}}{X}     &  \TIMES{(H \To X)}{\descop{D}{X}}
\end{array}
}\]


\[\stk{
\DescD : \Desc \\
\begin{array}{@{}ll}
\DescD \mapsto \DSigma{}{} & (\EnumT [ \DUnit, \DSigma{}{}, \DIndx{}, \DHindx{}{} ]) \\
                           & \left[\begin{array}{l}
                                   \DUnit                                \\
                                   \DSigma{\Set}{(\LAM{\V{X}} \DHindx{X}{\DUnit})} \\
                                   \DIndx{\DUnit}                                  \\
                                   \DSigma{\Set}{(\LAM{\_} \DIndx{\DUnit})}
                                   \end{array}
                             \right]
\end{array}
}
\]


\subsubsection{Final move}

\begin{structure}
<- Subtlety: translation of [ ... ]
    -> Let us do it manually
        -> Code with problem for the motive of switch
\end{structure}

\begin{structure}
<- The magician trick
    <- Our problem is to give a motive for switch
        /> We perfectly know what it ought to be: \_ -> DescD
    -> Solution: extend the type theory with a special purpose switchD
        -> Only extension required to the type theory!
        -> Hidden away to the user by the syntactic sugar
            -> Sufficient to ensure unavailability as a raw operator
            <- Another instance of type propagation
\end{structure}

\begin{structure}
<- Generic programming now!
    <- Desc is just data
        -> Can be manipulated
    <- Free induction scheme on Desc
        -> Ability to inspect data-types
        -> Ability to program on data-types
\end{structure}

\subsubsection{Desc, atomically}

\begin{structure}
<- Adding hindx have introduced some duplication
    <- indx == hindx 1
    -> We can factor out commonalities 
        /> Obtain an equivalent presentation
        /> Still embeddable (refer to the Agda model)
\end{structure}

\begin{structure}
<- Give new presentation [figure]
    <- hindx have introduced the notion of function space: 'Pi
    <- hindx and indx are both composed by a binary product and a left open term: 'x and 'id 
    -> Straightforward translation to the new system [equation]
\end{structure}

\[
\stk{
\data \Desc : \Set \where \\
\;\;\begin{array}{@{}l@{\::\:}l@{\quad}l}
    \DId            & \Desc                                   \\
    \DUnit          & \Desc                                   \\
    \DProd{}{}      & \PI{\V{D}, \V{D'}}{\Desc} \Desc         \\
    \DSigma         & \PI{\V{S}}{\Set} \PIS{S \To \Desc} \Desc \\
    \DPi            & \PI{\V{S}}{\Set} \PIS{S \To \Desc} \Desc 
\end{array}
}
\]

\[\stk{
\descop{\_\:}{} : \Desc \To \Set \To \Set \\
\begin{array}{@{}l@{\:=\:\:}ll}
\descop{\DId}{X}          &  X                                           \\
\descop{\DUnit}{X}        &  \Unit                                       \\
\descop{\DProd{D}{D'}}{X} &  \TIMES{\descop{D}{X}}{\descop{D'}{X}}       \\
\descop{\DSigma{S}{D}}{X} &  \SIGMA{\V{s}}{S} \descop{D\: s}{X}                \\
\descop{\DPi{S}{D}}{X}    &  \PI{\V{s}}{S} \descop{D\: s}{X}            
\end{array}
}\]

\[\begin{array}{l@{\:\mapsto\:\:}l}
\DIndx{D}         & \DProd{\DId}{D}                      \\
\DHindx{H}{D}     & \DProd{(\DPi{H}{(\LAM{\_} \DId)})}{D}
\end{array}
\]


\subsection{The generic catamorphism}

\begin{structure}
<- Making cata
    <- Present the type signature
    <- Starts with a call to generic induction
        <- induction on Desc!
        /> Show types at hand
        -> Explain how to use inductive hypothesis
    <- Implement the 'replace' function
    -> Dependent-typeless catamorphism 
\end{structure}

This \(\F{induction}\) operator is the natural dependent elimination
principle, but we might also benefit from the traditional \emph{catamorphism}
or `fold operator' which accompanies a (weakly) initial algebra. We should
like to have
\[\stk{
\F{cata} : \PITEL{D}{\Desc}
           \PI{T}{\Set}
           (\descop{D}{T}\To T) \To 
           \Mu{D} \To T \\
\F{cata}\: D\: T\: f \mapsto
  \F{induction}\: D\: (\LAM{\_}T)\: (\LAM{xs\:ts} f\: ?)
}\]
but what should \(?\) be? We have \(xs:\descop{D}{\Mu{D}}\)
and \(ts:\All{D}{(\Mu{D})}{(\LAM{\_}T)}{xs}\), so surely we can construct
an element of \(\descop{D}{T}\) by replacing each recursive component from
\(xs\) with its counterpart from \(ts\).

\[\stk{
\F{replace} : \stk{\PITEL{D}{\Desc}
                   \PITEL{X,Y}{\Set}\\
                   \PI{xs}{\descop{D}{X}} 
                   \All{D}{X}{(\LAM{\_}Y)}{xs} \To
                   \descop{D}{Y}} \\
\F{replace}\: \DUnit\:          X\: Y\: \Void\:          \Void          \mapsto \Void \\
\F{replace}\: (\DSigma{S}{D})\: X\: Y\: \pair{s}{xs}{}\: ys             \mapsto
    \pair{s}{\F{replace}\: {D~s}\: X\: Y\: xs\: ys}{}         \\
\F{replace}\: (\DIndx{D})\:     X\: Y\: \pair{x}{xs}{}\: \pair{y}{ys}{} \mapsto
    \pair{y}{\F{replace}\: D\: X\: Y\: xs\: ys}{}
}\]



\begin{structure}
<- Deriving generic functions
    <- Taking a Desc and computing a function
        <- Desc comes equipped with an induction principle
        -> Ability to compute more functions from it
            -> More generic functions
    <- Inspecting data-types
        <- All described by a Desc code
        -> Ability to explore the code
            <- Desc equipped with an induction principle
            -> Build new objects based on that structure
\end{structure}

\subsection{The generic Free Monad}

\begin{structure}
!!! EARLIER !!!
<- Tagged description
    <- Form TDesc = List (UId x Desc) [equation]
    <- Follow usual sums-of-product presentation of data-type
        <- Finite set of constructors
        <- Then whatever you want
    -> Any Desc data-type can be turned into this form
        -> No loss of expressive power
        /> Garantee a ``constructor form''
\end{structure}

A \emph{tagged} description is given by an inhabitant of
\[
 \F{TagDesc} \mapsto \List{\TIMES{\UId}{\Desc}}
\]


Datatypes specified in the conventional `sum of products' style naturally give rise to tagged descriptions. Of course, every description can be dorced into this style with a singleton choice of tag.


\begin{structure}
<- A generic program: the free monad construction
    <- Recall free monad construction in Haskell
        -> Based on a functor F
    <- Note that the free monad construction is itself defined by a functor
        -> Extract it
    <- Encode it in the Desc world [equation]
        <- F is the Desc we start with
        <- The free monad functor is what we have just defined
        <- [\_]* : Desc -> Set -> Desc
           [\_]* D X = 'cons ['var ('sigma X (\_ -> '1))] D
        -> Mu does the fix-point
\end{structure}


We may then implement the \emph{free monad} construction as a
transformation on \(\F{TagDesc}\).
\[\stk{
\FreeMonad{\_} : \F{TagDesc} \To \Set \To \F{TagDesc} \\
\FreeMonad{\pair{E}{D}{}}\:X \mapsto
\pair{\ListCons{\:\DVar{}}{E}}{\pair{\DConst{X}}{D}{}}{}
}\]
Of course, we must equip the resulting datatypes with operations delivering a monadic interface. As usual, \(\LAM{\x}\DVar{\x}\) performs the r\^ole of `return', embedding variables into terms. We shall also need a generic \emph{substitution} operator.


\begin{structure}
<- A generic program: monadic substitution [equation]
    <- subst : \forall T X Y. mu ([T]* X) -> (X -> mu ([T]* Y)) -> mu ([T]* Y)
        -> Using Fold
    -> Consequences
        <- We have free monad data-type
            <- Term + variables
        <- We have monad operations
            <- Return / var
            <- Substitution / bind
\end{structure}

\begin{structure}
<- Deriving new data-structure and functions on them
    <- Computing the Free Monad of a data-type
        <- Derive new data-structure from previous one
            <- It is just code
        /> New data-structure comes with some equipment
    <- Computing new functions on computed data-types
        <- If data comes with structure, we ought to be able to capture it
            <- Induction on Desc
            -> Ability to compute over data
\end{structure}
