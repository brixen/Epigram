\section{A universe of simple data-types}

\begin{structure}
<- Why starting with simple data-types
    <- For pedagogical purposes
        <- Data-types as we know them every day
        /> Target dependent types
    -> Cut down version of Induction Recursion
        -> Presentation evolves independently as we go to dependent types
\end{structure}

\subsection{The power of $\Sigma$}

\begin{structure}
<- The duality of Sigma
    <- Sigma generalizes sum over arbitrary arities
        -> \Sigma A B == \Sigma_{x : A} B x
    <- Sigma generalizes product to have a dependant second component
        -> \Sigma A B == (x : A) \times (B x)
\end{structure}

\begin{structure}
<- Data-types in the simply-typed world
    -> "sums-of-product"
        <- Sum of constructors
        <- Product of arguments
<- Data-types in the dependently-typed world
    -> "sigmas-of-sigmas"
    /> Need ability to manipulate these sigmas
        -> Define a Code for data-types
        -> Together with a sigma-based Interpretation
\end{structure}

\subsection{A universe of descriptions}

\begin{structure}
<- Introduction to Universe construction
    <- Define a Code
        -> Name objects
    <- Define an Interpretation of codes into the type theory
        -> Give a semantics to objects
    -> Ability to manipulate code
    -> Ability to compute with these objects
\end{structure}

\begin{structure}
<- Justification of the code 
    <- [both figures]: cannot be read separately
    <- Mimic the standard grammar of data-types description
        -> Just as we already know it
        <- '\Sigma for making sigmas-of-sigmas
        <- 'indx for exhibiting the functoriality
            -> For recursive arguments
        <- '1 for end of description
\end{structure}

\[
\stk{
\data \Desc : \Set \where \\
\;\;\begin{array}{@{}l@{\::\:}l@{\quad}l}
    \DUnit          & \Desc \\
    \DSigma         & \PI{\V{S}}{\Set} \PIS{S \To \Desc} \Desc \\
    \DIndx          & \Desc \To \Desc
\end{array}
}
\]

\[\stk{
\descop{\_\:}{} : \Desc \To \Set \To \Set \\
\begin{array}{@{}ll@{\:=\:\:}ll}
\descop{\DUnit}{& X}        &  \Unit                                       \\
\descop{\DSigma{S}{D}}{& X} &  \SIGMAS{\V{s} : S}{\descop{D~s}{X}}         \\
\descop{\DIndx{D}}{& X}     &  \TIMES{X}{\descop{D}{X}}
\end{array}
}\]


\subsection{Examples}

\begin{structure}
<- Nat
    <- Sum of zero, suc
    <- zero case: done
    <- suc case: leave open and done
    -> NatD Z = 1 + Z
\end{structure}

\[\stk{
\NatD : \Desc \\
\NatD \mapsto \DSigma{(\EnumT [ \NatZero, \NatSuc{} ])}
                     {[ \DUnit \quad (\DIndx{\DUnit}) ]}
}\]


\begin{structure}
<- List
    <- Sum of nil, cons
    <- nil case: done
    <- cons case: product of X with leave open and done
    -> ListD X Z = 1 + X * Z
\end{structure}

\[\stk{
\ListD : \Set \To \Desc \\
\ListD \: X \mapsto \DSigma{(\EnumT [ \ListNil, \ListCons ])}
                           {[ \DUnit \quad (\DSigma{X}{\LAM{\_} \DIndx{\DUnit}}) ]}
}\]


\begin{structure}
<- Tree
    <- sum of leaf, node
    <- leaf case: done
    <- node case: product of X with two leave open and done
    -> TreeD X Z = 1 + X * Z * Z
\end{structure}

\[\stk{
\TreeD : \Set \To \Desc \\
\TreeD \: X \mapsto \DSigma{(\EnumT [ \TreeLeaf, \TreeNode ])}
                           {[ \DUnit \quad (\DSigma{X}{\LAM{\_} \DIndx{(\DIndx{\DUnit})}}) ]}
}\]


\begin{structure}
<- Fictive object [figure 'data Desc']
    -> Must be read as a type signature
    -> See further for its actual implementation
        <- Subject to our levitation exercise
\end{structure}

\subsection{Fix-point}

\begin{structure}
<- Build the fix-point of functors
    <- See examples: need to build their initial algebra
    -> Extend the type theory with Mu/Con [figure]
        <- Straightforward definition of a fixpoint
            <- Interpret D with (Mu D) as subobjects
\end{structure}

\[
\Rule{\Gamma \vdash \Bhab{D}{\Desc}}
     {\Gamma \vdash \Bhab{\Mu{D}}{\Set}} \qquad
\Rule{\Gamma \vdash \Bhab{D}{\Desc} \quad 
      \Gamma \vdash \Bhab{x}{\descop{D}{(\Mu{D})}}}
     {\Gamma \vdash \Bhab{\Con{x}}{\Mu{D}}}
\]


\begin{structure}
<- Extending type propagation
    <- Data-type declaration turns into definitions
        -> Straightforward translation to Desc
        -> Creation of a variable referring to the structure
    <- Push Mu to an applied name [figure]
        -> Direct integration into the type propagation machinery
    <- Labelled Mu
        /> Just mention the possibility of labelling, no details required
        -> For the user, objects have names rather than Mu of codes
    -> Coded presentation is practical
        <- The user never see a code
\end{structure}

\begin{structure}
<- Elimination on Mu
    <- We are used to foldD : \forall X. (desc D X -> X) -> mu D -> X
        /> Not dependent
        -> Cannot express some (which one again?) properties
    -> Develop a dependent induction
        <- Everywhere/All
        <- Induction
    -> *Generic*
    ???
\end{structure}

\[
\F{cata} : \PITEL{D}{\Desc}
           \PI{T}{\Set}
           (\descop{D}{T}\To T) \To 
           \Mu{D} \To T 
\]

\[
\begin{array}{lcl}
\F{induction} & : & \PITEL{D}{\Desc}                \\
              &   & \PITEL{P}{\Mu{D} \To \Set}      \\
              &   & \PITEL{m}{\PI{xs}{\descop{D}{(\Mu{D})}} \All{D}{(\Mu{D})}{P}{xs} \To P (\Con{xs})} \\
              &   & \PI{x}{\Mu{D}} P x
\end{array}
\]
