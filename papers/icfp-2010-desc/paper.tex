\documentclass[preprint, authoryear]{sigplanconf}

\usepackage{amsmath}
\usepackage{verbatim}

%% Structure
\newenvironment{structure}{\footnotesize\verbatim}{\endverbatim}
%\newenvironment{structure}{\comment}{\endcomment}


\begin{document}

%\conferenceinfo{WXYZ '05}{date, City.} 
%\copyrightyear{2005} 
%\copyrightdata{[to be supplied]} 

%\titlebanner{banner above paper title}        % These are ignored unless
%\preprintfooter{short description of paper}   % 'preprint' option specified.



%%%%%%%%%%%%%%%%%%%%%%%%%%%%%%%%%%%%%%%%%%%%%%%%%%%%%%%%%%%%%%%%
%% Title
%%%%%%%%%%%%%%%%%%%%%%%%%%%%%%%%%%%%%%%%%%%%%%%%%%%%%%%%%%%%%%%%


\title{Generic programming is just programming}
% Or something mentionning levitation. Feel free to change. 
% \subtitle{Subtitle Text, if any}


%% Alphabetical ordering.
\authorinfo{James Chapman}
           {Institute of Cybernetics, Tallinn University of Technology}
           {james@cs.ioc.ee}
\authorinfo{Pierre-\'{E}variste Dagand \\ Conor McBride}
           {University of Strathclyde}
           {\{dagand,conor\}@cis.strath.ac.uk}
\authorinfo{Peter Morris}
           {University of Nottingham}
           {pwm@cs.nott.ac.uk}


\maketitle


%%%%%%%%%%%%%%%%%%%%%%%%%%%%%%%%%%%%%%%%%%%%%%%%%%%%%%%%%%%%%%%%
%% Abstract
%%%%%%%%%%%%%%%%%%%%%%%%%%%%%%%%%%%%%%%%%%%%%%%%%%%%%%%%%%%%%%%%


\begin{abstract}
This is the text of the abstract.
\end{abstract}

%\category{CR-number}{subcategory}{third-level}

%\terms
%term1, term2

%\keywords
%keyword1, keyword2

%%%%%%%%%%%%%%%%%%%%%%%%%%%%%%%%%%%%%%%%%%%%%%%%%%%%%%%%%%%%%%%%
%% Introduction
%%%%%%%%%%%%%%%%%%%%%%%%%%%%%%%%%%%%%%%%%%%%%%%%%%%%%%%%%%%%%%%%

\section{Introduction}

The text of the paper begins here. \cite{morris:spf}

ACM bullet points, we offer:
\begin{itemize}
\item a closed presentation of data-types (no generativity requires)
  subsuming standard inductive families (some popular extensions
  excluded for now)
\item descriptions of data-types are first-class (indeed,
  self-encoded)
\item ``generic programming is just programming''
\item first serious attempt to design a language for generic
  programming, except possibly Lisp
\end{itemize}



%%%%%%%%%%%%%%%%%%%%%%%%%%%%%%%%%%%%%%%%%%%%%%%%%%%%%%%%%%%%%%%%
%% Desc
%%%%%%%%%%%%%%%%%%%%%%%%%%%%%%%%%%%%%%%%%%%%%%%%%%%%%%%%%%%%%%%%

\section{A simple universe of descriptions}

\begin{structure}
<- a universe of simple inductive types
    -> nil, arg, ind, hind
<- levitation
\end{structure}

%%%%%%%%%%%%%%%%%%%%%%%%%%%%%%%%%%%%%%%%%%%%%%%%%%%%%%%%%%%%%%%%
%% IDesc
%%%%%%%%%%%%%%%%%%%%%%%%%%%%%%%%%%%%%%%%%%%%%%%%%%%%%%%%%%%%%%%%

\section{Indexing descriptions}


\begin{structure}
<- enumerations
<- indexing
    -> nil, arg, ind, hind
    -> var, Pi, Sigma, :-, x, Sigma_f
<- closure under \box and \diamond
<- Brady optimization as desc transformation
<- ``standard'' descriptions start with Sigma_f
    -> SIDesc : I -> (E : EnumU) -> Branches(E , \ e -> IDesc I )
<- ``standard'' descriptions closed under the free monad constructions
<- IDesc is just such a free monad
\end{structure}


%%%%%%%%%%%%%%%%%%%%%%%%%%%%%%%%%%%%%%%%%%%%%%%%%%%%%%%%%%%%%%%%
%% Appendices
%%%%%%%%%%%%%%%%%%%%%%%%%%%%%%%%%%%%%%%%%%%%%%%%%%%%%%%%%%%%%%%%

% \appendix
% \section{Appendix Title}

% This is the text of the appendix, if you need one.


%%%%%%%%%%%%%%%%%%%%%%%%%%%%%%%%%%%%%%%%%%%%%%%%%%%%%%%%%%%%%%%%
%% Acknowledgments
%%%%%%%%%%%%%%%%%%%%%%%%%%%%%%%%%%%%%%%%%%%%%%%%%%%%%%%%%%%%%%%%

% \acks

% Acknowledgments, if needed.


%%%%%%%%%%%%%%%%%%%%%%%%%%%%%%%%%%%%%%%%%%%%%%%%%%%%%%%%%%%%%%%%
%% Bibliography
%%%%%%%%%%%%%%%%%%%%%%%%%%%%%%%%%%%%%%%%%%%%%%%%%%%%%%%%%%%%%%%%


\bibliography{paper}
\bibliographystyle{abbrvnat}

% The bibliography should be embedded for final submission.
%\begin{thebibliography}{}
%\softraggedright
%\end{thebibliography}

\end{document}
