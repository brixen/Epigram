\documentclass[preprint, authoryear]{sigplanconf}

\usepackage{amsmath}
\usepackage{verbatim}

%% Structure
\newenvironment{structure}{\footnotesize\verbatim}{\endverbatim}
%\newenvironment{structure}{\comment}{\endcomment}


\begin{document}

%\conferenceinfo{WXYZ '05}{date, City.} 
%\copyrightyear{2005} 
%\copyrightdata{[to be supplied]} 

%\titlebanner{banner above paper title}        % These are ignored unless
%\preprintfooter{short description of paper}   % 'preprint' option specified.



%%%%%%%%%%%%%%%%%%%%%%%%%%%%%%%%%%%%%%%%%%%%%%%%%%%%%%%%%%%%%%%%
%% Title
%%%%%%%%%%%%%%%%%%%%%%%%%%%%%%%%%%%%%%%%%%%%%%%%%%%%%%%%%%%%%%%%


\title{Generic programming is just programming}
% Or something mentionning levitation. Feel free to change. 
% \subtitle{Subtitle Text, if any}


%% Alphabetical ordering.
\authorinfo{James Chapman}
           {Institute of Cybernetics, Tallinn University of Technology}
           {james@cs.ioc.ee}
\authorinfo{Pierre-\'{E}variste Dagand \\ Conor McBride}
           {University of Strathclyde}
           {\{dagand,conor\}@cis.strath.ac.uk}
\authorinfo{Peter Morris}
           {University of Nottingham}
           {pwm@cs.nott.ac.uk}


\maketitle


%%%%%%%%%%%%%%%%%%%%%%%%%%%%%%%%%%%%%%%%%%%%%%%%%%%%%%%%%%%%%%%%
%% Abstract
%%%%%%%%%%%%%%%%%%%%%%%%%%%%%%%%%%%%%%%%%%%%%%%%%%%%%%%%%%%%%%%%


\begin{abstract}
This is the text of the abstract.
\end{abstract}

%\category{CR-number}{subcategory}{third-level}

%\terms
%term1, term2

%\keywords
%keyword1, keyword2



%%%%%%%%%%%%%%%%%%%%%%%%%%%%%%%%%%%%%%%%%%%%%%%%%%%%%%%%%%%%%%%%%%%%%%%%%%%%%
%%% Inference Rules (some ancient macros by Conor)                        %%%
%%%%%%%%%%%%%%%%%%%%%%%%%%%%%%%%%%%%%%%%%%%%%%%%%%%%%%%%%%%%%%%%%%%%%%%%%%%%%

\newlength{\rulevgap}
\setlength{\rulevgap}{0.05in}
\newlength{\ruleheight}
\newlength{\ruledepth}
\newsavebox{\rulebox}
\newlength{\GapLength}
\newcommand{\gap}[1]{\settowidth{\GapLength}{#1} \hspace*{\GapLength}}
\newcommand{\dotstep}[2]{\begin{tabular}[b]{@{}c@{}}
                         #1\\$\vdots$\\#2
                         \end{tabular}}
\newlength{\fracwid}
\newcommand{\dotfrac}[2]{\settowidth{\fracwid}{$\frac{#1}{#2}$}
                         \addtolength{\fracwid}{0.1in}
                         \begin{tabular}[b]{@{}c@{}}
                         $#1$\\
                         \parbox[c][0.02in][t]{\fracwid}{\dotfill} \\
                         $#2$\\
                         \end{tabular}}
\newcommand{\Rule}[2]{\savebox{\rulebox}[\width][b]                         %
                              {\( \frac{\raisebox{0in} {\( #1 \)}}       %
                                       {\raisebox{-0.03in}{\( #2 \)}} \)}   %
                      \settoheight{\ruleheight}{\usebox{\rulebox}}          %
                      \addtolength{\ruleheight}{\rulevgap}                  %
                      \settodepth{\ruledepth}{\usebox{\rulebox}}            %
                      \addtolength{\ruledepth}{\rulevgap}                   %
                      \raisebox{0in}[\ruleheight][\ruledepth]               %
                               {\usebox{\rulebox}}}
\newcommand{\Case}[2]{\savebox{\rulebox}[\width][b]                         %
                              {\( \dotfrac{\raisebox{0in} {\( #1 \)}}       %
                                       {\raisebox{-0.03in}{\( #2 \)}} \)}   %
                      \settoheight{\ruleheight}{\usebox{\rulebox}}          %
                      \addtolength{\ruleheight}{\rulevgap}                  %
                      \settodepth{\ruledepth}{\usebox{\rulebox}}            %
                      \addtolength{\ruledepth}{\rulevgap}                   %
                      \raisebox{0in}[\ruleheight][\ruledepth]               %
                               {\usebox{\rulebox}}}
\newcommand{\Axiom}[1]{\savebox{\rulebox}[\width][b]                        %
                               {$\frac{}{\raisebox{-0.03in}{$#1$}}$}        %
                      \settoheight{\ruleheight}{\usebox{\rulebox}}          %
                      \addtolength{\ruleheight}{\rulevgap}                  %
                      \settodepth{\ruledepth}{\usebox{\rulebox}}            %
                      \addtolength{\ruledepth}{\rulevgap}                   %
                      \raisebox{0in}[\ruleheight][\ruledepth]               %
                               {\usebox{\rulebox}}}
\newcommand{\RuleSide}[3]{\gap{\mbox{$#2$}} \hspace*{0.1in}               %
                            \Rule{#1}{#3}                          %
                          \hspace*{0.1in}\mbox{$#2$}}
\newcommand{\AxiomSide}[2]{\gap{\mbox{$#1$}} \hspace*{0.1in}              %
                             \Axiom{#2}                            %
                           \hspace*{0.1in}\mbox{$#1$}}
\newcommand{\RULE}[1]{\textbf{#1}}
\newcommand{\hg}{\hspace{0.2in}}


%%%%%%%%%%%%%%%%%%%%%%%%%%%%%%%%%%%%%%%%%%%%%%%%%%%%%%%%%%%%%%%%
%% Introduction
%%%%%%%%%%%%%%%%%%%%%%%%%%%%%%%%%%%%%%%%%%%%%%%%%%%%%%%%%%%%%%%%

\section{Introduction}

The text of the paper begins here. \cite{morris:spf}

ACM bullet points, we offer:
\begin{itemize}
\item a closed presentation of data-types (no generativity requires)
  subsuming standard inductive families (some popular extensions
  excluded for now)
\item descriptions of data-types are first-class (indeed,
  self-encoded)
\item ``generic programming is just programming''
\item first serious attempt to design a language for generic
  programming, except possibly Lisp
\end{itemize}

\begin{structure}
<- presentation of the Core theory
    -> judgemental equality (extensional (a la OTT?))
    /> not Epigram specific 
        <- having OTT just buys you more stuffs
\end{structure}

\newcommand{\Valid}{\textsc{valid}}
\newcommand{\Set}{\textsc{Set}}
\newcommand{\Type}[1]{{#1}:\Set}
\newcommand{\x}{x}
\newcommand{\bnd}[2]{{#1}\!:\!{#2}}
\newcommand{\xS}{\bnd{x}{S}}
\newcommand{\stkl}[1]{\begin{array}[b]{@{}l@{}}#1\end{array}}
\newcommand{\stkc}[1]{\begin{array}[b]{@{}c@{}}#1\end{array}}
\newcommand{\pr}[2]{\left[{#1},{#2}\right]}

\[
\begin{array}{ll}
\Gamma\vdash\Valid & \mbox{\(\Gamma\) is a valid context, giving types to
                    variables} \\
\Gamma\vdash t:T & \mbox{term \(t\) has type \(T\) in context \(\Gamma\)} \\
\Gamma\vdash s\equiv t : T & \mbox{\(s\) and \(t\) are equal at type \(T\)
   in context \(\Gamma\)} \\
\end{array}
\]

The systems of inference rules will be formulated to ensure that that the
following implications always hold by induction on derivations.
\[\begin{array}{l@{\;\;\Rightarrow\;\;}l}
\Gamma\vdash t:T & \Gamma\vdash\Valid \;\wedge\; \Gamma\vdash\Type{T} \\
\Gamma\vdash s\equiv t:T & \Gamma\vdash s:T \;\wedge\; \Gamma\vdash t:T \\
\Gamma;\xS;\Delta \vdash J &
  \Gamma\vdash s:S \;\Rightarrow\; \Gamma;\Delta[s/x] \vdash J[s/x] \\
\end{array}\]


\[
\Axiom{\vdash\Valid}\qquad
\Rule{\Gamma\vdash\Type{S}}
     {\Gamma;\xS\vdash\Valid}\;x\not\in\Gamma
\]

\[\stkc{\stkl{
\Rule{\Gamma\Valid}
     {\Gamma\vdash\Type{\Set}}\qquad
\Rule{\Gamma\vdash\Type{S}\quad\Gamma,\xS\vdash\Type{T}}
     {\Gamma\vdash\Type{(\xS)\to T, (\xS)\times T} }}
}\]

\[
\Rule{\Gamma;\xS;\Delta\vdash\Valid}
     {\Gamma;\xS;\Delta\vdash x:S}\qquad
\Rule{\Gamma\vdash s:S \quad \Gamma\vdash\Type{S\equiv T}}
     {\Gamma\vdash s:T}
\]

\[
\Rule{\stkl{\Gamma\vdash\Type{S}\\\Gamma;\xS\vdash t:T}}
     {\Gamma\vdash \lambda_S x.\,t:(\xS)\to T} \qquad
\Rule{\stkl{\Gamma\vdash f:(\xS)\to T \\
      \Gamma\vdash s:S}}
     {\Gamma\vdash f\:s \vdash T[s/x]} \qquad
\Rule{\stkl{\Gamma\vdash\Type{S}\quad\Gamma;\xS\vdash t:T\\
      \Gamma\vdash s:S}}
     {\Gamma\vdash (\lambda_S x.\,t)\:s \equiv t[s/x] \vdash T[s/x]}
\]

\[\Rule{\Gamma\vdash s:S \quad \Gamma;\xS\vdash\Type{T}\quad
      \Gamma\vdash t:T[s/x]}
     {\Gamma\vdash \pr{s}{t}_{x.T} : (\xS)\times T} \qquad
\Rule{\Gamma\vdash p:(\xS)\times T}
     {\Gamma\vdash \pi_0 p : S} \qquad
\Rule{\Gamma\vdash p:(\xS)\times T}
     {\Gamma\vdash \pi_1 p : T[\pi_0 p/x]}
\]


%%%%%%%%%%%%%%%%%%%%%%%%%%%%%%%%%%%%%%%%%%%%%%%%%%%%%%%%%%%%%%%%
%% Desc
%%%%%%%%%%%%%%%%%%%%%%%%%%%%%%%%%%%%%%%%%%%%%%%%%%%%%%%%%%%%%%%%

\section{A simple universe of descriptions}

\begin{structure}
<- a universe of simple inductive types
    -> nil, arg, ind, hind
<- levitation
\end{structure}

%%%%%%%%%%%%%%%%%%%%%%%%%%%%%%%%%%%%%%%%%%%%%%%%%%%%%%%%%%%%%%%%
%% IDesc
%%%%%%%%%%%%%%%%%%%%%%%%%%%%%%%%%%%%%%%%%%%%%%%%%%%%%%%%%%%%%%%%

\section{Indexing descriptions}


\begin{structure}
<- enumerations
<- indexing
    -> nil, arg, ind, hind
    -> var, Pi, Sigma, :-, x, Sigma_f
<- closure under \box and \diamond
<- Brady optimization as desc transformation
<- ``standard'' descriptions start with Sigma_f
    -> SIDesc : I -> (E : EnumU) -> Branches(E , \ e -> IDesc I )
<- ``standard'' descriptions closed under the free monad constructions
<- IDesc is just such a free monad
\end{structure}

%%%%%%%%%%%%%%%%%%%%%%%%%%%%%%%%%%%%%%%%%%%%%%%%%%%%%%%%%%%%%%%%
%% Discussion (?)
%%%%%%%%%%%%%%%%%%%%%%%%%%%%%%%%%%%%%%%%%%%%%%%%%%%%%%%%%%%%%%%%

\section{Discussion}

\begin{structure}
<- universe stratification
    -> how would IDesc play with it?
\end{structure}


%%%%%%%%%%%%%%%%%%%%%%%%%%%%%%%%%%%%%%%%%%%%%%%%%%%%%%%%%%%%%%%%
%% Conclusion
%%%%%%%%%%%%%%%%%%%%%%%%%%%%%%%%%%%%%%%%%%%%%%%%%%%%%%%%%%%%%%%%

\section{Conclusion}

\begin{structure}
\end{structure}


%%%%%%%%%%%%%%%%%%%%%%%%%%%%%%%%%%%%%%%%%%%%%%%%%%%%%%%%%%%%%%%%
%% Appendices
%%%%%%%%%%%%%%%%%%%%%%%%%%%%%%%%%%%%%%%%%%%%%%%%%%%%%%%%%%%%%%%%

% \appendix
% \section{Appendix Title}

% This is the text of the appendix, if you need one.


%%%%%%%%%%%%%%%%%%%%%%%%%%%%%%%%%%%%%%%%%%%%%%%%%%%%%%%%%%%%%%%%
%% Acknowledgments
%%%%%%%%%%%%%%%%%%%%%%%%%%%%%%%%%%%%%%%%%%%%%%%%%%%%%%%%%%%%%%%%

% \acks

% Acknowledgments, if needed.


%%%%%%%%%%%%%%%%%%%%%%%%%%%%%%%%%%%%%%%%%%%%%%%%%%%%%%%%%%%%%%%%
%% Bibliography
%%%%%%%%%%%%%%%%%%%%%%%%%%%%%%%%%%%%%%%%%%%%%%%%%%%%%%%%%%%%%%%%


\bibliography{paper}
\bibliographystyle{abbrvnat}

% The bibliography should be embedded for final submission.
%\begin{thebibliography}{}
%\softraggedright
%\end{thebibliography}

\end{document}
