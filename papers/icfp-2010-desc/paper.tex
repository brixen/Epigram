\documentclass[preprint
              , authoryear
%              , onecolumn
              ]{sigplanconf}

\usepackage{amsmath}
\usepackage{verbatim}
\usepackage{pig}

%%%%%%%%%%%%%%%%%%%%%%%%%%%%%%%%%%%%%%%%%%%%%%%%%%%%%%%%%%%%%%%%%%%%%%%%%%%%%
%%% Inference Rules (some ancient macros by Conor)                        %%%
%%%%%%%%%%%%%%%%%%%%%%%%%%%%%%%%%%%%%%%%%%%%%%%%%%%%%%%%%%%%%%%%%%%%%%%%%%%%%

\newlength{\rulevgap}
\setlength{\rulevgap}{0.05in}
\newlength{\ruleheight}
\newlength{\ruledepth}
\newsavebox{\rulebox}
\newlength{\GapLength}
\newcommand{\gap}[1]{\settowidth{\GapLength}{#1} \hspace*{\GapLength}}
\newcommand{\dotstep}[2]{\begin{tabular}[b]{@{}c@{}}
                         #1\\$\vdots$\\#2
                         \end{tabular}}
\newlength{\fracwid}
\newcommand{\dotfrac}[2]{\settowidth{\fracwid}{$\frac{#1}{#2}$}
                         \addtolength{\fracwid}{0.1in}
                         \begin{tabular}[b]{@{}c@{}}
                         $#1$\\
                         \parbox[c][0.02in][t]{\fracwid}{\dotfill} \\
                         $#2$\\
                         \end{tabular}}
\newcommand{\Rule}[2]{\savebox{\rulebox}[\width][b]                         %
                              {\( \frac{\raisebox{0in} {\( #1 \)}}       %
                                       {\raisebox{-0.03in}{\( #2 \)}} \)}   %
                      \settoheight{\ruleheight}{\usebox{\rulebox}}          %
                      \addtolength{\ruleheight}{\rulevgap}                  %
                      \settodepth{\ruledepth}{\usebox{\rulebox}}            %
                      \addtolength{\ruledepth}{\rulevgap}                   %
                      \raisebox{0in}[\ruleheight][\ruledepth]               %
                               {\usebox{\rulebox}}}
\newcommand{\Case}[2]{\savebox{\rulebox}[\width][b]                         %
                              {\( \dotfrac{\raisebox{0in} {\( #1 \)}}       %
                                       {\raisebox{-0.03in}{\( #2 \)}} \)}   %
                      \settoheight{\ruleheight}{\usebox{\rulebox}}          %
                      \addtolength{\ruleheight}{\rulevgap}                  %
                      \settodepth{\ruledepth}{\usebox{\rulebox}}            %
                      \addtolength{\ruledepth}{\rulevgap}                   %
                      \raisebox{0in}[\ruleheight][\ruledepth]               %
                               {\usebox{\rulebox}}}
\newcommand{\Axiom}[1]{\savebox{\rulebox}[\width][b]                        %
                               {$\frac{}{\raisebox{-0.03in}{$#1$}}$}        %
                      \settoheight{\ruleheight}{\usebox{\rulebox}}          %
                      \addtolength{\ruleheight}{\rulevgap}                  %
                      \settodepth{\ruledepth}{\usebox{\rulebox}}            %
                      \addtolength{\ruledepth}{\rulevgap}                   %
                      \raisebox{0in}[\ruleheight][\ruledepth]               %
                               {\usebox{\rulebox}}}
\newcommand{\RuleSide}[3]{\gap{\mbox{$#2$}} \hspace*{0.1in}               %
                            \Rule{#1}{#3}                          %
                          \hspace*{0.1in}\mbox{$#2$}}
\newcommand{\AxiomSide}[2]{\gap{\mbox{$#1$}} \hspace*{0.1in}              %
                             \AxiomB{#2}                            %
                           \hspace*{0.1in}\mbox{$#1$}}
\newcommand{\RULE}[1]{\textbf{#1}}
\newcommand{\hg}{\hspace{0.2in}}


%% Structure
\newenvironment{structure}{\footnotesize\verbatim}{\endverbatim}
%\newenvironment{structure}{\comment}{\endcomment}

%% Written bits of Structure
\newenvironment{wstructure}{\comment}{\endcomment}

%% Comments
%% \setlength{\marginparwidth}{0.7in}
%% \newcommand{\note}[1]{\-\marginpar[\raggedright\footnotesize #1]%
%%                                   {\raggedright\footnotesize #1}}
\newcommand{\note}[1]{}

%% Syntax
\newcommand{\bind}{\emph{bind}}
\newcommand{\return}{\emph{return}}

\begin{document}

\ColourEpigram

%\conferenceinfo{WXYZ '05}{date, City.} 
%\copyrightyear{2005} 
%\copyrightdata{[to be supplied]} 

%\titlebanner{banner above paper title}        % These are ignored unless
%\preprintfooter{short description of paper}   % 'preprint' option specified.



%%%%%%%%%%%%%%%%%%%%%%%%%%%%%%%%%%%%%%%%%%%%%%%%%%%%%%%%%%%%%%%%
%% Title
%%%%%%%%%%%%%%%%%%%%%%%%%%%%%%%%%%%%%%%%%%%%%%%%%%%%%%%%%%%%%%%%


\title{The Gentle Art of Levitation}


%% Alphabetical ordering.
\authorinfo{James Chapman}
           {Institute of Cybernetics, Tallinn University of Technology}
           {james@cs.ioc.ee}
\authorinfo{Pierre-\'{E}variste Dagand \\ Conor McBride}
           {University of Strathclyde}
           {\{dagand,conor\}@cis.strath.ac.uk}
\authorinfo{Peter Morris}
           {University of Nottingham}
           {pwm@cs.nott.ac.uk}


\maketitle


%%%%%%%%%%%%%%%%%%%%%%%%%%%%%%%%%%%%%%%%%%%%%%%%%%%%%%%%%%%%%%%%
%% Abstract
%%%%%%%%%%%%%%%%%%%%%%%%%%%%%%%%%%%%%%%%%%%%%%%%%%%%%%%%%%%%%%%%


\begin{abstract}
  We present a closed dependent type theory whose inductive datatypes
  are presented not by a scheme for generative datatype declaration,
  but by encoding in a \emph{universe}. Each inductive datatype is
  thus given by interpreting its \emph{description}---an ordinary
  first-class value in a datatype of descriptions. Moreover, the
  latter itself has a description. By presenting datatypes via such a
  self-encoding universe, datatype-generic programming becomes
  ordinary programming. We show some of the resulting generic
  operations and deploy them in particular, useful ways on the
  datatype of datatype descriptions itself. Surprisingly this
  apparently self-supporting setup is achievable without paradox or
  infinite regress.
\end{abstract}

%\category{CR-number}{subcategory}{third-level}

%\terms
%term1, term2

%\keywords
%keyword1, keyword2





%%%%%%%%%%%%%%%%%%%%%%%%%%%%%%%%%%%%%%%%%%%%%%%%%%%%%%%%%%%%%%%%
%% Introduction
%%%%%%%%%%%%%%%%%%%%%%%%%%%%%%%%%%%%%%%%%%%%%%%%%%%%%%%%%%%%%%%%

\section{Introduction}

Dependent datatypes, such as the ubiquitous vectors (lists indexed by
length) express \emph{relative} notions of data validity. They allow
us to function in a complex world with a higher standard of basic
hygiene than is practical with the context-free datatypes of ML-like
languages. Dependent type systems, as found in
Agda~\cite{norell:agda}, Coq~\cite{coq},
Epigram~\cite{mcbride.mckinna:view-from-the-left}, and contemporary
Haskell~\cite{spj:gadt}, are beginning to make themselves useful. As
with rope, the engineering benefits of type indexing sometimes
outweigh the difficulties you can arrange with enough of it.

%Dependent types are an appealing technique for building safer and more
%reliable software. By giving types more expressive power, the
%developer is able to encode more precise invariants in the types. As a
%result, more bugs are caught automatically, during
%type-checking. Because of this benefit, dependently-typed systems have
%flourished, such as Generalized Abstract Data-Types (GADT) in
%Haskell~\cite{spj:gadt}, Agda~\cite{norell:agda},
%Ynot~\cite{morrisett:ynot}, or Epigram~\cite{pigs:epigram}, to name
%but a few.

\begin{wstructure}
<- Describe the problem
    <- Data-types in dependent-type theory
        <- Much more precise
            <- More powerful type-system
            -> Stronger safety guarantees
\end{wstructure}

%In this paper, we will focus on data-types in such systems. Indeed,
%the expressive power of the type-system has a direct impact on
%data-types. Because types can \emph{depend} on terms, our data-types
%can be made more precise. The typical example is vectors, which type
%depend on the size of the vector. Having more precision about
%data-types, we can write safer code: taking the $\CN{head}$ of a
%vector is \emph{ensured} to succeed whenever its type states that it
%is a non-empty vector. This property is automatically enforced by the
%type-checker.

\begin{wstructure}
        <- Equipped with elimination principle
            <- Defining functions over them
            <- Making proofs over them
\end{wstructure}

%Moreover, in total programming systems, such as Agda, Ynot, or
%Epigram, data-types come equipped with an elimination principle: while
%a data-type definition introduces new type formers in the theory, we
%need an eliminator to dispose of them. Looking through the
%Curry-Howard lenses, the elimination principle corresponds to an
%induction principle associated with the data-type. To program over our
%data-types, we rely on their induction principle, guaranteeing the
%well-foundedness of our definition.

\begin{wstructure}
    <- Agda standard library [Nisse file]
        <- x implementations of natural numbers
        <- y implementations of lists
        -> Painful duplication of code and functionality
            <- Types are (slightly) different
                -> Same functions need to be re-implemented 
        -> Crucial need for ``genericity''
\end{wstructure}

The blessing of expressing just the right type for the job can also be
a curse. Where once we might have had a small collection of basic
datatypes and a large library, we now must cope with a cornucopia of
finely confected structures, subtly designed, subtly different. The
basic vector equipment is much like that for lists, but we implement
it separately, often retyping the same code. The Agda standard
library~\cite{nisse:asl}, for example, sports a writhing mass of
list-like structures, including vectors, bounded-length lists,
difference lists, reflexive-transitive closures---the list is
petrifying. Here, we seek equipment to tame this gorgon's head with
\emph{reflection}.

\begin{wstructure}
        /> Coq, Agda: external notion
            <- Not first-class citizen
            -> Cannot "compute" with them
            -> No reflection for data-types (?)
\end{wstructure}

The business of belonging to a datatype is itself a notion
relative to the type's \emph{declaration}. Most typed functional
languages, including those with dependent types, feature a datatype
declaration construct, external to and extending the language for
defining values and programs. However, dependent type systems also
allow us to reflect types as the image of a function from a set of
`codes'---a \emph{universe construction}~\cite{martin-lof:itt}. 
Computing with codes, we expose operations on and
relationships between the types they reflect. Here, we adopt
the universe as our guiding design principle. We abolish the
datatype declaration construct, by reflecting it as a datatype of
datatype descriptions which, moreover, \emph{describes itself}. This
apparently self-supporting construction is a trick, of course, but
we shall show the art of it. We contribute


%As in the simply-typed world, the definition of
%data-types is processed by a meta-theoretical engine, before being
%reifed by extending the type theory with the corresponding type
%formers and elimination principle. Because of this external apparatus,
%data-type definition is not \emph{first-class}: we cannot compute with
%them, such as making new data-types from previous data-types. 
%
%This is a rather harsh limitation, in particular in a
%dependently-typed system. Indeed, reflection~\cite{allen:reflection,
%  gregoire:ring-solver} is at the heart of many dependently-typed
%programming techniques. Not having first-class data-type definitions,
%we have to give up reflection for data-types.

\begin{wstructure}
    <- Dependent types offer new programming techniques
        <- Eg.: universe construction
        /> State of the art haunted by the simply-typed paradigm
            -> Generative
            -> Non reflective
\end{wstructure}

%However, we do not think that we are condemned to such fate. The
%external presentation of data-types is an heritage of the simply-typed
%paradigm. Dependently-typed systems have more to offer. Indeed, new
%programming techniques, unavailable in a simply-typed setting,
%arises. One of them is \emph{universe
%  construction}~\cite{martin-lof:itt}. We shall see how this technique
%help us overcoming the limitations of the standard, non reflective and
%generative presentation of data-types.

\begin{wstructure}
<- State contributions
    <- Closed presentation of data-types 
        -> No generativity requires
        -> Subsuming standard inductive families 
            /> Some popular extensions excluded for now
    <- Descriptions of data-types are first-class 
        <- Self-encoded [Section sec:desc-levitate]
    <- ``generic programming is just programming''
        <- Ability to inspect data-type definition
            -> Write program over them
        <- A generic program works over a class of data-types (???)
            -> Capture this class by common structure
            -> Write a program over this common code
    <- Design a language for generic programming
        -> First serious attempt
            /> except possibly Lisp
                <- ???
\end{wstructure}

%In this paper, we propose a new approach to building data-types in a
%dependent-type theory. Our contributions are the following:

\begin{itemize}
\item a \emph{closed} type theory, equipped with a universe of
  inductive datatypes and a generic induction principle for programming
  and proof, extensible only with \emph{definitions};
\item a \emph{self-encoding} of the universe codes as a datatype in the
  universe---datatype generic programming is just programming;
\item a bidirectional \emph{type propagation} mechanism to conceal
  artefacts of the encoding, restoring
  a convenient presentation of data;
\item examples of generic operations and constructions over our universe,
  notably the \emph{free monad} construction;
\item a language sustaining datatype generic programming \emph{directly},
  with no need to mediate between declared datatypes and some isomorphic
  model or `view'.
\end{itemize}

We study two universes as a means to explore this novel way to equip a
programming language with its datatypes. We warm up with a universe of
\emph{simple} datatypes, just sufficient to describe itself. Once we
have learned this art, we scale up to \emph{indexed} datatypes,
encompassing the inductive families~\cite{dybjer:families,luo:utt}
found in Coq and Epigram, and delivering experiments in generic
programming with applications to the datatype of codes itself.

We aim to deliver proof of concept, showing that a closed theory with
a self-encoding universe of datatypes can be made practicable, but we
are sure there are bigger and better universes waiting for a similar
treatment. Benke, Dybjer and
Jansson~\cite{benke:universe-generic-prog} provide a useful survey of
the possibilities, including extension to inductive-recursive
definition, whose closed-form presentation~\cite{dybjer:axiom-ir,
  dybjer:ir-initial-algebra} is both an inspiration for the present
enterprise, and a direction for future study.

The work of Morris, Altenkirch and
Ghani~\cite{morris:PhD,morris:spf,alti:lics09} on
(indexed) containers has informed our style of encoding and the
equipment we choose to develop, but the details here reflect pragmatic
concerns about intensional properties which demand care in
practice. We have thus been able to implement our work as the basis
for datatypes in the Epigram 2 prototype~\cite{pigs:epigram}. We
have also developed a \emph{stratified} model of our coding scheme
in Agda.




%\item We present a basic type-theory and extend it with a universe of
%  finite sets (Section~\ref{sec:type-theory}). We show how coding can
%  be made practical by putting types at work ;
%\item We give a closed presentation of inductive data-types, through a
%  universe of descriptions (Section~\ref{sec:universe-desc}). This
%  first universe has the expressive power of simple inductive
%  types. Being closed, this presentation does not require
%  generativity, hence the type theory remains unchanged when
%  data-types are introduced ;
%\item We present a self-encoding of the universe of description inside
%  itself (Section~\ref{sec:desc-levitate}). As a consequence,
%  description of data-types appears as first-class object in the type
%  theory. We illustrate the benefit of a first-order presentation by
%  implementing a generic catamorphism as well as a generic free monad
%  construction, together with its monadic operations ;
%\item We index the universe of descriptions, to subsume standard
%  inductive families (Section~\ref{sec:indexing-desc}). In this
%  setting, we develop several examples of dependently-typed
%  data-structure and some generic operations over them ;
%\item We have implemented this technology in the Epigram programming
%  language. This is, we believe, the first attempt to design a
%  language for generic programming, Lisp having opened the way. We
%  propose and demonstrate with several examples that generic
%  programming is just programming. Because data-types are described by
%  code, we can finally program with them. As a consequence, generic
%  programs are implemented as functions built from the data-type
%  definition.
%\end{itemize}

%%%%%%%%%%%%%%%%%%%%%%%%%%%%%%%%%%%%%%%%%%%%%%%%%%%%%%%%%%%%%%%%
%% The Type Theory
%%%%%%%%%%%%%%%%%%%%%%%%%%%%%%%%%%%%%%%%%%%%%%%%%%%%%%%%%%%%%%%%


\section{The Type Theory}
\label{sec:type-theory}

One challenge in writing this paper is to extricate our account of datatypes from what else is new in Epigram 2. In fact, we demand relatively little from the setup, so we shall start with a `vanilla' theory and add just what we need. The reader accustomed to dependent types will recognise the basis of her favourite system; for those less familiar, we try to keep the presentation self-contained.

\subsection{Base theory}

\begin{wstructure}
<- Presentation of the formalism
    <- Standard presentation
        -> No novelty here
    <- 3 judgments [equation]
        -> Context validity
        -> Typing judgements
        -> Equality judgements
\end{wstructure}

We adopt a traditional presentation for our type
theory, with three mutually defined systems of judgments:
\emph{context validity}, \emph{typing}, and \emph{equality},
with the following forms:
%
\[
\begin{array}{ll}
\G \vdash \Valid                & \mbox{\(\G\) is a valid context, giving types to variables} \\
\G \vdash \Bhab{\M{t}}{\M{T}}           & \mbox{term \(\M{t}\) has type \(\M{T}\) in context \(\G\)} \\
\G \vdash \Bhab{\M{s} \equiv \M{t}}{\M{T}}  & \mbox{\(\M{s}\) and \(\M{t}\) are equal at type \(\M{T}\) in context \(\G\)} \\
\end{array}
\]

\begin{wstructure}
    <- Invariants [equation]
        -> By induction on derivations
\end{wstructure}

The rules are formulated to ensure that the
following `sanity checks' hold by induction on derivations
%
\[
\begin{array}{l@{\;\Rightarrow\;\;}l}
\G            \vdash \Bhab{\M{t}}{\M{T}}            
    & \G \vdash \Valid\; \wedge\; \G\vdash\Type{\M{T}} \\
\G            \vdash s \equiv \Bhab{\M{t}}{\M{T}}   
    & \G \vdash \Bhab{s}{T} \;\wedge\; \G\vdash \Bhab{\M{t}}{\M{T}}
\end{array} \]
%
and that judgments \(\M{J}\) are preserved by well-typed instantiation.
%
\[
\G ; \xS ; \Delta \vdash \M{J}            \;\Rightarrow\;      
    \G \vdash \Bhab{\M{s}}{\M{S}} \;\Rightarrow\; 
          \G ; \Delta[\M{s}/\x] \vdash \M{J}[\M{s}/\x] 
\]

%We are not going to prove the validity of these invariants. They
%follow rather straightforwardly from the induction rules. For formal
%proofs, we refer the reader to standard presentations of type theory,
%such as Luo's seminal work \cite{luo:utt}.

\begin{wstructure}
    <- Judgemental equality
        <- Presentation independent of particular implementation choice
        -> Model in Agda, intensional
        -> Used in Epigram, OTT
\end{wstructure}

We specify equality as a judgment, leaving open the details of its implementation, requiring only a congruence including ordinary computation (\(\beta\)-rules), decided, e.g., by testing \(\alpha\)-equivalence of \(\beta\)-normal forms~\cite{DBLP:journals/jfp/Adams06}. Coquand and Abel feature prominently in a literature of richer equalities, involving \(\eta\)-expansion, proof-irrelevance and other attractions~\cite{DBLP:journals/scp/Coquand96,DBLP:conf/tlca/AbelCP09}. Agda and Epigram 2 support such features, Coq currently does not, but they are surplus to requirements here.

% Therefore, we are not tied to a particular implementation
%choice. In particular, our system has been modelled in Agda, which
%features an intensional equality. On the other hand, it is used in
%Epigram, whose equality has a slightly extensional
%flavor~\cite{altenkirch:ott}. We expect users of fully extensional
%systems to also find their way through this presentation.

\begin{wstructure}
<- Context validity [figure no longer]
    <- Not much to be said
\end{wstructure}

Context validity ensures that variables inhabit well-formed
sets.
%
\[
%% Empty context validity
\Axiom{\vdash \Valid}
\qquad
%% Extend context
\Rule{\G       \vdash \Type{\M{S}}}
     {\G ; \xS \vdash \Valid}\;\x\not\in\G
\]
%
\begin{wstructure}
<- Typing judgements [figure]
    <- Set in Set
        -> For simplicity of presentation
        -> Assume that a valid stratification can be inferred
            <- Harper-Pollack, Luo, Courant
        -> See later discussion
    <- Standard presentation of Pi and Sigma types
\end{wstructure}
%
The basic typing rules
% (Fig.~\ref{fig:typing-judgements})
for tuples and functions are also standard, save that we locally adopt
\(\Set:\Set\) for presentational purposes. Usual techniques to resolve
this \emph{typical ambiguity} apply~\cite{harper:implicit-universe,
  luo:utt, courant:explicit-universe}. A formal treatment of
stratification for our system is a matter of
ongoing work.
%%  putting presentation before paradox~\cite{girard:set-in-set}.
%% The usual remedies apply, \emph{stratifying}
%% \(\Set\)~\cite{harper:implicit-universe, luo:utt,
%%   courant:explicit-universe}.
%%
\[\stkc{
%% Context
\Rule{\Gamma ; \xS ; \Delta \vdash \Valid}
     {\Gamma ; \xS ; \Delta \vdash \Bhab{x}{S}}
\qquad
%% Conversion
\Rule{\Gamma \vdash \Bhab{s}{S} \quad 
      \Gamma \vdash \Type{S \equiv T}}
     {\Gamma \vdash \Bhab{s}{T}}
\\
%% Girard's favorite
\Rule{\Gamma \vdash \Valid}
     {\Gamma \vdash \Type{\Set}}
\qquad
%% Pi-Sigma
\Rule{\Gamma       \vdash \Type{S} \quad
      \Gamma ; \xS \vdash \Type{T}}
     {\Gamma \vdash \Type{\PIS{\xS} T, \SIGMAS{\xS} T}}
\\
%% %% Prop
%% \Rule{\Gamma \vdash \Bhab{q}{\Prop}}
%%      {\Gamma \vdash \Type{\prf{q}}}
%% \qquad
%% %% True
%% \Rule{\Gamma \vdash \Valid}
%%      {\Gamma \vdash \Bhab{\True}{\Prop}}
%% \\
%% Unit
\Rule{\Gamma \vdash \Valid}
     {\Gamma \vdash \Bhab{\Unit}{\Set}}
\qquad
%% Void
\Rule{\Gamma \vdash \Valid}
     {\Gamma \vdash \Bhab{\Void}{\Unit}}
\\
%% Lambda
\Rule{\stkl{\Gamma       \vdash \Type{S} \\
            \Gamma ; \xS \vdash \Bhab{t}{T}}}
     {\Gamma \vdash \Bhab{\PLAM{\x}{S} t}{\PIS{\xS} T}}
\qquad
%% Application
\Rule{\stkl{\Gamma \vdash \Bhab{f}{\PIS{\xS} T} \\
            \Gamma \vdash \Bhab{s}{S}}}
     {\Gamma \vdash \Bhab{f\: s}{T[s/x]}} 
\\
%% Pair
\Rule{\Gamma       \vdash \Bhab{s}{S} \quad 
      \Gamma ; \xS \vdash \Bhab{T}{\Set}    \quad
      \Gamma       \vdash \Bhab{t}{T[s/x]}}
     {\Gamma \vdash \Bhab{\pair{s}{t}{x.T}}{\SIGMAS{\xS} T}}
\\
%% First projection
\Rule{\Gamma \vdash \Bhab{p}{\SIGMAS{\xS} T}}
     {\Gamma \vdash \Bhab{\fst{p}}{S}} 
\qquad
%% Second projection
\Rule{\Gamma \vdash \Bhab{p}{\SIGMAS{\xS} T}}
     {\Gamma \vdash \Bhab{\snd{p}}{T[\fst{p}/x]}}
\\
}\]


\paragraph{Notation.} We subscript information needed for type synthesis
but not type checking, e.g., the domain of a \(\LAMBINDER\)-abstraction,
and suppress it informally where clear. Square brackets denote tuples,
with a LISP-like right-nesting convention: \(\sqr{a\;b}\) abbreviates
\(\pair{a}{\pair{b}{\void}{}}{}\).

%We recognise the standard presentation of $\Pi$ and $\Sigma$
%types, respectively inhabited by lambda terms and dependent
%pairs. Naturally, there are rules for function application and
%projections of $\Sigma$-types. Equal types can be substituted, thanks
%to the conversion rule.

%For the sake of presentation, we postulate a $\Set$ in $\Set$
%rule. Having this rule makes our type theory inconsistent, by Girard's
%paradox~\cite{girard:set-in-set}. However, it has been
%shown~\cite{harper:implicit-universe, luo:utt,
%  courant:explicit-universe} that a valid stratification can be
%inferred, automatically or semi-automatically. In the remaining of our
%presentation, we will assume that such a stratification exists, even
%though we will keep it implicit. We shall discuss this assumption in
%Section~\ref{sec:discussion}.

%\begin{figure}
%
%\[\stkc{
%% Context
\Rule{\Gamma ; \xS ; \Delta \vdash \Valid}
     {\Gamma ; \xS ; \Delta \vdash \Bhab{x}{S}}
\qquad
%% Conversion
\Rule{\Gamma \vdash \Bhab{s}{S} \quad 
      \Gamma \vdash \Type{S \equiv T}}
     {\Gamma \vdash \Bhab{s}{T}}
\\
%% Girard's favorite
\Rule{\Gamma \vdash \Valid}
     {\Gamma \vdash \Type{\Set}}
\qquad
%% Pi-Sigma
\Rule{\Gamma       \vdash \Type{S} \quad
      \Gamma ; \xS \vdash \Type{T}}
     {\Gamma \vdash \Type{\PIS{\xS} T, \SIGMAS{\xS} T}}
\\
%% %% Prop
%% \Rule{\Gamma \vdash \Bhab{q}{\Prop}}
%%      {\Gamma \vdash \Type{\prf{q}}}
%% \qquad
%% %% True
%% \Rule{\Gamma \vdash \Valid}
%%      {\Gamma \vdash \Bhab{\True}{\Prop}}
%% \\
%% Unit
\Rule{\Gamma \vdash \Valid}
     {\Gamma \vdash \Bhab{\Unit}{\Set}}
\qquad
%% Void
\Rule{\Gamma \vdash \Valid}
     {\Gamma \vdash \Bhab{\Void}{\Unit}}
\\
%% Lambda
\Rule{\stkl{\Gamma       \vdash \Type{S} \\
            \Gamma ; \xS \vdash \Bhab{t}{T}}}
     {\Gamma \vdash \Bhab{\PLAM{\x}{S} t}{\PIS{\xS} T}}
\qquad
%% Application
\Rule{\stkl{\Gamma \vdash \Bhab{f}{\PIS{\xS} T} \\
            \Gamma \vdash \Bhab{s}{S}}}
     {\Gamma \vdash \Bhab{f\: s}{T[s/x]}} 
\\
%% Pair
\Rule{\Gamma       \vdash \Bhab{s}{S} \quad 
      \Gamma ; \xS \vdash \Bhab{T}{\Set}    \quad
      \Gamma       \vdash \Bhab{t}{T[s/x]}}
     {\Gamma \vdash \Bhab{\pair{s}{t}{x.T}}{\SIGMAS{\xS} T}}
\\
%% First projection
\Rule{\Gamma \vdash \Bhab{p}{\SIGMAS{\xS} T}}
     {\Gamma \vdash \Bhab{\fst{p}}{S}} 
\qquad
%% Second projection
\Rule{\Gamma \vdash \Bhab{p}{\SIGMAS{\xS} T}}
     {\Gamma \vdash \Bhab{\snd{p}}{T[\fst{p}/x]}}
\\
}\]

%
%\caption{Typing judgements}
%\label{fig:typing-judgements}
%
%\end{figure}


\begin{wstructure}
<- Judgemental equality [figure]
    <- symmetry, reflexivity, and transitivity
    <- beta-rules for lambda and pair
    <- xi-rule for functions
    -> Agnostic in the notion of equality
        <- Doesn't rely on a ``propositional'' equality
        -> Key: wide applicability of our proposal
\end{wstructure}

The judgmental equality comprises the computational rules below,
closed under reflexivity,
symmetry, transitivity and structural congruence, even under binders.
We omit the mundane rules which ensure these closure properties for
reasons of space.
\[\stkc{
%% %% Reflexivity
%% \Rule{\Gamma \vdash \Bhab{x}{T}}
%%      {\Gamma \vdash \Bhab{x \equiv x}{T}}
%% \qquad
%% %% Symmetry
%% \Rule{\Gamma \vdash \Bhab{x \equiv y}{T}}
%%      {\Gamma \vdash \Bhab{y \equiv x}{T}}
%% \qquad
%% %% Transitivity
%% \Rule{\stkl{\Gamma \vdash \Bhab{z \equiv y}{T} \\
%%             \Gamma \vdash \Bhab{y \equiv z}{T} }}
%%      {\Gamma \vdash \Bhab{x \equiv z}{T}}
%% \\
%% Beta-reduction
\Rule{\stkl{\Gamma       \vdash \Type{\M{S}} \quad
            \Gamma ; \xS \vdash \Bhab{\M{t}}{\M{T}} \\
            \Gamma       \vdash \Bhab{\M{s}}{\M{S}}}}
     {\Gamma \vdash \Bhab{(\PLAM{\x}{\M{S}} \M{t})\:\M{s} \equiv \M{t}[\M{s}/\x]}{\M{T}[\M{s}/\x]}}
\\
%% Xi rule
%% \Rule{\Gamma       \vdash \Type{S} \quad
%%       \Gamma ; \xS \vdash \Bhab{t \equiv t'}{T}}
%%      {\Gamma \vdash \Bhab{(\PLAM{\x}{S} t) \equiv (\PLAM{\x}{S} t')}{\PIS{\xS} T}}
%% \\
%% Projections
\Rule{\stkl{\Gamma                 \vdash \Bhab{\M{s}}{\M{S}} \quad
            \Gamma ; \xS           \vdash \Bhab{\M{T}}{\Set} \\
            \Gamma ; \Bhab{\M{s}}{\M{S}}   \vdash \Bhab{\M{t}}{\M{T}[\M{s}/\x]}}}
     {\Gamma \vdash \Bhab{\fst{(\pair{\M{s}}{\M{t}}{\x.\M{T}})} \equiv \M{s}}{\M{S}}}
\qquad
\Rule{\stkl{\Gamma               \vdash \Bhab{\M{s}}{\M{S}} \quad
            \Gamma ; \xS         \vdash \Bhab{\M{T}}{\Set} \\
            \Gamma ; \Bhab{\M{s}}{\M{S}} \vdash \Bhab{\M{t}}{\M{T}[\M{s}/\x]}}}
     {\Gamma \vdash \Bhab{\snd{(\pair{\M{s}}{\M{t}}{\x.\M{T}})} \equiv \M{t}}{\M{T}[\M{s}/\x]}}
}\]

Given a suitable stratification of \(\Set\), the
computation rules yield a terminating evaluation procedure, ensuring
the decidability of equality and thence type checking.

%Finally, we define the rules governing judgemental equality in
%Figure~\ref{fig:judgemental-equality}. We implicitly assume that
%judgemental equality respects symmetry, reflexivity, and
%transitivity. We capture the computational behavior of the language
%through the $\beta$-rules for function application and pairs. Finally,
%we implicitly assume that it respects purely syntactic and structural
%equality. This includes equality under lambda ($\xi$-rule).

%Crucially, being judgemental, this presentation is agnostic in the
%notion of equality actually implemented. Indeed, our typing and
%equality judgements do not rely on a ``propositional'' equality. This
%freedom is a key point in favour of the wide applicability of our
%proposal. This judgemental presentation must be read as a
%\emph{specification}: our proposal works with any propositional
%equality satisfying this specification. Moreover, our lightweight
%requirements do not endanger decidability of equality-checking.
%Obviously, when implementing our technology in an existing
%type-theory, some opportunities arise. We will present some of them
%along the course of the paper.


%\begin{figure}

%\[\stkc{
%% %% Reflexivity
%% \Rule{\Gamma \vdash \Bhab{x}{T}}
%%      {\Gamma \vdash \Bhab{x \equiv x}{T}}
%% \qquad
%% %% Symmetry
%% \Rule{\Gamma \vdash \Bhab{x \equiv y}{T}}
%%      {\Gamma \vdash \Bhab{y \equiv x}{T}}
%% \qquad
%% %% Transitivity
%% \Rule{\stkl{\Gamma \vdash \Bhab{z \equiv y}{T} \\
%%             \Gamma \vdash \Bhab{y \equiv z}{T} }}
%%      {\Gamma \vdash \Bhab{x \equiv z}{T}}
%% \\
%% Beta-reduction
\Rule{\stkl{\Gamma       \vdash \Type{\M{S}} \quad
            \Gamma ; \xS \vdash \Bhab{\M{t}}{\M{T}} \\
            \Gamma       \vdash \Bhab{\M{s}}{\M{S}}}}
     {\Gamma \vdash \Bhab{(\PLAM{\x}{\M{S}} \M{t})\:\M{s} \equiv \M{t}[\M{s}/\x]}{\M{T}[\M{s}/\x]}}
\\
%% Xi rule
%% \Rule{\Gamma       \vdash \Type{S} \quad
%%       \Gamma ; \xS \vdash \Bhab{t \equiv t'}{T}}
%%      {\Gamma \vdash \Bhab{(\PLAM{\x}{S} t) \equiv (\PLAM{\x}{S} t')}{\PIS{\xS} T}}
%% \\
%% Projections
\Rule{\stkl{\Gamma                 \vdash \Bhab{\M{s}}{\M{S}} \quad
            \Gamma ; \xS           \vdash \Bhab{\M{T}}{\Set} \\
            \Gamma ; \Bhab{\M{s}}{\M{S}}   \vdash \Bhab{\M{t}}{\M{T}[\M{s}/\x]}}}
     {\Gamma \vdash \Bhab{\fst{(\pair{\M{s}}{\M{t}}{\x.\M{T}})} \equiv \M{s}}{\M{S}}}
\qquad
\Rule{\stkl{\Gamma               \vdash \Bhab{\M{s}}{\M{S}} \quad
            \Gamma ; \xS         \vdash \Bhab{\M{T}}{\Set} \\
            \Gamma ; \Bhab{\M{s}}{\M{S}} \vdash \Bhab{\M{t}}{\M{T}[\M{s}/\x]}}}
     {\Gamma \vdash \Bhab{\snd{(\pair{\M{s}}{\M{t}}{\x.\M{T}})} \equiv \M{t}}{\M{T}[\M{s}/\x]}}
}\]

%
%\caption{Judgemental equality}
%\label{fig:judgemental-equality}
%
%\end{figure}



\begin{wstructure}
!!! Need Help !!!
<- Meta-theoretical properties
    <- Assuming a stratified discipline
    <> The point here is to reassert that dependent types are not evil, 
       there is no non-terminating type-checker, or such horrible lies <>
    -> Strongly normalising
        -> Every program terminates
    -> Type-checking terminates
    ???
\end{wstructure}


%This completes our presentation of the type theory. Assuming a
%stratified discipline of universe, the system we have described enjoy
%some very strong meta-theoretical properties. Unlike simply typed
%languages, such as Haskell, dependently-typed systems are
%\emph{strongly normalising}: every program that type-checks
%terminates. Moreover, type-checking is decidable and can therefore be
%implemented by a terminating algorithm.
%\note{Need some care here. Expansion would be good too. I wanted to
%  carry the intuition that we are not the bad guys with a
%  non-terminating type-checker.}

\subsection{Finite enumerations of tags}
\label{sec:finite-sets}

\begin{wstructure}
<- Motivation
    <- Finite sets could be encoded with Unit and Bool
        /> Hinder the ability to name things
    <- W-types considered harmful?
        ???
    -> For convenience
        <- Named elements
        <- Referring by name instead of code
        -> Types as coding presentation
            /> Also as coding representation!
\end{wstructure}

It is time for our first example of a \emph{universe}. You might
want to offer a choice of named constructors in your datatypes: we shall
equip you with sets of tags to choose from. Our plan
is to implement (by extending the theory, or by encoding) the signature
%
\[
  \Type{\EnumU}\qquad \Type{\EnumT{(\Bhab{\M{E}}{\EnumU})}}
\]
%
where some value \(E:\EnumU\) in the `enumeration universe' describes
a type of tag choices \(\EnumT{E}\). We shall need
some tags---valid identifiers, marked to indicate that
they are data, not variables scoped and substitutable---so we hardwire these
rules:
%
\[
%% UId
\Rule{\Gamma \vdash \Valid}
     {\Gamma \vdash \Type{\UId}}
\qquad
%% Tag
\Rule{\Gamma \vdash \Valid}
     {\Gamma \vdash \Bhab{\Tag{\V{s}}}{\UId}}\;\V{s}\: \mbox{a valid identifier}
\]
%
Let us describe enumerations as lists of tags, with signature:
%
\[
\Bhab{\NilE}{\EnumU}\qquad
\Bhab{\ConsE{(\Bhab{\M{t}}{\UId})}{(\Bhab{\M{E}}{\EnumU})}}{\EnumU}
\]
%
What are the \emph{values} in \(\EnumT{E}\)? Formally, we represent
the choice of a tag as a numerical index into \(E\), via new rules:
%
\[
%% Ze
\Rule{\Gamma \vdash \Valid}
     {\Gamma \vdash \Bhab{\Ze}{\EnumT{(\ConsE{\M{t}}{\M{E}})}}} 
\qquad
%% Su
\Rule{\Gamma \vdash \Bhab{\M{n}}{\EnumT{\M{E}}}}
     {\Gamma \vdash \Bhab{\Su{\M{n}}}{\EnumT{(\ConsE{\M{t}}{\M{E}})}}}
\]
%
However, we expect that in practice, you might rather refer to these
values \emph{by tag}, and we shall ensure that this is possible in due course.

%As a motivating example, we are now going to extend the type theory
%with a notion of finite set. One could argue that there is no need for
%such an extension: finite sets, just as any data-structure, can be
%encoded inside the type theory. A well-known example of such encoding
%is the Church encoding of natural numbers, which is isomorphic to
%finite sets. \note{Shall we talk about W-types encoding?}

%However, using encodings is impractical. In the case of finite sets,
%for instance, we would like to name the elements of the sets. Then, we
%need to be able to manipulate these elements by their name, instead of
%their encoding. While we are able to give names to encodings, it is
%extremely tedious to map the encodings back to a name. Whereas these
%objects have a structure, the structure is lost during the encoding,
%when they become anonymous inhabitants of a $\Pi$ or $\Sigma$-type.

%In the simply-typed world, we are used to see types as a coding
%presentation -- presentation of invariants, presentation of
%properties. In the dependently-typed world, we also learn to use types
%as a coding representation: finite sets being good citizens, they
%ought to be democratically represented at the type level. As we will
%see, this gives us the ability to name and manipulate them (this is
%were the democracy analogy goes crazy, I think).
%\note{Did I got the coding presentation vs. coding representation
%  story right? No.}

\begin{wstructure}
<- Implementation [figure]      
    <- Tags
        -> Purely informational token
    <- EnumU
        -> Universe of finite sets
    <- EnumT e
        -> Elements of finite set e
\end{wstructure}

%The specification of finite sets is presented in
%Figure~\ref{fig:typing-finite-set}. It is composed of three
%components. First, we define tags as inhabitants of the $\UId$ type. A
%tag is solely an informative token, used for diagnostic
%purposes. Finite sets inhabits the $\EnumU$ type. Unfolding the
%definition, we get that a finite set is a list of tags. Finally,
%elements of a finite set $\V{u}$ belong to the corresponding $\EnumT{\V{u}}$
%type. Intuitively, it corresponds to an index -- a number -- pointing
%to an element of $\V{u}$.

%\begin{figure}

%\[\stkc{
%% UId
\Rule{\Gamma \vdash \Valid}
     {\Gamma \vdash \Type{\UId}}
\qquad
%% Tag
\Rule{\Gamma \vdash \Valid}
     {\Gamma \vdash \Bhab{\Tag{\V{s}}}{\UId}}\;\V{s} \mbox{ unique identifier}
\\
%% EnumU
\Rule{\Gamma \vdash \Valid}
     {\Gamma \vdash \Type{\EnumU}} 
\qquad
%% EnumT
\Rule{\Gamma \vdash \Bhab{\V{e}}{\EnumU}}
     {\Gamma \vdash \Type{\EnumT{\V{e}}}} 
\\
%% NilE
\Rule{\Gamma \vdash \Valid}
     {\Gamma \vdash \Bhab{\NilE}{\EnumU}} 
\qquad
%% ConsE
\Rule{\Gamma \vdash \Bhab{\V{t}}{\UId} \quad
      \Gamma \vdash \Bhab{\V{e}}{\EnumU}}
     {\Gamma \vdash \Bhab{\ConsE{\V{t}}{\V{e}}}{\EnumU}}
\\
%% Ze
\Rule{\Gamma \vdash \Valid}
     {\Gamma \vdash \Bhab{\Ze}{\EnumT{\ConsE{\V{t}}{\V{e}}}}} 
\qquad
%% Su
\Rule{\Gamma \vdash \Bhab{\V{n}}{\EnumT{\V{e}}}}
     {\Gamma \vdash \Bhab{\Su{\V{n}}}{\EnumT{\ConsE{\V{t}}{\V{e}}}}}
}\]


%\caption{Typing rules for finite sets}
%\label{fig:typing-finite-set}

%\end{figure}


\begin{wstructure}
<- Equipment
    <- \spi operator
        <- Equivalent of Pi on finite sets
        <- First argument: (finite) domain
        <- Second argument: for each element of the domain, a co-domain
        -> Inhabitant of \spi: right-nested tuple of solutions
            <- Skip code for space reasons
    <- switch operator
        <- case analyses over x
        <- index into the \spi tuple to retrieve the corresponding result
\end{wstructure}

Enumerations come with further machinery. Each \(\EnumT{E}\) needs an
eliminator, allowing us to branch according to a tag choice. Formally,
whenever we need such new computational facilities, we add primitive
operators to the type theory and extend the judgmental equality with
their computational behavior. However, for compactness and readability, we
shall write these operators as functional programs (much as we
model them in Agda).

We first define the `small product' $\SYMBspi$ operator:
%
\[\stk{
%% spi
\spi{}{}: \PITEL{\V{E}}{\EnumU}\PITEL{\V{P}}{\EnumT{\V{E}} \To \Set} \To \Set \\
\begin{array}{@{}l@{\:}l@{\;\;\mapsto\;\;}l}
\spi{\NilE}{& \V{P}}        & \Unit \\
\spi{(\ConsE{\V{t}}{\V{E}})}{& \V{P}} & \TIMES{\V{P}\: \Ze}{\spi{\V{E}}{\LAM{\V{x}} \V{P}\: (\Su{\V{x}})}}
\end{array}
}\]
%
This builds a right-nested
tuple type, packing a $\V{P}\:\V{i}$ value for each $\V{i}$ in the given
domain. The step case exposes our notational convention that
binders scope rightwards as far as possible. These tuples are `jump tables', tabulating dependently
typed functions.
We give this functional
interpretation---the eliminator we need---by the
$\SYMBswitch$ operator, which, unsurprisingly, iterates projection:
%
\[\stk{
%% switch
\begin{array}{@{}ll}
\SYMBswitch : \PITEL{\V{E}}{\EnumU}
               \PITEL{\V{P}}{\EnumT{\V{E}} \To \Set} \To
              \spi{\V{E}}{\V{P}} \To
               \PITEL{\V{x}}{\EnumT{\V{E}}} \To \V{P}\: \x
\end{array} \\
\begin{array}{@{}l@{\:\mapsto\:\:}l}
\switch{(\ConsE{\V{t}}{\V{E}})}{\V{P}}{\V{b}}{\Ze}      & \fst{\V{b}} \\
\switch{(\ConsE{\V{t}}{\V{E}})}{\V{P}}{\V{b}}{(\Su{\V{x}})} & \switch{\V{E}}{(\LAM{\V{x}} \V{P}
  (\Su{\V{x}}))}{(\snd{\V{b}})}{\V{x}}
\end{array}
}\]

%Again, there is a clear equivalent in the full-$\Set$ world: function
%application. The operational behaviour of $\F{switch}$ is
%straightforward: $\V{x}$ is peeled off as we move deeper inside the nested
%tuple $\V{b}$. When $\V{x}$ equals $\Ze$, we simply return the value we are
%pointing to.

\begin{wstructure}
<- Equivalent to having a function space over finite sets
    /> Made non-obvious by low-level encodings
        <- General issue with codes
             -> Need to provide an attractive presentation to the user
    -> Types seem to obfuscate our reading
        <- Provide ``too much'' information
        /> False impression: information is actually waiting to be used more widely
        -> See next Section
\end{wstructure}

The $\SYMBspi$ and $\SYMBswitch$ operators deliver dependent
elimination for finite enumerations, but are rather awkward to
use directly. We do not write the range for a \(\LAMBINDER\)-abstraction, so
it is galling to supply \(\V{P}\) for functions defined by $\SYMBswitch$.
Let us therefore find a way to recover the tedious details of the
encoding from types.

%, they also
%come with a notion of finite function space. However, we had to
%extract that intuition from the type, by a careful reading. This seems
%to contradict our argument in favour of types for coding
%representation. Here, we are overflown by low-level, very precise type
%information.

%However, our situation is significantly different from the one we
%faced with encoded data: while we were suffering from a crucial lack
%of information, we are now facing too much information, hence losing
%focus. This is a general issue with the usage of codes, as they convey
%much more information than what the developer is willing to see. 

%As we will see in the following section, there exists a cure to this
%problem. In a nutshell, instead of being overflown by typing
%information, we will put it at work, automatically. The consequence is
%that, in such system, working with codes is \emph{practical}: one
%should not be worried by information overload, but how to use it as
%much as possible. Therefore, we should not be afraid of using codes for
%practical purposes.


\subsection{Type propagation}
\label{sec:type-propagation}

\begin{wstructure}
<- Bidirectional type-checking [ref. Turner,Pierce]
    -> Separating type-checking from type synthesis
    <- Type checking: push terms into types
        <- Example: |Pi S T :>: \ x . t| allows us to drop annotation on lambda
    <- Type synthesis: pull types out of terms
        <- Example: |x : S l- x :<: S| gives us the type of x
\end{wstructure}

Our approach to tidying the coding cruft is deeply rooted in the
bidirectional presentation of type checking from Pierce and
Turner~\cite{pierce:bidirectional-tc}. They divide
type inference into two communicating components. In
\emph{type synthesis}, types are \emph{pulled} out of terms. A
typical example is a variable in the context:
%
\[
\Rule{\G ; \xS ; \Delta \vdash \Valid}
     {\G ; \xS ; \Delta \vdash \Bhab{\V{x}}{\M{S}}}
\]
%
Because the context stores the type of the variable, we can extract the
type whenever the variable is used.

On the other hand, in the \emph{type checking} phase, types are
\emph{pushed} into terms. We are handed a type together with a term,
our task consists of checking that the type admits the term. In doing
so, we can and should use the information
provided by the type. Therefore, we can relax our requirements on the
term. Consider \(\LAMBINDER\)-abstraction:
%
\[
\Rule{\G       \vdash \Type{\M{S}} \quad
      \G ; \xS \vdash \Bhab{\M{t}}{\M{T}}}
     {\G \vdash \Bhab{\PLAM{\x}{\M{S}} \M{t}}{\PIS{\xS} \M{T}}}
\]
%
The official rules require an annotation specifying the domain.
However, in type \emph{checking}, the \(\Pi\)-type we push in determines
the domain, so we can drop the annotation.

\begin{wstructure}
<- Formalisation: type propagation
    <- Motivation
        -> High-level syntax
            -> exprIn: types are pushed in
                <- Subject to type *checking*
            -> exprEx: types are pulled from
                <- Subject to type *synthesis*
        -> Translated into our low-level type theory
        -> Presented as judgements
    -> Presentation mirrors typing rule of [figure] 
        -> Ignore identical judgements
\end{wstructure}

We adapt this idea, yielding a \emph{type
propagation} system, whose purpose is to elaborate compact
\emph{expressions} into the terms of our underlying type theory, much
as in the definition of Epigram
1~\cite{mcbride.mckinna:view-from-the-left}.  We divide expressions
into two syntactic categories: $\exprIn$ into which types are pushed,
and $\exprEx$ from which types are extracted. In the
bidirectional spirit, the $\exprIn$ are subject to type
\emph{checking}, while the $\exprEx$---variables and elimination
forms---admit type \emph{synthesis}. We embed $\exprEx$ into
$\exprIn$, demanding that the synthesised type coincides with the type
proposed. The other direction---only necessary to apply
abstractions or project from pairs---takes a type annotation.

% As the presentation largely
%mirrors the inference rules of the type theory, we will ignore the
%judgments that are identical. We refer our reader to the associated
%technical report~\cite{chapman:desc-tech-report} for the complete
%system of rules.

\begin{wstructure}
<- Type synthesis [figure]
    <- Pull a type out of an exprEx
    <- Result in a full term, together with its type
    -> Do *not* need to specify types
        -> Extracting a term from the context
        -> Function application
        -> Projections
\end{wstructure}

Type synthesis (Fig.~\ref{fig:type-synthesis}) is the \emph{source} of
types. It follows the \(\exprEx\) syntax, delivering both the
elaborated term and its type. Terms and expressions never mix: e.g.,
for application, we instantiate the range with the \emph{term}
delivered by checking the argument \emph{expression}. Hardwired
operators are checked as variables.

\begin{figure}
\[\stkc{
%% Form
\boxed{\Gamma \Vdash \propag{\exprEx}{\pull{\CN{term}}{\CN{type}}}}
\\
\\
%% Reversal
\Rule{\Gamma \Vdash \propag{\push{T}{\Set}}
                           {T'} \quad
      \Gamma \Vdash \propag{\push{t}{T'}}
                           {t'}}
     {\Gamma \Vdash \propag{(\Bhab{t}{T})}
                           {\pull{t'}{T'}}} 
\\
%% Context
\Rule{\Gamma ; \xS ; \Delta \vdash \Valid}
      {\Gamma ; \xS ; \Delta \Vdash \propag{\x}
                                           {\pull{\x}{\M{S}}}}
\qquad
%% Application
\Rule{\stkl{\Gamma \Vdash \propag{\M{f}}
                                 {\pull{\M{f}\M{'}}{\PIS{\xS}{\M{T}}}} \\
            \Gamma \Vdash \propag{\push{\M{s}}{\M{S}}}
                                 {\M{s'}}}}
     {\Gamma \Vdash \propag{\M{f}\: \M{s}}{\pull{\M{f'}\: \M{s'}}{\M{T} [\M{s'}/\x]}}} 
\\
%% First projection
\Rule{\Gamma \Vdash \propag{\M{p}}
                           {\pull{\M{p'}}{\SIGMAS{\xS}{\M{T}}}}}
     {\Gamma \Vdash \propag{\fst{\M{p}}}
                           {\pull{\fst{\M{p'}}}{\M{S}}}} \qquad 
%% Second projection
\Rule{\Gamma \Vdash \propag{\M{p}}
                           {\pull{\M{p'}}{\SIGMAS{\xS}{\M{T}}}}}
     {\Gamma \Vdash \propag{\snd{\M{p}}}
                           {\pull{\snd{\M{p'}}}{\M{T} [\fst{\M{p'}}/\x]}}}
}\]

\caption{Type synthesis}
\label{fig:type-synthesis}
\end{figure}


\begin{wstructure}
<- Type checking [figure]
    <- Push a type in an exprIn
    <- Result in a full term
    -> *Use* the type to build the term!
        -> Domain and co-domain propagation for Pi and Sigma
        -> Translation of 'tags into EnumTs
        -> Translation of ['tags ...] into EnumUs
        -> Finite function space into switch
\end{wstructure}

Dually, type checking judgments (Fig.~\ref{fig:type-checking})
are \emph{sinks} for types. From an
$\exprIn$ and a type pushed into it, they elaborate a low-level
term, extracting information from the type. Note that we inductively ensure the following `sanity checks':
%
\[\stkc{
\Gamma\Vdash\propag{e}{\pull{t}{T}} \Rightarrow
  \Gamma\vdash t:T \\
\Gamma\Vdash\push{\propag{e}{t}}{T} \Rightarrow
  \Gamma\vdash t:T
}\]

Canonical set-formers are \emph{checked}: we could exploit
\(\Set:\Set\) to give them synthesis rules, but this would prejudice
our future stratification plans. Note that abstraction and pairing are
free of annotation, as promised. Most of the propagation rules are
unremarkably structural: we have omitted some mundane rules which just
follow the pattern, e.g., for \(\UId\).

\begin{figure}
\[\stkc{
%% Form
\boxed{\Gamma \Vdash \propag{\push{\CN{exprIn}}{\CN{type}}}{\CN{term}}} 
\\
\\
%% Set in Set
%% \Axiom{\Gamma \Vdash \propag{\push{\Set}{\Set}}
%%                             {\Set}}
%% \\
%% Prop
%% \Rule{\Gamma \Vdash \propag{\push{q}{\Prop}}
%%                            {q'}}
%%      {\Gamma \Vdash \propag{\push{\prf{q}}{\Set}}
%%                            {\prf{q'}}}
%% \qquad
%% True
%% \Axiom{\Gamma \Vdash \propag{\push{\True}{\Prop}}
%%                             {\True}}
%% \\
%% Pi
%% \Rule{\Gamma \Vdash \propag{\push{S}{\Set}}
%%                            {S'} \quad
%%       \Gamma \Vdash \propag{\push{T}{S' \To \Set}}
%%                            {T'}}
%%      {\Gamma \Vdash \propag{\push{\PiTy{S}{T}}{\Set}}
%%                            {\PiTy{S'}{T'}}} 
%% \\
%% %% Sigma
%% \Rule{\Gamma \Vdash \propag{\push{S}{\Set}}
%%                            {S'} \quad
%%       \Gamma \Vdash \propag{\push{T}{S' \To \Set}}
%%                            {T'}}
%%      {\Gamma \Vdash \propag{\push{\SigmaTy{S}{T}}{\Set}}
%%                            {\SigmaTy{S'}{T'}}}
%% \\
%% Lambda
\Rule{\Gamma ; \xS \Vdash \propag{\push{\V{t}}{\V{T}}}
                                 {\V{t'}}}
     {\Gamma \Vdash \propag{\push{\LAM{\x} \V{t}}{\PIS{\xS}{\V{T}}}}
                           {\PLAM{\x}{\V{S}} \V{t'}}} 
\\
%% Pair
\Rule{\stkl{ \Gamma \Vdash \propag{\push{\V{s}}{\V{S}}}
                                  {\V{s'}} \\
             \Gamma \Vdash \propag{\push{\V{t}}{\V{T}[\V{s'}/\V{x}]}}
                                  {\V{t'}}}}
     {\Gamma \Vdash \propag{\push{\pair{\V{s}}{\V{t}}{}}{\SIGMAS{\xS}{\V{T}}}}
                           {\pair{\V{s'}}{\V{t'}}{\V{T}}}}
\\
%% EnumU
%% \Axiom{\Gamma \Vdash \propag{\push{\EnumU}{\Set}}
%%                             {\EnumU}} 
%% \qquad
%% %% EnumT
%% \Rule{\Gamma \Vdash \propag{\push{e}{\EnumU}}
%%                            {e'}}
%%      {\Gamma \Vdash \propag{\push{\EnumT{e}}{\Set}}
%%                            {\EnumT{e'}}}
%% \\
%% Tag
\Axiom{\Gamma \Vdash \propag{\push{\etag{t_0}}{\EnumT{(\ConsE{\etag{t_1}}{\V{e}})}}}
                            {\Ze}}\;\etag{t_0} = \etag{t_1}
\\
\Rule{\Gamma \Vdash \propag{\push{\etag{t_0}}{\EnumT{\V{e}}}}
                           {\V{n}}}
     {\Gamma \Vdash \propag{\push{\etag{t_0}}{\EnumT{(\ConsE{\etag{t_1}}{\V{e}})}}}
                            {\Su{\V{n}}}}\;\etag{t_0} \neq \etag{t_1}
\\
%% EnumU
\Axiom{\Gamma \Vdash \propag{\push{[]}{\EnumU}}
                            {\NilE}}
\\
\Rule{\Gamma \Vdash \propag{\push{\V{ts}}{\EnumU}}
                                 {\V{cs}}}
     {\Gamma \Vdash \propag{\push{[ \etag{t_1},\: \V{ts} ]}{\EnumU}}
                                 {\ConsE{\etag{t_1}}{\V{cs}}}}
\\
%% Switch
\Rule{\Gamma \Vdash \propag{\push{\V{t}}{\spi{\V{e}}{\V{P}}}}
                           {\V{t'}}}
     {\Gamma \Vdash \begin{array}{@{}l} 
                        \propag{\push{\V{t}}{\PI{\V{x}}{\EnumT{\V{e}}} \V{P}\:\V{x}}}
                               {\\ \PLAM{\V{x}}{(\EnumT{\V{e}})} \switch{\V{e}}{\V{P}}{\V{t'}}{\V{x}}}
                    \end{array}}\;\mbox{$\V{t}$ is $[]$ or $[\_,\_]$}
%% \\
%% Conversion
%% \Rule{\Gamma \Vdash \propag{s}
%%                            {\pull{s'}{S}} \quad 
%%       \Gamma \Vdash \push{S \equiv T}{\Set}}
%%      {\Gamma \Vdash \propag{\push{s}{T}}
%%                            {s'}}
}\]

\caption{Type checking}
\label{fig:type-checking}
\end{figure}

However, we also add abbreviations. We write \(\spl{\M{f}}\),
pronounced `uncurry \(\M{f}\)' for the function which takes a pair and
feeds it to \(\M{f}\) one component at a time, letting us name
them individually. Now, for the finite enumerations, we go to work.

Firstly, we present the codes for enumerations as right-nested tuples
which, by our LISP convention, we write as unpunctuated lists of tags
\(\sqr{\etag{t_0}\ldots\etag{t_n}}\).
Secondly, we can denote an element \emph{by its
name}: the type pushed in allows us to recover the numerical
index. We retain the numerical forms to facilitate
\emph{generic} operations and ensure that shadowing is punished
fittingly, not fatally.
Finally, we express functions from enumerations as tuples.
Any tuple-form, \(\void\) or \(\pair{\_}{\_}{}\), is accepted by the
function space---the generalised product---if it is accepted by the
small product. Propagation fills in the appeal to $\SYMBswitch$,
copying the range information.

Our interactive development tools also perform the
reverse transformation for intelligible output.
The encoding of any specific enumeration is thus hidden by these
translations. Only, and rightly, in enumeration-generic programs is the
encoding exposed.

\begin{wstructure}
<- Summary
    -> Not a novel technique [refs?]
        /> Used as a boilerplate scrapper
    -> Make dealing with codes *practical*
        <- Example: Finite sets/finite function space
        -> We should not restrain our self in using codes
            <- We know how to present them to the user
-> Will extend this machinery in further sections
\end{wstructure}

%In this section, we have developed a type propagation system based on
%bidirectional type-checking. Using bidirectional type-checking as a
%boilerplate scrapper is a well-known
%technique~\cite{pierce:bidirectional-tc,
%  xi:bidirectional-tc-bound-array, chlipala:strict-bidirectional-tc}
%\note{Everybody ok with the citations?}. In our case, we have shown
%how to instrument bidirectionality to rationalise the expressivity of
%dependent types. We have illustrated our approach with finite sets. We
%have abstracted away the low-level presentation of finite sets,
%offering a convenient syntax instead.

%This example teaches us that we should not be afraid of codes, as soon
%as type information is available. We have shown how to rationalise
%this information in a formal presentation. Hence, we have shown that
%programming with such objects is practical. As we introduce more codes
%in our theory, we will show how to extend the framework we have
%developed so far.

Our type propagation mechanism does no constraint
solving, just copying, so it is just the thin end of the elaboration wedge.
It can afford us this `assembly
language' level of civilisation as \(\EnumU\) universe
specifies not only the \emph{representation} of the
low-level values in each set as bounded numbers, but also the
\emph{presentation} of these values as high-level tags. To encode only
the former, we should merely need the \emph{size} of enumerations, but
we extract more work from these types by making them more informative.
We have also, \emph{en passant}, distinguished enumerations which have
the same cardinality but describe distinct notions:
\(\EnumT{\sqr{\etag{\CN{red}}\,\etag{\CN{blue}}}}\) is not
\(\EnumT{\sqr{\etag{\CN{green}}\,\etag{\CN{orange}}}}\).


%%%%%%%%%%%%%%%%%%%%%%%%%%%%%%%%%%%%%%%%%%%%%%%%%%%%%%%%%%%%%%%%
%% A Universe of simple data-types
%%%%%%%%%%%%%%%%%%%%%%%%%%%%%%%%%%%%%%%%%%%%%%%%%%%%%%%%%%%%%%%%

\section{A Universe of Inductive Datatypes}
\label{sec:universe-desc}

\begin{wstructure}
<- Why starting with simple datatypes
    <- For pedagogical purposes
        <- datatypes as we know them every day
        /> Target dependent types
    -> Cut down version of Induction Recursion
        -> Presentation evolves independently as we go to dependent types
\end{wstructure}

In this section, we describe an implementation of inductive types, as
we know them in ML-like languages. By working with familiar datatypes,
we hope to focus on the delivery mechanism, warming up gently to the
indexed datatypes we really want.  % Our encoding is inspired by
Dybjer and Setzer's closed formulation of
induction-recursion~\cite{dybjer:axiom-ir}, but without the
`-recursion'.  An impredicative Church-style encoding of datatypes is
not adequate for dependently typed programming, as although such
encodings present data as non-dependent eliminators, they do not
support dependent \emph{induction}~\cite{geuvers:induction-not-derivable}.
Whilst the \(\LAMBINDER\)-calculus captures all that data can \emph{do},
it cannot ultimately delimit all that data can \emph{be}.

\subsection{The power of $\Sigma$}

\begin{wstructure}
<- The duality of Sigma
    <- Sigma generalises sum over arbitrary arities
        -> \Sigma A B == \Sigma_{x : A} B x
    <- Sigma generalises product to have a dependant second component
        -> \Sigma A B == (x : A) \times (B x)
\end{wstructure}

In dependently typed languages, $\Sigma$-types can be interpreted as
two different generalisations. This duality is reflected in the
notation we can find in the literature. The notation
$\blue{\Sigma}_{\x : \M{A}} (\M{B}\: \x)$ stresses that
$\Sigma$-types are `dependent sums', generalising of sums over
arbitrary arities, where simply typed languages have finite sums.

On the other hand, our choice of notation $\SIGMA{\x}{\M{A}}
(\M{B}\:\x)$ emphasises that $\Sigma$-types generalise products,
with the type of the second component depending on the value of the
first, where simply typed languages do not express such relative validity.

\begin{wstructure}
<- datatypes in the simply-typed world
    -> "sums-of-product"
        <- Sum of constructors
        <- Product of arguments
<- datatypes in the dependently-typed world
    -> "sigmas-of-sigmas"
    /> Need ability to manipulate these sigmas
        -> Define a Code for datatypes
        -> Together with a sigma-based Interpretation
\end{wstructure}

In ML-like languages, datatypes are presented as a
\emph{sum-of-products}. A datatype is defined by a finite sum of
constructors, each carrying a product of arguments. To embrace
these datatypes, we have to capture this grammar.
With dependent types, the notion of sum-of-products translates into
\emph{sigmas-of-sigmas}.


\subsection{The universe of descriptions}
\label{sec:desc-universe}

\begin{wstructure}
<- Introduction to Universe construction
    <- Define a Code
        -> Name objects
    <- Define an Interpretation of codes into the type theory
        -> Give a semantics to objects
    -> Ability to manipulate code
    -> Ability to compute with these objects
\end{wstructure}

While sigmas-of-sigmas can give a \emph{semantics} for the
sum-of-products structure in each node of the tree-like values in a
datatype, we need to account somehow for the recursive structure which
ties these nodes together. Not for the first time, we do this by
constructing a \emph{universe}~\cite{martin-lof:itt}. Universes
are ubiquitous in dependently typed
programming~\cite{benke:universe-generic-prog, oury:power-of-pi},
but here we seek to exploit them as the foundation of our notion
of datatypes.

%The key idea behind universe construction is our ability to make names
%by defining new types. These names are called \emph{codes}. By
%defining a set of codes, we define the syntax of a language. To give
%it a semantics, we \emph{interpret} these codes back into the type
%theory. Hence, codes act as labels, while the type theory provides the
%computational machinery. Codes being mere labels, we can inspect them,
%hence regaining structural information. We can also compute over them:
%deriving new codes from others, or even functions on them.

\begin{wstructure}
<- Justification of the code 
    <- [both figures]: cannot be read separately
    <- Mimic the standard grammar of datatypes description
        -> Just as we already know it
        <- '\Sigma for making sigmas-of-sigmas
        <- 'indx for exhibiting the functoriality
            -> For recursive arguments
        <- '1 for end of description
\end{wstructure}

To add inductive types to our type theory, we build a universe of
datatype \emph{descriptions} by implementing the signature presented
in Figure~\ref{fig:desc_universe}, with codes mimicking the grammar of
datatype declarations. We can read a description $\V{D}:\Desc$ as a
`pattern functor' on $\Set$, with $\SYMBdescop{\V{D}}$ its action on
an object, \(\V{X}\), soon to be instantiated recursively.

Descriptions are sequential structures, terminated by $\DUnit$, indicating
the empty tuple. To build
sigmas-of-sigmas, we define a $\SYMBDSigma$ code, interpreted as a
$\Sigma$-type. To request a recursive component, we have $\DIndx{\V{D}}$,
where \(\V{D}\) describes the rest of the node.

\begin{figure}
\[\stk{
\begin{array}{l}
\Desc : \Set\\
\DUnit : \Desc\\
\DSigma{(\Bhab{\V{S}}{\Set})}{(\Bhab{\V{D}}{\V{S} \To \Desc})} : \Desc\\
\DIndx{(\Bhab{\V{D}}{\Desc})} : \Desc \\
\end{array}
\smallskip\\
\begin{array}{l@{\V{X}\:\mapsto\:}l}     
\multicolumn{2}{l}{\descop{\_\:}{} : \Desc \To \Set \To \Set} \\
 \descop{\DUnit}{} &  \Unit \\
 \descop{\DSigma{\V{S}}{\V{D}}}{} &
     \SIGMAS{\Bhab{\V{s}}{\V{S}}}{\descop{\V{D}\,\V{s}}{\!\V{X}}}  \\
\descop{\DIndx{\V{D}}}{}  &  \TIMES{\V{X}}{\descop{\V{D}}{\V{X}}}
\end{array}
}\]

\caption{Universe of Descriptions}
\label{fig:desc_universe}
 
\end{figure}

You may have noticed that we are a little coy about this presentation,
writing of `implementing a signature' without clarifying how. A viable
approach would simply be to extend the theory with constants for the
constructors and an operator for \(\descop{\V{D}}\). However, in
Section~\ref{sec:desc-levitate}, you will see what we actually do.  In
the meantime, let us first gain some intuition for its use by
developing some examples.

\subsection{Examples}
\label{sec:desc-examples}

\begin{wstructure}
<- Nat
    <- Sum of zero, suc
    <- zero case: done
    <- suc case: leave open and done
    -> NatD Z = 1 + Z
\end{wstructure}

We begin with the natural numbers, now working in the high-level
expression language of Section~\ref{sec:type-propagation}, exploiting
type propagation.
%
\[\stk{
\NatD : \Desc \\
\NatD \mapsto \DSigma{\EnumT{\sqr{\NatZero\: \SYMBNatSuc }}}
                     {\sqr{ \DUnit \quad (\DIndx{\DUnit}) }}
}\]
%
Let us explain its construction. First, we use $\SYMBDSigma$ to
give a choice between the $\NatZero$ and $\SYMBNatSuc$ constructors.
What follows depends on this choice, so we write the function
computing the rest of the description in tuple notation.  In the
$\NatZero$ case, we reach the end of the description. In the
$\SYMBNatSuc$ case, we attach one recursive argument and close the
description. Translating the \(\Sigma\) to a binary sum, we have
effectively described the functor:
%
\[    \NatD\: \V{Z} \mapsto \Unit \mathop{\D{+}} \V{Z}    \]
Correspondingly, we can see the injections to the sum:
\[
\sqr{\NatZero}:\descop{\NatD}Z \qquad
\sqr{\NatSuc{(\Bhab{z}{Z})}}:\descop{\NatD}Z
\]

\begin{wstructure}
<- List
    <- Sum of nil, cons
    <- nil case: done
    <- cons case: product of X with leave open and done
    -> ListD X Z = 1 + X * Z
\end{wstructure}

With a small change to this definition, we obtain the pattern functor
for lists:
%
\[\stk{
\ListD : \Set \To \Desc \\
\ListD \: \V{X} \mapsto
 \DSigma{\EnumT{\sqr{ \ListNil\: \SYMBListCons }}}
         {\sqr{ \DUnit \quad (\DSigma{\V{X}}{\LAM{\_} \DIndx{\DUnit}}) }}
}\]
%
The $\SYMBNatSuc$ constructor is turned into a proper
$\SYMBListCons$, taking an argument in $\V{X}$ followed by a
recursive argument. This code describes the following functor:
%
\[    \ListD\: \V{X}\: \V{Z} \mapsto \Unit \mathop{\D{+}} \V{X} \D{\ensuremath{\times}} \V{Z}     \]

\begin{wstructure}
<- Tree
    <- sum of leaf, node
    <- leaf case: done
    <- node case: product of X with two leave open and done
    -> TreeD X Z = 1 + X * Z * Z
\end{wstructure}

Finally, we are not limited to one recursive argument. This is
demonstrated by our description of node-labelled binary trees:
%
\[\stk{
\TreeD : \Set \To \Desc \\
\begin{array}{@{}ll}
\TreeD \: \V{X} \mapsto \DSigma{ & \EnumT{\sqr{ \TreeLeaf\: \TreeNode }} \\}
  { & \sqr{ \DUnit \quad
       ( \DIndx{(\DSigma{\V{X}}{\LAM{\_}\DIndx{\DUnit}})})} }
\end{array}
}\]
%
Again, we are one evolutionary step away from $\ListD$. However,
instead of a single call to the induction code, we add another. The
interpretation of this code corresponds to the following functor:
%
\[    \TreeD\: \V{X}\: \V{Z} \mapsto \Unit \mathop{\D{+}} 
          \V{Z} \Prod \V{X}  \Prod \V{Z}     \]


\begin{wstructure}
<- Tagged description
    <- Form TDesc = List (UId x Desc) [equation]
    <- Follow usual sums-of-product presentation of datatype
        <- Finite set of constructors
        <- Then whatever you want
    -> Any Desc datatype can be turned into this form
        -> No loss of expressive power
        /> Guarantee a ``constructor form''
\end{wstructure}

From the examples above, we observe that datatypes are defined by a
$\SYMBDSigma$ whose first argument enumerates the constructors. We
call codes fitting this pattern \emph{tagged} descriptions. Again,
this is a clear reminder of the sum-of-products style. Every
description can be forced into this style with a singleton constructor
if necessary. We characterise tagged descriptions thus:
\[\stk{
 \TagDesc : \Set \\
 \TagDesc \mapsto \SIGMA{\V{E}}{\EnumU} (\spi{\V{E}}{(\LAM{\_} \Desc)})
\smallskip\\
\SYMBtoDesc : \TagDesc\To\Desc \\
\SYMBtoDesc \mapsto
\spl{\LAM{\V{E}}\LAM{\V{D}}
\DSigma{\EnumT{\V{E}}}{(\F{switch}\:\V{E}\:(\LAM{\_}\Desc)\:\V{D})}}
}\]
It is not a great stretch to imagine that the traditional datatype
declaration syntax might desugar to the \emph{definition} of a datatype
via a tagged description.

\begin{wstructure}
<- Fictive object [figure 'data Desc']
    -> Must be read as a type signature
    -> See further for its actual implementation
        <- Subject to our levitation exercise
\end{wstructure}

\subsection{The least fixpoint}
\label{sec:desc-fix-point}

\begin{wstructure}
<- Build the fixpoint of functors
    <- See examples: need to build their initial algebra
    -> Extend the type theory with Mu/Con [figure]
        <- Straightforward definition of a fixpoint
            <- Interpret D with (Mu D) as sub-objects
\end{wstructure}


So far, we have built pattern functors with our \(\Desc\) universe.
Being polynomial functors, they all admit a least fixpoint, which we
now construct by \emph{tying the knot}: the element type abstracted
by the functor is now instantiated recursively:
%
\[
\Rule{\Gamma \vdash \Bhab{\M{D}}{\Desc}}
     {\Gamma \vdash \Bhab{\Mu{\M{D}}}{\Set}} \qquad
\Rule{\Gamma \vdash \Bhab{\M{D}}{\Desc} \quad 
      \Gamma \vdash \Bhab{\M{d}}{\descop{\M{D}}{(\Mu{\M{D}})}}}
     {\Gamma \vdash \Bhab{\Con{\M{d}}}{\Mu{\M{D}}}}
\]

\begin{wstructure}
<- Elimination on Mu
    <- We are used to foldD : \forall X. (desc D X -> X) -> mu D -> X
        /> Not dependent
        -> Cannot express some (which one again?) properties
    -> Develop a dependent induction
        <- Everywhere/All
        <- Induction
    -> *Generic*
    ???
\end{wstructure}

We can now build datatypes and their elements, e.g.:
\[\stk{
\Nat\mapsto\Mu{(\toDesc{\pair{\sqr{\NatZero\: \SYMBNatSuc }}
                       {\sqr{ \DUnit \quad (\DIndx{\DUnit})}}{}})}:\Set
\\
\Con{\sqr{\NatZero}}:\Nat \qquad
\Con{\sqr{\NatSuc{(\Bhab{n}{\Nat})}}}:\Nat
}\]

But how shall we compute with our data?  We should expect an
elimination principle. Following a categorical intuition, we might
provide the `fold', or `iterator', or `catamorphism':
%
\[
\F{cata} : \PITEL{\V{D}}{\Desc}
           \PI{\V{T}}{\Set}
           (\descop{\V{D}}{\V{T}} \To \V{T}) \To 
           \Mu{\V{D}} \To \V{T} 
\]

However, iteration is inadequate for \emph{dependent} computation.
We need \emph{induction} to write functions whose type depends on inductive
data. Following \citet{benke:universe-generic-prog}, we adopt the following:
%
\[\stk{
\F{ind} : \stk{ \PITEL{\V{D}}{\Desc}
                    \PI{\V{P}}{\Mu{\V{D}} \To \Set}         \\
               (\PI{\V{d}}{\descop{\V{D}}{(\Mu{\V{D}})}}       
                \All{\V{D}}{(\Mu{\V{D}})}{\V{P}}{\V{d}} \To \V{P} (\Con{\V{d}})) \To \\
               \PI{\V{x}}{\Mu{\V{D}}} \V{P} \V{x} 
} \\
\F{ind}\, \V{D}\, \V{P}\, \V{m}\, (\Con{\V{d}}) = 
    \V{m}\, \V{d}\, (\all{\V{D}}{(\Mu{\V{D}})}{\V{P}}
                           {(\F{ind}\, \V{D}\, \V{P}\, \V{m})}
                           {\V{d}})
}\]
%
Here, $\All{\V{D}}{\V{X}}{\V{P}}{\V{d}}$ states that
$\Bhab{\V{P}}{\V{X} \To \Set}$ holds for every subobject $\V{x}:\V{X}$ in \(\V{D}\), and \(\all{\V{D}}{\V{X}}{\V{P}}{\V{p}}{\V{d}}\) is a
`dependent map', applying
some \(\Bhab{\V{p}}{\PIS{\Bhab{\x}{\V{X}}}\V{P}\,\V{x}}\) to each \(\x\)
contained in \(\V{d}\). The full
definition (including an extra case, introduced shortly) is presented in
Figure~\ref{fig:all-predicates}. Note that
\(\F{ind}\) is our first generic
operation over descriptions, albeit a hardwired operator.
Any datatype we define automatically
comes with an induction principle.

We note that the very same functors \(\SYMBdescop{\M{D}}\) also admit greatest
fixpoints, and we have indeed implemented coinductive types this way, but
that is a story for another time.

\begin{figure*}

\[
\begin{array}{ll}
%%
\stk{
\begin{array}{@{}ll}
\SYMBAll : & \PITEL{\V{D}}{\Desc}
             \PITEL{\V{X}}{\Set}
             \PITEL{\V{P}}{\V{X} \To \Set} \\
           & \PI{\V{xs}}{\descop{\V{D}}{\V{X}}} 
             \Set 
\end{array} \\
\begin{array}{@{}l@{}l@{\:=\:\:}l}
\All{\DUnit}{& \V{X}}{\V{P}}{\Void} &
    \Unit \\
\All{(\DSigma{\V{S}}{\V{D}})}{& \V{X}}{\V{P}}{\pair{\V{s}}{\V{d}}{}} &
    \All{(\V{D}\: \V{s})}{\V{X}}{\V{P}}{\V{d}} \\
\All{(\DIndx{\V{D}})}{& \V{X}}{\V{P}}{\pair{\V{x}}{\V{d}}{}} &
    \TIMES{\V{P}\: \V{x}}{\All{\V{D}}{\V{X}}{\V{P}}{\V{d}}} \\
\All{(\DHindx{\V{H}}{\V{D}})}{& \V{X}}{\V{P}}{\pair{\V{f}}{\V{d}}{}} &
    \TIMES{(\PI{\V{h}}{\V{H}} \V{P}\: (\V{f}\: \V{h}))}
          {\All{\V{D}}{\V{X}}{\V{P}}{\V{d}}}
\end{array}
}
&
%%
\stk{
\begin{array}{@{}ll}
\SYMBall : & \PITEL{\V{D}}{\Desc}
             \PITEL{\V{X}}{\Set}
             \PITEL{\V{P}}{\V{X} \To \Set} \\
           & \PITEL{\V{p}}{\PI{\V{x}}{\V{X}} \V{P}\: \V{x}}
             \PI{\V{xs}}{\descop{\V{D}}{\V{X}}} 
             \All{\V{D}}{\V{X}}{\V{P}}{\V{xs}} 
\end{array} \\
\begin{array}{@{}l@{}l@{\:=\:\:}l}
\all{\DUnit}{& \V{X}}{\V{P}}{\V{p}}{\Void} &
    \Void \\
\all{(\DSigma{\V{S}}{\V{D}})}{& \V{X}}{\V{P}}{\V{p}}{\pair{\V{s}}{\V{d}}{}} &
    \all{(\V{D}\: \V{s})}{\V{X}}{\V{P}}{\V{p}}{\V{d}} \\
\all{(\DIndx{\V{D}})}{& \V{X}}{\V{P}}{\V{p}}{\pair{\V{x}}{\V{d}}{}} &
    \pair{\V{p}\: \V{x}}
         {\all{\V{D}}{\V{X}}{\V{P}}{\V{p}}{\V{d}}}{} \\
\all{(\DHindx{\V{H}}{\V{D}})}{& \V{X}}{\V{P}}{\V{p}}{\pair{\V{f}}{\V{d}}{}} &
    \pair{\LAM{\V{h}} \V{p}\: (\V{f}\: \V{h})}
         {\all{\V{D}}{\V{X}}{\V{P}}{\V{p}}{\V{d}}}{}
\end{array}
\end{array}
}
\]

\caption{Defining and collecting inductive hypotheses}
\label{fig:all-predicates}

\end{figure*}


\subsection{Extending type propagation}

\begin{wstructure}
<- Extending type propagation
    <- datatype declaration turns into definitions
        -> Straightforward translation to Desc
        -> Creation of a variable referring to the structure
    <- Labelled Mu
        /> Just mention the possibility of labelling, no details required
        -> For the user, objects have names rather than Mu of codes
    <- Push Mu to an applied name [figure]
        -> Direct integration into the type propagation machinery
    -> Coded presentation is practical
        <- The user never see a code
\end{wstructure}


We have now enough machinery to build and manipulate inductive
types at a low level. Let us now apply cosmetic surgery
to the syntactic overhead. We extend type checking of
expressions:
%
\[
\Rule{\Gamma\Vdash\propag{\push{\etag{c}}{\EnumT{\M{E}}}}{n} \quad
\Gamma\Vdash  
\propag{
  \push{\sqr{\vec{t}}}
   {\descop{\M{D}\: n}{(\Mu{(\DSigma{\EnumT{\M{E}}}{\M{D}})})}}}
        {\M{t'}}}
     {\Gamma\Vdash\propag{\push{\etag{c}\: \vec{\M{t}}}
                   {\Mu{(\DSigma{\EnumT{\M{E}}}{\M{D}})}}}
             {\Con{\pair{n}{\M{t'}}{}}}}
\]
%
Here $\etag{c}\:\vec{t}$ denotes a tag `applied' to a sequence
of arguments, and \(\sqr{\vec{t}}\) that sequence's repackaging as
a right-nested tuple. Now we can just write data directly.
\[
\NatZero:\Nat\qquad \NatSuc{(\Bhab{n}{\Nat})}:\Nat
\]
Once again, the type explains the legible presentation, as well as
the low-level representation.

We may also simplify appeals to induction by type propagation, as we
have done with functions from pairs and enumerations.
\[
\Rule{\Gamma\Vdash\stk{
\propag{\push{\M{f}}
             {\PI{\V{d}}{\descop{\M{D}}{(\Mu{\M{D}})}}       
                \All{\M{D}}{(\Mu{\M{D}})}{(\PLAM{\x}{\Mu{\M{D}}}\M{P})}{\V{d}} \To \M{P}[\Con{\V{d}}/\x]\\}}
       {\M{f'}}}}
{\Gamma\Vdash
\propag{\push{\sind{\M{f}}}
             {\PIS{\Bhab{\x}{\Mu{\M{D}}}}\M{P} }}
       {\F{ind}\:\M{D}\:(\PLAM{\x}{\Mu{\M{D}}}\M{P})\:\M{f'}}}
\]
This abbreviation is no substitute for the dependent pattern
matching to which we are entitled in a high-level language built on top
of this theory~\cite{goguen:pattern-matching}, but it does at least make
`assembly language' programming mercifully brief, if hieroglyphic.
\[\stk{
\F{plus}:\Nat\To\Nat\To\Nat \\
\F{plus}\mapsto\sind{\spl{\sqr{(\LAM{\_}\LAM{\_}\LAM{\V{y}}\V{y})
  \quad(\LAM{\_}\spl{\LAM{\V{h}}\LAM{\_}\LAM{\V{y}}
    \NatSuc{(\V{h}\:\V{y})}})}}}
}\]

This concludes our introduction to the universe of datatype descriptions.
We have encoded sum-of-products datatypes from the simply-typed world
as data and equipped them with computation. We have also made sure
to hide the details by type propagation.


%%%%%%%%%%%%%%%%%%%%%%%%%%%%%%%%%%%%%%%%%%%%%%%%%%%%%%%%%%%%%%%%
%% Levitating the universe of descriptions
%%%%%%%%%%%%%%%%%%%%%%%%%%%%%%%%%%%%%%%%%%%%%%%%%%%%%%%%%%%%%%%%

\section{Levitating the universe of descriptions}
\label{sec:desc-levitate}

In this section, we will fulfil several promises. One promise we made
was to effectively implement the code of the $\Desc$ universe. Another
promise we tacitly was the existence of the type formers for finite
sets. Despite being a perilous pedagogical exercise for the authors,
we hope to convey to the reader the dizzy feeling of levitation,
without the falling.

\subsection{Implementing finite sets}

\begin{wstructure}
<- Recall typing rules of 1st section
    -> Make clear they were just promises
    -> Can be implemented now
        <- Simply List UId
\end{wstructure}

In Section~\ref{sec:finite-sets}, we have given a specification of
finite sets. It consists in the objects presented in
Figure~\ref{fig:typing-finite-set}. We are going to implement the
$\EnumU$ type former and its constructors. Let us recall their
specification:

\[\stkc{
%% EnumU
\Rule{\Gamma \vdash \Valid}
     {\Gamma \vdash \Type{\EnumU}} 
\\
%% NilE
\Rule{\Gamma \vdash \Valid}
     {\Gamma \vdash \Bhab{\NilE}{\EnumU}} 
\qquad
%% ConsE
\Rule{\Gamma \vdash \Bhab{t}{\UId} \quad
      \Gamma \vdash \Bhab{e}{\EnumU}}
     {\Gamma \vdash \Bhab{\ConsE{t}{e}}{\EnumU}}
}\]

The names $\NilE$ and $\ConsE{}{}$ are absolutely not a
coincidence. $\EnumU$ is indeed isomorphic to a list of $\UId$s. In
Section~\ref{sec:desc-examples}, we have shown how to build lists in
the universe of description. Therefore, the actual implementation of
$\EnumU$ boils down to the following, trivial definition:

\[\stk{
\EnumU : \Set \\
\EnumU \mapsto \Mu{(\ListD~\UId)}
}\]


\begin{wstructure}
<- Consequences
    -> Type theory doesn't need to be extended with EnumU, NilE, and ConsE
        <- EnumU == Mu EnumUD
        <- NilE, ConsE are just tags
    -> Do not need a specific \spi eliminator
        <- \spi is an instance of the generic eliminator
            <- Code?
    -> Anything else remains the same (switch, EnumT, 0, 1+)
\end{wstructure}

Let us examine the consequences of our act. First of all, we discover
that the type theory does not actually need to be extended with the
type former $\EnumU$, the constructors $\NilE$ and $\ConsE$. We have
just defined $\EnumU$ above, by fix-point over the signature functor
$\ListD$. $\NilE$ and $\ConsE$ are simply the $\ListNil$ and
$\ListCons$ constructors of lists.

Another interesting consequence concerns the $\spi{}{}$ operator: it
can actually be implemented with the generic $\F{induction}$
principle. From an implementation perspective, this means that we
remove an operator hard-coded in the host language. This operator is
simply implemented in the dependently-typed target language, as any
other function.

Apart from these changes, we are left with implementing the other
components of finite sets. This consists in $\EnumT$, $\Ze$, $\Su$, as
well as the $\F{switch}$ operator. Note that the actual implementation
of $\EnumU$ does not influence our implementation of $\EnumT$: be they
hard-coded or codes in the $\Desc$ universe, the $\EnumU$ objects just
behave similarly. We have witnessed this effect when carrying this
operation in Epigram, moving from an hard-coded presentation to a
self-hosted one. Absolutely no change to the $\EnumT$ objects was
required.

\begin{wstructure}
<- Summary of the operation
    <- The content of the type theory is exactly the same
        <- before == after
    /> type naming scheme condenses
        <- Replace named constructors by codes in the universe of data-types
    -> Our next step is a similar move (in essence)
        /> Condenses the entire naming scheme of data-types
\end{wstructure}

In this section, we have replaced a low-level presentation of finite
sets by a self-hosted one, expressed in the universe of
descriptions. However, formally, the content of the type theory
remains unchanged: objects which were present before the modification
are still there. Conversely, we have not introduced any spurious
object.

If not on the content, this modification had an effect on the
names. The type naming scheme of the type theory has condensed: named
type formers ($\EnumU$) and constructors ($\NilE$ and $\ConsE$) are
now replaced by codes and their fix-point in the universe of
descriptions. In essence, our next step is similar: we are going to
condense the entire naming scheme of data-types \emph{in itself}.

\subsection{Implementing descriptions}

\begin{wstructure}
<- Realising our promises
    <- We are going to implement Desc
    /> Desc is itself a data-type
        <- Same situation as EnumU
            <- We want to benefit from generic operations
        -> It ought to be encoded in itself
\end{wstructure}

We shall now fulfil our promises for $\Desc$. We are going to
implement the codes of the universe of descriptions. Interestingly, in
and by itself, a code is nothing but a data-type. We are in the same
situation than with $\EnumU$: we ought to be able to describe the
codes of $\Desc$ in $\Desc$ itself. Hence, this code would be a
first-class citizen, born with the standard, generic equipment of
data-types.

\subsubsection{First attempt}

\begin{wstructure}
<- A partial implementation
    <- '1 and 'indx are easy
    <- 'sigma is partially doable
        /> lack the ability to do an higher-order inductive call
    -> Show partial code [figure]
\end{wstructure}

Our first tentative is the following:

\[\stk{
\DescD : \Desc \\
\begin{array}{@{}ll}
\DescD \mapsto \DSigma{}{} & (\EnumT [ \DUnit, \DSigma{}{}, \DIndx{} ])  \\
                           & \left[\begin{array}{l}
                                   \DUnit                                \\
                                   \DSigma{\Set}{(\LAM{\V{S}} ???)}      \\
                                   \DIndx{\DUnit}                        \\
                                   \end{array}
                             \right]
\end{array}
}\]

Let us explain how we have proceeded and the obstacle we face. Much as
the data-types we have seen so far, we first have to choose a
constructor. Constructors correspond to the code's constructors:
$\DUnit$, $\DSigma$, and $\DIndx$. The reader will notice that the
tagged notation we have used so far for $\Desc$ constructors now fully
make sense: these were actually the tags we are defining. \note{This
  sentence badly needs a rephrasing.}

The case of $\DUnit$ is trivial: we immediately reach the end of the
data-type. $\DUnit$ has no payload. The case of $\DIndx{}$ comes
naturally: it only takes a recursive argument -- a description. The
case of $\DSigma$ is problematic. Recall the specification of
$\DSigma$:

\[    \DSigma{}{} : \PI{\V{S}}{\Set} \PIS{S \To \Desc} \Desc      \]

So, we first pack an element $S$ of $\Set$. Then, we would like to
express a notion of recursive argument \emph{indexed} by $S$. Because
our presentation is entirely first-order so far, we are not able to
express this notion. We shall remedy to this situation.


\subsubsection{Second attempt}

\begin{wstructure}
<- Extending the universe of description
    -> With higher-order induction
    <- Intuition: index elements in X by H, and go on reading
        -> indx is isomorph to hindx for H = 1
    /> Keep indx
        <- First order!
        -> Extensionally equal to hindx 1
        /> Practically, definitional equality on Sigma/Pi cannot cope with it
    -> Show DescD code
\end{wstructure}

In order to capture a notion of higher-order induction, we need to
extend our universe of descriptions. This consists in adding a code
$\DHindx$ that takes an indexing set $H$. The code and its
interpretation are presented in
Figure~\ref{fig:hindx_desc}. Intuitively, $\DHindx$ use the elements
of $H$ to index as many recursive arguments. 

\begin{figure*}

\[
\begin{array}{ll}
\stk{
\data \Desc : \Set \where \\
\;\;\begin{array}{@{}l@{\::\:\:}l@{\quad}l}
    \ldots          & \:\:\ldots \\
    \DHindx         & \PI{H}{Set} \Desc \To \Desc
\end{array}
}
&
\stk{
\descop{\_\:}{} : \Desc \To \Set \To \Set \\
\begin{array}{@{}l@{\:=\:\:}ll}
\ldots                        &  \ldots \\
\descop{\DHindx{H}{D}}{X}     &  \TIMES{(H \To X)}{\descop{D}{X}}
\end{array}
}
\end{array}
\]

\caption{Higher-order universe of descriptions}
\label{fig:hindx_desc}

\end{figure*}


Note that $\DIndx$ is isomorphic to $\DHindx{\Unit}$. However, we
tolerate this duplication and do not replace $\DIndx$ by the more
general $\DHindx$. Indeed, unlike its counterpart, $\DIndx$ is
first-order. Therefore, while both codes are \emph{extensionally} the
same, in most practical implementation they would be dealt with rather
differently. While definitional equality can cope with first order
objects, the functional presentation introduced by $\DHindx$ is
unlikely to be amenable to a purely definitional treatment.

Equipped with $\DHindx$, we can describe the extended code of our
universe of data-types, as shown in Figure~\ref{fig:desc-levitate}.
The $\DUnit$ and $\DIndx$ cases remains unchanged, as expected. We
successfully describe the $\DSigma$ case, by a simple appeal to the
higher-order induction on $S$. The $\DHindx$ case consists in packing
the $H$ in $\Set$ with a recursive argument.

In a first glance, we have achieved our goal. We have described the
codes of the universe of description. Taking the fix-point of this
object gives us $\Desc$, up to isomorphism. We have implemented our
code of data-types as an object levitating inside itself. However,
this levitation operation, just as any magic trick, relies on an
invisible cable. Let us reveal it, hence finishing our implementation.

\begin{figure}

\[\stk{
\DescD : \Desc \\
\begin{array}{@{}ll}
\DescD \mapsto \DSigma{}{} & (\EnumT [ \DUnit, \DSigma{}{}, \DIndx{}, \DHindx{}{} ]) \\
                           & \left[\begin{array}{l}
                                   \DUnit                                            \\
                                   \DSigma{\Set}{(\LAM{\V{S}} \DHindx{S}{\DUnit})}   \\
                                   \DIndx{\DUnit}                                    \\
                                   \DSigma{\Set}{(\LAM{\V{H}} \DIndx{\DUnit})}
                                   \end{array}
                             \right]
\end{array}
}\]

\caption{Levitating description of $\Desc$}
\label{fig:desc-levitate}

\end{figure}

\subsubsection{Final move}

\begin{wstructure}
<- Subtlety: translation of [ ... ]
    -> Let us do it manually
        -> Code with problem for the motive of switch
\end{wstructure}

Our reader might be slightly confused to learn that the trick is
visible in the definition of $\DescD$
(Fig.\ref{fig:desc-levitate}). Or rather, it is made invisible by
careful usage of type propagation. Indeed, let us try to elaborate
this term down to the low-level type theory. The $\EnumT [ \ldots ]$
construct elaborates to a finite set $e$ in $\EnumU$, inhabited by the
codes. Then, we can type-check the case definition, between square
brackets $[ \ldots ]$. This term $f$ is pushed into the type
$\EnumT{e} \To \Desc$. This corresponds to a finite function
definition, as formalised in Figure~\ref{fig:type-checking}.

What happens if we unfold the definition? We ought to build the following term:

\[
\PLAM{x}{(\EnumT{e})} \switch{e}{(\LAM{\_} \Mu{\DescD})}{\pi^f}{x}
\]

But this is quite problematic. We are still in the process of
constructing $\DescD$, and the motive of $\F{switch}$ is abruptly
begging for this very same $\DescD$. Despite our willingness, we
cannot materialise such motive.

\begin{wstructure}
<- The magician trick
    <- Our problem is to give a motive for switch
        /> We perfectly know what it ought to be: \_ -> DescD
    -> Solution: extend the type theory with a special purpose switchD
        -> Only extension required to the type theory!
        -> Hidden away to the user by the syntactic sugar
            -> Sufficient to ensure unavailability as a raw operator
            <- Another instance of type propagation
\end{wstructure}

If we cannot make it happen, we simply do not make it happen. We
perfectly know what the motive ought to be. Consequently, we extend
the type theory with a special-purpose operator:

\[
\begin{array}{@{}ll}
%% switchD
\F{switchD} : & \PITEL{\V{e}}{\EnumU}               
                \PITEL{\V{b}}{\spi{e}{\LAM{\_} \Desc}}
                \PITEL{\V{x}}{\EnumT{e}} \To \Desc
\end{array}
\]

The entire work of the magician stands here, in this extension. One
could be worried by the availability of such operations in our type
theory. Indeed, it would be inconvenient for the user to have to
decide between two variants. The solution to this problem is, once
again, to use the type propagation system. Just as the $\F{switch}$
operator, $\F{switchD}$ can be automatically elaborated. To this end,
we extend the type-checking rules (Fig.~\ref{fig:type-checking}) with
the following inference rule:

\[
\Rule{\Gamma \Vdash \propag{\push{t}{\spiD{e}}}
                           {t'}}
     {\Gamma \Vdash \begin{array}{@{}l}
                        \propag{\push{t}{\EnumT{e} \To \Desc}}
                               {\\ \PLAM{x}{(\EnumT{e})} \switchD{e}{t'}{x}}
                    \end{array}
     }\;\mbox{t is $[]$ or $[a,b]$}
\]


\begin{wstructure}
<- Generic programming now!
    <- Desc is just data
        -> Can be manipulated
    <- Free induction scheme on Desc
        -> Ability to inspect data-types
        -> Ability to program on data-types
\end{wstructure}


\note{
Maybe a bit of perspective would be good: 
Everything stands up with the Mu.
Induction principle defined over Mu-things.
}

This concludes our levitation work. Beyond its pedagogical value, this
exercise has several practical outcomes. First of all, this exercise
reveals that the $\Desc$ universe is just plain data. Just as any
piece of data, it can therefore be inspected and
manipulated. Moreover, it is expressed in the $\Desc$ universe. As a
consequence, it is equipped, for free, with an induction
principle. So, our ability to inspect and program with $\Desc$ is not
restricted to the meta-language: we now have all the necessary
equipment in the target language to \emph{program} over data-types. In
this setting, \emph{generic programming is just about programming}.


\subsection{The generic catamorphism}

\begin{wstructure}
<- Making cata
    <- Present the type signature
    <- Starts with a call to generic induction
        <- induction on Desc!
        /> Show types at hand
        -> Explain how to use inductive hypothesis
    <- Implement the 'replace' function
    -> Dependent-typeless catamorphism 
\end{wstructure}

In Section~\ref{sec:desc-fix-point}, we have argued for the
implementation of a dependently-typed $\F{induction}$ principle,
instead of the more traditional catamorphism. However, in some
circumstances, the full-power of a dependent elimination is not
necessary. In the following, we propose to derive the catamorphism
from the generic induction principle. 

The catamorphism can be characterised as an induction on the
description $D$, with a non-dependent motive targeting $T$. Given a
node $xs$ and the induction hypotheses, the method ought to build an
element of $T$. Provided that we know how to make an element of
$\descop{D}{T}$, this step will be performed by the algebra $f$. Let
us take a look at this jigsaw:

\newcommand{\cata}{\F{cata}}

\[\stk{
\cata : \PITEL{D}{\Desc}
           \PI{T}{\Set}
           (\descop{D}{T} \To T) \To 
           \Mu{D} \To T \\
\cata\: D\: T\: f \mapsto
  \F{induction}\: D\: (\LAM{\_}T)\: (\LAM{xs\:hs} f\: ???)
}\]

We are left with implementing the \(???\) program. Recall that we have
\(\Bhab{xs}{\descop{D}{\Mu{D}}}\) and
\(\Bhab{hs}{\All{D}{(\Mu{D})}{(\LAM{\_} T)}{xs}}\) at hand. Our goal
is to make an element of \(\descop{D}{T}\). Intuitively, $xs$ is of
the right shape, but its sub-elements are of the wrong type. On the
other hand, for each sub-element of $xs$, $hs$ gives us the
corresponding element in $T$. Hence the name ``induction
hypotheses''. Therefore, to construct an element of \(\descop{D}{T}\),
we replace the recursive components of \(xs\) by their counterpart in
\(hs\):

\[\stk{
\F{replace} : \stk{\PITEL{D}{\Desc}
                   \PITEL{X,Y}{\Set}\\
                   \PI{xs}{\descop{D}{X}} 
                   \All{D}{X}{(\LAM{\_}Y)}{xs} \To
                   \descop{D}{Y}} \\
\F{replace}\: \DUnit\:          X\: Y\: \Void\:          \Void          \mapsto \Void \\
\F{replace}\: (\DSigma{S}{D})\: X\: Y\: \pair{s}{xs}{}\: ys             \mapsto
    \pair{s}{\F{replace}\: {D~s}\: X\: Y\: xs\: ys}{}                                 \\
\F{replace}\: (\DIndx{D})\:     X\: Y\: \pair{x}{xs}{}\: \pair{y}{ys}{} \mapsto
    \pair{y}{\F{replace}\: D\: X\: Y\: xs\: ys}{}
}\]

Filling the \(???\) with \(\F{replace}\: D\: (\Mu{D})\: T\: xs\: hs\) closes the
problem. In the type theory, we have built a generic catamorphism. Any
data-type will now come equipped with this operation, for free.

\begin{wstructure}
<- Deriving generic functions
    <- Taking a Desc and computing a function
        <- Desc comes equipped with an induction principle
        -> Ability to compute more functions from it
            -> More generic functions
    <- Inspecting data-types
        <- All described byu a Desc code
        -> Ability to explore the code
            <- Desc equipped with an induction principle
            -> Build new objects based on that structure
\end{wstructure}

With this example, we have shown how we can derive a generic
operation, the catamorphism, from a pre-existing generic operation,
the induction principle. This has been made possible by our ability to
manipulate descriptions as first-class objects: the catamorphism is,
basically, a function mapping a $\Desc$ to a data-type specific
operation.

Moreover, the $\F{replace}$ function demonstrates the benefit of an
approach based on universes. The data-types living in the universe of
descriptions, we are able to \emph{inspect} them. As shown by
$\F{replace}$, it is easy to explore these structures, as well as
building new ones.

\subsection{The generic Free Monad}
\label{sec:desc-free-monad}

\begin{wstructure}
<- A generic program: the free monad construction
    <- Recall free monad construction in Haskell
        -> Based on a functor F
    <- Note that the free monad construction is itself defined by a functor
        -> Extract it
\end{wstructure}

In this section, we will turn to a more ambitious generic operation on
data-type. Given a functor, represented as a tagged description, we
build the free monad over this functor.

\newcommand{\FMFreeMonad}{\D{FreeMonad}}
\newcommand{\FMFreeMonadD}{\D{FreeMonadD}}
\newcommand{\FMVar}{\C{Var}}
\newcommand{\FMComposite}{\C{Composite}}

Let us recall the free monad construction. Given a functor $F$, the
free monad over $F$ is defined by the following data-type:

\[
\stk{
\data \FMFreeMonad : \PITEL{\V{F}}{\Set \To \Set} 
                     \PI{\V{X}}{\Set} 
                     \Set 
\where \\
\;\;\begin{array}{@{}l@{\::\:}l@{\quad}l}
    \FMVar           & X \To \FMFreeMonad\: F\: X                            \\
    \FMComposite     & F (\FMFreeMonad\: F\: X) \To \FMFreeMonad\: F\: X    
\end{array}
}
\]

\note{Recall monadic structure of this object}

Being an inductive type, this $\FMFreeMonad$ data-type is itself
defined by a functor signature. It is given by:

\[
\FMFreeMonadD\: F\: X\: Z \mapsto X + F Z
\]

\begin{wstructure}
    <- Encode it in the Desc world [equation]
        <- F is the Desc we start with
        <- The free monad functor is what we have just defined
        <- [\_]* : Desc -> Set -> Desc
           [\_]* D X = 'cons ['var ('sigma X (\_ -> '1))] D
        -> Mu does the fix-point
\end{wstructure}

In our setting, the functor will be represented by a tagged
description. The free monad construction will take such tagged
description, a set $X$ of variables, and will compute the tagged
description of the corresponding free monad. Implementing this
function is surprisingly easy:

\[\stk{
\FreeMonad{\_} : \TagDesc \To \Set \To \TagDesc \\
\FreeMonad{\pair{E}{D}{}}\:X \mapsto
    \pair{\ListCons{\DVar{}}{E}}
         {\pair{\DSigma{X}{\DUnit}}{D}{}}{}
}\]

We simply add a constructor, $\DVar{}$, and define its argument to be
a $\DSigma{X}{\DUnit}$, that is an element of $X$. We keep $E$ and $D$
as they were, hence leaving the functor unchanged. Unfolding the
interpretation of this definition, we convince ourself that this
corresponds to the functor $\FMFreeMonadD$. The fix-point operation
ties the knot and gives us the full-blown free monad construction.

\begin{wstructure}
<- A generic program: monadic substitution [equation]
    <- subst : \forall T X Y. mu ([T]* X) -> (X -> mu ([T]* Y)) -> mu ([T]* Y)
        -> Using Fold
\end{wstructure}

Of course, we must equip the resulting data-types with operations
delivering a monadic interface. As usual, \(\LAM{\x}\DVar{\x}\)
performs the r\^ole of ``return'', embedding variables into terms. The
``bind'' operation corresponds to \emph{substitution}. We will now
implement it, as a generic function.

Our implementation will appeal to the $\cata$ function developed
previously. So, let us write down the types, and fill as much
arguments to $\cata$ as possible:

\newcommand{\subst}{\F{subst}}
\newcommand{\apply}{\F{apply}}

\note{This is not quite Mu of a tagged description}

\[\stk{
\begin{array}{@{}ll}
\subst : & \PITEL{\V{D}}{\TagDesc}
           \PI{\V{X}, \V{Y}}{\Set} \\
         & \Mu{(\toDesc{\FreeMonad{D}{X}})} \To
           (\V{X} \To \Mu{(\toDesc{\FreeMonad{D}{Y}})}) \To
           \Mu{(\toDesc{\FreeMonad{D}{Y}})} 
\end{array} \\
\subst\: D\: X\: Y\: x\: \sigma \mapsto
  \cata\: (\toDesc{\FreeMonad{D}{X}})\: 
          (\Mu{(\toDesc{\FreeMonad{D}{Y}})})\: 
          ???\: 
          x
}\]

Our task is therefore to implement the algebra of the
catamorphism. Intuitively, its role is to catch appearances of
$\DVar{x}$ and replace them by $\sigma x$. Otherwise, it should simply
appeal to the induction hypothesis. This corresponds to the following
definition:

\[\stk{
\begin{array}{@{}ll}
\apply : & \PITEL{\V{D}}{\TagDesc} 
           \PI{\V{X}, \V{Y}}{\Set} \\
         & (\V{X} \To \Mu{(\toDesc{\FreeMonad{D}{X}})}) \To
           \descop{\toDesc{\FreeMonad{D}{X}}}{\Mu{(\toDesc{\FreeMonad{D}{Y}})}} \To
           \Mu{(\toDesc{\FreeMonad{D}{Y}})}
\\
\end{array} \\
\begin{array}{@{}l@{\:\:\mapsto\:}l}
\apply\: D\: X\: Y\: \sigma\: \pair{\DVar{}}{x}{}   & \sigma\: x                   \\
\apply\: D\: X\: Y\: \sigma\: \pair{\etag{t}}{ys}{} & \Con{\pair{\etag{t}}{ys}{}}
\end{array}
}\]

\begin{wstructure}
    -> Consequences
        <- We have free monad data-type
            <- Term + variables
        <- We have monad operations
            <- Return / var
            <- Substitution / bind
\end{wstructure}

Filling the $???$ sub-goal with $\apply\: D\: X\: Y\: \sigma$
completes the implementation. To sum up, we have implemented the free
monad construction for an arbitrary tagged description. This gives our
user the ability, for any data-type, to extend it with a notion of
variable. Then, we have equipped this structure with the corresponding
monadic operation, ``bind'' and ``return''. Whereas the ``return'' is
a trivial variable former, the ``bind'' operation corresponds to the
substitution of variables by terms. We have shown how to implement it,
for any free monad.


\begin{wstructure}
<- Deriving new data-structure and functions on them
    <- Computing the Free Monad of a data-type
        <- Derive new data-structure from previous one
            <- It is just code
        /> New data-structure comes with some equipment
    <- Computing new functions on computed data-types
        <- If data comes with structure, we ought to be able to capture it
            <- Induction on Desc
            -> Ability to compute over data
\end{wstructure}

With the free monad construction, we have seen two kind of generic
operations. Firstly, we have derived a new data-structure from another
one: we make the free monad from its underlying functor. To do so, we
crucially rely on the fact that data-types are nothing but codes. We
are therefore entitled to modify this code and, in this case, extend
it. Extending a data-type might give rise to a more structured
object, as was the case here.

So, secondly, we have equipped this new data-type with its inherent
structure: the bind and return operations. We have been able to build
them as generic functions. Here, we rely on our ability to compute
over descriptions, thanks to the induction principle.


%%%%%%%%%%%%%%%%%%%%%%%%%%%%%%%%%%%%%%%%%%%%%%%%%%%%%%%%%%%%%%%%
%% Indexing descriptions
%%%%%%%%%%%%%%%%%%%%%%%%%%%%%%%%%%%%%%%%%%%%%%%%%%%%%%%%%%%%%%%%

\section{Indexing descriptions}
\label{sec:indexing-desc}

\begin{wstructure}
!!! Need Help !!!
<- Motivation
    <- Desc: expressivity of simply-typed data-types: inductive types
        <- Values do not influence types
    /> Example: Vectors
        <- Cannot be defined by just induction
            <- Vectors of all size need to be defined at the *same* time
            -> Defined as a *family* of types
                -> Index
        -> I -> IDesc I: Inductive family
    ???
\end{wstructure}

\note{ Need care: motivating motivation of indexing. }

So far, we have explored the well-known realm of inductive types. We
have built upon the experience gained in using simply-typed languages,
such as Haskell or OCaml. In our dependent setting, we provided these
data-types by the mean of $\Desc$, a universe of descriptions. 

While inductive types are the alpha and omega of simply-typed
languages, evolving in a dependent setting fosters new
opportunities. The typical example is bounded lists, also known as
vectors. A vector is a list decorated by its length. Having this
information prevents hazardous operations, such as taking the head of
an empty vector: the head function only takes vectors of length
$\NatSuc{n}$, as enforced by its type. This is made possible by the
very specificity of dependently-typed systems: terms -- the length --
are allowed to influence types -- the vector type.

However, this new class of objects cannot be defined by mere
induction. In the case of vectors, for instance, we have to define the
whole \emph{family} of vector in one go: vectors of all size needs to
be defined at the same time. Where simply-typed languages have
inductive types, in dependently typed programming we the basic grammar of
data-types should be that of inductive families of types. To this end, we
rely on \emph{indexing}. We are going to transform the $\Desc$ universe into
$\IDesc$, the universe of indexed descriptions. We express inductive families
by the $I \To \IDesc{I}$ type.  \note{pwm: Vec might not be the best
motivating example here, since it doesn't appear in the paper (more on that
later). Come to think of it, this is entirely too vague without spelling out
a concrete example.} 

\subsection{Desc, atomically}

\begin{wstructure}
<- Adding hindx have introduced some duplication
    <- indx == hindx 1
    -> We can factor out commonalities 
        /> Obtain an equivalent presentation
        /> Still embeddable (refer to the Agda model)
\end{wstructure}

Before moving on indexed descriptions, we have to carry out some
maintenance work on descriptions. We presented $\Desc$ as the grammar
of inductive types. Hence, the codes closely follow this grammar. In
the following, we adopt an alternative presentation. With $\DSigma$,
we are actually \emph{quoting} a standard type-former, namely

$$\Bhab{\Sigma}{\PI{\V{S}}{\Set} (S \To \Set) \To \Set}$$

In the alternative presentation, we go further and present all our
codes as quotations of standard type-formers. This presentation is
shown in Figure~\ref{fig:type-former-desc}. The reader will notice the
introduction of $\DProd{}{}$ that overlaps with $\DSigma{}{}$. Unlike
$\DSigma{}{}$, $\DProd{}{}$ is first-order, hence amenable to a
definitional treatment. \note{We could replace $\DUnit$ by $\DConst$
  here, instead of doing it in the following section. This is also a
  story about being first-order, so that would make sense, isn't it?}

This reorganisation is strictly equivalent to the previous one
(Fig.~\ref{fig:hindx_desc}). Just as the previous version, it is also
self-descriptive. We refer the reader to the companion technical
report for details. In this finer-grained presentation, we can define
$\DIndx$ and $\DHindx$ as follow:

\[\begin{array}{l@{\:\mapsto\:\:}l}
\DIndx{D}         & \DProd{\DId}{D}                      \\
\DHindx{H}{D}     & \DProd{(\DPi{H}{(\LAM{\_} \DId)})}{D}
\end{array}
\]

Consequently, the examples previously developed can be
straightforwardly translated into this new presentation. For example,
here is the new definition of $\NatD$:

\[\stk{
\NatD : \Desc \\
\NatD \mapsto \DSigma{(\EnumT{[ \NatZero, \NatSuc{} ]})}
                     {[ \DUnit \quad \DId ]}
}\]


In the following, we adopt this last version as our de
facto universe of inductive types. In particular, we are going to
evolve this presentation into an indexed one.

\note{Shall we talk about the Type Theory being Desc Zero? or such story?}

\begin{figure}

\[\stk{
\begin{array}{ll}
\stk{
\data \Desc : \Set \where \\
\;\;\begin{array}{@{}l@{\::\:\:}l@{\quad}l}
    \DId            & \Desc                                   \\
    \DUnit          & \Desc                                   \\
    \DProd{}{}      & \PI{\V{D}, \V{D'}}{\Desc} \Desc         \\
    \DSigma         & \PI{\V{S}}{\Set} \PIS{S \To \Desc} \Desc \\
    \DPi            & \PI{\V{S}}{\Set} \PIS{S \To \Desc} \Desc 
\end{array}
}
\vspace{0.2in}
\\
\stk{
\descop{\_\:}{} : \Desc \To \Set \To \Set \\
\begin{array}{@{}l@{\:=\:\:}ll}
\descop{\DId}{X}          &  X                                           \\
\descop{\DUnit}{X}        &  \Unit                                       \\
\descop{\DProd{D}{D'}}{X} &  \TIMES{\descop{D}{X}}{\descop{D'}{X}}       \\
\descop{\DSigma{S}{D}}{X} &  \SIGMA{\V{s}}{S} \descop{D\: s}{X}                \\
\descop{\DPi{S}{D}}{X}    &  \PI{\V{s}}{S} \descop{D\: s}{X}            
\end{array}
}
\end{array}
}\]

\caption{Universe of descriptions based on Type-formers}
\label{fig:type-former-desc}

\end{figure}

\subsection{From Desc to IDesc}

\begin{wstructure}
<- Labelling Id
    <- We had: data Desc : Set -> Set
    -> We want: data IDesc : (I -> Set) -> Set
        <- Indexed functor (?)
        -> It is sufficient to label Id
            <- Where the functor is built
\end{wstructure}

In the previous section, we have presented the $\Desc$ universe as the
grammar of functors in the category $\Set$. We have seen how to encode
inductive types in this setting. To encode an inductive family indexed
by $\Bhab{\V{I}}{\Set}$, we rely on functors in the category
$\Set^I$. We call these \emph{indexed functors}. 
\note{pwm: What's our story \emph{vis-\`{a}-vis} the 2 notions of indexed
functor at play here: $(I\to\Set)\to\Set$ and $(I\to\Set)\to(I\to\Set)$?}  
In the following, we
implement $I \To \IDesc{I}$ as a grammar for indexed functors:

\[\stk{
\data \IDesc{} (\Bhab{\V{I}}{\Set}) : \Set \where \\
\;\;\ldots \\
\\
\idescop{\_\:}{}{} : \PI{\V{I}}{\Set} \IDesc{I} \To (\V{I} \To \Set) \To \Set    \\
\ldots
}\]

This generalisation can be achieved by a minor evolution of $\Desc$
(Fig.~\ref{fig:type-former-desc}). Indeed, the functorial nature of the
atomic $\Desc$ is entirely captured by the $\DId$ code. We get an indexed
functor by turning $\DId$ into an index-dependent $\DVar$ code:

\[\stk{
\data \IDesc{} (\Bhab{\V{I}}{\Set}) : \Set \where \\
\;\;\begin{array}{@{}l@{\::\:\:}l@{\quad}l}
    \DVar{}         & I \To \IDesc{I}                                   \\
    \ldots          & \ldots
\end{array} \\
\\
\idescop{\_\:}{}{} :_{\PI{\V{I}}{\Set}} \IDesc{I} \To (\V{I} \To \Set) \To \Set        \\
\begin{array}{@{}l@{\:=\:\:}ll}
\idescop{\DVar{i}}{I}{P}      &  P~i                                                 \\
\ldots                        &  \ldots
\end{array}
}\]

\begin{wstructure}
<- Also replacing '1 by 'const  [figure]
    <- For convenience
        <- 'const X equivalent to 'sigma X (\_ -> '1)
        /> Easier to abstract
            <- Extensionally same
            /> 'const more useful in practice
\end{wstructure}

The resulting code of $\IDesc$ is presented in
Figure~\ref{fig:idesc}. The reader will notice that we have also
replaced $\DUnit$ by a more general $\DConst$ code. Whereas $\DUnit$
was interpreted as the unit set, $\DConst{X}$ is interpreted as $X$,
for any $\Bhab{X}{\Set}$. Extensionally, $\DConst{X}$ and
$\DSigma{X}{\DUnit}$ are equivalent. However, $\DConst$ is more
succinct. More importantly, $\DConst$ is \emph{first-order}, unlike
its equivalent encoding. From a definitional perspective, we are
giving more opportunities to the type-system, hence reducing the
burden on the programmer.

\begin{figure}

\[\stk{
\begin{array}{ll}
\stk{
\data \IDesc{} (\Bhab{\V{I}}{\Set}) : \Set \where \\
\;\;\begin{array}{@{}l@{\::\:\:}l@{\quad}l}
    \DVar{}         & I \To \IDesc{I}                                   \\
    \DConst{}       & \Set \To \IDesc{I}                                \\
    \DProd{}{}      & \PI{\V{D}, \V{D'}}{\IDesc{I}} \IDesc{I}           \\
    \DSigma         & \PI{\V{S}}{\Set} \PIS{S \To \IDesc{I}} \IDesc{I}  \\
    \DPi            & \PI{\V{S}}{\Set} \PIS{S \To \IDesc{I}} \IDesc{I} 
\end{array}
}
\vspace{0.2in}
\\
\stk{
\idescop{\_\:}{}{} :_{\PI{\V{I}}{\Set}} \IDesc{I} \To (\V{I} \To \Set) \To \Set                  \\
\begin{array}{@{}l@{\:=\:\:}ll}
\idescop{\DVar{i}}{I}{P}      &  P~i                                                 \\
\idescop{\DConst{X}}{I}{P}    &  X                                                   \\
\idescop{\DProd{D}{D'}}{I}{P} &  \TIMES{\idescop{D}{I}{P}}{\idescop{D'}{I}{P}}       \\
\idescop{\DSigma{S}{D}}{I}{P} &  \SIGMA{\V{s}}{S} \idescop{D\: s}{I}{P}                    \\
\idescop{\DPi{S}{D}}{I}{P}    &  \PI{\V{s}}{S} \idescop{D\: s}{I}{P}            
\end{array}
}
\end{array}
}\]

\caption{Universe of indexed description}
\label{fig:idesc}

\end{figure}

\subsection{Tagged indexed descriptions}

\begin{wstructure}
<- Tagged constructor choice
    <- Index available when defining the data-type
        -> Can influence the choice of constructors
        -> Dependently-typed data-types
            <- term (index) influence types
    -> Two parts
        <- Always on the menu
            <- E : EnumU 
            <- ED : E -> IDesc I
        <- Index-dependent
            <- F : I -> EnumU 
            <- FD : (i : I) -> spi (F i) (\_ -> IDesc I)
        -> taggedIDesc I == Sigma E ED x Sigma F FD
        -> toIDesc : (I : Set) -> taggedIDesc I -> (I -> IDesc I)
\end{wstructure}

In Section~\ref{sec:desc-examples}, we have defined a \emph{tagged}
form for descriptions. This format follows the usual presentation of
inductive types as sum-of-product. With indexed description, this
definition can be generalized. Indeed, when defining an indexed
data-type, we have access to this index. Therefore, we can use this
index to influence the choice of constructors. This captures the
essence of dependent data-types: a term -- the index -- has the
ability to influence the data-type.

For convenience, we divide a tagged indexed description in two parts:
first, the constructors that do not depend on the index; then, the
constructors that do. The non-dependent part mirrors the definition
for non-indexed descriptions: we are provided a finite choice of
constructors. The index-depend part simply indexes the choice of
constructors by $I$. Hence, by inspecting the index, it is possible to
enable or disable the constructors. 

\[\stk{
 \TagIDesc{I}  \mapsto \TIMES{\ATagIDesc{I}}{\ITagIDesc{I}} \\
 \ATagIDesc{I} \mapsto \SIGMA{\V{E}}{\EnumU} (\PI{\V{i}}{I} \spi{\V{E}}{(\LAM{\_} \IDesc{I})}) \\
 \ITagIDesc{I} \mapsto \SIGMA{\V{F}}{I \To \EnumU} (\PI{\V{i}}{I} \spi{(F\: i)}{(\LAM{\_} \IDesc{I})}) 
}\]

In the case of vectors, for instance, for the index $\NatZero$, we
would only propose the constructor $\ListNil$. Similarly, for
$\NatSuc{n}$, we only propose the constructor $\ListCons{}{}$. We will
see more instances of this pattern in the following.
\note{We're talking about vectors again, are we \emph{really} not going to
define them?}



\note{Should we say something about IMu, fixed-points, and their
  elimination before moving on? If so, where should we put it?}


\subsubsection{Examples}
\label{sec:idesc-examples}

\paragraph{Natural numbers:}

\begin{wstructure}
<- Nat
    -> [equation]
    <- Non-indexed types lives in IDesc 1
        -> This applies to all previous examples
\end{wstructure}

In order to gain some intuition with $\IDesc$, let us re-implement the
signature functor of natural numbers. We refer the reader to
Section~\ref{sec:desc-examples} for its incarnation in the universe of
descriptions. The code is the following:

\[\stk{
\NatD : \IDesc{\Unit} \\
\NatD \mapsto \DSigma{(\EnumT{[ \NatZero, \NatSuc{} ]})}
                     {[ \DConst{\Unit} \quad (\DVar{\Void}) ]}
}\]

Because $\Nat$ is just an inductive type, we make no use of the index:
it is the trivial inhabitant of $\Unit$. Therefore, the recursive
argument is materialised by $\DVar{\Void}$, where we were using $\DId$
in the $\Desc$ presentation. This transformation applies to all
inductive types: we have not lost in expressive power during the
transition.

\paragraph{Indexed descriptions:}

\begin{wstructure}
<- Levitation [figure]
    <- Following Desc encoding
        /> Note: simple data-type
            -> Live in IDesc 1
    -> Behind the scene, relies on the special purpose switchD
\end{wstructure}

A mandatory exercise consists in describing $\IDesc$ in itself. With
the experience gained in Section~\ref{sec:desc-levitate}, this
levitation does not pose any issue:

\[\stk{
\IDescD : \PI{\V{I}}{\Set} \IDesc{\Unit} \\
\begin{array}{@{}ll}
\IDescD~I \mapsto \DSigma{}{} & (\EnumT \red{[} \DVar{},
                                          \DConst{},
                                          \DProd{}{},
                                          \DSigma{}{}, 
                                          \DPi{}{} \red{]}) \\
                              & \bigRedBracket{\begin{array}{l}
                                      \DConst{I}                  \\
                                      \DConst{\Set}               \\
                                      \DProd{\DVar{\Void}}{\DVar{\Void}}  \\
                                      \DSigma{\Set}{(\LAM{S} \DPi{S}{(\LAM{\_} \DVar{\Void})})} \\
                                      \DSigma{\Set}{(\LAM{S} \DPi{S}{(\LAM{\_} \DVar{\Void})})}
                                   \end{array}
                             }
\end{array}
}\]

Note that this description also lives in $\IDesc{\Unit}$. Indeed,
for any $\V{I}$, $\IDesc~\V{I}$ is merely an inductive type. As for the
$\Desc$ levitation, the finite function space $[ \ldots ]$ elaborates into a
special purpose $\F{switchID}$ operator. 

\paragraph{Typed expressions:}

\begin{wstructure}
<- Hutton's razor
    <- Types
        <- 'Nat
        <- 'Bool
    <- Term [figure]
        <- val : Val 'a -> 'a  for Val : Ty -> Set, mapping to Nat and Bool
        <- cond : 'Bool -> a -> a -> a
        <- plus : 'Nat -> 'Nat -> 'Nat
        <- le : 'Nat -> 'Nat -> 'Bool
\end{wstructure}

%% Types
\newcommand{\Ty}{\C{Ty}}
\newcommand{\Ebool}{\etag{\CN{bool}}}
\newcommand{\Enat}{\etag{\CN{nat}}}

%% Constructors
\newcommand{\Eval}[1]{\etag{\CN{val}}~#1}
\newcommand{\Econd}[3]{\etag{\CN{cond}}~#1~#2~#3}
\newcommand{\Eplus}[2]{\etag{\CN{plus}}~#1~#2}
\newcommand{\Ele}[2]{\etag{\CN{le}}~#1~#2}

%% Index mapper (terminology?)
\newcommand{\Val}[1]{\D{Val}~#1}
\newcommand{\Var}[2]{\D{Var}_{#1}~#2}

%% Hutton expressions
\newcommand{\HExprD}{\C{ExprD}}
\newcommand{\HExprAD}{\C{ExprAlwaysD}}
\newcommand{\HExprID}{\C{ExprIndexedD}}
\newcommand{\HExprVarD}[1]{\C{ExprD}_{\C{Var},#1}}
\newcommand{\HExprFreeD}{\C{ExprD}^{\C{Free}}}
\newcommand{\HExprAFreeD}{\C{ExprAlwaysD}^{\C{Free}}}

So far, the examples we have seen lives in $\IDesc{\Unit}$, hence are
not using any indexing. We are going to define a syntax for a small
typed language. We consider two types, natural numbers and booleans:

\[
\Ty \mapsto \EnumT{[\Enat, \Ebool]}
\]

An expression of this language is either a value, a conditional
expression, addition of numbers, or comparison of numbers. Informally,
their type is the following:

\[
\begin{array}{l@{\::\:\:}l}
\Econd{}{}{}     & \forall \Bhab{ty}{\Ty} . \Ebool \To ty \To ty \To ty  \\ 
\Eplus{}{}       & \Enat \To \Enat \To \Enat                           \\
\Ele{}{}         & \Enat \To \Enat \To \Ebool                          \\
\Eval{}          & \forall \Bhab{ty}{\Ty} . \Val{ty} \To ty
\end{array}
\]

The function $\Val{}$, used in the definition of $\Eval{}$, simply
maps an object type $ty$ to the corresponding type in the host
language. Hence, the argument of $\Eval{}$ are ensured to be of the
expected type. We assume $\Nat$ and $\Bool$ represent natural numbers
and booleans in the host language. Further, we also assume the
existence of an addition and comparison operator in $\Nat$,
respectively named $\F{plusHost}$ and $\F{leHost}$. We define $\Val{}$
as follow:

\[\stk{
\Val{} : \Ty \To \Set \\
\begin{array}{@{}l@{\:=\:\:}l}
\Val{\Enat}   & \Nat \\
\Val{\Ebool}  & \Bool
\end{array}
}\]

In our universe of descriptions, this data-type is represented by a
tagged indexed description. We use the index to carry the type: the
resulting description is indexed by $\Ty$. We observe that some
constructors are ``polymorph'', namely $\Econd{}{}{}$ and $\Eval{}$. On the
other hand, the $\Eplus$ and $\Ele$ constructors are
index-dependent. $\Eplus$ is defined if and only if the result type --
the index -- is $\Enat$, whereas $\Ele$ is defined if and only if the
index is $\Ebool$. The actual code precisely follows this intuition,
as shown in Figure~\ref{fig:hexpr-full}. For brevity, we use an
informal $\case{\ldots}{\ldots}$ notation, simulating a definition by
pattern-matching. Formally, this corresponds to a call to the
$\switch{}{}{}{}$ eliminator.

\begin{figure*}

\[\stkc{
\stk{
\HExprD : \TagIDesc{\Ty} \\
\HExprD \mapsto ( \HExprAD , \HExprID ) \\
} \\
\\
\begin{array}{ll}
\stk{
\HExprAD : \ATagIDesc{\Ty} \\
\HExprAD \mapsto \bigRedBracket{
                 \begin{array}{l}
                   \EnumT{[\Eval{}, \Econd{}{}{}]} \red{,} \\
                   \LAM{ty}
                   \bigRedBracket{
                   \begin{array}{l}
                   \DConst{(\Val{ty})} \\
                   \DProd{\DVar{\Ebool}}{\DProd{\DVar{ty}}{\DVar{ty}}}
                   \end{array}
                   }
                 \end{array}
                 }
\\
\\
} &
\stk{
\HExprID : \ITagIDesc{\Ty} \\
\HExprID \mapsto \bigRedBracket{
                 \begin{array}{l}
                   \LAM{ty} \case{ty}{\Enat \To \EnumT{[\Eplus{}{}]} \\ \Ebool \To \EnumT{[\Ele{}{}]}} \red{,} \\
                   \LAM{ty}
                   \case{ty}{
                     \Enat \To \DProd{\DVar{nat}}{\DVar{nat}} \\
                     \Ebool \To \DProd{\DVar{nat}}{\DVar{nat}}
                   }
                   \end{array}
                   }
}
\end{array}
}\]

\caption{Syntax of typed expressions}
\label{fig:hexpr-full}

\end{figure*}

\begin{wstructure}
    -> evaluation: IMu TermD -> Val
        -> it is a catamorphism
            <- Look closer at the type
        -> implementation [code]
            <- Just define one reduction step
            -> cata does the rest
                /> cata is for free!
\end{wstructure}

\newcommand{\evalH}{\F{eval}_{\green{\Downarrow}}}
\newcommand{\evalOne}{\F{eval}_{\green{\downarrow}}}

Having implemented the syntax, we would like to describe its
semantics. To do so, we implement an evaluator. The type of the
evaluator is:

\[
\evalH : \PI{\V{ty}}{\Ty} 
         \IMu{\Ty}{\toIDesc{\HExprD}}{ty} \To
         \Val{ty}
\]

The type of $\F{eval}$ is strikingly similar to a
catamorphism. Indeed, implementing a single step of evaluation -- the
algebra -- is sufficient, as $\F{cataI}$ \note{We have not defined
  \F{cataI}!} gives, for free, the full evaluator. The implementation
is as follow:


\[\stk{
\evalOne : \PI{\V{ty}}{\Ty} \idescop{\toIDesc{\HExprD}\: ty}{\Ty}{ty} \To {\Val{ty}} \\
\begin{array}{@{}l@{\:=\:\:}l}
\evalOne\: \_\: \pair{\Eval{}}{x}{}                                         & x \\
\evalOne\: \_\: \pair{\Econd{}{}{}}{\pair{\BoolTrue}{\pair{x}{\_}{}}{}}{}   & x \\
\evalOne\: \_\: \pair{\Econd{}{}{}}{\pair{\BoolFalse}{\pair{\_}{y}{}}{}}{}  & x \\
\evalOne\: \Enat\: \pair{\Eplus{}{}}{\pair{x}{y}{}}{}                       & \F{plusHost}\: x y \\
\evalOne\: \Ebool\: \pair{\Ele{}{}}{\pair{x}{y}{}}{}                        & \F{leHost}\: x y 
\end{array} \\
\\
\evalH : \PI{\V{ty}}{\Ty} 
           \IMu{\Ty}{\toIDesc{\HExprD}}{ty} \To
           \Val{ty} \\
\evalH\: ty\: term = \F{cataI}\: \Ty\: 
                                 \toIDesc{\HExprD}\: 
                                 \Val{}\: 
                                 \evalOne\: 
                                 ty\: 
                                 term
}\]
\note{pwm: $\evalOne$ rather illustrates my point about the 2 types of
  IFunc floating around, $\idescop{\_}{}{}$ is defined using the 1st
  notion, and used it here as if it were the 2nd. pierre: in this
  case, it was a typo. But you're right, there is two notions and we
  probably want to clarify where these things live.}

\begin{wstructure}
    /> Closed term
        <- only constants and operations on them
        -> Extend Val with Var : Ty -> Set, mapping to EnumU
            -> Open term
            -> Language of well-typed terms
                <- By construction
\end{wstructure}

Hence, we have defined the syntax of a typed language of arithmetic
and boolean operations. We have given its semantics through an
evaluation function. Provided a one step semantic of the language, the
big step evaluation is granted without effort thanks to the generic
catamorphism. 

However, so far, we are only able to define and manipulate
\emph{closed} terms. By abstracting over $\Val{}$, it is possible to
build and manipulate \emph{open} terms, that is terms with symbolic
variables. On the model of $\Val{}$, we define $\Var{}{}$:

\[\stk{
\Var{}{} : \EnumU \To \Ty \To \Set \\
\Var{dom}{\_} = \EnumT{dom}
}\]

Whereas $\Val{}$ was mapping the type to the corresponding host type,
$\Var{}{}$ maps types to a finite set. The finite set -- the context
-- contains closed terms. A variable is therefore a $\Eval{}$ that
contains a pointer to a particular element of the finite set -- an
element of $\EnumT{dom}$. The extra argument to $\Var{}{}$ is the
domain of this context.

Consequently, replacing $\Val{ty}$ by
$(\SUM{\Val{ty}}{\Var{dom}{ty}})$ in Figure~\ref{fig:hexpr-full} turns
the language of closed term into a language of opened terms with
constants. For readability, we will abbreviate $\LAM{ty}
\SUM{\Val{ty}}{\Var{dom}{ty}}$ into $\SUM{\Val{}}{\Var{}{}}$ \note{Is
  it polite and comprehensible to ask that?} This defines a new
indexed description, called $\HExprVarD{dom}$.

\begin{wstructure}
        <- evaluator, with a context
            -> First, close variables
                -> Perform assignment
                <- subst [code]
\end{wstructure}

\newcommand{\discharge}{\F{discharge}}

Again, we would like to give a semantics to this extended language. We
proceed in two steps: first, we replace the variables by their value
in the context; then, we evaluate the resulting closed term. Thanks to
$\evalH$, we are already able to solve the second problem. Let us
focus on discharging variables from the context. Again, we can
subdivide this problem: first, we need the ability to discharge a
single variable from the context; then, we apply this $\discharge$
function on every variables in the term.

The $\discharge$ function is relative to the required type, the domain
of the context, and a context containing values of the corresponding
type. Its action is to map variables to their value in context, and
directly return constant values. This corresponds to the following
function:

\[\stk{
\begin{array}{@{}ll}
\discharge : & \PITEL{\V{ty}}{\Ty}
               \PITEL{\V{dom}}{\EnumU} \\
             & \PI{\V{\gamma}}{\spi{dom}{(\LAM{\_} \IMu{\Ty}{\toIDesc{\HExprVarD{dom}}}{ty})}} \\
             & (\SUM{\Val{ty}}{\Var{dom}{ty}}) \To
               \IMu{\Ty}{\toIDesc{\HExprD}}{ty} 
\end{array} \\
\begin{array}{@{}l@{\:\mapsto\:\:}l}
\discharge\: ty\: dom\: \gamma\: (\SumLeft x)  & \Con{\pair{\Eval{}}{x}{}} \\
\discharge\: ty\: dom\: \gamma\: (\SumRight v) & \switch{dom}{(\LAM{\_} \IMu{\Ty}{\toIDesc{\HExprVarD{dom}}}{ty})}{\gamma}{v}
\end{array}
}\]

\begin{wstructure}
            /> Then, perform subst everywhere in the term
                -> Show type [code]
                /> This is a bind!?
                -> There is some more structure 
                    -> We should try to get it
\end{wstructure}

Having implemented the local $\discharge$ operation, we are left with
applying it over all variables of the term. The type of this operation
is the following:

\newcommand{\substH}{\F{substExpr}}
\newcommand{\domNat}{dom_{\CN{nat}}}
\newcommand{\domBool}{dom_{\CN{bool}}}
\newcommand{\gammaNat}{\gamma_{\CN{nat}}}
\newcommand{\gammaBool}{\gamma_{\CN{bool}}}

\[
\begin{array}{@{}ll}
\substH  : & \PITEL{\V{\domNat}}{\EnumU} 
             \PITEL{\V{\gammaNat}}{\spi{\domNat}{(\LAM{\_} \IMu{\Ty}{\toIDesc{\HExprVarD{\domNat}}}{\Enat})}} \\
           & \PITEL{\V{\domBool}}{\EnumU} 
             \PITEL{\V{\gammaBool}}{\spi{\domBool}{(\LAM{\_} \IMu{\Ty}{\toIDesc{\HExprVarD{\domBool}}}{\Ebool})}} \\
          &  \begin{array}{@{}ll}
             \PI{\V{\sigma}}{& \PITEL{\V{\domNat}}{\EnumU} \\
                             & \PITEL{\V{\gammaNat}}
                                     {\spi{\domNat}{(\LAM{\_} \IMu{\Ty}{\toIDesc{\HExprVarD{\domNat}}}{\Enat})}} \\
                             & \PITEL{\V{\domBool}}{\EnumU} \\
                             & \PITEL{\V{\gammaBool}}
                                     {\spi{\domBool}{(\LAM{\_} \IMu{\Ty}{\toIDesc{\HExprVarD{\domBool}}}{\Ebool})}} \\
                             & \PI{\V{ty}}{\Ty} (\SUM{\Val{ty}}{\Var{dom_{ty}}{ty}}) \To
                               \IMu{\Ty}{(\toIDesc{\HExprD})}{ty}}
             \end{array}\\
          & \PI{\V{ty}}{\Ty}
            \IMu{\Ty}{\toIDesc{\HExprVarD{dom_{ty}}}}{\V{ty}} \To
            \IMu{\Ty}{\toIDesc{\HExprD}}{\V{ty}}
\end{array}
\]

Where $dom_{ty}$ is short for
 $\case{ty}{\Enat  \To \domNat
         \\ \Ebool \To \domBool}$

Abstracting away the book-keeping introduced by the context, this
definition looks familiar. Indeed, it is extremely similar to a
monadic \bind. This is not surprising as we are defining a first-order
syntax with variables: our data-type enjoys more structure than what
we are given. In particular, this is reminiscent to a free monad,
where $\Eval{}$ is the \return\ introducing variables. The substitution
$\sigma$ is implemented from $\discharge$, by picking the domain and
context corresponding to the type:

\[
\sigma\: \domNat\: \gammaNat\: \domBool\: \gammaBool\: ty\: var \mapsto 
    \discharge\: ty\: dom_{ty}\: \gamma_{ty}\: var 
\]

Instead of implementing $\substH$ in this special case, we shall
implement it in a generic setting, and express our example in that
setting. However, before entering the realm of generic structures, let
us further explore the universe of indexed descriptions.

\subsection{Constrained constructors}

\begin{wstructure}
!!! Need Help !!!
<- Assuming a suitable notion of definitional equality for I
    ???
\end{wstructure}

In the following, we ask $I$ to come equipped with a notion of
definitional equality. Hence, we can legibly equates elements of $I$.

\note{ I will need some tutorial on this ad-hoc notion of definitional
  equality. Or what Conor's meant with that. }

\begin{wstructure}
<- Fin [figure]
    <- Presentation with equality constraints
    <- Example of constrained constructor
        -> Constraints are translated into equalities
    ???
\end{wstructure}
 
Another common dependent data-type is $\Fin$, the finite sets of
numbers. $\Fin$ is a typical example of an inductive family using
\emph{constrained constructors}. The constructors $\FinZero$ and
$\FinSuc$ are only defined for an index distinct from $\NatZero$:


\[
\stk{
\data \Fin{} : \PI{\V{n}}{\Nat} \Set \where \\
\;\;\begin{array}{@{}l@{\::\:\:}l@{\quad}l}
    \FinZero_n      & \Fin{(\NatSuc{n})}   \\
    \FinSuc{}_n     & \Fin{n} \To \Fin{(\NatSuc{n})}
\end{array}
}
\]

One way to code constrained data-type is to appeal to equality. The
constraints are therefore captured by equations in the data-type. In
this case, we obtain the following definition:

\[\stk{
\FinD : \Nat \To \IDesc{\Nat} \\
\begin{array}{@{}ll}
\FinD\: n = \DSigma{}{} & (\EnumT{[ \FinZero , \FinSuc{} ]}) \\
                        & \bigRedBracket{
                          \begin{array}{l}
                            \DSigma{\Nat}{\LAM{m} \DConst{(n \PropEq \NatSuc{m})}} \\
                            \DSigma{\Nat}{\LAM{m} \DProd{\DVar{m}}{\DConst{(n \PropEq \NatSuc{m})}}}
                          \end{array}
                          }
\end{array}
}\]

Intuitively, the constraint is captured as follows. First, we store an
element $m$ of $\Nat$, thanks to a $\DSigma$. However, the constraint
stipulates that $m$ cannot be \emph{any} natural numbers: it must be
``the index minus one''. This translates into the constraint $n
\PropEq \NatSuc{m}$. 

\note{ Anything else? }

\begin{wstructure}
<- Vectors
    Do we treat them in the end? 
    What can we say here we haven't with Fin?
\end{wstructure}

\note{ I suspect we don't implement vectors, space-wise }
\note{pwm: I suspect we don't get away with leaving them out}

\begin{wstructure}
<- GADT-style data-type definition
    <- We directly support it
        <- Constraints turned into equalities
            <- Fin example
    <- Again, smooth transition from "today" data-types to advanced ones
        <- Same grammar
        /> More power
    /> Equations in data-types are not a blessing
        <- Complexify induction principle
        <- Lose match between data-type definition and the elimination form
\end{wstructure}

The $\Fin$ example is also a well-known example of Generalised
Abstract Data-Type (GADT) in Haskell. It comes as no surprise that our
universe of indexed description supports this restricted class of
dependent data-types. As we have shown with $\Fin$, constrained
constructors are straightforwardly translated into equality
constraints.

While we support the same grammar, indexed descriptions embraces a
much wider class of data-types. Indeed, we pose no constraint apart
from strict positivity\note{pwm: Is this actually the 1st mention of this
phrase? I think that should be rectified :)}. This suggests that, based on
the familiar grammar of GADTs, we can simply lift off the restrictions and
translates the user's input to a code in our universe of indexed description. 

However, one could, quite rightfully, be worried by the introduction
of equations in the picture. In the context of programming, these
equations will come back and haunt us, begging for proofs. Hence,
whereas the user typed a constructor-constrained data-type, the
elimination form will involve equations and their proofs. This
introduces a loophole in our abstraction we should try to address.

\note{Is that convincing? Worth saying? Anything else?}

\begin{wstructure}
!!! Need Help !!!
<- Brady optimisation: forcing
    <- Source to source translation
    <- Able to remove some constraints
    -> Example: Fin [figure]
    ??? More technical detail needed
\end{wstructure}

To get a clue of a solution, let us look back at $\Fin$. We note that
the equation is introduced because we are \emph{storing} the index of
the inductive family, through $m$. Hence the constraint. However,
\emph{inductive families need not store their
  indices}~\cite{brady:index-inductive-families}\note{ I'm not
  familiar with this, I need to get into the paper. }. Applying the
\emph{forcing} optimisation to our definition of $\Fin$ gives the
following, equivalent definition:

\[\stk{
\FinD : \Nat \To \IDesc{\Nat} \\
\begin{array}{@{}llll}
\FinD\: 0            & = & \multicolumn{2}{l}{\DSigma{\Void}{\Void} } \\
\FinD\: (\NatSuc{n}) & = & \DSigma{}{} & (\EnumT{[ \FinZero , \FinSuc{} ]}) \\
                     &   &             & \bigRedBracket{
                                         \begin{array}{l}
                                         \DConst{\Unit} \\
                                         \DVar{n}
                                         \end{array}
                                         }
\end{array}
}\]

Equations have simply disappeared! We should precise that forcing a
description is not guaranteed to remove all constraints. It is subject
of future work to see if constraints can be entirely eradicated, or
presented more conveniently to the user. Finally, it is worth
mentioning that the forcing optimisation is a \emph{source-to-source}
transformation of the description.

\subsection{Free IMonad}

\begin{wstructure}
<- Variation on a theme: free imonad construction
    <- Recall existence of generic free monad construction
    -> Present its generalisation to IDesc [equation]
        <- \I -> IDesc I as describing an indexed endofunctor
        <- Free monad construction
    -> Still a suitable, generic notion of substitution
        <- show type signature
        <- show implementation?? (space! space!)
\end{wstructure}

In this section, we develop more structure for our indexed
data-types. In Section~\ref{sec:desc-free-monad}, we have constructed
a free monad operation on simple descriptions. Building on this
experience, we are going to present its equivalent in the indexed
world. The process the same. Namely, given an indexed functor, we
derive the indexed functor coding its free monad:
\note{pwm: Whoa there. Maybe we should say something about IMonads in
general before we get to this point?}

\[\stk{
\begin{array}{ll}
\FreeIMonad{\_}{} : & \PITEL{\V{I}}{\Set}
                      \PITEL{\V{X}}{\V{I} \To \Set} \\
                    & \PITEL{\V{R}}{\TagIDesc{I}} \To
                      \TagIDesc{I}
\end{array} \\
\FreeIMonad{(E,F)}{I}{X} \mapsto
    \pair{\pair{\ListCons{\DVar{}}{(\fst{E})}} 
               {\LAM{i}
                \pair{(\DConst{(X\: i)})}
                     {((\snd{E})\: i)}{}}{}}
         {F}{}
}\]

Just as in the universe of descriptions, this construction comes with
an obvious \return and a substitution operation, the \bind. Its type
signature is the following:

\newcommand{\substI}{\F{substI}}

\[
\begin{array}{@{}ll}
\substI : & \PITEL{\V{I}}{\Set}
            \PITEL{\V{X}, \V{Y}}{\V{I} \To \Set}
            \PITEL{\V{R}}{\TagIDesc{\V{I}}} \\
          & \PITEL{\V{\sigma}}{\PI{\V{i}}{\V{I}} \V{X}\:\V{i} \To 
                               \IMu{I}{(\toIDesc{\FreeIMonad{\V{R}}{\V{I}}{\V{Y}}})}{\V{i}}} \\
          & \PITEL{\V{i}}{\V{I}}
            \PITEL{\V{D}}{\IMu{I}{(\toIDesc{\FreeIMonad{\V{R}}{\V{I}}{\V{X}}})}{\V{i}}} \To
            \IMu{I}{(\toIDesc{\FreeIMonad{\V{R}}{\V{I}}{\V{Y}}})}{\V{i}}
\end{array}
\]


\subsubsection{Examples}

\paragraph{Typed expressions:}

\begin{wstructure}
<- Hutton's razor is a free monad
    <- substI was our candidate bind
        -> Massage the definition of expr to get it for free
    <- Finding the functor
        <- 'val is the return
        <- The other components are the action
        -> Updated tagged description [figure]
    -> Compute the free monad Hutton * X
        -> With X = Val: We get back our closed terms
        -> With X = Val + Var vars: We get back our open terms
\end{wstructure}

In Section~\ref{sec:idesc-examples}, we had the intuition that our
datatypes $\HExprD$ and $\HExprVarD{dom}$ enjoy a monadic structure. We had
identified the variable substitution operation as the \bind of a free
monad. The definition of $\substI$ above confirm our intuition: using
$\substI$, we should easily obtain $\substH$. To do so, we first have
to massage the definition of our data-type, to exhibit its monadic
structure.

As previously mentioned, we identify $\Eval{}$ as the \return of the
free monad, while the other components are the action of the monad. As
a result, the definition is similar to $\HExprD$ presented in
Figure~\ref{fig:hexpr-full}, at the exception of $\HExprAD$ that is
replaced by $\HExprAFreeD$:

\[\stk{
\HExprAFreeD : \ATagIDesc{\Ty} \\
\HExprAFreeD \mapsto \bigRedBracket{
                 \begin{array}{l}
                   \EnumT{\red{[}\Econd{}{}{}\red{]}} \red{,} \\
                   \LAM{ty}
                   \bigRedBracket{
                   \begin{array}{l}
                   \DProd{\DVar{\Ebool}}{\DProd{\DVar{ty}}{\DVar{ty}}}
                   \end{array}
                   }
                 \end{array}
                 }
}\]

The resulting data-type is called $\HExprFreeD$. By a simple unfolding
of definition, we note that $\FreeIMonad{\HExprFreeD}{\Ty}{\Val{}}$
corresponds to the syntax of closed terms, $\HExprD$. Similarly,
$$\FreeIMonad{\HExprFreeD}{\Ty}{(\LAM{ty} \SUM{\Val{ty}}{\Var{dom}{ty}})}$$
corresponds to expressions with variables, $\HExprVarD{dom}$.

\begin{wstructure}
    /> On open terms, we get a substitution
        -> Apply substI on assgnmt
            -> Have a well-typed language
            -> Get a safe evaluator 
                <- for well-typed terms 
                <- in well-typed contexts
\end{wstructure}

The evaluator for closed term we implemented in
Section~\ref{sec:idesc-examples} remains unchanged. It reduces closed
terms in $\FreeIMonad{\HExprFreeD}{\Ty}{\Val{}}\: ty$ to values in
$\Val{ty}$. We are therefore left with implementing $\substH$. We
simply have to fill in the right arguments to $\substI$, the type
guiding us:


\[\stk{
\substH\: \domNat\: \domBool\: 
            \gammaNat\: \gammaBool\:
            \sigma\: 
            ty\: 
            term \mapsto  \\
\;\;\ \begin{array}{ll}
       \substI\: & \Ty\: 
                  (\SUM{\Val{}}{\Var{dom_{ty}}{}})\: 
                  \Val{}\:
                  \HExprFreeD\:  \\
                &
                  (\sigma\: \domNat\: \domBool\: \gammaNat\: \gammaBool)\:
                  ty\:
                  term
      \end{array}
}\]


Hence, we have completed our implementation of the evaluator for open
terms. We started with a well-typed language of arithmetical
expressions. We have seen how to take advantage of indexes to only
present type-safe constructors. Then, we have implemented an evaluator
for closed term, based on the generic catamorphism function. Having
implemented an open term representation, we wanted a substitution
operation, in order to close open terms in a context. Again, we have
implemented this operation with the generic substitution
operator. Hence, without much efforts, we have described the syntax of
a well-typed language, together with its semantics.

\begin{wstructure}
<- IDesc
    <- Another instance of free monad
        <- Var: Return
        <- Remaining: (not even indexed) functor
    -> Madness just starts
        <- map operation
        <- [ subst sigma D ] X = [D] (\x -> [subst sigma x] X)
        ???
\end{wstructure}

\paragraph{Indexed description:}

Another instance of free monad is $\IDesc$ itself. Indeed, $\DVar$ is
nothing but the \return. The remaining constructors are simply the
signature functor, trivially indexed by $\Unit$. Hence, we describe
this signature functor by the following code:

\[\stk{
\IDescFreeD : \PI{\V{I}}{\Set} \IDesc{\Unit} \\
\begin{array}{@{}ll}
\IDescFreeD~I \mapsto \DSigma{}{} & (\EnumT \red{[} \DConst{},
                                              \DProd{}{},
                                              \DSigma{}{}, 
                                              \DPi{}{} \red{]}) \\
                                  & \bigRedBracket{\begin{array}{l}
                                        \DConst{\Set}               \\
                                        \DProd{\DVar{\Void}}{\DVar{\Void}}  \\
                                        \DSigma{\Set}{(\LAM{S} \DPi{S}{(\LAM{\_} \DVar{\Void})})} \\
                                        \DSigma{\Set}{(\LAM{S} \DPi{S}{(\LAM{\_} \DVar{\Void})})}
                                    \end{array}}
\end{array}
}\]

Then, we get $\IDesc$ by building its free monad:

\[\stk{
\IDescD : \PI{\V{I}}{\Set} \IDesc{\Unit} \\
\IDescD\: I \mapsto \FreeIMonad{\red{[}\IDescFreeD\red{]}}{\Unit}{I}\: I
}\]


\note{ Should we stop here? Or show off some madness? }


%%%%%%%%%%%%%%%%%%%%%%%%%%%%%%%%%%%%%%%%%%%%%%%%%%%%%%%%%%%%%%%%
%% Discussion
%%%%%%%%%%%%%%%%%%%%%%%%%%%%%%%%%%%%%%%%%%%%%%%%%%%%%%%%%%%%%%%%

\section{Discussion}
\label{sec:discussion}

\subsection{Universe stratification}

\begin{wstructure}
!!! Need Help !!!
<- Universe stratification
    <- Stratified agda model
        <- Fully stratified
        <- Proof of iso between host and embedding
    ???
\end{wstructure}

As presented, our type theory suffers from a major weakness. Indeed,
we are subject to Girard's paradox, as we assume that $\Set$ lives in
$\Set$. We made that choice for presentational convenience, as
universe stratification is orthogonal to our work. Nonetheless, our
universe of description rather naturally leads itself to
stratification. Unsurprisingly, $\IDesc{\!}$ at level $l$ is of type
$\Set^{\blue{l+1}}$. Similarly, the interpretation of $\IDesc{\!}$ at
level $l$ is an object of type $\Set^{\blue{l}}$:

\[\stk{
\data \IDesc{\!}^{\blue{l}} (\Bhab{\V{I}}{\Set^{\blue{l+1}}}) : \Set^{\blue{l+1}} \where \\
\;\;\ldots \\
\\
\idescop{\_\:}{}{}^{\blue{l}} : \PI{\V{I}}{\Set^{\blue{l+1}}} \IDesc{{\!}^{\blue{l}}\V{I}} \To (\V{I} \To \Set^{\blue{l}}) \To \Set^{\blue{l}}    \\
\ldots
}\]

Consequently, we can implement the operations and examples developed
above. We refer the reader to our Agda implementation, which take
advantage of set polymorphism to implement the universe of indexed
descriptions at any level. Further, we have coded $\IDesc{\!}$ in
itself and have proved the isomorphism between the host and the
embedded universes.

\subsection{Related Work}

\begin{structure}
!!! Need Help !!!
<- Comparison with Induction Recursion
    ???
\end{structure}


\begin{wstructure}
!!! Need Help !!!
<- Related Work
    <- Generic in simply-typed functional languages
        <- PolyP \cite{jansson:polyp}
        <- Generic Haskell \cite{hinze:generic-haskell}
        <- Scratch your boilerplate \cite{spj:syb}
\end{wstructure}

Generic programming is a vast topic. We refer our reader to Garcia et
al.~\cite{garcia:generic-comparative-study} for a broad overview of
generic support in various languages. In the sole context of Haskell,
there is a myriad of proposals. These approaches are presented and
compared in Hinze et al.~\cite{hinze:generic-approach-comparative} and
Rodriguez et al.~\cite{rodriguez:generic-libs-comparative}.

In particular, our approach is similar in spirit with polytypic
programming, as initiated by PolyP~\cite{jansson:polyp}. Indeed,
generic functions, in our system, are built by induction on the
pattern functor. Unlike PolyP, we do not have to resort to pre-process
data-type definitions: our data-types are, natively, nothing but
codes.

Our approach also support the Generic
Haskell~\cite{hinze:generic-haskell} model. This model, based on
type-indexed data types, enables computing new data-types from
others. This is natural in our system, as data-types descriptions are
first-class.

Another generic programming framework is Scrap Your
Boilerplate~\cite{spj:syb} (SYB). Our proposal is different in various
ways. The corner stone of SYB is the \emph{spine} view of data-type
constructors. A data-type is a spine composed by a constructor applied
to some arguments. Further, this spine is equipped with some
combinators including, primarily, an iterator. In this setting,
generic programs are written by composing these combinators. This
relies on a $\CN{Typeable}$ type-class, allowing dynamic dispatch to
data-type specific operations. As a result, SYB is not reflexive: it
is restricted to data-types instanciating $\CN{Typeable}$. Moreover,
it forbids type-indexed data types: it can only define generic
functions.


\begin{wstructure}
    <- Generic in dependent types
        <- Norell \cite{norell:msc-thesis}
        <- Polytypic prog in Coq \cite{verbruggen:polytype-coq}
        <- Universes for generic prog \cite{benke:universe-generic-prog}
\end{wstructure}

The interest in generic programming for dependent types is not new
either. Norell~\cite{norell:msc-thesis} have shown the benefit of
polytypic programming in the setting of Alfa, a precursor of
Agda. Similarly, Verbruggen et al.~\cite{verbruggen:polytype-prog-coq,
  verbruggen:polytype-coq} have developed a framework for polytypic
programming in the Coq theorem prover. However, these works aim at
\emph{modelling} PolyP or Generic Haskell in a dependently-typed
setting, for the purpose of proving correctness properties of Haskell
code. Our approach is different in that we aim at building a
foundation for data-types, in a dependently-typed theory, for a
dependently-typed system.

Closer to us is the work by Benke et
al.~\cite{benke:universe-generic-prog}. This seminal work introduced
the usage of universes for developing generic programs. Our own
universes are rather similar to theirs: our universe of descriptions
is similar to their universe of iterated induction, and our universe
of indexed descriptions is isomorphic to their universe of finitary
indexed induction. This is not surprising, as we share the same source
of inspiration, namely induction recursion.

However, we differ in several ways. First, they adopt a generative
perspective: each universe extends the base type theory with both type
formers and elimination rules. Thanks to the levitation, we only rely
on a generic induction and a specialised
$\switchD{\!}{\!}{\!}$. Second, the authors do not tackle the issue of
\emph{programming} with such codes: it is not obvious, to us, how one
can define, manipulate, and program over these coded data-types. With
descriptions, we have shown how to abstract away codes and present a
convenient and familiar presentation to the developer. Finally, the
authors often resort to an extensional notion of equality, while we
have given an equality-agnostic presentation. Beside, our presentation
is arranged so as to use definitional equality as much as
possible. Hence, in practice, the developer is relieved from many
proof obligations.


%%%%%%%%%%%%%%%%%%%%%%%%%%%%%%%%%%%%%%%%%%%%%%%%%%%%%%%%%%%%%%%%
%% Conclusion
%%%%%%%%%%%%%%%%%%%%%%%%%%%%%%%%%%%%%%%%%%%%%%%%%%%%%%%%%%%%%%%%

\section{Conclusion}

\begin{wstructure}
<- System developed in a reasonable theory
    <- Pi, Sigma, Finite sets
    /> No assumption about the equality
    -> Low requirement / high applicability
\end{wstructure}

\begin{wstructure}
<- Formalize a rationnalized presentation of types
    <- Working directly with codes is not practical
    -> Bidirectional type-checking
        <- Type information flows during type checking/type synthesis
        -> Elaboration turns high-level expressions to low-level terms
    -> Should not be afraid by codes
\end{wstructure}

In this paper, we have presented a universe of datatypes for a
dependently-typed language. To ensure the generality of our proposal,
this system has been built in a familiar type theory, with no
assumption about the underlying propositional equality. Because our
approach is extensively using codes for universes, we have given a
rationalised presentation of codes. Thanks to type propagation, we
make practical the usage of codes for datatypes.

\begin{wstructure}
<- Dependently-typed presentation of simple inductive types
    <- Universe of descriptions
        <- Based on the specificity of dependent types
            <- Universe of codes
            <- Sigma types
        -> External fix-point and generic induction scheme
    <- Rationalised by type propagation
        -> Developer does not see the code
    <- Self-describing
        <- Step-by-step exposition
        -> Minimal extension to the type theory
            <- Just need fix-point and induction
        -> Closed presentation of datatype
            -> Non generative
        -> datatype is just data
    <- Generic programming is just programming
        <- Generic catamorphism
        <- Generic free monad
\end{wstructure}

To introduce our approach, we have presented a universe of
description. This universe has the expressive power of simple
inductive types, as we can find them in simply typed languages. Going
one step further, we present an implementation of this universe as a
self-described object. Hence, for a minimal extension of the
type-theory, we get a closed, self-describing presentation of
datatypes, where datatypes are just data.

\begin{wstructure}
<- Indexed descriptions for dependent datatypes
    <- Presented as a slight generalisation of Desc
        <- Just add indexing
    <- Develop several examples of datatypes
        <- Typed syntax
        <- Constrained datatype a la GADT
    <- Generic indexed programming
        <- Indexed free monad
        <- Substitution
\end{wstructure}

To model dependent datatypes, we generalise our presentation to
support indexing. The universe of indexed descriptions thus built
encompasses inductive families. Again, this universe is
self-describing. Moreover, we develop several examples of dependent
datatypes and generic functions over them.

\begin{wstructure}
<- All of this without cheating
    <- Admit a correct stratification
    <- Terminating
    <- Strictly-positive types
\end{wstructure}

We have presented a self-describing, self-hosted universe for
datatypes. We have shown the benefit of such approach, by our ability
to reflect datatypes in our type-theory. This fosters a new way of
considering generic programming: just as programming. Moreover,
despite its egg-and-chicken nature, this presentation is free of
paradox: it has been formalised in Agda, admitting a correct
stratification.

\paragraph{Future Work:} As such, our indexed description universe does 
not cover several extensions of inductive families. One of these is
induction-recursion. An interesting question is to locate indexed
descriptions in the spectrum between inductive families and indexed
induction-recursion. Another popular extension we plan to consider is
to allow internal fix-points.

Also, we have presented a generic notion of syntax with variables,
thanks to the free monad construction. Further, we would like to
explore the notion of syntax with binding. Interestingly, introducing
internal fix-points in our universe turns it into such syntax with
binding. Once again, levitation would reveal itself extremely
convenient by providing generic tools to handle binding.

%%%%%%%%%%%%%%%%%%%%%%%%%%%%%%%%%%%%%%%%%%%%%%%%%%%%%%%%%%%%%%%%
%% Appendices
%%%%%%%%%%%%%%%%%%%%%%%%%%%%%%%%%%%%%%%%%%%%%%%%%%%%%%%%%%%%%%%%

% \appendix
% \section{Appendix Title}

% This is the text of the appendix, if you need one.


%%%%%%%%%%%%%%%%%%%%%%%%%%%%%%%%%%%%%%%%%%%%%%%%%%%%%%%%%%%%%%%%
%% Acknowledgments
%%%%%%%%%%%%%%%%%%%%%%%%%%%%%%%%%%%%%%%%%%%%%%%%%%%%%%%%%%%%%%%%

% \acks

% Acknowledgments, if needed.


%%%%%%%%%%%%%%%%%%%%%%%%%%%%%%%%%%%%%%%%%%%%%%%%%%%%%%%%%%%%%%%%
%% Bibliography
%%%%%%%%%%%%%%%%%%%%%%%%%%%%%%%%%%%%%%%%%%%%%%%%%%%%%%%%%%%%%%%%


\bibliography{paper}
\bibliographystyle{abbrvnat}

% The bibliography should be embedded for final submission.
%\begin{thebibliography}{}
%\softraggedright
%\end{thebibliography}

\end{document}
