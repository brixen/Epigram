\section{A Universe of Inductive Families}
\label{sec:indexing-desc}

\newcommand{\vtup}[2]{\bigRedBracket{\begin{array}{@{}#1@{}}#2\end{array}}}

\begin{wstructure}
!!! Need Help !!!
<- Motivation
    <- Desc: expressivity of simply-typed datatypes: inductive types
        <- Values do not influence types
    /> Example: Vectors
        <- Cannot be defined by just induction
            <- Vectors of all size need to be defined at the *same* time
            -> Defined as a *family* of types
                -> Index
        -> I -> IDesc I: Inductive family
    ???
\end{wstructure}

So far, we have explored the realm of inductive types, building on
intuition from ML-like datatypes, using type dependency as a
descriptive tool in $\Desc$ and its interpretation. Let us now make
dependent types the object as well as the means of our study.

Dependent datatypes provide a way to work at higher level of
precision \emph{a priori}, reducing the sources of failure we
might otherwise need to manage. For the perennial
example, consider \emph{vectors}---lists indexed by length. By
making length explicit in the type, we can prevent hazardous
operations (the type of `head' demands vectors of length
$\NatSuc{\V{n}}$) and offer stronger guarantees (pointwise
addition of $\V{n}$-vectors yields an $\V{n}$-vector).

However, these datatypes are not \emph{individually} inductive. For
instance, we have to define the whole \emph{family} of vectors
mutually, in one go. In dependently typed languages, the basic grammar
of datatypes is that of inductive families. To capture this grammar,
we must account for \emph{indexing}.

%% \subsection{Desc, atomically}
%% \label{sec:idesc-atomic-desc}

%% \begin{wstructure}
%% [Outdated: type former presentation instead]
%% <- Adding hindx have introduced some duplication
%%     <- indx == hindx 1
%%     -> We can factor out commonalities 
%%         /> Obtain an equivalent presentation
%%         /> Still embeddable (refer to the Agda model)
%% \end{wstructure}

%% \begin{wstructure}
%% <- Also replacing '1 by 'const  [figure]
%%     <- For convenience
%%         <- 'const X equivalent to 'sigma X (\_ -> '1)
%%         /> Easier to abstract
%%             <- Extensionally same
%%             /> 'const more useful in practice
%% \end{wstructure}

%% Before moving on to indexed descriptions, we have to carry out some
%% maintenance work on descriptions. We presented $\Desc$ as the grammar
%% of inductive types. Hence, the codes closely follow this grammar. In
%% the following, we adopt an alternative presentation. With
%% $\DSigma{\!}{\!}$, we are actually \emph{quoting} a standard
%% type-former, namely

%% $$\Bhab{\Sigma}{\PI{\V{S}}{\Set} \PI{\V{S}}{\Set} \Set}$$

%% In the alternative presentation, we go further and present all our
%% codes as quotations of standard type-formers. This presentation is
%% shown in Figure~\ref{fig:type-former-desc}. The reader will notice
%% that we replace $\DUnit$ by a more general $\DConst{\!}$ code. Whereas
%% $\DUnit$ was interpreted as the unit set, $\DConst{\V{X}}$ is
%% interpreted as $\V{X}$, for any $\Bhab{\V{X}}{\Set}$. Extensionally,
%% $\DConst{\V{X}}$ and $\DSigma{\V{X}}{\DUnit}$ are equivalent. However,
%% $\DConst{\!}$ is more succinct. More importantly, $\DConst$ is
%% \emph{first-order}, unlike its equivalent encoding. From a
%% definitional perspective, we are giving more opportunities to the
%% type-system, hence reducing the burden on the programmer. For the same
%% reason, we introduce $\DProd{\!}{\!}$ that overlaps with
%% $\DSigma{\!}{\!}$.

%% This reorganisation is strictly equivalent to the previous one
%% (Fig.~\ref{fig:hindx_desc}). Just as the previous version, it is also
%% self-descriptive. We refer the reader to the companion technical
%% report for details. In this finer-grained presentation, we can define
%% $\DIndx{\!}$ and $\DHindx{\!}{\!}$ as follow:

%% \[\begin{array}{l@{\:\mapsto\:\:}l}
%% \DIndx{\V{D}}         & \DProd{\DId}{\V{D}}                      \\
%% \DHindx{\V{H}}{\V{D}}     & \DProd{(\DPi{\V{H}}{(\LAM{\_} \DId)})}{\V{D}}
%% \end{array}
%% \]

%% Consequently, the examples previously developed can be
%% straightforwardly translated into this new presentation. For example,
%% here is the new definition of $\NatD$:

%% \[\stk{
%% \NatD : \Desc \\
%% \NatD \mapsto \DSigma{(\EnumT{[ \NatZero, \NatSuc{\!} ]})}
%%                      {[ \DUnit \quad \DId ]}
%% }\]


%% In the following, we adopt this last version as our de
%% facto universe of inductive types. In particular, we are going to
%% evolve this presentation into an indexed one.

%% \note{Shall we talk about the Type Theory being Desc Zero? or such story?}

%% \begin{figure}

%% \[\stk{
%% \begin{array}{ll}
%% \stk{
%% \data \Desc : \Set \where                                      \\
%% \;\;\begin{array}{@{}l@{\::\:\:}l@{\quad}l}
%%     \DId            & \Desc                                    \\
%%     \DConst{\!}     & \Set \To \Desc                           \\
%%     \DProd{\!}{\!}  & \PI{\V{D}, \V{D'}}{\Desc} \Desc          \\
%%     \DSigma{\!}{\!} & \PI{\V{S}}{\Set} \PIS{\V{S} \To \Desc} \Desc \\
%%     \DPi{\!}{\!}    & \PI{\V{S}}{\Set} \PIS{\V{S} \To \Desc} \Desc 
%% \end{array}
%% }
%% \vspace{0.2in}
%% \\
%% \stk{
%% \descop{\_\:}{} : \Desc \To \Set \To \Set \\
%% \begin{array}{@{}l@{\:=\:\:}ll}
%% \descop{\DId}{\V{X}}          &  \V{X}                                           \\
%% \descop{\DConst{\V{Z}}}{\V{X}}    &  \V{Z}                                           \\
%% \descop{\DProd{\V{D}}{\V{D'}}}{\V{X}} &  \TIMES{\descop{\V{D}}{\V{X}}}{\descop{\V{D}\V{'}}{\V{X}}}       \\
%% \descop{\DSigma{\V{S}}{\V{D}}}{\V{X}} &  \SIGMA{\V{s}}{\V{S}} \descop{\V{D}\: \V{s}}{\V{X}}          \\
%% \descop{\DPi{\V{S}}{\V{D}}}{\V{X}}    &  \PI{\V{s}}{\V{S}} \descop{\V{D}\: \V{s}}{\V{X}}            
%% \end{array}
%% }
%% \end{array}
%% }\]

%% \caption{Universe of descriptions based on Type-formers}
%% \label{fig:type-former-desc}

%% \end{figure}

\subsection{The universe of indexed descriptions}

\begin{wstructure}
<- Labelling Id
    <- We had: data Desc : Set -> Set
    -> We want: data IDesc : (I -> Set) -> Set
        <- Indexed functor (?)
        -> It is sufficient to label Id
            <- Where the functor is built
\end{wstructure}


\newcommand{\DotTo}{\mathop{\blue{\dot{\rightarrow}}}}

We presented the $\Desc$ universe as a
grammar of strictly positive endofunctors on $\Set$ and developed
inductive types by taking a fixpoint. To describe inductive families
indexed by some $\Bhab{\V{I}}{\Set}$, we play a similar game with
endofunctors on the category $\Set^{\V{I}}$,
families of sets \(\V{X},\V{Y}:\V{I}\To\Set\) for objects, and for morphisms,
families of functions in \(\V{X}\DotTo\V{Y}\), defined pointwise:
\[
\V{X}\DotTo\V{Y} \mapsto \PI{\V{i}}{\V{I}}\V{X}\:\V{i}\To\V{Y}\:\V{i}
\]

An \emph{indexed functor} in $\Set^{\V{I}}\To\Set^{\V{J}}$ has the
flavour of a device driver, characterising `responses' to a given
request in \(\V{J}\) where we may in turn make `subrequests' at indices
chosen from \(\V{I}\). When we use indexed functors to define inductive
families of datatypes,
\(\V{I}\) and \(\V{J}\) coincide: we explain how to make a node fit a given
index, including subnodes at chosen indices. E.g., if we are asked for a
vector of length 3, we choose to ask in turn for a tail of length 2.

To code up valid notions of response to a given request, we introduce
$\SYMBIDesc$ and its interpretation:
%
\[\stk{
\IDesc{(\Bhab{\V{I}}{\Set})} : \Set \smallskip \\
\idescop{\_}{}{} : _{\PI{\V{I}}{\Set}} \IDesc{\V{I}} \To (\V{I} \To \Set) \To \Set    \\
}\]

An \(\IDesc{\V{I}}\) specifies just \emph{one} response, but a
request-to-response \emph{function},
$\V{R}:\V{I} \To \IDesc{\V{I}}$, yields a strictly positive endofunctor
\[
  \LAM{\V{X}} \LAM{\V{i}} \idescop{\V{R}\:\V{i}}{\V{I}}{\V{X}} :
  \Set^{\V{I}} \To \Set^{\V{I}}
\]
whose fixpoint we then take:
%
\[\stkl{
\Rule{\Gamma \vdash \Bhab{\V{I}}{\Set} \qquad
      \Gamma \vdash \Bhab{\V{R}}{\V{I} \To \IDesc{\V{I}}}}
     {\Gamma \vdash \Bhab{\SYMBIMu_{\V{I}}{\V{R}}}{\V{I}\To\Set}} \qquad
\\
\Rule{\begin{array}{l@{\qquad}l}
          \Gamma \vdash \Bhab{\V{I}}{\Set} &
          \Gamma \vdash \Bhab{\V{R}}{\V{I} \To \IDesc{\V{I}}} \\
          \Gamma \vdash \Bhab{\V{i}}{\V{I}} &
          \Gamma \vdash \Bhab{\V{x}}{\idescop{\V{R}\:\V{i}}{\V{I}}{(\SYMBIMu_{\V{I}}{\V{R}})}}
      \end{array}}
     {\Gamma \vdash \Bhab{\Con{\V{x}}}{\IMu{\V{I}}{\V{R}}{\V{i}}}}
}\]

\newcommand{\upgrade}{\F{upgrade}}
\newcommand{\inductionI}{\F{indI}}
\newcommand{\cataI}{\F{cataI}}

We define the $\SYMBIDesc$ grammar in Figure~\ref{fig:idesc},
delivering only \emph{strictly positive} families. As well as
indexing our descriptions, we have refactored a little, adopting
a more compositional algebra of codes, where $\Desc$ is
biased towards the right-nested tuples. We now have
\(\DVar{i}\) for recursive `subrequests' at a chosen index \(i\),
with tupling by right-associative
\(\DProd{}{}\) and higher-order branching
by \(\DPi{\!}{\!}\).  Upgrade your old $\Desc$
to a trivially indexed $\IDesc{\Unit}$ as follows!
\[\begin{array}{@{}ll}
\upgrade :\Desc & \To\IDesc{\Unit} \\
\upgrade\:\DUnit & \mapsto \DConst{\Unit} \\
\upgrade\:(\DSigma{\V{S}}{\V{D}}) &
   \mapsto \DSigma{\V{S}}{\LAM{\V{s}}\upgrade\:(\V{D}\:\V{s})} \\
\upgrade\:(\DIndx{\V{D}}) & \mapsto
  \DProd{\DVar{\Void}}{\upgrade\:\V{D}} \\
\upgrade\:(\DHindx{\V{H}}{\V{D}}) & \mapsto
  \DProd{(\DPi{\V{H}}{\LAM{\_}\DVar{\Void}})}{\upgrade\:\V{D}} \\
\end{array}\]

To deliver induction for indexed datatypes, we need the `holds everywhere'
machinery. We present $\SYMBAllI$ and $\SYMBallI$ in
Figure~\ref{fig:allI-predicates}, with a twist---where
$\Desc$ admits the $\SYMBall$ construction, $\SYMBIDesc$ is \emph{closed}
under it! The $\SYMBAllI$
operator for a description indexed on \(\V{I}\) is strictly positive in
turn, and has a description indexed on
 some \(\SIGMA{\V{i}}{\V{I}}{\V{X}\: \V{i}}\).
Induction on indexed descriptions is then hardwired thus:
%
\[\stk{
\begin{array}{@{}ll}
\inductionI : & _{\PI{\V{I}}{\Set}}
                   \PITEL{\V{R}}{\V{I} \To \IDesc{\V{I}}}
                   \PI{\V{P}}{(\SIGMA{\V{i}}{\V{I}}{\IMu{\V{I}}{\V{R}}{\V{i}}}) \To \Set} \\
                 & (      \PITEL{\V{i}}{\V{I}} 
                          \PI{\V{xs}}{\idescop{\V{R}\: \V{i}}{\V{I}}{(\SYMBIMu_{\V{I}}{\V{R}})}} \\
                 & \   \idescop{\AllI{}
                                     {(\V{R}\: \V{i})}
                                     {(\SYMBIMu_{\V{I}}{\V{R}})}
                                     {\V{xs}}}
                               {}
                               {\V{P}} \To
                       \V{P}\: \pair{\V{i}}{\Con{\V{xs}}}{}) \To \\
                 & \PITEL{\V{i}}{\V{I}}
                   \PI{\V{x}}{\IMu{\V{I}}{\V{R}}{\V{i}}}
                   \V{P}\: \pair{\V{i}}{\V{x}}{}
\end{array} \\
\inductionI\: \V{R}\: \V{P}\: \V{m}\: \V{i}\: (\Con{\V{xs}}) \mapsto 
    \V{m}\: \V{i}\: \V{xs}\: (\allI{}
                                    {\V{R}\: \V{i}}
                                    {(\SYMBIMu_{\V{I}}{\V{R}})}
                                    {\V{P}}
                                    {(\spl{\LAM{\V{i}}\LAM{\V{xs}} \inductionI\: \V{R}\: \V{P}\: \V{m}})}
                                    {\V{xs}})
}\]
%
The generic catamorphism, $\cataI$, is constructed from $\inductionI$
as before. Its type becomes more elaborated, to deal with the
indexing:
%
\[
\begin{array}{@{}l@{}l}
\cataI :& \PITEL{\V{I}}{\Set}
          \PITEL{\V{R}}{\V{I} \To \IDesc{\V{I}}} \\
        & \PI{\V{T}}{\V{I} \To \Set}
          (\PI{\V{i}}{\V{I}}{\idescop{\V{R}\: \V{i}}{}{\V{T}} \To \V{T}\: \V{i}}) \To
          \SYMBIMu_{\V{I}}{\V{R}} \DotTo \V{T}
\end{array}
\]


\begin{figure*}

\[
\begin{array}{ll}
%%
\stk{
\begin{array}{@{}ll}
\SYMBAllI : & _{\PI{\V{I}}{\Set}}
              \PITEL{\V{D}}{\IDesc{\V{I}}}
              \PI{\V{X}}{\V{I} \To \Set} \\
            & \idescop{\V{D}}{\V{I}}{\V{X}} \To
              \IDesc{(\SIGMA{\V{i}}{\V{I}}{\V{X}\: \V{i}})}
\end{array} \\
\begin{array}{@{}l@{}l@{\:\mapsto\:\:}l}
\AllI{\:}{(\DVar{\V{i}})}{& \V{X}}{\V{x}} &
    \DVar{\pair{\V{i}}{\V{x}}{}} \\
\AllI{\:}{(\DConst{\V{K}})}{& \V{X}}{\V{k}} &
    \DConst{\Unit} \\
\AllI{\:}{(\DProd{\V{D}}{\V{D'}})}{& \V{X}}{\pair{\V{d}}{\V{d'}}{}} &
    \DProd{\AllI{}{\V{D}}{\V{X}}{\V{d}}}{\AllI{}{\V{D'}}{\V{X}}{\V{d'}}} \\
\AllI{\:}{(\DSigma{\V{S}}{\V{D}})}{& \V{X}}{\pair{\V{s}}{\V{d}}{}} &
    \AllI{}{(\V{D}\: \V{s})}{\V{X}}{\V{d}} \\
\AllI{\:}{(\DPi{\V{S}}{\V{D}})}{& \V{X}}{\V{f}} &
    \DPi{\V{S}}{\LAM{\V{s}} \AllI{}{(\V{D}\: \V{s})}{\V{X}}{(\V{f}\: \V{s})}}
\end{array}
}
&
%%
\stk{
\begin{array}{@{}ll}
\SYMBallI : & _{\PI{\V{I}}{\Set}}
              \PITEL{\V{D}}{\IDesc{\V{I}}}
              \PITEL{\V{X}}{\V{I} \To \Set} 
              \PI{\V{P}}{(\SIGMA{\V{i}}{\V{I}}{\V{X}\: \V{i}}) \To \Set} \\
            & (\PI{\V{x}}{\SIGMA{\V{i}}{\V{I}}{\V{X}\: \V{i}}} \V{P}\: \V{x}) \To
              \PI{\V{xs}}{\idescop{\V{D}}{\V{I}}{\V{X}}} 
              \idescop{\AllI{}{\V{D}}{\V{X}}{\V{xs}}}{}{\V{P}}
\end{array} \\
\begin{array}{@{}l@{}l@{\:\mapsto\:\:}l}
\allI{\:}{(\DVar{\V{i}})}{& \V{X}}{\V{P}}{\V{p}}{\V{x}} &
    \V{p}\: \pair{\V{i}}{\V{x}}{} \\
\allI{\:}{(\DConst{\V{K}})}{& \V{X}}{\V{P}}{\V{p}}{\V{k}} &
    \void \\
\allI{\:}{(\DProd{\V{D}}{\V{D'}})}{& \V{X}}{\V{P}}{\V{p}}{\pair{\V{d}}{\V{d'}}{}} &
    \pair{\allI{}{\V{D}}{\V{X}}{\V{P}}{\V{p}}{\V{d}}}
         {\allI{}{\V{D'}}{\V{X}}{\V{P}}{\V{p}}{\V{d'}}}{} \\
\allI{\:}{(\DSigma{\V{S}}{\V{D}})}{& \V{X}}{\V{P}}{\V{p}}{\pair{\V{s}}{\V{d}}{}} &
    \allI{}{(\V{D}\: \V{s})}{\V{X}}{\V{P}}{\V{p}}{\V{d}} \\
\allI{\:}{(\DPi{\V{S}}{\V{D}})}{& \V{X}}{\V{P}}{\V{p}}{\V{f}} &
    \LAM{\V{a}}\allI{}{(\V{D}\: \V{a})}{\V{X}}{\V{P}}{\V{p}}{(\V{f}\: \V{a})}
\end{array}
\end{array}
}
\]

\caption{Indexed induction predicates}
\label{fig:allI-predicates}

\end{figure*}


\begin{figure}

\[\stk{\begin{array}{@{}ll}
\IDesc{(\Bhab{\V{I}}{\Set})} &: \Set \\
\DVar{(\Bhab{\V{i}}{\V{I}})} &: \IDesc{\V{I}} \\
\DConst{(\Bhab{\V{A}}{\Set})} &:\IDesc{\V{I}}       \\
\DProd{(\Bhab{\V{D}}{\IDesc{\V{I}}})}{(\Bhab{\V{D}}{\IDesc{\V{I}}})}
  & :\IDesc{\V{I}}       \\
\DSigma{(\Bhab{\V{S}}{\Set})}{(\Bhab{\V{D}}{\V{S}\To\IDesc{\V{I}}})}
& : \IDesc{\V{I}}  \\
\DPi{(\Bhab{\V{S}}{\Set})}{(\Bhab{\V{D}}{\V{S}\To\IDesc{\V{I}}})}
& : \IDesc{\V{I}}  \\
\end{array}\smallskip \\
\idescop{\_\:}{}{} :_{\PI{\V{I}}{\Set}} \IDesc{\V{I}} \To (\V{\V{I}} \To \Set) \To \Set                  \\
\begin{array}{@{}l@{\V{X}}@{\:\mapsto\:\:}ll}
\idescop{\DVar{\V{i}}}{\V{I}}{&}      &  \V{X}\: \V{i}                                           \\
\idescop{\DConst{\V{K}}}{\V{I}}{&}    &  \V{K}                                                   \\
\idescop{\DProd{\V{D}}{\V{D'}}}{\V{I}}{&} &  \TIMES{\idescop{\V{D}}{\V{I}}{\V{X}}}{\idescop{\V{D'}}{\V{I}}{\V{X}}}       \\
\idescop{\DSigma{\V{S}}{\V{D}}}{\V{I}}{&} &  \SIGMA{\V{s}}{\V{S}} \idescop{\V{D}\: \V{s}}{\V{I}}{\V{X}}                    \\
\idescop{\DPi{\V{S}}{\V{D}}}{\V{I}}{&}    &  \PI{\V{s}}{\V{S}} \idescop{\V{D}\: \V{s}}{\V{I}}{\V{X}}            
\end{array}
}
\]

\caption{Universe of indexed descriptions}
\label{fig:idesc}

\end{figure}



\subsection{Examples}
\label{sec:idesc-examples}

\paragraph{Natural numbers:}

\begin{wstructure}
<- Nat
    -> [equation]
    <- Non-indexed types lives in IDesc 1
        -> This applies to all previous examples
\end{wstructure}

For basic reassurance, we \(\upgrade\:\NatD\):
%
\[\stk{
\upgrade\:\NatD : \IDesc{\Unit} \\
\upgrade\:\NatD \mapsto \DSigma{(\EnumT{\sqr{\NatZero\: \SYMBNatSuc}})}
                     {\sqr{(\DConst{\Unit}) \; 
                           (\DProd{\DVar{\Void}}{\DConst{\Unit}})}}
}\]
%
Note that trailing \(\Unit\)'s keep our right-nested, \(\void\)-terminated
tuple structure, and with it our elaboration machinery.
We can similarly \(\upgrade\) any inductive type.
Moreover, \(\IDesc{I}\) can now code a bunch of mutually
inductive types, if \(I\) enumerates the
bunch~\cite{paulin:habilitation, yakushev:mutual-def}.



\paragraph{Indexed descriptions:}

\begin{wstructure}
<- Levitation [figure]
    <- Following Desc encoding
        /> Note: simple datatype
            -> Live in IDesc 1
    -> Behind the scene, relies on the special purpose switchD
\end{wstructure}

Note that $\IDesc{\V{I}}$ is a plain inductive type, parametrized
by \(\V{I}\), but indexed trivially.
%
\[\stk{
\IDescD : \PI{\V{I}}{\Set} \IDesc{\Unit} \\
\IDescD\: \V{I} \mapsto \SYMBDSigma \\
\quad
 \EnumT\vtup{r}{\SYMBDVar\\
                \SYMBDConst\\
                \DProd{}{}\\
                \SYMBDSigma\\
                \SYMBDPi}
         \; \vtup{l@{}l}{
  (\DProd{\DConst{\V{I}} &}{\DConst{\Unit}})                  \\
  (\DProd{\DConst{\Set}  &}{\DConst{\Unit}})                  \\
  (\DProd{\DVar{\Void}}{\DProd{\DVar{\Void}&}{\DConst{\Unit}}})  \\
  (\DSigma{\Set}{\LAM{\V{S}}
     \DProd{( \DPi{\V{S}}{\LAM{\_} \DVar{\Void}}) &}{\DConst{\Unit}}})     \\
  (\DSigma{\Set}{\LAM{\V{S}}
     \DProd{( \DPi{\V{S}}{\LAM{\_} \DVar{\Void}}) &}{\DConst{\Unit}}})     \\
                                   }
}\]

Therefore, this universe is self-describing and can be
levitated. As before, we rely on a special purpose $\F{switchID}$
operator to build the finite function $\bigRedBracket{\ldots}$
without mentioning \(\SYMBIDesc\).

\paragraph{Vectors:}

\newcommand{\VecD}{\F{VecD}}
\newcommand{\VecNil}{\etag{\CN{vnil}}}
\newcommand{\SYMBVecCons}{\etag{\CN{vcons}}\xspace}
\newcommand{\VecCons}[2]{\SYMBVecCons\:#1\:#2}

So far, our examples live in $\IDesc{\Unit}$, with no interesting
indexing. Let us at least have vectors. Recall
that the constructors $\VecNil$ and $\SYMBVecCons$ are defined only for
$\NatZero$ and $\NatSuc$ respectively:
%
\[
\stk{
\data \D{Vec}\: \PITEL{\V{X}}{\Set} : \PI{\V{i}}{\Nat} \Set \where \\
\;\;\begin{array}{@{}l@{\::\:\:}l@{\quad}l}
    \VecNil          & \D{Vec}\:\V{X}\:{\NatZero}   \\
    \SYMBVecCons & _{\PI{\V{n}}{\Nat}}\V{X} \To \D{Vec}\:{\V{X}}\:{\V{n}} \To \D{Vec}\:{\V{X}}\:{(\NatSuc{\V{n}})}
\end{array}
}
\]

One way to code constrained datatypes is to appeal to a suitable
notion of propositional equality \(\PropEq\) on indices. The
constraints are expressed as `Henry Ford' equations in the datatype.
For vectors:
%
\[\stk{
\VecD : \Set \To \Nat \To \IDesc{\Nat} \\
\VecD\: \V{X}\: \V{i} \mapsto \SYMBDSigma\\
\quad
\EnumT{\vtup{r}{\VecNil\\ \SYMBVecCons}}
\; \vtup{r}{
                            (\DConst{(\NatZero\PropEq\V{i})}) \\
 ( \DSigma{\Nat}{\LAM{\V{n}}
   \DProd{\DConst{\V{X}}}
     {\DProd{\DVar{\V{n}}}{\DConst{(\NatSuc{\V{n}}\PropEq\V{i})}})}}
                          }
}\]

You may choose $\VecNil$ for any index you like as long as it is
$\NatZero$; in the $\SYMBVecCons$ case, the length of the tail is
given explicitly, and the index $\V{i}$ must be one more. Our previous
\(\Unit\)-terminated tuple types can now be seen as the trivial case
of constraint-terminated tuple types, with elaboration supplying the
witnesses when trivial.

In this paper, we remain anxiously agnostic about
propositional equality. Any will do, according to
conviction; many variations are popular. The
homogeneous identity type used in Coq is ill-suited to
dependent types, but its heterogeneous variant (forming equations
regardless of type) allows the translation of pattern
matching with structural recursion to
\(\F{indI}\)~\cite{goguen:pattern-matching}. The
extensional equality of \citet{altenkirch:ott} also sustains the translation.

\begin{wstructure}
!!! Need Help !!!
<- Brady optimisation: forcing
    <- Source to source translation
    <- Able to remove some constraints
    -> Example: Fin [figure]
    ??? More technical detail needed
\end{wstructure}

However, sometimes, the equations are redundant. 
Looking back at $\D{Vec}$, we find that the equations constrain
the choice of constructor and stored tail index retrospectively.
But \emph{inductive families need not store their
  indices}~\cite{brady:index-inductive-families}!  If we
analyse the incoming index, we can tidy our description of $\D{Vec}$
as follows:
%
\[\stk{
\VecD \:\PITEL{\V{X}}{\Set} : \Nat \To \IDesc{\Nat} \\
\begin{array}{@{}lll}
\VecD\:\V{X}\: \NatZero     & \mapsto & \DConst{\Unit} \\
\VecD\:\V{X}\: (\NatSuc{\V{n}}) & \mapsto &
 \DProd{\DConst{\V{X}}}{\DVar{\V{n}}}
\end{array}
                                       
}\]
%
The constructors and equations have simply disappeared. A similar
example is $\SYMBFin$ (bounded numbers), specified by:
%
\[
\stk{
\data \SYMBFin : \PI{\V{n}}{\Nat} \Set \where \\
\;\;\begin{array}{@{}l@{\::\:\:}l@{\quad}l}
    \FinZero      & _{\PI{\V{n}}{\Nat}}\Fin{(\NatSuc{\V{n}})}   \\
    \SYMBFinSuc   & _{\PI{\V{n}}{\Nat}}\Fin{\V{n}} \To \Fin{(\NatSuc{\V{n}})}
\end{array}
}\]
%
In this case, we can eliminate equations but not constructors, since both
$\FinZero$ and $\SYMBFinSuc$ both target $\SYMBNatSuc$:
%
\[\stk{
\FinD : \Nat \To \IDesc{\Nat} \\
\begin{array}{@{}lll}
\FinD\: \NatZero         & \mapsto & \DSigma{\EnumT{\Void}}{\Void} \\
\FinD\: (\NatSuc{\V{n}}) & \mapsto & \DSigma{\EnumT{\sqr{\FinZero\: \SYMBFinSuc}}}
                                            {\sqr{(\DConst{\Unit})\: (\DVar{\V{n}})}}
\end{array}
}\]

This technique of extracting information by case analysis on indices
applies to descriptions exactly where Brady's `forcing' and
`detagging' optimisations apply in compilation. They eliminate just
those constructors, indices and constraints which are redundant even
in \emph{open} computation. In \emph{closed} computation, where proofs
can be trusted, all constraints are dropped.


\paragraph{Tagged indexed descriptions:}

\newcommand{\SYMBmuide}{\D{\({\mu}^{\!+}\)}\xspace}
\newcommand{\muide}[2]{\SYMBmuide\!\!_{#1}\:#2}

Let us reflect this index analysis technique.
We can divide a description of tagged indexed data in two: first, the
constructors that do not depend on the index; then, the constructors
that do. The non-dependent part mirrors the definition for non-indexed
descriptions. The index-dependent part simply indexes the choice of
constructors by $\V{I}$. Hence, by inspecting the index, it is
possible to vary the `menu' of constructors.
%
\[
\begin{array}{@{}l@{\:\mapsto\:\:}l}
 \TagIDesc{\V{I}}  & \TIMES{\ATagIDesc{\V{I}}}{\ITagIDesc{\V{I}}} \\
 \ATagIDesc{\V{I}} & \SIGMA{\V{E}}{\EnumU} \PI{\V{i}}{\V{I}} \spi{\V{E}}{\LAM{\_} \IDesc{\V{I}}} \\
 \ITagIDesc{\V{I}} & 
     \SIGMA{\V{F}}{\V{I} \To \EnumU} \PI{\V{i}}{\V{I}} \spi{(\V{F}\: \V{i})}{\LAM{\_} \IDesc{\V{I}}} 
\end{array}
\]

\begin{wstructure}
<- Vectors
    Do we treat them in the end? 
    What can we say here we haven't with Fin?
\end{wstructure}

In the case of a tagged $\D{Vec}$, for instance, for the index
$\NatZero$, we would only propose the constructor
$\ListNil$. Similarly, for $\NatSuc{n}$, we would only propose the
constructor $\SYMBListCons$.

We write $\toIDesc{\V{D}}\:\V{i}$ to denote the $\IDesc{\V{I}}$
computed from the tagged indexed description $\V{D}$ at index
$\V{i}$. Its expansion is similar to the definition of \(\SYMBtoDesc\)
for tagged descriptions, except that it must also append the two parts.
We again write $\muide{\V{I}}{\V{D}}$ for
$\IMu{\V{I}}{(\toIDesc{\V{D}})}$.

\paragraph{Typed expressions:}

\begin{wstructure}
<- Hutton's razor
    <- Types
        <- 'Nat
        <- 'Bool
    <- Term [figure]
        <- val : Val 'a -> 'a  for Val : Ty -> Set, mapping to Nat and Bool
        <- cond : 'Bool -> a -> a -> a
        <- plus : 'Nat -> 'Nat -> 'Nat
        <- le : 'Nat -> 'Nat -> 'Bool
\end{wstructure}

%% Types
\newcommand{\Ty}{\D{Ty}}
\newcommand{\Ebool}{\etag{\CN{bool}}}
\newcommand{\Enat}{\etag{\CN{nat}}}

%% Constructors
\newcommand{\SYMBEval}{\etag{\CN{val}}\xspace}
\newcommand{\SYMBEvar}{\etag{\CN{var}}\xspace}
\newcommand{\Eval}[1]{\SYMBEval\:#1}
\newcommand{\SYMBEcond}{\etag{\CN{cond}}\xspace}
\newcommand{\Econd}[3]{\SYMBEcond\:#1\:#2\:#3}
\newcommand{\SYMBEplus}{\etag{\CN{plus}}\xspace}
\newcommand{\Eplus}[2]{\SYMBEplus\:#1\:#2}
\newcommand{\SYMBEle}{\etag{\CN{le}}\xspace}
\newcommand{\Ele}[2]{\SYMBEle\:#1\:#2}

%% Index mapper (terminology?)
\newcommand{\SYMBVal}{\F{Val}\xspace}
\newcommand{\Val}[1]{\SYMBVal\:#1}
\newcommand{\SYMBVar}{\F{Var}\xspace}
\newcommand{\Var}[2]{\SYMBVar\: #1\: #2}

%% Hutton expressions
\newcommand{\HExprD}{\F{ExprD}}
\newcommand{\HExprAD}{\F{ExprAD}}
\newcommand{\HExprID}{\F{ExprID}}
\newcommand{\HExprVarD}[1]{\C{ExprD}_{\F{Var},#1}}
\newcommand{\HExprFreeD}{\C{ExprD}^{\C{Free}}}
\newcommand{\HExprAFreeD}{\C{ExprAD}^{\C{Free}}}

We are going to define a syntax for a small language with
two types, natural numbers and booleans:
%
\[
\Ty \mapsto \EnumT{\sqr{\Enat\: \Ebool}}
\]

\newcommand{\plusHost}{\mathop{\green{+_{\mathrm{H}}}}}
\newcommand{\leHost}{\mathop{\green{\leq_{\mathrm{H}}}}}


This language has values, conditional expression, addition and
comparison. Informally, their types are:
%
\[
\begin{array}{l@{\::\:\:}l}
\SYMBEval            & \Val{\V{ty}} \To \V{ty} \\
\SYMBEcond           & \Ebool \To \V{ty} \To \V{ty} \To \V{ty}  \\ 
\end{array}
\qquad
\begin{array}{l@{\::\:\:}l}
\SYMBEplus           & \Enat \To \Enat \To \Enat                           \\
\SYMBEle             & \Enat \To \Enat \To \Ebool                          \\
\end{array}
\]
%
The function $\SYMBVal$ interprets object language types in the
host language, so that arguments to $\SYMBEval$ fit their
expected type.
%
\[\stk{
\SYMBVal : \Ty \To \Set \\
\begin{array}{@{}l@{\:\mapsto\:\:}l}
\Val{\Enat}   & \Nat \\
\Val{\Ebool}  & \Bool
\end{array}
}\]
%
We take $\Nat$ and $\Bool$ to represent natural numbers and Booleans
in the host language, equipped with addition $\plusHost$ and
comparison $\leHost$.

We express our syntax as a tagged indexed description, indexing over
object language types $\Ty$. We note that some constructors are always
available, namely $\SYMBEval$ and $\SYMBEcond$. On the other hand,
$\SYMBEplus$ and $\SYMBEle$ constructors are index-dependent, with
$\SYMBEplus$ available just when building a $\Enat$, $\SYMBEle$ just
for $\Ebool$. The code, below, reflects this intuition, with the first
component uniformly offering $\SYMBEval$ and $\SYMBEcond$, the second
selectively offering $\SYMBEplus$ or $\SYMBEle$.
%
%%% \begin{figure}
%
\[\stk{
\stk{
\HExprD : \TagIDesc{\Ty} \\
\HExprD \mapsto \sqr{ \HExprAD , \HExprID } \\
} \smallskip\\
\stk{
\HExprAD : \ATagIDesc{\Ty} \\
\HExprAD \mapsto \vtup{l}{
   {\vtup{r}{\SYMBEval\\ \SYMBEcond \,}} \red{,} \;
      \LAM{\V{ty}}
      \vtup{l@{\:}l}{
      \DProd{\DConst{(\Val{\V{ty}})}&}{\DConst{\Unit}} \\
      \DProd{\DProd{\DVar{\Ebool}}{\DProd{\DVar{\V{ty}}}{\DVar{\V{ty}}}}&}
        {\DConst{\Unit}} \\
     }
   }
\smallskip\\
} 
\\
\stk{
\HExprID : \ITagIDesc{\Ty} \\
\HExprID \mapsto \vtup{l}{
                   \vtup{r}{\sqr{\SYMBEplus} \\ \sqr{\SYMBEle}} \red{,} \;
  \LAM{\_} \sqr{\DProd{\DProd{\DVar{\Enat}}{\DVar{\Enat}}}{\DConst{\Unit}}}
                   }
}
}\]

%%%\caption{Syntax of typed expressions}
%%%\label{fig:hexpr-full}

%%%\end{figure}

\newcommand{\evalH}{\F{eval}_{\green{\Downarrow}}}
\newcommand{\evalOne}{\F{eval}_{\green{\downarrow}}}

Given the syntax, let us supply the semantics. We implement an
evaluator as a catamorphism:
%
\[\stk{
\evalH : \PI{\V{ty}}{\Ty} 
         \muide{\Ty}{\HExprD}\: \V{ty} \To
         \Val{\V{ty}} \\
\evalH\: \V{ty}\: \V{term} \mapsto \cataI_{\Ty} \:
                                 (\toIDesc{\HExprD})\: 
                                 \SYMBVal\: 
                                 \evalOne\: 
                                 \V{ty}\: 
                                 \V{term}
}\]
%
To finish the job, we must supply the algebra which implements a single
step of evaluation, given subexpressions evaluated already.
%
\[\stk{
\evalOne : \PI{\V{ty}}{\Ty}
 \idescop{(\toIDesc{\HExprD})\:\V{ty}}{\Ty}{\SYMBVal}
           \To {\Val{\V{ty}}} \\
\begin{array}{@{}l@{}c@{}l@{\:\mapsto\:\:}l}
\evalOne\: & \_\: & (\SYMBEval\;\V{x})                                            & \V{x} \\
\evalOne\: & \_\: & (\SYMBEcond\:\BoolTrue\:\V{x}\:\_)   & \V{x} \\
\evalOne\: & \_\: & (\SYMBEcond\:\BoolFalse\:\_\:\V{y})  & \V{y} \\
\evalOne\: & \Enat\: & (\SYMBEplus\:\V{x}\:\V{y})   & \V{x} \plusHost \V{y} \\
\evalOne\: & \Ebool\: & (\SYMBEle\:\V{x}\:\V{y})  & \V{x} \leHost \V{y} 
\end{array}
}\]

\begin{wstructure}
    /> Closed term
        <- only constants and operations on them
        -> Extend Val with Var : Ty -> Set, mapping to EnumU
            -> Open term
            -> Language of well-typed terms
                <- By construction
\end{wstructure}

Hence, we have a type-safe syntax and a tagless interpreter for our
language, in the spirit
of~\citet{augustsson.carlsson:dependent.interpreter}, with help from
the generic catamorphism. However, so far, we are only able to define
and manipulate \emph{closed} terms. Adding variables, it is possible
to build and manipulate \emph{open} terms, that is, terms in a
context. We shall get this representation, for free, thanks to the
\emph{free indexed monad} construction.


\subsection{Free indexed monad}

\begin{wstructure}
<- Variation on a theme: free imonad construction
    <- Recall existence of generic free monad construction
    -> Present its generalisation to IDesc [equation]
        <- \I -> IDesc I as describing an indexed endofunctor
        <- Free monad construction
    -> Still a suitable, generic notion of substitution
        <- show type signature
        <- show implementation?? (space! space!)
\end{wstructure}

In Section~\ref{sec:desc-free-monad}, we have built a free monad
operation for simple descriptions. The process is similar in the
indexed world. Namely, given an indexed functor, we derive the indexed
functor coding its free monad: \note{pwm: Whoa there. Maybe we should
  say something about IMonads in general before we get to this point?}
%
\[\stk{
\begin{array}{ll}
\FreeIMonad{\_}{} : & _{\PI{\V{I}}{\Set}}
                     \PITEL{\V{R}}{\TagIDesc{\V{I}}} 
                     \PITEL{\V{X}}{\V{I} \To \Set}\To 
                      \TagIDesc{\V{I}}
\end{array} \\
\FreeIMonad{\pair{\V{E}}{\V{F}}{}}{\V{I}}{\V{R}} \mapsto
    \pair{\pair{\ListCons{\SYMBDVar}{(\fst{\V{E}})}} 
               {\LAM{\V{i}}
                \pair{\DConst{(\V{R}\: \V{i})}}
                     {(\snd{\V{E}})\: \V{i}}{}}{}}
         {\V{F}}{}
}\]


\newcommand{\substI}{\F{substI}}


Just as in the universe of descriptions, this construction comes with
an obvious \return and a substitution operation, the \bind. Its
definition is the following:
%
\[\stk{
\begin{array}{@{}ll}
\substI : & _{\PI{\V{I}}{\Set}}
            \PI{\V{X}, \V{Y}}{\V{I} \To \Set}
            \PITEL{\V{R}}{\TagIDesc{\V{I}}} \\
          & (\V{X} \DotTo 
             \muide{\V{I}}{(\FreeIMonad{\V{R}}{\V{I}}{\V{Y}})}) \To 
             \muide{\V{I}}{(\FreeIMonad{\V{R}}{\V{I}}{\V{X}})} \DotTo
             \muide{\V{I}}{(\FreeIMonad{\V{R}}{\V{I}}{\V{Y}})}
\end{array} \\
\substI\: \V{X}\: \V{Y}\: \V{R}\: \V{\sigma}\: \V{i}\: \V{t} \mapsto \\
\qquad    \cataI_{\V{I}}\: (\toIDesc{\FreeIMonad{\V{R}}{}{\V{X}}})\:
                      (\muide{\V{Y}}{(\FreeIMonad{\V{R}}{}{\V{Y}})})\:
                      (\F{applyI}\: \V{R}\: \V{X}\: \V{Y}\: \V{\sigma})\:
                      \V{i}\:
                      \V{t} 
}\]
% 
Where  $\F{applyI}$ is defined as follow:
%
\[\stk{
\begin{array}{@{}ll}
\F{applyI} : & _{\PI{\V{I}}{\Set}}
            \PITEL{\V{R}}{\TagIDesc{\V{I}}}
            \PI{\V{X}, \V{Y}}{\V{I} \To \Set} \\
          & (\V{X} \DotTo \muide{\V{I}}{(\FreeIMonad{\V{R}}{\V{I}}{\V{Y}})}{}) \To \\
          & \idescop{\toIDesc{\FreeIMonad{\V{R}}
                                         {\V{I}}
                                         {\V{X}}}}
                    {\V{I}}
                    {\muide{\V{I}}{(\FreeIMonad{\V{R}}{\V{I}}{\V{Y}})}} \DotTo 
            \muide{\V{I}}{(\FreeIMonad{\V{R}}{\V{I}}{\V{Y}})}{}
\end{array} \\
\begin{array}{@{}l@{\:\mapsto\:\:}l}
\F{applyI}\: \V{R}\: \V{X}\: \V{Y}\: \V{\sigma}\: \V{i}\: \pair{\SYMBDVar}{\V{x}}{}   & \V{\sigma}\: \V{i}\: \V{x}                   \\
\F{applyI}\: \V{R}\: \V{X}\: \V{Y}\: \V{\sigma}\: \V{i}\: \pair{\V{c}}{\V{ys}}{} & \Con{\pair{\V{c}}{\V{ys}}{}}
\end{array}
}\]
 
The subscripted types corresponds to implicit arguments that can be
automatically inferred, hence do not have to be typed in. Let us now
consider two examples of free indexed monads.


\paragraph{Typed expressions:}

\begin{wstructure}
    /> Closed term
        <- only constants and operations on them
        -> Extend Val with Var : Ty -> Set, mapping to EnumU
            -> Open term
            -> Language of well-typed terms
                <- By construction
\end{wstructure}

\newcommand{\Ctxt}{\D{Context}}
\newcommand{\SYMBCtxtEmpty}{\C{[]}\xspace}
\newcommand{\CtxtEmpty}{\SYMBCtxtEmpty}
\newcommand{\SYMBCtxtSnoc}{\C{snoc}\xspace}
\newcommand{\CtxtSnoc}[2]{\SYMBCtxtSnoc\:#1\:#2}
\newcommand{\SYMBEnv}{\F{Env}}
\newcommand{\Env}[1]{\SYMBEnv\: #1}
\newcommand{\SYMBlookup}{\F{lookup}}
\newcommand{\lookup}[4]{\SYMBlookup\: #1\: #2\: #3\: #4}

In the previous section, we have presented a language of closed
arithmetic expressions. Using the free monad construction, we are
going to extend this construction to open terms. An open term is
defined with respect to a context, represented by a snoc-list of
types:
%
\[
\begin{array}{@{}l@{\::\:\:}l@{\quad}l}
\Ctxt           & \Set \\
\SYMBCtxtEmpty  & \Ctxt \\
\SYMBCtxtSnoc   & \Ctxt \To \Ty \To \Ctxt
\end{array}
\]
%
An environment realises the context, associating a value for each type:
%
\[
\stk{
\SYMBEnv : \Ctxt \To \Set \\
\begin{array}{@{}l@{\:\:\mapsto\:\:}l}
\Env{\CtxtEmpty}                & \Unit \\
\Env{(\CtxtSnoc{\V{G}}{\V{S}})} & \TIMES{\Env{\V{G}}}{\Val{\V{S}}}
\end{array}
}\]
%
In this setting, we define $\SYMBVar$ by:
%
\[\stk{
\Var{}{} : \Ctxt \To \Ty \To \Set \\
\begin{array}{@{}ll@{\:\:\mapsto\:\:}l}
\Var{\CtxtEmpty}{& \V{T}}                & 
    \Void \\
\Var{(\CtxtSnoc{\V{G}}{\V{S}})}{& \V{T}} & 
    \SUM{(\Var{\V{G}}{\V{T}})}{(\V{S} \PropEq \V{T})}
\end{array}
}\]
%
While $\SYMBVal$ maps the type to the corresponding host type,
$\SYMBVar$ indexes a value in the context, obtaining a proof that the
types match. The $\SYMBlookup$ function precisely follow this
semantics:
%
\[\stk{
\SYMBlookup : \PI{\V{G}}{\Ctxt} 
          \Env{\V{G}} \To 
          \PI{\V{T}}{\Ty} 
          \Var{\V{G}}{\V{T}} \To
          \Val{\V{T}} \\
\begin{array}{@{}l@{}l@{}l@{}l@{}lll}
\lookup{& (\CtxtSnoc{\V{G}}{.T})}{& \pair{\V{g}}{\V{t}}{}}{& \V{T}}{& (\SumRight{\C{refl}})} & \mapsto & \V{t} \\
\lookup{& (\CtxtSnoc{\V{G}}{\V{S}})}{& \pair{\V{g}}{\V{t}}{}}{& \V{T}}{& (\SumLeft{\V{x}})} & \mapsto & \lookup{\V{G}}{\V{g}}{\V{T}}{\V{x}} 
\end{array}
}\]

\newcommand{\SYMBEmpty}{\F{Empty}\xspace}
\newcommand{\Empty}[1]{\SYMBEmpty\:#1}

\newcommand{\SYMBopenTerm}{\F{openTm}\xspace}
\newcommand{\openTerm}[1]{\SYMBopenTerm\: #1}
\newcommand{\closeTerm}{\F{closeTm}}

\newcommand{\update}{\F{update}}

Consequently, taking the free monad of \(\HExprD\) by \(\SYMBVar\:
\V{G}\), we obtain the language of open terms in a context \(\V{G}\):
%
\[
\openTerm{\V{G}} \mapsto \FreeIMonad{\HExprD}{\Ty}{(\SYMBVar\:\V{G})}
\]
%
In this setting, the language of closed terms corresponds to the free
monad assigning an empty set of values to variables
%
\[
\closeTerm \mapsto \FreeIMonad{\HExprD}{\Ty}{\SYMBEmpty}
\quad
\mbox{where}
\quad
\stk{
\SYMBEmpty : \Ty \To \Set \\
\begin{array}{@{}l@{\:\:\mapsto\:\:}l}
\Empty{\_}   & \Zero \\
\end{array}
}\]
%
Allowing variables from an empty set is much like forbidding variables,
so \(\closeTerm\) and \(\HExprD\) describe isomorphic
datatypes. Correspndingly, you can update your old \(\HExprD\) to a
shiny \(\closeTerm\) with \(\update\):
%
\[\stk{
\update : \muide{\Ty}{\HExprD} \DotTo \muide{\Ty}{\closeTerm} \\
\begin{array}{@{}l@{}l}
\update\: \V{ty}\: \V{tm} \mapsto \cataI_{\Ty}\: & (\toIDesc{\HExprD})\:
                                                  (\muide{\Ty}{\closeTerm})\: \\
                                                & (\LAM{\_} \LAM{\pair{\V{tag}}{\V{tm}}{}} \Con{\pair{\Su{\V{tag}}}{\V{tm}}{}})\:
                                                  \V{ty}\:
                                                  \V{tm}
\end{array}
}\]
% 
The other direction of the isomorphism is straightforward, the
$\SYMBDVar$ case being impossible. Therefore, we are entitled to
reuse the $\evalH$ function to define the semantics of $\closeTerm$.

\newcommand{\discharge}{\F{discharge}}

Now we would like to give a semantics to the open term language. We
proceed in two steps: first, we substitute variables by their value in
the context; then, we evaluate the resulting closed term. Thanks to
$\evalH$, the second problem is already solved. Let us focus on
substituting variables from the context. Again, we can subdivide this
problem: first, discharging a single variable from the context; then,
applying this $\discharge$ function on every variables in the term.

The $\discharge$ function is relative to the required type and a
context of the right type. Its action is to map values to themself,
and variables to their value in context. This corresponds to the
following function:
%
\[\stk{
\begin{array}{@{}ll}
\discharge : & \PI{\V{G}}{\Ctxt}
               \Env{\V{G}} \To 
               \Var{\V{G}}{} \DotTo
               \muide{\Ty}{\closeTerm}
\end{array} \\
\begin{array}{@{}l@{\:\mapsto\:\:}l}
\discharge\: \V{G}\: \V{g}\: \V{ty}\: \V{v} &
    \Con{\pair{\SYMBEval}{\lookup{\V{G}}{\V{g}}{\V{ty}}{\V{v}}}{}}
\end{array}
}\]

\begin{wstructure}
            /> Then, perform subst everywhere in the term
                -> Show type [code]
                /> This is a bind!?
                -> There is some more structure 
                    -> We should try to get it
\end{wstructure}

\newcommand{\substH}{\F{substExpr}}

We are now left with applying $\discharge$ over all variables of the
term.  We simply have to fill in the right arguments to $\substI$, the
type guiding us:
%
\[
\stk{
\begin{array}{@{}ll}
\substH  : & \PI{\V{G}}{\Ctxt} \\
           & (\Var{\V{G}}{} \DotTo
              \muide{\Ty}{\closeTerm}) \DotTo \\
          & \muide{\Ty}{(\openTerm{\V{G}})} \DotTo 
            \muide{\Ty}{\closeTerm}
\end{array} \\
\substH\: \V{G}\:
          \V{ty}\:          
          \V{g}\:
          \V{\sigma}\: 
          \V{tm} \mapsto  
\substI_{\Ty}\:
               (\SYMBVar\: \V{G})\: 
               \SYMBEmpty\:
               \HExprD\: 
               \V{\sigma}\:
               \V{ty}\:
               \V{tm}
}\]

Hence completing our implementation of the open terms interpreter.
We have defined a well-typed language of arithmetical expressions,
taking advantage of indexing. Then, we have implemented an evaluator
for closed term, based on the generic catamorphism function. Using the
free monad construction, we have automatically derived the language of
open terms. Using its monadic structure, we have implemented the
interpreter for open terms in context. Without much effort, we
have described the syntax of a well-typed language, together with its
semantics.


\paragraph{Indexed descriptions:}

An interesting instance of free monad is $\SYMBIDesc$ itself. Indeed,
$\SYMBDVar$ is nothing but the \return. The remaining constructors form
the carrier functor, trivially indexed by $\Unit$. The carrier functor
is described as follow:
%
\[\stk{
\IDescFreeD : \ATagIDesc{\Unit} \\
\begin{array}{@{}ll}
\IDescFreeD \mapsto \bigRedBracket{\begin{array}{l}
                                \sqr{\SYMBDConst\:\:
                                     \DProd\:\:
                                     \SYMBDSigma\:\:
                                     \SYMBDPi} \red{,}\\
                                  \LAM{\_}\bigRedBracket{\begin{array}{l}
                                        \DConst{\Set}               \\
                                        \DProd{\DVar{\Void}}{\DVar{\Void}}  \\
                                        \DSigma{\Set}{(\LAM{\V{S}} \DPi{\V{S}}{(\LAM{\_} \DVar{\Void})})} \\
                                        \DSigma{\Set}{(\LAM{\V{S}} \DPi{\V{S}}{(\LAM{\_} \DVar{\Void})})}
                                    \end{array}}\end{array}}
\end{array}
}\]
%
Then, we get $\IDesc$ by building its free monad:
%
\[\stk{
\IDescD : \PI{\V{I}}{\Set} \TagIDesc{\Unit} \\
\IDescD\: \V{I} \mapsto \FreeIMonad{\red{[}\IDescFreeD\red{,[}\LAM{\_}\sqr{}\red{,}\LAM{\_}\red{[]]]}}{\Unit}\LAM{\_}\V{I}
}\]

The fact that indexed descriptions are closed under substitution
is potentially of considerable utility, if we can exploit this fact:
\[
\idescop{\V{\sigma} \V{D}}{\V{J}}{\V{X}} 
    \mapsto 
        \idescop{\V{D}}
                {\V{I}}
                {\LAM{\V{i}}
                     {\idescop{\V{\sigma} \V{i}}
                              {\V{J}}
                              {\V{X}}}}
        \quad \mbox{where}\;\V{\sigma}:\V{I}\To\IDesc{\V{J}}
\]
By observing that a description can be decomposed via substitution, we
split its meaning into a superstructure of substructures, e.g. a
`database containing salaries', ready for traversal operations
preserving the former and targeting the latter. We shall now turn to
some practical use of the free monad construction.
 
