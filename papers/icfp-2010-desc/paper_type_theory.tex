\section{The Type Theory}

Our work is deeply rooted in the realm of dependently-typed functional
languages. In particular, we have developed this technology in our
on-going work on Epigram~\cite{pigs:epigram}. However, in this paper,
we strive to present our technology in the broadest setting
possible. To this end, we are going to present a type theory from the
ground-up.

To the reader new to type theory, this first section will provide a
gentle introduction to dependently-typed systems. To the reader
familiar with type theory, she will recognise the basis of her
favourite type theory, be it Agda or Coq for instance. This scene sets,
we have a common ground for discussion.

\subsection{Base theory}

\begin{wstructure}
<- Presentation of the formalism
    <- Standard presentation
        -> No novelty here
    <- 3 judgments [equation]
        -> Context validity
        -> Typing judgements
        -> Equality judgements
\end{wstructure}

In the following, we adopt a standard presentation for our type
theory. It is defined by three forms of judgements: context validity
judgements, typing judgements, and equality judgements. Formally,
these judgements are presented as follow:

\[
\begin{array}{ll}
\Gamma \vdash \Valid                & \mbox{\(\Gamma\) is a valid context, giving types to variables} \\
\Gamma \vdash \Bhab{t}{T}           & \mbox{term \(t\) has type \(T\) in context \(\Gamma\)} \\
\Gamma \vdash \Bhab{s \equiv t}{T}  & \mbox{\(s\) and \(t\) are equal at type \(T\) in context \(\Gamma\)} \\
\end{array}
\]

\begin{wstructure}
    <- Invariants [equation]
        -> By induction on derivations
\end{wstructure}

Each of these judgements is populated by a respective system of
inferences rules. These rules are formulated so as to ensure that the
following implications always hold by induction on derivations:

\[
\begin{array}{l@{\;\Rightarrow\;\;}l}
\Gamma            \vdash \Bhab{t}{T}            
    & \Gamma \vdash \Valid\; \wedge\; \Gamma\vdash\Type{T} \\
\Gamma            \vdash s \equiv \Bhab{t}{T}   
    & \Gamma \vdash \Bhab{s}{T} \;\wedge\; \Gamma\vdash \Bhab{t}{T} \\
\Gamma ; \xS ; \Delta \vdash J                      
    & \Gamma \vdash \Bhab{s}{S} \;\Rightarrow\; 
          \Gamma ; \Delta[s/x] \vdash J[s/x] \\
\end{array}
\]

We are not going to prove the validity of these invariants. They
follow rather straightforwardly from the induction rules. For formal
proofs, we refer the reader to standard presentations of type theory,
such as Luo's seminal work \cite{luo:utt}.

\begin{wstructure}
    <- Judgemental equality
        <- Presentation independent of particular implementation choice
        -> Model in Agda, intentional
        -> Used in Epigram, OTT
\end{wstructure}

Noticeably, we have adopted a judgemental presentation of
equality. Therefore, we are not tied to a particular implementation
choice. In particular, our system has been modelled in Agda, which
features an intentional equality. On the other hand, it is used in
Epigram, which equality has a slightly extensional
flavor~\cite{altenkirch:ott}. We expect users of fully extensional
systems to also find their way through this presentation.

\begin{wstructure}
<- Context validity [figure]
    <- Not much to be said
\end{wstructure}

\begin{figure}

\[
%% Empty context validity
\Axiom{\vdash \Valid}
\qquad
%% Extend context
\Rule{\Gamma       \vdash \Type{S}}
     {\Gamma ; \xS \vdash \Valid}\;x\not\in\Gamma
\]

\caption{Context validity}
\label{fig:context-validity}
\end{figure}

The rules ensuring Context Validity (Fig.~\ref{fig:context-validity})
are unsurprisingly common. They simply guarantee the validity of the
empty context, as well as the linear extension of a context by a
well-typed variable.

\begin{wstructure}
<- Typing judgements [figure]
    <- Set in Set
        -> For simplicity of presentation
        -> Assume that a valid stratification can be inferred
            <- Harper-Pollack, Luo, Courant
        -> See later discussion
    <- Standard presentation of Pi and Sigma types
\end{wstructure}

The typing judgements (Fig.~\ref{fig:typing-judgements}) are just as
familiar. We recognise the standard presentation of $\Pi$ and $\Sigma$
types, respectively inhabited by lambda terms and dependent
pairs. Naturally, there are rules for function application and
projections of $\Sigma$-types. Equal types can be substituted, thanks
to the conversion rule.

For the sake of presentation, we postulate a $\Set$ in $\Set$
rule. Having this rule makes our type theory inconsistent, by Girard's
paradox~\cite{girard:set-in-set}. However, it has been
shown~\cite{harper:implicit-universe, luo:utt,
  courant:explicit-universe} that a valid stratification can be
inferred, automatically or semi-automatically. In the remaining of our
presentation, we will assume that such a stratification exists, even
though we will keep it implicit. We shall discuss this assumption in
Section~\ref{sec:discussion}.

\begin{figure}

\[\stkc{
%% Context
\Rule{\Gamma ; \xS ; \Delta \vdash \Valid}
     {\Gamma ; \xS ; \Delta \vdash \Bhab{x}{S}}
\qquad
%% Conversion
\Rule{\Gamma \vdash \Bhab{s}{S} \quad 
      \Gamma \vdash \Type{S \equiv T}}
     {\Gamma \vdash \Bhab{s}{T}}
\\
%% Girard's favorite
\Rule{\Gamma \vdash \Valid}
     {\Gamma \vdash \Type{\Set}}
\qquad
%% Pi-Sigma
\Rule{\Gamma       \vdash \Type{S} \quad
      \Gamma ; \xS \vdash \Type{T}}
     {\Gamma \vdash \Type{\PIS{\xS} T, \SIGMAS{\xS} T}}
\\
%% %% Prop
%% \Rule{\Gamma \vdash \Bhab{q}{\Prop}}
%%      {\Gamma \vdash \Type{\prf{q}}}
%% \qquad
%% %% True
%% \Rule{\Gamma \vdash \Valid}
%%      {\Gamma \vdash \Bhab{\True}{\Prop}}
%% \\
%% Unit
\Rule{\Gamma \vdash \Valid}
     {\Gamma \vdash \Bhab{\Unit}{\Set}}
\qquad
%% Void
\Rule{\Gamma \vdash \Valid}
     {\Gamma \vdash \Bhab{\Void}{\Unit}}
\\
%% Lambda
\Rule{\stkl{\Gamma       \vdash \Type{S} \\
            \Gamma ; \xS \vdash \Bhab{t}{T}}}
     {\Gamma \vdash \Bhab{\PLAM{\x}{S} t}{\PIS{\xS} T}}
\qquad
%% Application
\Rule{\stkl{\Gamma \vdash \Bhab{f}{\PIS{\xS} T} \\
            \Gamma \vdash \Bhab{s}{S}}}
     {\Gamma \vdash \Bhab{f\: s}{T[s/x]}} 
\\
%% Pair
\Rule{\Gamma       \vdash \Bhab{s}{S} \quad 
      \Gamma ; \xS \vdash \Bhab{T}{\Set}    \quad
      \Gamma       \vdash \Bhab{t}{T[s/x]}}
     {\Gamma \vdash \Bhab{\pair{s}{t}{x.T}}{\SIGMAS{\xS} T}}
\\
%% First projection
\Rule{\Gamma \vdash \Bhab{p}{\SIGMAS{\xS} T}}
     {\Gamma \vdash \Bhab{\fst{p}}{S}} 
\qquad
%% Second projection
\Rule{\Gamma \vdash \Bhab{p}{\SIGMAS{\xS} T}}
     {\Gamma \vdash \Bhab{\snd{p}}{T[\fst{p}/x]}}
\\
}\]


\caption{Typing judgements}
\label{fig:typing-judgements}

\end{figure}


\begin{wstructure}
<- Judgemental equality [figure]
    <- symmetry, reflexivity, and transitivity
    <- beta-rules for lambda and pair
    <- xi-rule for functions
    -> Agnostic in the notion of equality
        <- Doesn't rely on a ``propositional'' equality
        -> Key: wide applicability of our proposal
\end{wstructure}

Finally, we define the rules governing judgemental equality in
Figure~\ref{fig:judgemental-equality}. We implicitly assume that
judgemental equality respects symmetry, reflexivity, and
transitivity. We capture the computational behavior of the language
through the $\beta$-rules for function application and pairs. Finally,
we implicitly assume that it respects purely syntactic and structural
equality. This includes equality under lambda ($\xi$-rule).

Crucially, being judgemental, this presentation is agnostic in the
notion of equality actually implemented. Indeed, our typing and
equality judgements do not rely on a ``propositional'' equality. This
freedom is a key point in favour of the wide applicability of our
proposal. This judgemental presentation must be read as a
\emph{specification}: our proposal works with any propositional
equality satisfying this specification. Moreover, our lightweight
requirements do not endanger decidability of equality-checking.
Obviously, when implementing our technology in an existing
type-theory, some opportunities arise. We will present some of them
along the course of the paper.


\begin{figure}

\[\stkc{
%% %% Reflexivity
%% \Rule{\Gamma \vdash \Bhab{x}{T}}
%%      {\Gamma \vdash \Bhab{x \equiv x}{T}}
%% \qquad
%% %% Symmetry
%% \Rule{\Gamma \vdash \Bhab{x \equiv y}{T}}
%%      {\Gamma \vdash \Bhab{y \equiv x}{T}}
%% \qquad
%% %% Transitivity
%% \Rule{\stkl{\Gamma \vdash \Bhab{z \equiv y}{T} \\
%%             \Gamma \vdash \Bhab{y \equiv z}{T} }}
%%      {\Gamma \vdash \Bhab{x \equiv z}{T}}
%% \\
%% Beta-reduction
\Rule{\stkl{\Gamma       \vdash \Type{\M{S}} \quad
            \Gamma ; \xS \vdash \Bhab{\M{t}}{\M{T}} \\
            \Gamma       \vdash \Bhab{\M{s}}{\M{S}}}}
     {\Gamma \vdash \Bhab{(\PLAM{\x}{\M{S}} \M{t})\:\M{s} \equiv \M{t}[\M{s}/\x]}{\M{T}[\M{s}/\x]}}
\\
%% Xi rule
%% \Rule{\Gamma       \vdash \Type{S} \quad
%%       \Gamma ; \xS \vdash \Bhab{t \equiv t'}{T}}
%%      {\Gamma \vdash \Bhab{(\PLAM{\x}{S} t) \equiv (\PLAM{\x}{S} t')}{\PIS{\xS} T}}
%% \\
%% Projections
\Rule{\stkl{\Gamma                 \vdash \Bhab{\M{s}}{\M{S}} \quad
            \Gamma ; \xS           \vdash \Bhab{\M{T}}{\Set} \\
            \Gamma ; \Bhab{\M{s}}{\M{S}}   \vdash \Bhab{\M{t}}{\M{T}[\M{s}/\x]}}}
     {\Gamma \vdash \Bhab{\fst{(\pair{\M{s}}{\M{t}}{\x.\M{T}})} \equiv \M{s}}{\M{S}}}
\qquad
\Rule{\stkl{\Gamma               \vdash \Bhab{\M{s}}{\M{S}} \quad
            \Gamma ; \xS         \vdash \Bhab{\M{T}}{\Set} \\
            \Gamma ; \Bhab{\M{s}}{\M{S}} \vdash \Bhab{\M{t}}{\M{T}[\M{s}/\x]}}}
     {\Gamma \vdash \Bhab{\snd{(\pair{\M{s}}{\M{t}}{\x.\M{T}})} \equiv \M{t}}{\M{T}[\M{s}/\x]}}
}\]


\caption{Judgemental equality}
\label{fig:judgemental-equality}

\end{figure}



\begin{wstructure}
!!! Need Help !!!
<- Meta-theoretical properties
    <- Assuming a stratified discipline
    <> The point here is to reassert that dependent types are not evil, 
       there is no non-terminating type-checker, or such horrible lies <>
    -> Strongly normalising
        -> Every program terminates
    -> Type-checking terminates
    ???
\end{wstructure}

This completes our presentation of the type theory. Assuming a
stratified discipline of universe, the system we have described enjoy
some very strong meta-theoretical properties. Unlike simply typed
languages, such as Haskell, dependently-typed systems are
\emph{strongly normalising}: every program that type-checks
terminates. Moreover, type-checking is decidable and can therefore be
implemented by a terminating algorithm.

\note{Need some care here. Expansion would be good too. I wanted to
  carry the intuition that we are not the bad guys with a
  non-terminating type-checker.}

\subsection{Finite sets}
\label{sec:finite-sets}

\begin{wstructure}
<- Motivation
    <- Finite sets could be encoded with Unit and Bool
        /> Hinder the ability to name things
    <- W-types considered harmful?
        ???
    -> For convenience
        <- Named elements
        <- Referring by name instead of code
        -> Types as coding presentation
            /> Also as coding representation!
\end{wstructure}

As a motivating example, we are now going to extend the type theory
with a notion of finite set. One could argue that there is no need for
such an extension: finite sets, just as any data-structure, can be
encoded inside the type theory. A well-known example of such encoding
is the Church encoding of natural numbers, which is isomorphic to
finite sets. 

% Another example is Martin-Löf's W-types~\cite{}.

\note{Shall we talk about W-types encoding?}

However, using encodings is impractical. In the case of finite sets,
for instance, we would like to name the elements of the sets. Then, we
need to be able to manipulate these elements by their name, instead of
their encoding. While we are able to give names to encodings, it is
extremely tedious to map the encodings back to a name. Whereas these
objects have a structure, the structure is lost during the encoding,
when they become anonymous inhabitants of a $\Pi$ or $\Sigma$-type.

In the simply-typed world, we are used to see types as a coding
presentation -- presentation of invariants, presentation of
properties. In the dependently-typed world, we also learn to use types
as a coding representation: finite sets being good citizens, they
ought to be democratically represented at the type level. As we will
see, this gives us the ability to name and manipulate them (this is
were the democracy analogy goes crazy, I think).

\note{Did I got the coding presentation vs. coding representation
  story right?}

\begin{wstructure}
<- Implementation [figure]      
    <- Tags
        -> Purely informational token
    <- EnumU
        -> Universe of finite sets
    <- EnumT e
        -> Elements of finite set e
\end{wstructure}

The specification of finite sets is presented in
Figure~\ref{fig:typing-finite-set}. It is composed of three
components. First, we define tags as inhabitants of the $\UId$ type. A
tag is solely an informative token, used for diagnostic
purposes. Finite sets inhabits the $\EnumU$ type. Unfolding the
definition, we get that a finite set is a list of tags. Finally,
elements of a finite set $u$ belong to the corresponding $\EnumT{u}$
type. Intuitively, it corresponds to an index -- a number -- pointing
to an element of $u$.

\begin{figure}

\[\stkc{
%% UId
\Rule{\Gamma \vdash \Valid}
     {\Gamma \vdash \Type{\UId}}
\qquad
%% Tag
\Rule{\Gamma \vdash \Valid}
     {\Gamma \vdash \Bhab{\Tag{\V{s}}}{\UId}}\;\V{s} \mbox{ unique identifier}
\\
%% EnumU
\Rule{\Gamma \vdash \Valid}
     {\Gamma \vdash \Type{\EnumU}} 
\qquad
%% EnumT
\Rule{\Gamma \vdash \Bhab{\V{e}}{\EnumU}}
     {\Gamma \vdash \Type{\EnumT{\V{e}}}} 
\\
%% NilE
\Rule{\Gamma \vdash \Valid}
     {\Gamma \vdash \Bhab{\NilE}{\EnumU}} 
\qquad
%% ConsE
\Rule{\Gamma \vdash \Bhab{\V{t}}{\UId} \quad
      \Gamma \vdash \Bhab{\V{e}}{\EnumU}}
     {\Gamma \vdash \Bhab{\ConsE{\V{t}}{\V{e}}}{\EnumU}}
\\
%% Ze
\Rule{\Gamma \vdash \Valid}
     {\Gamma \vdash \Bhab{\Ze}{\EnumT{\ConsE{\V{t}}{\V{e}}}}} 
\qquad
%% Su
\Rule{\Gamma \vdash \Bhab{\V{n}}{\EnumT{\V{e}}}}
     {\Gamma \vdash \Bhab{\Su{\V{n}}}{\EnumT{\ConsE{\V{t}}{\V{e}}}}}
}\]


\caption{Typing rules for finite sets}
\label{fig:typing-finite-set}

\end{figure}


\begin{wstructure}
<- Equipment
    <- \spi operator
        <- Equivalent of Pi on finite sets
        <- First argument: (finite) domain
        <- Second argument: for each element of the domain, a co-domain
        -> Inhabitant of \spi: right-nested tuple of solutions
            <- Skip code for space reasons
    <- switch operator
        <- case analyses over x
        <- index into the \spi tuple to retrieve the corresponding result
\end{wstructure}

The type of finite sets comes with further machinery. Indeed, to
manipulate the $\EnumU$ and $\EnumT{}$ objects, we need to equip our
type theory with an appropriate \emph{elimination
  principle}. Formally, as we add types for data -- such as finite
sets here --, we need such primitive operators to compute with those
data, and corresponding extension of the judgemental equality with
their computational behavior. For compactness and readability, we
shall present these operators as functional programs, much as we
implement them in our Agda model. Hence, we first define the
$\spi{}{}$ operator:

\[\stk{
%% spi
\spi{}{} : \PITEL{\V{e}}{\EnumU}
           \PITEL{\V{P}}{\EnumT{e} \To \Set} \To \Set \\
\begin{array}{@{}ll@{\:\mapsto\:\:}l}
\spi{\NilE}{& P}        & \Unit \\
\spi{(\ConsE{t}{e})}{& P} & \TIMES{P\: \Ze}{\spi{e}{(\LAM{x} P\: (\Su{x}))}}
\end{array}
}\]

A closer look to its type should reveal a certain familiarity with its
equivalent for sets, $\Pi$. Indeed, $\spi{}{}$ defines a \emph{finite
  function space}, mapping elements of finite sets to objects in
$\Set$. Intuitively, this operator build a right-nested tuple,
effectively mapping an object of $P\:i$ for each $i$ the finite
domain.

Having understood the meaning of $\spi{}{}$, the definition of the
$\F{switch}$ eliminator is not surprising:

\[\stk{
%% switch
\begin{array}{@{}ll}
\F{switch} : & \PITEL{\V{e}}{\EnumU}
               \PITEL{\V{P}}{\EnumT{e} \To \Set} \\
             & \PITEL{\V{b}}{\spi{e}{P}}
               \PITEL{\V{x}}{\EnumT{e}} \To P~x
\end{array} \\
\begin{array}{@{}l@{\:\mapsto\:\:}l}
\switch{(\ConsE{t}{e})}{P}{b}{\Ze}      & \fst{b} \\
\switch{(\ConsE{t}{e})}{P}{b}{(\Su{x})} & \switch{e}{(\LAM{x} P
  (\Su{x}))}{(\snd{b})}{x}
\end{array}
}\]

Again, there is a clear equivalent in the full-$\Set$ world: function
application. The operational behaviour of $\F{switch}$ is
straightforward: $x$ is peeled off as we move deeper inside the nested
tuple $b$. When $x$ equals $\Ze$, we simply return the value we are
pointing to.

\begin{wstructure}
<- Equivalent to having a function space over finite sets
    /> Made non-obvious by low-level encodings
        <- General issue with codes
             -> Need to provide an attractive presentation to the user
    -> Types seem to obfuscate our reading
        <- Provide ``too much'' information
        /> False impression: information is actually waiting to be used more widely
        -> See next Section
\end{wstructure}

As we have seen, finite sets are not just a way of making
data. Through the $\spi{}{}$ and $\F{switch}$ operators, they also
come with a notion of finite function space. However, we had to
extract that intuition from the type, by a careful reading. This seems
to contradict our argument in favour of types for coding
representation. Here, we are overflown by low-level, very precise type
information.

However, our situation is significantly different from the one we
faced with encoded data: while we were suffering from a crucial lack
of information, we are now facing too much information, hence loosing
focus. This is a general issue with the usage of codes, as they convey
much more information than what the user is willing to see. 

As we will see in the following section, there exists a cure to this
problem. In a nutshell, instead of being overflown by typing
information, we will put it at work, automatically. The consequence is
that, in such system, working with codes is \emph{practical}: one
should not be worried by information overload, but how to use it as
much as possible. Therefore, we should not be afraid by using code for
practical purposes.


\subsection{Type propagation}
\label{sec:type-propagation}

\begin{wstructure}
<- Bidirectional type-checking [ref. Turner,Pierce]
    -> Separating type-checking from type synthesis
    <- Type checking: push terms into types
        <- Example: |Pi S T :>: \ x . t| allows us to drop annotation on lambda
    <- Type synthesis: pull types out of terms
        <- Example: |x : S l- x :<: S| gives us the type of x
\end{wstructure}

Our proposal to tame the problem of codes is deeply rooted in the
bidirectional presentation of type-checking from Pierce and
Turner~\cite{pierce:bidirectional-tc}. In this formalism,
type-checking is divided into two complementary phases. In the
\emph{type inference} phase, types are \emph{pulled} out of terms. A
typical example is a variable in the context:

\[
\Axiom{\Gamma ; \xS ; \Delta \Vdash \propag{\x}
                                           {\pull{\x}{S}}}
\]

Because the variable and its type are stored in the context, whenever
this variable is used, we can pull its type out of it.

On the other hand, in the \emph{type checking} phase, terms are
\emph{pushed} into types. We are handed a type together with a term,
our task consists in checking that the term is indeed of this
type. Doing so, we are allowed, and actually encouraged, to use the
information provided by the type. Therefore, we can relax our
requirements on the term. A typical example is the lambda abstraction:

\[
\Rule{\Gamma ; \xS \Vdash \propag{\push{t}{T}}
                                 {t'}}
     {\Gamma \Vdash \propag{\push{\LAM{\x} t}{\PIS{\xS}{T}}}
                           {\PLAM{\x}{S} t'}} 
\]

In our formal presentation of the type theory, we were forced to
annotate lambda abstractions by the type of the domain. In a
bidirectional setting, the domain is provided by the $\Pi$ type we are
pushing the abstraction into. Hence we do not need the annotation.

\begin{wstructure}
<- Formalisation: type propagation
    <- Motivation
        -> High-level syntax
            -> exprIn: types are pushed in
                <- Subject to type *checking*
            -> exprEx: types are pulled from
                <- Subject to type *synthesis*
        -> Translated into our low-level type theory
        -> Presented as judgements
    -> Presentation mirrors typing rule of [figure] 
        -> Ignore identical judgements
\end{wstructure}

We adapt and formalise this \emph{type propagation} mechanism to our
richer type theory. First of all, we define a high-level syntax,
composed of \emph{expressions}. Expressions are divided into two
syntactic categories: $\exprIn$ into which types are pushed in, and
$\exprEx$ from which types are pulled out. Following the bidirectional
presentation, $\exprIn$ are subject to type \emph{checking}, while
$\exprEx$ are subject to type \emph{synthesis}.

\note{Why don't we need a context validity judgement, again?}

The purpose of type propagation is to \emph{elaborate} an high-level
expression into a term of the type theory, together with its
type. This presentation builds on the experience gained in
implementing Epigram 1~\cite{mcbride:view-from-the-left}. Being
entirely syntax driven, we present our formalism as a system of
syntax-directed inference rules. As the presentation largely mirrors
the inference rules of the type theory, we will ignore the judgements
that are identical. We refer our reader to the associated technical
report~\cite{chapman:desc-tech-report} for the complete system of
rules.

\begin{wstructure}
<- Type synthesis [figure]
    <- Pull a type out of an exprEx
    <- Result in a full term, together with its type
    -> Do *not* need to specify types
        -> Extracting a term from the context
        -> Function application
        -> Projections
\end{wstructure}

The type synthesis rules are presented
Figure~\ref{fig:type-synthesis}. As illustrated by the judgement form,
a type synthesis problem is provided an expression $\exprEx$ in a
context, which is elaborated into a low-level term together with its
type. These terms act as \emph{sources} for the type propagation
machinery: they make types out of expressions. We have already studied
the case of a variable in context. This also include function
application as well as projections. Further down the line, these
inferred types might propagate to other terms, thanks to type
checking.

\begin{figure}
\[\stkc{
%% Form
\boxed{\Gamma \Vdash \propag{\exprEx}{\pull{\CN{term}}{\CN{type}}}}
\\
\\
%% Reversal
\Rule{\Gamma \Vdash \propag{\push{T}{\Set}}
                           {T'} \quad
      \Gamma \Vdash \propag{\push{t}{T'}}
                           {t'}}
     {\Gamma \Vdash \propag{(\Bhab{t}{T})}
                           {\pull{t'}{T'}}} 
\\
%% Context
\Rule{\Gamma ; \xS ; \Delta \vdash \Valid}
      {\Gamma ; \xS ; \Delta \Vdash \propag{\x}
                                           {\pull{\x}{\M{S}}}}
\qquad
%% Application
\Rule{\stkl{\Gamma \Vdash \propag{\M{f}}
                                 {\pull{\M{f}\M{'}}{\PIS{\xS}{\M{T}}}} \\
            \Gamma \Vdash \propag{\push{\M{s}}{\M{S}}}
                                 {\M{s'}}}}
     {\Gamma \Vdash \propag{\M{f}\: \M{s}}{\pull{\M{f'}\: \M{s'}}{\M{T} [\M{s'}/\x]}}} 
\\
%% First projection
\Rule{\Gamma \Vdash \propag{\M{p}}
                           {\pull{\M{p'}}{\SIGMAS{\xS}{\M{T}}}}}
     {\Gamma \Vdash \propag{\fst{\M{p}}}
                           {\pull{\fst{\M{p'}}}{\M{S}}}} \qquad 
%% Second projection
\Rule{\Gamma \Vdash \propag{\M{p}}
                           {\pull{\M{p'}}{\SIGMAS{\xS}{\M{T}}}}}
     {\Gamma \Vdash \propag{\snd{\M{p}}}
                           {\pull{\snd{\M{p'}}}{\M{T} [\fst{\M{p'}}/\x]}}}
}\]

\caption{Type synthesis}
\label{fig:type-synthesis}
\end{figure}



\begin{wstructure}
<- Type checking [figure]
    <- Push a type in an exprIn
    <- Result in a full term
    -> *Use* the type to build the term!
        -> Domain and co-domain propagation for Pi and Sigma
        -> Translation of 'tags into EnumTs
        -> Translation of ['tags ...] into EnumUs
        -> Finite function space into switch
\end{wstructure}

Dually to type synthesis, type checking judgements
(Fig.~\ref{fig:type-checking}) act as \emph{sinks}. From an expression
$\exprIn$ and a type propagated down to it, they elaborate a low-level
term. The strength of this approach comes from the availability of the
type to drive the elaboration of expressions. Its benefit on lambda
abstraction and pairs is well-known. Let us explore the consequences
on our coding scheme for finite sets.

A first consequence is that we can actually denote elements of finite
sets \emph{by their name}. Indeed, such expression will be pushed into
a $\EnumT{u}$, where $\Bhab{u}{\EnumU}$ contains all the names
populating the finite set. It is therefore easy to elaborate a name
into the corresponding index.

A second consequence concerns the definition of a finite set. We can
abstract away the details of code and use a familiar syntax
instead. Hence, we use the syntax $[t_1 \ldots t_n]$ to denote the
finite sets composed of the named elements $t_1$, \ldots, $t_n$. The
translation to elements of $\EnumU$ is straightforward.

\note{This might deserve a more thorough explanation, with snippet of
  code.}

Finally, and more importantly, we are able to present our user an
abstraction of \emph{finite} function space. This work as follow. When
an expression $t$ is pushed into a function which domain is a finite
set, we instead elaborate it as a term in a small $\pi$-type. That is,
a finite function. To realise our promise of a big $\Pi$-type, we
elaborate a call to the $\F{switch}$ eliminator. 

The abstraction we have developed in this section are entirely
transparent to the user. In the case of finite function space, the
user has never to explicitly \emph{call} the $\F{switch}$
eliminator. For diagnosis purposes, a similar, reverse mechanism turns
low-level terms into high-level expressions. Hence, this system, which
complete presentation is out of the scope of this paper, seals the
low-level type theory behind the high-level expressions.

\begin{figure}
\[\stkc{
%% Form
\boxed{\Gamma \Vdash \propag{\push{\CN{exprIn}}{\CN{type}}}{\CN{term}}} 
\\
\\
%% Set in Set
%% \Axiom{\Gamma \Vdash \propag{\push{\Set}{\Set}}
%%                             {\Set}}
%% \\
%% Prop
%% \Rule{\Gamma \Vdash \propag{\push{q}{\Prop}}
%%                            {q'}}
%%      {\Gamma \Vdash \propag{\push{\prf{q}}{\Set}}
%%                            {\prf{q'}}}
%% \qquad
%% True
%% \Axiom{\Gamma \Vdash \propag{\push{\True}{\Prop}}
%%                             {\True}}
%% \\
%% Pi
%% \Rule{\Gamma \Vdash \propag{\push{S}{\Set}}
%%                            {S'} \quad
%%       \Gamma \Vdash \propag{\push{T}{S' \To \Set}}
%%                            {T'}}
%%      {\Gamma \Vdash \propag{\push{\PiTy{S}{T}}{\Set}}
%%                            {\PiTy{S'}{T'}}} 
%% \\
%% %% Sigma
%% \Rule{\Gamma \Vdash \propag{\push{S}{\Set}}
%%                            {S'} \quad
%%       \Gamma \Vdash \propag{\push{T}{S' \To \Set}}
%%                            {T'}}
%%      {\Gamma \Vdash \propag{\push{\SigmaTy{S}{T}}{\Set}}
%%                            {\SigmaTy{S'}{T'}}}
%% \\
%% Lambda
\Rule{\Gamma ; \xS \Vdash \propag{\push{\V{t}}{\V{T}}}
                                 {\V{t'}}}
     {\Gamma \Vdash \propag{\push{\LAM{\x} \V{t}}{\PIS{\xS}{\V{T}}}}
                           {\PLAM{\x}{\V{S}} \V{t'}}} 
\\
%% Pair
\Rule{\stkl{ \Gamma \Vdash \propag{\push{\V{s}}{\V{S}}}
                                  {\V{s'}} \\
             \Gamma \Vdash \propag{\push{\V{t}}{\V{T}[\V{s'}/\V{x}]}}
                                  {\V{t'}}}}
     {\Gamma \Vdash \propag{\push{\pair{\V{s}}{\V{t}}{}}{\SIGMAS{\xS}{\V{T}}}}
                           {\pair{\V{s'}}{\V{t'}}{\V{T}}}}
\\
%% EnumU
%% \Axiom{\Gamma \Vdash \propag{\push{\EnumU}{\Set}}
%%                             {\EnumU}} 
%% \qquad
%% %% EnumT
%% \Rule{\Gamma \Vdash \propag{\push{e}{\EnumU}}
%%                            {e'}}
%%      {\Gamma \Vdash \propag{\push{\EnumT{e}}{\Set}}
%%                            {\EnumT{e'}}}
%% \\
%% Tag
\Axiom{\Gamma \Vdash \propag{\push{\etag{t_0}}{\EnumT{(\ConsE{\etag{t_1}}{\V{e}})}}}
                            {\Ze}}\;\etag{t_0} = \etag{t_1}
\\
\Rule{\Gamma \Vdash \propag{\push{\etag{t_0}}{\EnumT{\V{e}}}}
                           {\V{n}}}
     {\Gamma \Vdash \propag{\push{\etag{t_0}}{\EnumT{(\ConsE{\etag{t_1}}{\V{e}})}}}
                            {\Su{\V{n}}}}\;\etag{t_0} \neq \etag{t_1}
\\
%% EnumU
\Axiom{\Gamma \Vdash \propag{\push{[]}{\EnumU}}
                            {\NilE}}
\\
\Rule{\Gamma \Vdash \propag{\push{\V{ts}}{\EnumU}}
                                 {\V{cs}}}
     {\Gamma \Vdash \propag{\push{[ \etag{t_1},\: \V{ts} ]}{\EnumU}}
                                 {\ConsE{\etag{t_1}}{\V{cs}}}}
\\
%% Switch
\Rule{\Gamma \Vdash \propag{\push{\V{t}}{\spi{\V{e}}{\V{P}}}}
                           {\V{t'}}}
     {\Gamma \Vdash \begin{array}{@{}l} 
                        \propag{\push{\V{t}}{\PI{\V{x}}{\EnumT{\V{e}}} \V{P}\:\V{x}}}
                               {\\ \PLAM{\V{x}}{(\EnumT{\V{e}})} \switch{\V{e}}{\V{P}}{\V{t'}}{\V{x}}}
                    \end{array}}\;\mbox{$\V{t}$ is $[]$ or $[\_,\_]$}
%% \\
%% Conversion
%% \Rule{\Gamma \Vdash \propag{s}
%%                            {\pull{s'}{S}} \quad 
%%       \Gamma \Vdash \push{S \equiv T}{\Set}}
%%      {\Gamma \Vdash \propag{\push{s}{T}}
%%                            {s'}}
}\]

\caption{Type checking}
\label{fig:type-checking}
\end{figure}

\begin{wstructure}
<- Summary
    -> Not a novel technique [refs?]
        /> Used as a boilerplate scrapper
    -> Make dealing with codes *practical*
        <- Example: Finite sets/finite function space
        -> We should not restrain our self in using codes
            <- We know how to present them to the user
-> Will extend this machinery in further sections
\end{wstructure}

In this section, we have developed a type propagation system based on
bidirectional type-checking. Using bidirectional type-checking as a
boilerplate scrapper is a well-known technique~\cite{who?} \note{I
  would need some citation of people using bidirectional presentation
  of their type theory}. In our case, we have shown how to instrument
bidirectionality to rationalise the expressivity of dependent
types. We have illustrated our approach with finite sets. We have
abstracted away the low-level presentation of finite sets, offering a
convenient syntax instead.

This example teaches us that we should not be afraid by codes, as soon
as type information is available. We have shown how to rationalise
this information in a formal presentation. Hence, we have proved that
programming with such objects is practical. As we introduce more codes
in our theory, we will show how to extend the framework we have
developed so far.
