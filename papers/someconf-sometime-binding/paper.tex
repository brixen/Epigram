\begin{wstructure}
<- Untyped lambda terms [figure]
    <- Indexed by Nat
        <- De Bruijn "encoding"
    <- Variable in Fin n
        -> Correct by construction lambda term
    /> Lack the monadic structure [Altenkirch & Reus ???]
        -> Miss substitution operator
        /> Will be fixed later
\end{wstructure}

Our next example is, for once, building an inductive family. We
present the implementation of untyped lambda terms, as a family
indexed by natural numbers. The natural numbers play the role of De
Bruijn indexes. The indexed description is the following:

\[\stk{
\LambdaTD : \Nat \To \IDesc{\Nat} \\
\begin{array}{@{}ll}
\LambdaTD\: n = \DSigma{}{} & (\EnumT [ \LambdaTVar, \LambdaTApp, \LambdaTLam ]) \\
                            & \left[\begin{array}{l}
                                  \DConst{(\Fin{n})} \\
                                  \DProd{\DVar{n}}{\DVar{n}} \\
                                  \DVar{(\NatSuc{n})}
                              \end{array}\right]
\end{array}
}\]

A variable is any number between $0$ and $n$, hence forbidding the
reference of out-of-scope binders. Making a lambda increases by one
the possible index choice of the underlying term. A function
application applies a term with $n+1$ binders, resulting in a term
with $n$ binders.

We have therefore given a representation of lambda terms as a
data-type. Any inhabitant of this data-type is a valid lambda term,
forbidding out-of-scope reference in variables. While we have the
data-type, we can regret the absence of structure. As shown by
Altenkirch and Reus~\cite{altenkirch:monadic-lambda}, the lambda terms
actually enjoy a monadic structure. In this setting, the ``bind''
corresponds to substitution. We shall remedy to this situation
shortly.




\begin{wstructure}
<- Untyped lambda terms
    <- Obvious instance of free monad
        <- Variable: Return
        <- App and Abst: indexed functor
        <- Bind: substitution
    -> Get a correct-by-construction lambda term
        /> With substitution for free!
        <- We should not pay for the structure which *is* here
\end{wstructure}

In Section~\ref{sec:idesc-examples}, we were missing the monadic
structure of lambda terms. In particular, we wish we were given a
substitution principle for free. Above, we have generically built such
substitution operation for free monads. Can we be rewarded for our
effort? Let us define the following $\LambdaTFreeD$ description,
largely inspired by $\LambdaTD$:

\[\stk{
\LambdaTFreeD : \Nat \To \IDesc{\Nat} \\
\begin{array}{@{}ll}
\LambdaTFreeD\: n = \DSigma{}{} & (\EnumT [ \LambdaTApp, \LambdaTLam ]) \\
                                & \left[\begin{array}{l}
                                  \DProd{\DVar{n}}{\DVar{n}} \\
                                  \DVar{(\NatSuc{n})}
                                  \end{array}\right]
\end{array}
}\]

In essence, we have forgotten variables, just keeping the nodes of
lambda terms: abstraction and application. Our hope is that taking the
free monad of this indexed functor gives us our lambda terms back:

\note{ FreeMonad takes a damned tagged description }

\[\stk{
\LambdaTD : \Nat \To \IDesc{\Nat} \\
\LambdaTD\: \mapsto \FreeIMonad{[\LambdaTFreeD]}{\Nat}{\Fin{}}
}\]

By a straightforward unfolding of definitions, the reader will
convince herself that we recover the initial definition. However, this
time, we have a substitution operation. In summary, we have done less
work in defining $\LambdaTFreeD$ but we have pointed out its
structure, so, rightfully, we have been rewarded in our effort. The
lesson is that, when structure there is, one should be allowed to get
it for free.
