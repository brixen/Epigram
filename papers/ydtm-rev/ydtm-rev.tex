\documentclass{fundam}

\usepackage{url,fontlock,alltt,latexsym}
%%%%%%%%%%%%%%%%%%%%%%%%%%%%%%%%%%%%%%%%%%%%%%%%%%%%%%%%%%%%%%%%%%%%%%%%%%%%%
%%% Inference Rules (some ancient macros by Conor)                        %%%
%%%%%%%%%%%%%%%%%%%%%%%%%%%%%%%%%%%%%%%%%%%%%%%%%%%%%%%%%%%%%%%%%%%%%%%%%%%%%

\newlength{\rulevgap}
\setlength{\rulevgap}{0.05in}
\newlength{\ruleheight}
\newlength{\ruledepth}
\newsavebox{\rulebox}
\newlength{\GapLength}
\newcommand{\gap}[1]{\settowidth{\GapLength}{#1} \hspace*{\GapLength}}
\newcommand{\dotstep}[2]{\begin{tabular}[b]{@{}c@{}}
                         #1\\$\vdots$\\#2
                         \end{tabular}}
\newlength{\fracwid}
\newcommand{\dotfrac}[2]{\settowidth{\fracwid}{$\frac{#1}{#2}$}
                         \addtolength{\fracwid}{0.1in}
                         \begin{tabular}[b]{@{}c@{}}
                         $#1$\\
                         \parbox[c][0.02in][t]{\fracwid}{\dotfill} \\
                         $#2$\\
                         \end{tabular}}
\newcommand{\Rule}[2]{\savebox{\rulebox}[\width][b]                         %
                              {\( \frac{\raisebox{0in} {\( #1 \)}}       %
                                       {\raisebox{-0.03in}{\( #2 \)}} \)}   %
                      \settoheight{\ruleheight}{\usebox{\rulebox}}          %
                      \addtolength{\ruleheight}{\rulevgap}                  %
                      \settodepth{\ruledepth}{\usebox{\rulebox}}            %
                      \addtolength{\ruledepth}{\rulevgap}                   %
                      \raisebox{0in}[\ruleheight][\ruledepth]               %
                               {\usebox{\rulebox}}}
\newcommand{\Case}[2]{\savebox{\rulebox}[\width][b]                         %
                              {\( \dotfrac{\raisebox{0in} {\( #1 \)}}       %
                                       {\raisebox{-0.03in}{\( #2 \)}} \)}   %
                      \settoheight{\ruleheight}{\usebox{\rulebox}}          %
                      \addtolength{\ruleheight}{\rulevgap}                  %
                      \settodepth{\ruledepth}{\usebox{\rulebox}}            %
                      \addtolength{\ruledepth}{\rulevgap}                   %
                      \raisebox{0in}[\ruleheight][\ruledepth]               %
                               {\usebox{\rulebox}}}
\newcommand{\Axiom}[1]{\savebox{\rulebox}[\width][b]                        %
                               {$\frac{}{\raisebox{-0.03in}{$#1$}}$}        %
                      \settoheight{\ruleheight}{\usebox{\rulebox}}          %
                      \addtolength{\ruleheight}{\rulevgap}                  %
                      \settodepth{\ruledepth}{\usebox{\rulebox}}            %
                      \addtolength{\ruledepth}{\rulevgap}                   %
                      \raisebox{0in}[\ruleheight][\ruledepth]               %
                               {\usebox{\rulebox}}}
\newcommand{\RuleSide}[3]{\gap{\mbox{$#2$}} \hspace*{0.1in}               %
                            \Rule{#1}{#3}                          %
                          \hspace*{0.1in}\mbox{$#2$}}
\newcommand{\AxiomSide}[2]{\gap{\mbox{$#1$}} \hspace*{0.1in}              %
                             \AxiomB{#2}                            %
                           \hspace*{0.1in}\mbox{$#1$}}
\newcommand{\RULE}[1]{\textbf{#1}}
\newcommand{\hg}{\hspace{0.2in}}

\input{library.ltx}
\NatPackage

\begin{document}
\title{Why Dependent Types Matter}
\author{Thorsten Altenkirch, Conor McBride and James McKinna}
\maketitle

\begin{abstract}
  We exhibit the rationale behind the design of Epigram, a dependently
  typed programming language and interactive program development
  system, using refinements of a well known program---merge sort---as a
  running example. We discuss its relationship with other proposals
  to introduce aspects of dependent types into functional programming
  languages and sketch some topics for further work in this area.
\end{abstract}

\section{Introduction}

Types matter. That's what they're for---to classify data with respect
to criteria which matter: how they should be stored in memory, whether
they can be safely passed as inputs to a given operation, even who is
allowed to see them. Dependent types are types expressed in terms of
data, explicitly relating their inhabitants to that data. As such,
they enable you to express more of what matters about data.  While
conventional type systems allow us to validate our programs with
respect to a fixed set of criteria, dependent types are much more
flexible, they realize a continuum of precision from the basic
assertions we are used to expect from types up to a complete
specification of the program's behaviour. It is the programmer's
choice to what degree he wants to exploit the expressiveness of such a
powerful type discipline. While the price for formally certified
software may be high, it is good to know that we can pay it in
installments and that we are free to decide how far we want to go.
Dependent types reduce certification to type checking, hence they
provide a means to convince others that the assertions we make about
our programs are correct. Dependently typed programs are, by their
nature, proof carrying code \cite{necula1996,hamid03:safpcc}.

%Dependent
%types are better at mattering on your behalf, and that is why we hope
%they might matter to you.

Functional programmers have started to incorporate many aspects of
dependent types into novel type systems using \emph{generalized
  algebraic data types} and \emph{singleton types}. Indeed, we share
Sheard's vision \cite{sheard-04} of closing the \emph{semantic gap}
between programs and their properties.
While Sheard's language $\Omega$mega approaches this goal by an
evolutionary step from current functional languages like Haskell, we
are proposing a more radical departure with Epigram, exploiting what
we have learnt from proof development tools like LEGO and COQ.

Epigram is a full dependently typed programming language defined by
McBride and McKinna \cite{ConorJames:vfl}, drawing on experience with
the LEGO system. McBride has implemented a prototype which is
available together with basic documentation
\cite{epigram-man,epigram-afp} from the Epigram homepage.%
\footnote{Currently {\tt http://sneezy.cs.nott.ac.uk/epigram/}.}
The prototype implements most of
the features discussed in this article, and we are continuing to
develop it to close the remaining gaps, improve performance and add
new features. Brady has implemented a compiler
\cite{edwin:thesis,edwin.conor.james:family.indices} for the Epigram
language, providing important technology for producing efficient code
from dependently typed programs.

\newcommand{\general}{\mathrm{g}\RW{eneral}}
\newcommand{\nil}{\DC{nil}}
\newcommand{\cons}{\mbox{\(\;\DC{:\!:}\;\)}}
\newcommand{\tup}{\;\times\;}

\begin{figure}[htb]
\[\AR{
\RW{data}\;\;
\Axiom{\TC{Nat}\hab\Type}
\;\;
\RW{where}
\;\;
\Axiom{\DC{0}\hab\TC{Nat}}
\;\;
\Rule{\vn\hab\TC{Nat}}
     {\DC{1+}\;\vn\hab\TC{Nat}}
\medskip \\
%
\RW{data}\;\;
\Axiom{\TC{Order}\hab\Type}
\;\;
\RW{where}
\;\;
\Axiom{\DC{le},\DC{ge}\hab\TC{Order}}
\medskip \\
%
\RW{data}\;\;
\Rule{\vX\hab\Type}
     {\TC{List}\;\vX\hab\Type}
\;\;
\RW{where}
\;\;
\Axiom{\DC{nil}\hab\TC{List}\;\vX}
\;\;
\Rule{\vx\hab\vX \quad
      \vxs\hab\TC{List}\;\vX}
     {\vx\cons\vxs\hab\TC{List}\;\vX}
\medskip \\
%
\RW{let}\;\;
\Rule{\vx,\vy\hab\TC{Nat}}
     {\FN{order}\;\vx\;\vy\hab\TC{Order}}
\qquad
\begin{array}[t]{@{}l@{\;}c@{\;}c@{\;}l}
\FN{order} & \vx & \vy &\BY\RW{rec}\;\vx \\
\;\;\FN{order} & \DC{0} & \vy & \cq\DC{le} \\
\;\;\FN{order} & (\DC{1+}\vx\Pp) & \DC{0} & \cq\DC{ge} \\
\;\;\FN{order} & (\DC{1+}\vx\Pp) & (\DC{1+}\vy\Pp) &
        \cq\FN{order}\;\vx\Pp\;\vy\Pp \\ 
\end{array}
\medskip \\
%
\RW{let}\;\;\AR{
\Rule{\vxs,\vys\hab\TC{List}\:\TC{Nat}}
     {\FN{merge}\;\vxs\;\vys\hab\TC{List}\;\TC{Nat}}
\smallskip \\
\begin{array}{@{}l@{\;}c@{\;}c@{\;}l}
\FN{merge}&\vxs&\vys&\BY\RW{rec}\;\vxs \\
\;\;  \FN{merge} & \nil & \vys & \cq \vys \\
\;\;  \FN{merge} & (\vx \cons \vxs\Pp) & \vys & \BY \RW{rec}\;\vys \\
\;\;  \;\;\FN{merge} & (\vx \cons \vxs\Pp) & \nil & \cq \vxs \\
\;\;  \;\;\FN{merge} & (\vx \cons \vxs\Pp) & (\vy \cons \vys\Pp) &
    \begin{array}[t]{|c@{\;}l}
      \FN{order}\;\vx\;\vy \\
      \DC{le} & \cq \vx \cons \FN{merge}\;\vxs\Pp\;\vys \\
      \DC{ge} & \cq \vy \cons \FN{merge}\;\vxs\;\vys\Pp \\
    \end{array}
\end{array}
}
\medskip\\
%
\RW{let}\;\;
\Rule{\vxs\hab\TC{List}\;\vX}
     {\FN{deal}\;\vxs\hab\TC{List}\;\vX \tup \TC{List}\;\vX}
\qquad
\begin{array}[t]{@{}l@{\;}c@{\;}l}
  \FN{deal} & \vxs & \BY \RW{rec}\;\vx \\
\;\;  \FN{deal} & \nil & \cq (\nil;\;\nil) \\
\;\;  \FN{deal} & (\vx \cons \nil) & \cq (\vx \cons \nil;\;\nil) \\
\;\;  \FN{deal} & (\vy \cons \vz \cons \vxs) &
    \begin{array}[t]{|c@{\;}l}
      \FN{deal}\;\vxs \\
      (\vys;\;\vzs) & \cq (\vy\cons\vys;\;\vz\cons\vzs)
    \end{array}
\end{array}
\medskip\\
%
\RW{let}\;\;\AR{
\Rule{\vxs\hab\TC{List}\;\TC{Nat}}
     {\FN{sort}\;\vxs\hab\TC{List}\;\TC{Nat}}
\qquad
\begin{array}[t]{@{}l@{\;}c@{\;}l}
  \FN{sort} & \vxs &\AR{
    \BY\general \\
    \begin{array}[t]{|c@{\;}l}
      \FN{deal}\;\vxs \\
      (\vys;\;\nil) & \cq \vys \\
      (\vys;\;\vz\cons\vzs) & \cq \FN{merge}\;(\FN{sort}\;\vys)
                                    \;(\FN{sort}\;(\vz\cons\vzs))
    \end{array}}
\end{array}
}
}\]
\caption{Merge-sort, generally\label{fig:mergesortgenerally}}
\end{figure}

In this article we exhibit the rationale behind Epigram's design,
using refinements of a well known program---merge sort---as a running
example.  Our starting point is the implementation shown in Figure
\ref{fig:mergesortgenerally}: it is written in Epigram, but it could have
been written in any functional language. We start by
revisiting the question of totality versus partiality in section
\ref{sec:partial}, showing how $\FN{sort}$ can be made
structurally recursive. Section \ref{sec:static} continues by
addressing the problem of how to maintain static invariants which is
illustrated by implementing a sized $\FN{sort}$. In section
\ref{sec:evidence} we how to use dependent types to maintain static
invariants about dynamic data, which is illustrated by implementing a
version of $\FN{sort}$ which certifies that its output is in order.
We look behind the curtains of the Epigram system in section
\ref{sec:tools} and discuss how dependent types support an extensible
system of programming patterns which include, but are not restricted to,
constructor case analysis and constructor guarded recursion; we also
discuss the design of the
interactive aspect of the Epigram programming environment.  Finally,
we describe areas of further work in \ref{sec:further} and
summarize our conclusions in section \ref{sec:concl}.

Before embarking on this programme let's quickly describe Epigram's
syntax (which may look unusual to functional programmers who have
grown up with languages like ML or Haskell), taking our
implementation of merge sort as an example.  Epigram uses a
two-dimensional syntax for declarations, based on natural deduction
rules, a choice which pays off once type
dependencies become more intricate.  E.g. the declaration of the constructor
$\DC{1+}$ for the datatype%
\footnote{While the declaration of $\TC{Nat}$ provides a convenient interface to the
type of natural numbers, there is no need to implement them using
a unary representation internally. Moreover, we shall also exploit the
syntactic
convenience of using the usual decimal notation to refer to elements of $\TC{Nat}$.}
$\Nat$ is equivalent to writing
$\DC{1+}\hab \TC{Nat}\To\TC{Nat}$.
The conclusion of a declaration visualizes its use---this is important
because parameters of any type can be implicit in Epigram. For example,
when declaring $\DC{nil}$ we do not declare $\vX$---Epigram figures out
from its usage that it must inhabit $\Type$, the type of types and internally
completes the declaration
\[ 
\Rule{\vX\hab\Type} {\DC{nil}_{\vX} \hab\TC{List}\;\vX}
\]
When presented with
$\DC{nil}$, Epigram uses the well known technology established by
Damas and Milner \cite{damas.milner:principal} to figure out the value
of this implicit parameter.  We can also make an implicit parameter
explicit by writing it as a subscript, eg., 
$\DC{nil}_{\TC{Nat}} \hab\TC{List}\;\TC{Nat}$.

Epigram programs are tree-structured: each node has a `left-hand side'
indicating a \emph{programming problem} and a `right hand side'
indicating either how to solve it outright by returning a value, $\TO
t$ ($\TO$ is pronounced `return'), or how to reduce it to subproblems
by deploying a programming
pattern indicated by the $\BY$ symbol ($\BY$ is pronounced `by').
Programming patterns include
structural recursion, like $\RW{rec}\;\vx$ in $\FN{order}$, $\general$
recursion and also case analysis. We suppress nonempty case analyses
for the sake of brevity---they can be recovered by the standard algorithm
\cite{augustsson:compiling.patterns}. If, as in the
$\FN{merge}$ function, we need to analyse the result of an
intermediate computation, we bring it to the left with the $|cdots$
construct ($|$ is pronounced `with').%
 \footnote{The current prototype doesn't yet support
  the suppression of $\RW{case}$ and it doesn't implement the $|$ notation.
  As a consequence its code is more verbose.} 
Here, $\FN{order}$ decides the
$\leq$ relation, returning a value in the enumeration $\TC{Order}$.

We have given a complete program, but Epigram can also typecheck and
evaluate incomplete programs with unfinished sections sitting in
\emph{sheds}, $\SHEDC{\cdots}$, where the typechecker is forbidden to
tread. Programs can be developed interactively, with the machine
showing the available context and the required type, wherever the
cursor may be. Moreover, it is Epigram which generates the left-hand
sides of programs from type information, each time a problem is
simplified with $\BY$ on the right.


\section{Related Work}

Dependent types are a central feature of Martin-L\"of's Type
Theory \cite{martinloef:intuitionistic,nordstrom:programming}, integrating constructive
logic and strongly typed functional programming. Type Theory and its
impredicative extension, the Calculus of Constructions, inspired many
type-based proof systems, such as NUPRL \cite{nuprl-book}, LEGO
\cite{luo.pollack:legomanual} and the widely used COQ
\cite{CoqManualV8}. Magnusson's ALF system \cite{magnusson.nordstrom:alf}
%\cite{magnusson94implementation} 
was not only the first system to
implement inductive families \cite{dybjer:families} and pattern
matching for dependent types \cite{coquand:patternmatching} and it
also pioneered the interactive style of type-driven program and proof
development which inspired Epigram.

Xi and Pfenning's DML (for Dependent ML) \cite{DML99} was an
impressive experiment in exploiting dependent types for a real
functional programming language. DML, like other \emph{Applied Type
  Systems} \cite{ATStypes03}, separates the world of indexing
expressions and programs, thereby 
keeping types unaffected from potentially badly behaved programs.  In contrast to
DML, Augustsson's implementation of the Cayenne language
\cite{augustsson:cayenne}, which also inspired the AGDA proof system
\cite{coquand99structured}, uses full dependency and doesn't
differentiate between static and dynamic types.

Nested types provide a poor man's approach to many indexed
data structures such as square matrices or well scoped $\lambda$-terms,
e.g. see \cite{Hin00Man}. McBride suggested a more general approach
\cite{conor:faking}, exploiting that Haskell's class system provides a
static logic programming language. Having realized the power of
indexed data structures, Cheney and Hinze \cite{ChH03Pha} proposed
to extend the type system to introduce a rich language of index
expressions leading to \emph{generalized
  algebraic datatypes} (GADTs), which are basically inductive families
in a programming language setting \cite{sheard-04,peyton-jones-wobbly}.

%Sheard's $\Omega$mega
%\cite{sheard-04} and \emph{wobbly types} \cite{peyton-jones-wobbly}
%are based on GADTs.


%While Cayenne
%was an important, and mostly successful experiment, Epigram goes beyond
%Cayenne in a number if important points, which: Epigram introduces a notion of
%totality which leads to a better behaved symbolic evaluator, it provides 
%support for inductively defined families and it provides a type-driven
%interactive development system, which seems essential to deal with the
%complexity of dependently type programming. We will discuss all these 
%issues in more detail below.

\section{Why we are Partial to Totality}
\label{sec:partial}

A popular myth, still to be found in learned articles and referee reports on
grant applications, is that dependent types and general recursion do not
mix. This is a misunderstanding, but it's an understandable one.
Let us examine the facts, beginning with the typing rule for application:

\[
\Rule{\vf\hab\forall\vx:\vS\TO\vT[\vx]\quad\vs\hab\vS}
     {\vf\;\vs\hab\vT[\vs]}
\]

It's clear from the premises that, as ever, to check an application we
need to compare the function domain and the argument type. It's also
clear from the rule's conclusion that these types may contain
expressions. If computation is to preserve typings, then
$\vf\;(\red{2}\green{+}\red{2})$ should have the same type as
$\vf\;\red{4}$, so $\vT[\red{2}\green{+}\red{2}]$ must \emph{be} the
same type as $\vT[\red{4}]$. To decide typechecking, we therefore need
to decide some kind of equivalence \emph{up to computation}. There are
various approaches to this problem.


\textbf{The totalitarian approach.}$\quad$
%
Some proof systems based on intensional type theory, including Coq and
Lego, forbid general recursion. As all computations terminate,
equality of types is just syntactic equality of their normal forms.
Decidability of typechecking is a consequence, but it's not the
primary motivation for this choice. As a proof method, general
recursion is wholly bogus---its type, $\forall\vP\TO(\vP\to\vP)\To\vP$
is a blatant lie. General recursion, non-exhaustive
patterns and other such non-total programming features compromise the
logical soundness of a proof system.
Trust is more important than termination in proof checking.

Of course, even in the absence of general recursion, it's possible to
write programs which take a long time---e.g. checking all the basic
configurations of four-colouring problems. That doesn't make dependent
typechecking necessarily intractable: the complexity of the programs
in types is entirely controlled by the programmer---the more you say,
the more you pay, but the more also you can be repaid in terms of
genericity, or precision, or brevity. Georges Gonthier's proof of the
four colour theorem \cite{gonthier:four} is made \emph{tractable} by
type-level computation, because it lets him avoid generating and
checking a separate proof for each configuration---the latter approach
would have involved at least as much work for the computer and a great
deal more work for Georges!


\textbf{The libertarian approach.}$\quad$
%
It's reasonable to allow arbitrary recursion in type-level programs,
provided you have some sort of cut-off mechanism which interrupts
loops when they happen in practice. This is the approach taken by the
Agda proof system, Cayenne and by Haskell with
`undecidable instances'---Haskell's overloading resolution amounts
to executing a kind of `compile-time Prolog'. Agda restores logical
soundness by a separate termination check, performed after typechecking.
The basic point is that you include recursive programs in types at your
own risk: mostly they're benign and typechecking behaves sensibly.

The legendary `loopiness' of dependent typechecking stems from the
particular way the libertarian approach was implemented in Cayenne. It's
perfectly reasonable to implement recursion via fixpoints in the
value-only run-times of functional programming languages, but Lennart
Augustsson's attempt to lift this to the open terms used in dependent
typechecking had an unintended consequence---when even a structurally
recursive function is stuck on a non-constructor input, you can still
expand the fixpoint, potentially putting the system into a
spin: this is intolerable, but it's not inevitable, as the other
`libertarian' systems have shown.

The pragmatic advantage of libertarianism is that we don't have to
care why a program works in order to start playing with it---it
seems a shame to ban certain programs at run-time just to protect
ourselves at compile-time. However, it also seems a shame to forsake
the certainties which totalitarianism supports.

\textbf{The mixed economy.}$\quad$
%
As both of the above have their merits, it seems
sensible to support both, as long as programs are clear about where
they stand.  Dependent types can be used to express schemes of
recursion in the form of induction principles, such as
constructor-guarded recursion for a given inductive datatype. In
Epigram, programs invoke recursion schemes explicitly---each recursive
call must be translatable to an `inductive hypothesis' arising from
such a scheme. Any program which employs only constructor-guarded
recursion, or some fancier scheme derived therefrom, is guaranteed
to be total.

However, general recursion also takes the form of an induction
principle (with a particularly generous inductive hypothesis). We can
make general recursion available, but only by explicit appeal to this
principle. If we suppress the compile-time computational behaviour of
general recursion, we can preserve decidability of typechecking, leaving
computation by explicitly total programs activated.

This opens a range of possibilities, as shown by the implementation of
merge-sort shown in figure \ref{fig:mergesortgenerally}. The
$\FN{order}$ing of the natural numbers with respect to $\le$ is a
one-step constructor guarded recursion. To $\FN{merge}$ two sorted
lists, we need a combination of one-step recursions on each
argument---for each successive element of the first list, we step
along the second list until its rightful place appears. The
$\FN{deal}$ing out of a list into two lists of (roughly) half the
length here exploits a two-step recursion, but this still fits within
the constructor-guarded scheme indicated by the keyword $\RW{rec}$.
However, $\FN{sort}$ performs a peculiar recursion via
$\FN{deal}$---it's not obvious yet how to justify this, so for now we give
up and use $\general$.\footnote{To see how subtle the justification
can be, try swapping the case analysis on $\FN{deal}\;\vxs$ so the
patterns are $(\nil;\;\vys)$ and $(\vz\cons\vzs;\;\vys)$.}


This approach rules nothing out, but it still allows us to observe
guaranteed totality where we can see the explanation. The notational
overhead is not large and could be reduced still further if we were to
install an Agda-style termination checker, inferring simple
explanations when the user omits them.  We could go even further,
pushing the total-versus-general distinction into types by treating
general recursion as an effect which we lock away safely in a monad.
For practical purposes, we should need a better notation for monadic
programming in the \emph{functional} style.  Both of these are active
topics of research.


\subsection{Totality is Good for more than the Soul}

The warm fuzzy feeling you get when you've persuaded your program to live
in a total programming language should not be underestimated. It's a
strong static guarantee---you can say that you've written a \emph{function}
without having to pretend that $\bot$ is a value.
But warm fuzzy feelings don't pay the rent: what are the practical benefits
of virtue?

Randy Pollack has often said to us `the point of writing a proof in a
strongly normalizing calculus is that you don't need to normalize it'.
When you have an expression of a given type in a total language, you
can guarantee that it will compute to a value: if you don't care what
that value is---as is usually the case with a proof---you have no need
to perform the computation. Now we know that we can integrate proofs
of logical properties into our programs at \emph{no run-time cost}.

This is particularly important with proofs of equations. Equality is
defined as follows:

\newcommand{\EQ}{\;\blue{=}\;}
\[
\RW{data}\;\;
\Rule{\vs\hab\vS\quad\vt\hab\vT}
     {\vs\EQ\vt\hab\Type}
\;\;\RW{where}\;\;
\Axiom{\DC{refl}\hab\vt\EQ\vt}
\]

An equation between types induces a coercion from one
to the other which is trivial if the proof is $\DC{refl}$.

\[
\RW{let}\;\;
\Rule{\vQ\hab\vS=\vT\quad \vs\hab\vS}
     {\{\vQ\}\vs\hab\vT}
\qquad
\{\DC{refl}\}\vt \cq\vt
\]

In a partial setting, we need to run $\vQ$ to check that it's
$\DC{refl}$, because trusting a false equation (like
$\TC{Nat}=\TC{Nat}\to\TC{Nat}$) induces a run-time type error.  When
$\vQ$ is total, the compiler can erase
$\{\vQ\}$. 
Contrast this with the proposal to represent type equations 
% current vogue for representing 
in Haskell by isomorphisms, e.g. \cite{BaSw02},  --- even though good programmers
always try to ensure that these functions turn out at run-time to be
functorial liftings of $\texttt{id}$, there is no way to guarantee
this to the compiler, so the isomorphisms must actually be executed.

Moreover, an optimising compiler can exploit totality in various
useful ways. The evaluation strategy of a total program is irrelevant,
so they can be run as strictly or lazily as a heuristic analysis might
suggest. Optimisations like replacing $\texttt{foldr}$ with $\texttt{foldl}$,
which only work with finite values, can be applied safely. Specialisation
by partial evaluation is untroubled by $\bot$. Further the explicit marking
of a program as structurally recursive is a clear invitation to apply
fusion techniques.

All in all, there is no bliss to be had from ignorance of totality;
there is no disadvantage to wisdom.


\subsection{Defusing General Recursion}

A recursive function which happens to be total will generally exploit some
sort of structure, but perhaps not the `native' structure of its inductive
arguments. The totality of the function can be made clear if that structure,
whatever it is, can be brought into the open and expressed inductively.

A typical `divide and conquer' recursion often becomes structural by
introducing an intermediate data structure which represents the
division of the input, built by a process of insertion, and collapsed
structurally by `conquering'. This intermediate data structure thus
corresponds to the control structure built by the original recursion,
which can be reconstructed by fusing the building with the collapse.

As David Turner observed \cite{turner:strong.functional}, defusing
quick-sort exposes the binary search tree structure.  The standard
example of a non-structural program is actually tree-sort---Rod
Burstall's first example of a structural program
\cite{burstall:induction}!

We play the same game with merge-sort in figure
\ref{fig:mergesortstructurally}. The `divide' phase deals out the
input to each sort into two inputs for sub-sorts (roughly) half the
size; the `conquer' phase merges the two sorted outputs into one
(roughly) twice the size. If we build a tree representing the sorting
processes, we find that each node deals its inputs fairly to its
subnodes, with the leaves having none or one.

\newcommand{\pev}{\DC{p0}}
\newcommand{\poo}{\DC{p1}}

\begin{figure}[htb]
\[\AR{
\RW{data}\;\;
\Axiom{\TC{Parity}\hab\Type}
\;\;
\RW{where}
\;\;
\Axiom{\pev,\poo\hab\TC{Parity}}
\medskip\\
%
\RW{data}\;\;
\Rule{\vX\hab\Type}
     {\TC{DealT}\;\vX\hab\Type}
\;\;
\RW{where}
\;\;
\Axiom{\DC{empT}\hab\TC{DealT}\;\vX}
\;\;
\Rule{\vx\hab\vX}
     {\DC{leafT}\;\vx\hab\TC{DealT}\;\vX}
\;\;
\Rule{\vp\hab\TC{Parity}\quad\vl,\vr\hab\TC{DealT}\;\vX}
     {\DC{nodeT}\;\vp\;\vl\;\vr \hab \TC{DealT}\;\vX}
\medskip\\
%
\RW{let}\;\;
\Rule{\vx\hab\vX\quad\vt\hab\TC{DealT}\;\vX}
     {\FN{insertT}\;\vx\;\vt\hab\TC{DealT}\;\vX}
\qquad
\begin{array}[t]{@{}l@{\;}c@{\;}c@{\;}l}
  \FN{insertT} & \vx & \vt & \BY \RW{rec}\;\vt \\
\;\;  \FN{insertT} & \vx & \DC{empT} & \cq \DC{leafT}\;\vx \\
\;\;   \FN{insertT} & \vx & (\DC{leafT}\;\vy) & \cq
    \DC{nodeT}\;\pev\; (\DC{leafT}\;\vy)\;(\DC{leafT}\;\vx)\\
\;\;   \FN{insertT} & \vx & (\DC{nodeT}\;\pev\;\vl\;\vr) & \cq
    \DC{nodeT}\;\poo\;(\FN{insertT}\;\vx\;\vl)\;\vr \\
\;\;   \FN{insertT} & \vx & (\DC{nodeT}\;\poo\;\vl\;\vr) & \cq
    \DC{nodeT}\;\pev\;\vl\;(\FN{insertT}\;\vx\;\vr)
\end{array}
\medskip\\
%
\RW{let}\;\;
\Rule{\vxs\hab\TC{List}\;\vX}
     {\FN{dealT}\;\vxs\hab\TC{DealT}\;\vX}
\qquad
\begin{array}[t]{@{}l@{\;}c@{\;}l}
  \FN{dealT} & \vxs & \BY \RW{rec}\;\vxs \\
\;\;  \FN{dealT} & \nil & \cq \DC{empT}\\
\;\;  \FN{dealT} & (\vx\cons\vxs) & \cq\FN{insertT}\;\vx\;(\FN{dealT}\:\vxs)
\end{array}
\medskip\\
%
\RW{let}\;\;
\Rule{\vt\hab\TC{DealT}\;\TC{Nat}}
     {\FN{mergeT}\;\vt\hab\TC{List}\;\TC{Nat}}
\qquad
\begin{array}[t]{@{}l@{\;}c@{\;}l}
\FN{mergeT} & \vt & \BY \RW{rec}\;\vt \\
\;\;\FN{mergeT} & \DC{empT} & \cq \nil \\
\;\;\FN{mergeT} & (\DC{leafT}\;\vx) & \cq \vx\cons\nil \\
\;\;\FN{mergeT} & (\DC{nodeT}\;\vp\;\vl\;\vr) & \cq
  \FN{merge}\;(\FN{mergeT}\;\vl)\;(\FN{mergeT}\;\vr)
\end{array}
\medskip\\
%
\RW{let}\;\;
\Rule{\vxs\hab\TC{List}\;\TC{Nat}}
     {\FN{sort}\;\vxs\hab\TC{List}\;\TC{Nat}}
\qquad
  \FN{sort} \; \cq \FN{mergeT}\cdot\FN{dealT}
}\]
\caption{Merge-sort, structurally
    (with $\FN{merge}$ as before)\label{fig:mergesortstructurally}}
\end{figure}

Correspondingly, a `dealing' is a binary tree with leaves of weight
zero or one, and nodes off balance by at most one---if we keep a
parity bit at each node, we shall know into which subnode the next
element should be dealt. In effect, we defuse the general recursion as
a composition of folds, with `divide' replacing $[\nil,(\!\cons\!)]$
by $[\DC{empT},\FN{insertT}]$ and `conquer' ($\FN{mergeT}$) replacing
$[\DC{empT},\DC{leafT},\DC{nodeT}]$ by
$[\nil,(\!\cons\nil),\lambda\vp\TO\FN{merge}]$.

Of course, there is no panacea: there are all sorts of ways to write
programs which conceal the structures by which they operate. A
dependent type system provides a rich language of datatypes in which
to expose these structures---other examples include evaluation for the
simply typed $\lambda$-calculus with primitive recursion on
$\TC{Nat}$, which gives a denotational semantics to both types and
terms, and first-order unification, which combines recursion on the
number of available variables with recursion on terms over those
variables.

If you care about totality, it's often easier to write a new program
which works with the relevant structure than to write a proof which
finds the structure which a $\general$-program is hiding. The best way
to tidy up the mess is not to make it in the first place, if you can
possibly avoid it.


\section{Maintaining Invariants by Static Indexing}
\label{sec:static}

An important aspect of many recent and innovative type systems is the
idea of \emph{indexing} datatypes in order to express and enforce
structural invariants. There are various ways this can be achieved: in
Epigram, we define \emph{datatype families} in the style of the Alf
system~\cite{dybjer:families}.  A good introductory example is given
by the \emph{vectors}---lists indexed with their length.

\newcommand{\VEC}{\TC{Vec}}
\newcommand{\vnil}{\DC{vnil}}
\newcommand{\vcons}{\DC{vcons}}
\[
\RW{data}\;\;
\Rule{\vn\hab\Nat\quad\vX\hab\Type}
     {\VEC\;\vn\;\vX\hab\Type}
\;\;
\RW{where}
\;\;
\Axiom{\vnil\hab\VEC\;\DC{0}\;\vX}
\;\;
\Rule{\vx\hab\vX\quad
      \vxs\hab\VEC\;\vn\;\vX}
     {\vcons\;\vx\;\vxs\hab\VEC\;(\DC{1+}\vn)\;\vX}
\]

Case analysis on inductive families~\cite{coquand:patternmatching}
involves \emph{unifying} the type of the scrutinee with the type of
each possible constructor pattern---those patterns for which
constructors clash are rejected as impossible, as in this notorious
example:

\[
\RW{let}\;\;
\Rule{\vxs\hab\VEC\;(\DC{1+}\vn)\;\vX}
     {\FN{vtail}\;\vxs\hab\VEC\;\vn\;\vX}
\qquad
\FN{vtail}\;(\vcons\;\vx\;\vxs) \;\cq\vxs
\]

Vectors admit operations which enforce and
maintain length invariants, such as this `vectorized application':

\newcommand{\vapp}{\;^{\FN{@}}\;}
\[
\RW{let}\;\;
\Rule{\vfs\hab\VEC\;\vn\;(\vS\to\vT)\quad
      \vsss\hab\VEC\;\vn\;\vS}
     {\vfs\vapp\vsss\hab\VEC\;\vn\;\vT}
\qquad
\AR{
\vfs\vapp\vsss\;\BY\RW{rec}\;\vfs \\
\;\;\begin{array}{l@{}c@{}l@{\;}l}
       \vnil & \vapp & \vnil & \cq \vnil \\
       \vcons\;\vf\;\vfs\Pp & \vapp & \vcons\;\vs\;\vsss\Pp & \cq
         \vcons\;(\vf\;\vs)\;(\vfs\Pp\vapp\vsss\Pp)
    \end{array}
}
\]

Sometimes, we need some operations on the indices in order to express
the type of an operation on indexed data. Concatenating vectors is a simple
example

\newcommand{\plus}{\;\FN{+}\;}
\newcommand{\vconc}{\;\FN{+\!+}\;}
\[\begin{array}{l@{\qquad}l}
\RW{let}\;\;
\Rule{\vm,\vn\hab\TC{Nat}}
     {\vm\plus\vn\hab\TC{Nat}}
&
\AR{
\vm\plus\vn\;\BY\RW{rec}\;\vm \\
\;\;\begin{array}{l@{}c@{}l@{\;}l}
       \DC{0} & \plus & \vn & \cq \vn \\
       (\DC{1+}\;\vm\Pp) & \plus & \vn & \cq\DC{1+}\;(\vm\Pp\plus\vn)
    \end{array}
}
\medskip\\
\RW{let}\;\;
\Rule{\vxs\hab\VEC\;\vm\;\vX\qquad\vys\hab\VEC\;\vn\;\vX}
     {\vxs\vconc\vys\hab\VEC\;(\vm\plus\vn)\;\vX}
&
\AR{
\vxs\plus\vys\;\BY\RW{rec}\;\vxs \\
\;\;\begin{array}{l@{}c@{}l@{\;}l}
       \vnil & \vconc & \vys & \cq \vys \\
       \vcons\;\vx\;\vxs\Pp & \vconc & \vys &
         \cq\vcons\;\vx\;(\vxs\Pp\vconc\vys)
    \end{array}
}
\end{array}\]

Note the importance of the index unification in the above example---it's
the instantiation of the first argument's length with $\DC{0}$ or
$(\DC{1+}\;\vm\Pp)$ which enables the length of the concatenation to
compute down to the length of the vectors we actually return in each case.

The fact that the length is some kind of data allows us to write
generic operations by computation over it. This example computes a
constant vector of the required size.

\[
\RW{let}\;\;
\Rule{\vx\hab\vX}
     {\FN{vec}_{\vn}\;\vx\hab\VEC\;\vn\;\vX}
\qquad
\AR{
\FN{vec}_{\vn}\;\vx\;\BY\RW{rec}\;\vn \\
\;\;\begin{array}{l@{\;}l@{\;}l}
      \FN{vec}_{\DC{0}} & \vx & \cq \vnil \\
      \FN{vec}_{(\DC{1+}\;\vn\Pp)} & \vx &
         \cq \vcons\;\vx\;(\FN{vec}_{\vn\Pp}\;\vx)
    \end{array}
}
\]

We write $\FN{vec}$'s length argument as a subscript to indicate that
it is usually to be left implicit when the $\FN{vec}$ function is
used.  For example, we can map a function $\vf$ across a vector $\vxs$
just by writing $\FN{vec}\:\vf\vapp\vxs$, because the type of
$\!\vapp\!$ will require the output of $\FN{vec}$ to have the same
length as $\vxs$.  The technology we have inherited from Damas and
Milner is now being used to infer \emph{value} parameters as well as
types.  The behaviours of $\FN{vec}$ and $\!\vapp\!$ combine
conveniently to give us `vectorized applicative programming' with size
invariants quietly maintained: transposition shows this in action.

\[
\RW{let}\;\;
\Rule{\VV{xij}\hab\VEC\;\vi\;(\VEC\;\vj\;\vX)}
     {\FN{xpose}\;\VV{xij}\hab\VEC\;\vj\;(\VEC\;\vi\;\vX)}
\qquad
\AR{
  \FN{xpose}\;\VV{xij}\;\BY\RW{rec}\;\VV{xij}\\
  \begin{array}{l@{\;}c@{\;}l}
    \FN{xpose} & \vnil & \cq \FN{vec}\;\vnil \\
    \FN{xpose} & (\vcons\;\VV{xj}\;\VV{xi}\Pp\!\vj) & \cq
      \FN{vec}\;\vcons\vapp\VV{xj}\vapp\FN{xpose}\;\VV{xi}\Pp\!\vj
  \end{array}
}
\]


\subsection{Static Indexing and Proofs}

In our definition of $\!\vconc\!$, we were lucky---the computation on
the vectors was in harmony with the computation on the numbers. We are
not always so lucky---if we try to reverse a vector by the usual
accumulating algorithm, we kick against the computational behaviour of
$\!\plus\!$ and the obvious program does not typecheck.

\[
\RW{let}\;\;
\Rule{\vxs\hab\VEC\;\vm\;\vX\qquad\vys\hab\VEC\;\vn\;\vX}
     {\FN{vrevacc}\;\vxs\;\vys\hab\VEC\;(\vm\plus\vn)\;\vX}
\qquad
\AR{
\FN{vrevacc}\;\vxs\;\vys\;\BY\RW{rec}\;\vxs \\
\;\;\begin{array}{l@{\;}c@{\;}l@{\;}l}
       \FN{vrevacc} & \vnil & \vys & \cq \vys \\
       \FN{vrevacc} & (\vcons\;\vx\;\vxs\Pp) & \vys &
         \cq\brownBG{$\FN{vrevacc}\;\vxs\Pp\;(\vcons\;\vx\;\vys)$}
    \end{array}
}
\]

The trouble is that the shaded expression has length
$\vm\Pp\plus(\DC{1+}\;\vn)$, and we require a length of
$\DC{1+}\;(\vm\Pp\plus\vn)$, where $\vxs\Pp$ and $\vys$ have lengths
$\vm\Pp$ and $\vn$ respectively. The fact that these two lengths are
the same does not follow directly from applying the computational
behaviour of $\!\plus\!$, rather it's an algebraic property for which
we can offer an inductive explanation.

\newcommand{\CONG}[1]{\left[#1\right>}
\newcommand{\GNOC}[1]{\left<#1\right]}

\[
\RW{let}\;\;\AR{
\Axiom{\FN{plusSuc}\;\vm\;\vn\hab
       \vm\plus(\DC{1+}\;\vn) \EQ \DC{1+}\;(\vm\plus\vn)}
\smallskip\\
\FN{plusSuc}\;\vm\;\vn\;\BY\RW{rec}\;\vm \\
\;\;\begin{array}{l@{\;}c@{\;}c@{\;}l}
      \FN{plusSuc} & \DC{0} & \vn & \cq\DC{refl} \\
      \FN{plusSuc} & (\DC{1+}\;\vm\Pp) & \vn & \cq
         \CONG{\FN{plusSuc}\;\vm\Pp\;\vn} \\
    \end{array}
}
\]

We write $\CONG{\vq}$ for the proof of an equation
$\vp[\vs]\EQ\vp[\vt]$ where $\vq\hab\vs\EQ\vt$ and
$\GNOC{\vq}$ for the symmetric proof of $\vp[\vt]\EQ\vp[\vs]$
Once we have this proof, we can fix our accumulating reverse:

\[
\RW{let}\;\;\AR{
\Rule{\vxs\hab\VEC\;\vm\;\vX\qquad\vys\hab\VEC\;\vn\;\vX}
     {\FN{vrevacc}_{\vm\;\vn}\;\vxs\;\vys\hab\VEC\;(\vm\plus\vn)\;\vX}
\smallskip\\
\FN{vrevacc}\;\vxs\;\vys\;\BY\RW{rec}\;\vxs \\
\;\;\begin{array}{l@{\;}c@{\;}l@{\;}l}
       \FN{vrevacc} & \vnil & \vys & \cq \vys \\
       \FN{vrevacc}_{(\DC{1+}\:\vm\Pp)\;\vn} & (\vcons\;\vx\;\vxs\Pp) & \vys &
         \cq\{\CONG{\FN{plusSuc}\;\vm\Pp\;\vn}\}\;
            \FN{vrevacc}\;\vxs\Pp\;(\vcons\;\vx\;\vys)
    \end{array}
}
\]

It is perhaps not surprising that to finish the job, we need another
lemma (whose proof is an unremarkable induction):

\[\AR{
\RW{let}\;\;
\Axiom{\FN{plusZero}\;\vn\hab\vn\plus\DC{0} \EQ \vn} \qquad \cdots
\medskip\\
\RW{let}\;\;
\Rule{\vxs\hab\VEC\;\vn\;\vX}
     {\FN{vrev}_{\vn}\;\vxs\hab\VEC\;\vn\;\vX}
\qquad
\FN{vrev}_{\vn}\;\vxs\;\cq \{\CONG{\FN{plusZero}\:\vn}\}\;
                            \FN{vrevacc}\;\vxs\;\vnil
}\]

As we have already mentioned, these proofs are erased at run-time---their
only r\^ole is to make the typechecker aware of more than it can figure
out by computation alone. Perhaps it's undesirable to `repair' programs with
proofs---our $\FN{vrev}$ is certainly more trouble to write than the
reversal of an unsized list, but this is just the work which has to
be done if you want to know that length is preserved. If we don't
care, we don't have to work so hard---we could just return a vector of
\emph{some} length, rather than the \emph{same} length:

\newcommand{\Gp}{\green{'}}
\[
\RW{let}\;\;\AR{
\Rule{\vxs\hab\VEC\;\vm\;\vX\qquad\vys\hab\VEC\;\vn\;\vX}
     {\FN{vrevacc}\Gp\;\vxs\;\vys\hab\exists_{\vl}\TO\VEC\;\vl\;\vX}
\smallskip\\
\FN{vrevacc}\Gp\;\vxs\;\vys\;\BY\RW{rec}\;\vxs \\
\;\;\begin{array}{l@{\;}c@{\;}l@{\;}l}
       \FN{vrevacc}\Gp & \vnil & \vys & \cq \vys \\
       \FN{vrevacc}\Gp & (\vcons\;\vx\;\vxs\Pp) & \vys &
           \cq \FN{vrevacc}\Gp\;\vxs\Pp\;(\vcons\;\vx\;\vys)
    \end{array}
}
\]

The notion of `some length' is expressed via an \emph{implicit
  existential} $\exists_{\vl}\TO\VEC\;\vl\;\vX$. The vector is packed
(inferring the witness) and unpacked (discarding the witness)
automatically. It allows us to recover a less dependent type by hiding
an index---here we recover ordinary lists.  We can write the old
programs with the old precision just as easily as before.

As with totality, we have enough language to negotiate a pragmatic
compromise, adopting more precise methods where the gain is worth the
work.  
%To reject this language is to presuppose the outcome of this
%negotiation in all cases---we are not capable of such courage.  
The
more prudent course, perhaps, is to try to make the unremarkable
proofs as cheap as possible. Xi and Pfenning's approach of equipping
the typechecker with decision procedures for standard problem classes
is a great help in practice: DML has no difficulty discharging the
elementary properties of $\!\plus\!$ we required above. We should certainly
emulate this functionality.

\subsection{Sized Merge-Sort}

We can provide a more substantial example of `the pragmatics of precision' by
rolling out size invariants across our development of merge-sort. We shall
replace the lists by vectors, and we shall seek to ensure that sorting
preserves the length of its input.

How will sizing affect $\FN{merge}$? The output length should be the
sum of the input lengths.

\[
\RW{let}\;\;\AR{
\Rule{\vxs\hab\VEC\;\vm\;\Nat\quad
      \vys\hab\VEC\;\vn\;\Nat}
     {\FN{merge}_{\vm\;\vn}\;\vxs\;\vys\hab\VEC\;(\vm\plus\vn)\;\Nat}
\smallskip \\
\begin{array}{@{}l@{\;}c@{\;}c@{\;}l}
\FN{merge}&\vxs&\vys&\BY\RW{rec}\;\vxs \\
\;\;  \FN{merge} & \vnil & \vys & \cq \vys \\
\;\;  \FN{merge} & (\vcons\;\vx\; \vxs\Pp) & \vys & \BY \RW{rec}\;\vys \\
\;\;  \;\;\FN{merge}_{(\DC{1+}\:\vm\Pp)\;\DC{0}}
        & (\vcons\;\vx\; \vxs\Pp) & \vnil &
        \cq \{\GNOC{\FN{plusZero}\;(\DC{1+}\:\vm\Pp)}\}\;\vxs \\
\;\;  \;\;\FN{merge}_{(\DC{1+}\:\vm\Pp)\;(\DC{1+}\:\vn\Pp)} &
      (\vcons\;\vx\; \vxs\Pp) & (\vcons\;\vy\; \vys\Pp)\\
  \multicolumn{4}{l}{\qquad
    \begin{array}[t]{|c@{\;}l}
      \FN{order}\;\vx\;\vy \\
      \DC{le} & \cq \vcons\;\vx \;(\FN{merge}\;\vxs\Pp\;\vys) \\
      \DC{ge} & \cq \vcons\;\vy \;
      (\{\GNOC{\FN{plusSuc}\;\vm\Pp\;\vn}\}\;
       \FN{merge}\;\vxs\;\vys\Pp) \\
    \end{array}
  }
\end{array}
}
\]

%Note that we need to use our lemmas the other way around when we are
%\emph{using} constructors on the right---in $\FN{vrev}$, we were putting
%constructors on the right.

We shall also need to add sizes to our intermediate $\TC{DealT}$ data
structure. The sizes for $\DC{empT}$ and $\DC{leafT}$ are obvious
enough, but what about $\DC{nodeT}$?  A useful clue is provided by the
\emph{balancing} invariant which our program preserves---the size of
the left subtree is either equal to that of the right subtree or just
one more, depending on the parity bit.  Let's write that down (we use
decimals to abbreviate numerical constants):

\newcommand{\pfog}[1]{\green{\hat{#1}}}
\[\AR{
\RW{let}\;\;\Rule{\vp\hab\TC{Parity}}
                 {\pfog{\vp}\hab\Nat}\quad
\AR{\pfog{\pev}\;\cq\DC{0}\\
    \pfog{\poo}\;\cq\DC{1}\\
   }
\medskip\\
\RW{data}\;\;
\Rule{\vn\hab\Nat\quad\vX\hab\Type}
     {\TC{DealT}\;\vn\;\vX\hab\Type}
\;\;
\RW{where}
\;\;
\AR{
\Axiom{\DC{empT}\hab\TC{DealT}\;\DC{0}\;\vX}
\;\;
\Rule{\vx\hab\vX}
     {\DC{leafT}\;\vx\hab\TC{DealT}\;\DC{1}\;\vX}
\\
\Rule{\vp\hab\TC{Parity}\quad\vl\hab\TC{DealT}\;(\pfog{\vp}\plus\vn)\;\vX
                        \quad\vr\hab\TC{DealT}\;\vn\;\vX}
     {\DC{nodeT}_{\vn}\;\vp\;\vl\;\vr \hab
       \TC{DealT}\;((\pfog{\vp}\plus\vn)\plus\vn)\;\vX}
}
}\]

There is more than one way to write down the size of a
$\DC{nodeT}$.\footnote{Indeed, the above does not ensure that
the subtrees of a node are nonempty---this can be done by replacing
$\vn$ with $(\DC{1+}\;\vn\Pp)$ in the type of $\DC{nodeT}$.} The
choice we make here is motivated by the $\FN{mergeT}$ operation, which now
has a more informative type:

\[
\RW{let}\;\;\AR{
\Rule{\vt\hab\TC{DealT}\;\vn\;\Nat}
     {\FN{mergeT}\;\vt\hab\VEC\;\vn\;\Nat}
\smallskip \\
\begin{array}{@{}l@{\;}c@{\;}l}
\FN{mergeT} & \vt & \BY \RW{rec}\;\vt \\
\;\;\FN{mergeT} & \DC{empT} & \cq \vnil \\
\;\;\FN{mergeT} & (\DC{leafT}\;\vx) & \cq \vcons\;\vx\;\nil \\
\;\;\FN{mergeT} & (\DC{nodeT}\;\vp\;\vl\;\vr) & \cq
  \FN{merge}\;(\FN{mergeT}\;\vl)\;(\FN{mergeT}\;\vr)
\end{array}
}
\]

We simply chose the return type for $\DC{nodeT}$ which made the old code
for $\FN{mergeT}$ go through as it stood, given the new type of
$\FN{merge}$! Of course, we shall pay when we come to write
$\FN{insertT}$---we could shift the burden the other way by taking the
size of a $\DC{nodeT}$ to be
$\pfog{\vp}\plus\vn\;\green{\ast}\;\DC{2}$.

\[
\RW{let}\;\;\AR{
\Rule{\vx\hab\vX\quad\vt\hab\TC{DealT}\;\vn\;\vX}
     {\FN{insertT}\;\vx\;\vt\hab\TC{DealT}\;(\DC{1+}\;\vn)\;\vX}
\smallskip \\
\begin{array}{@{}l@{\;}c@{\;}c@{\;}l}
  \FN{insertT} & \vx & \vt & \BY \RW{rec}\;\vt \\
\;\;  \FN{insertT} & \vx & \DC{empT} & \cq \DC{leafT}\;\vx \\
\;\;   \FN{insertT} & \vx & (\DC{leafT}\;\vy) & \cq
    \DC{nodeT}\;\pev\; (\DC{leafT}\;\vy)\;(\DC{leafT}\;\vx)\\
\;\;   \FN{insertT} & \vx & (\DC{nodeT}\;\pev\;\vl\;\vr) & \cq
    \DC{nodeT}\;\poo\;(\FN{insertT}\;\vx\;\vl)\;\vr \\
\;\;   \FN{insertT} & \vx & (\DC{nodeT}_{\vn}\;\poo\;\vl\;\vr) & \cq
    \{\CONG{\FN{plusSuc}\;\vn\;\vn}\}\;
    \DC{nodeT}_{(\DC{1+}\:\vn)}\;\pev\;\vl\;(\FN{insertT}\;\vx\;\vr)
\end{array}
}
\]

The damage is not too bad---we just have to appeal to algebraic properties of $\plus$,
to show that the constructed tree of size $(\DC{1+}\vn)\plus\vn$ fits the constraint 
$\DC{1+}(\vn\plus\vn)$.
%when we insert on the right, we still want
%the $\DC{1+}$ on the left!
This leaves $\FN{dealT}$ and $\FN{sort}$ with new types, but basically
the same code:

\[\AR{
\RW{let}\;\;
\Rule{\vxs\hab\VEC\;\vn\;\vX}
     {\FN{dealT}\;\vxs\hab\TC{DealT}\;\vn\;\vX}
\qquad
\begin{array}[t]{@{}l@{\;}c@{\;}l}
  \FN{dealT} & \vxs & \BY \RW{rec}\;\vxs \\
\;\;  \FN{dealT} & \vnil & \cq \DC{empT}\\
\;\;  \FN{dealT} & (\vcons\;\vx\;\vxs) &
      \cq\FN{insertT}\;\vx\;(\FN{dealT}\:\vxs)
\end{array}
\medskip\\
%
\RW{let}\;\;
\Rule{\vxs\hab\VEC\;\vn\;\Nat}
     {\FN{sort}\;\vxs\hab\VEC\;\vn\;\Nat}
\qquad
  \FN{sort} \; \cq \FN{mergeT}\cdot\FN{dealT}
}\]

It seems appropriate at this point to emphasize the importance of feedback
from the typechecker when doing a development like this. It's the typechecker
which tells you which lemmas you need and where you need to insert them,
and it's the constraints which arise during typechecking which tell you
how to engineer a data structure's indices so that its operations typecheck
where possible. The computational coincidences between the indices
we encounter are really a matter of care, not luck.


\subsection{A Question of Phase}

In our sorting example, the natural numbers $\TC{Nat}$ are playing two
separate parts---dynamically, they represent the data being sorted;
statically, they give sizes to the data structures involved in the
process. All the case analysis happens over the indexed datatypes,
rather than the indices themselves, so there is no need for sizes at
run-time. It would appear that, in this example at least, the data on
which types depend is entirely static. Although we need something like
the natural numbers within the language of types, should it be the
natural numbers? Perhaps it would be simpler if, as well as the natural
numbers, there was something else exactly like them.

It's not as ridiculous as it sounds: it enables us to keep the
workaday term language out of types, allowing dependency only on the
static things.  This keeps the type/term distinction in alignment with
the phase distinction, separating the static $\forall$ from
the dynamic $\to$, each with their own distinct notions of
abstraction, application and now of datatype.

Of course, static datatypes are not quite enough: our $\TC{DealT}$
datatype family relies on operations such as $\plus$. We need static
functions over static data, together with the associated partial
evaluation technology. We need to find design criteria for this
emerging static programming language. Will all our static data
structures be monomorphic? Do we believe that static data will never
need to be indexed itself? It would be bold to imagine that `yes' is
the answer to these questions.  At what point will we stop extending
the static language with a replica of the dynamic language?

We're beginning to see more and more of the same phenomema showing up
on either side of the phase distinction; we're even beginning to see
the potential emergence of a phase \emph{hierarchy}. Occam's razor
suggests that we should understand `data' and `function' once,
regardless of phase, since the phase distinction no longer
distinguishes where these notions arise.

But Occam's razor is a subjective instrument, so we cannot presume
that others will come to the same judgment as ourselves. We can,
however, examine the impact of the underlying presumption that `the
data on which types depend is entirely static'.


\section{Evidence is About Data; Evidence is Data}
\label{sec:evidence}

Type systems without dependency on dynamic data tend to satisfy the
\emph{replacement} property---any subexpression of a well typed
expression can be replaced by an alternative subexpression of the same
type in the same scope, and the whole will remain well typed. For
example, in Java or Haskell, you can always swap the $\texttt{then}$
and $\texttt{else}$ branches of conditionals and nothing will go
wrong---nothing of any static significance, anyway. The simplifying
assumption is that within any given type, one value is as good as
another. These type systems have no means to express the way that
different data mean different things, and should be treated accordingly
in different ways. That is why dependent types matter.

More specifically, we have seen how the apparatus of dependent types can
be used to maintain the length invariant in our sorting algorithm, but
that the length can essentially be regarded as a prior static notion
which the code must \emph{respect} dynamically.  What if we wanted to
guarantee the the output of our sorting algorithm is in order? The
order is not a prior static notion---that is why we need a sorting
algorithm---the order is \emph{established} by run-time testing of the
dynamic data. Can we observe this fact statically? As things stand,
$\TC{Order}$ does not matter: we can swap around the $\DC{le}$ and
$\DC{ge}$ outputs of our $\FN{order}$ test without affecting the well
typedness of $\FN{merge}$. How can we make $\TC{Order}$ matter? By
making $\TC{Order}$ a dependent type!


\subsection{Evidence of Ordering}

We replace the uninformative type $\TC{Order}$ by an inductive family with
$\TC{Order}\;\vx\;\vy$ expressing that  $\vx$ and $\vy$ can be ordered
and each possibility can be established with \emph{evidence}.
%There are two ways a given $\vx$ and $\vy$ can be ordered,
%and each possibility can be established with \emph{evidence}.

\newcommand{\LE}{\mathrel{\blue{\le}}}
\[\AR{
\RW{data}\;\;
\Rule{\vx,\vy\hab\Nat}
     {\vx\LE\vy\hab\Type}
\;\;\RW{where}\;\;
\orange{\cdots}
\\
\RW{data}\;\;
\Rule{\vx,\vy\hab\Nat}
     {\TC{Order}\;\vx\;\vy\hab\Type}
\;\;\RW{where}\;\;
\Rule{\VV{xley}\hab\vx\LE\vy}
     {\DC{le}\;\VV{xley}\hab\TC{Order}\;\vx\;\vy}
\;\;
\Rule{\VV{ylex}\hab\vy\LE\vx}
     {\DC{ge}\;\VV{ylex}\hab\TC{Order}\;\vx\;\vy}
}\]


\newcommand{\HOLE}{\yellowBG{[]}}
We shall give $\LE$ its constructors shortly. First, let us see what
happens to the $\FN{order}$ test---for a start, its type now tells us
what its output says about its input! We provide empty sheds $\HOLE$
which will be completed once we have decided how to represent $\LE$ proofs.
The freedom to delay implementation decisions while type checking
the rest of the code is an essential feature of Epigram which we 
further elaborate in the next section.

%We have left $\HOLE$s standing
%for the $\LE$ proofs.

\[
\RW{let}\;\;
\Axiom{\FN{order}\;\vx\;\vy\hab\TC{Order}\;\vx\;\vy}
\qquad
\begin{array}[t]{@{}l@{\;}c@{\;}c@{\;}l}
\FN{order} & \vx & \vy &\BY\RW{rec}\;\vx \\
\;\;\FN{order} & \DC{0} & \vy & \cq \DC{le}\;\HOLE \\
\;\;\FN{order} & (\DC{1+}\vx\Pp) & \DC{0} & \cq \DC{ge}\;\HOLE \\
\;\;\FN{order} & (\DC{1+}\vx\Pp) & (\DC{1+}\vy\Pp) &
    \begin{array}[t]{|c@{\;}l}
      \FN{order}\;\vx\Pp\;\vy\Pp \\
      \DC{le}\;\VV{x\Pp\!ley\Pp} & \cq \DC{le}\;\HOLE \\
      \DC{ge}\;\VV{y\Pp\!lex\Pp} & \cq \DC{ge}\;\HOLE \\
    \end{array}
\end{array}
\]

This program is almost as before, except that we cannot pass the recursive
call back directly---it is the ordering of $\vx\Pp$ and $\vy\Pp$, but
we need to order their successors. We can treat this partial program as
a clue to good constructors for $\LE$. What should we wish them to
be? Well, our first $\HOLE$ requires a proof that $\DC{0}\LE\vy$, so let
us have

\[
\Axiom{\DC{le0}\hab\DC{0}\LE\vy}
\]

Actually, that will also satisfy our second $\HOLE$, which needs a
proof of $\DC{0}\LE(\DC{1+}\vx\Pp)$. Our third $\HOLE$ requires us to
establish $(\DC{1+}\vx\Pp)\LE (\DC{1+}\vy\Pp)$, given
$\VV{x\Pp\!ley\Pp}\hab\vx\Pp\LE\vy\Pp$, and the fourth $\HOLE$ is
similar, so let us have

\[
\Rule{\VV{xley}\hab\vx\LE\vy}
     {\DC{leS}\;\VV{xley}\hab(\DC{1+}\vx\Pp)\LE (\DC{1+}\vy\Pp)}
\]

The two seem like a reasonable definition of $\LE$, and they certainly
enable us to fill in our $\HOLE$s.

\[
\RW{let}\;\;
\Axiom{\FN{order}\;\vx\;\vy\hab\TC{Order}\;\vx\;\vy}
\qquad
\begin{array}[t]{@{}l@{\;}c@{\;}c@{\;}l}
\FN{order} & \vx & \vy &\BY\RW{rec}\;\vx \\
\;\;\FN{order} & \DC{0} & \vy & \cq \DC{le}\;\DC{le0} \\
\;\;\FN{order} & (\DC{1+}\vx\Pp) & \DC{0} & \cq \DC{ge}\;\DC{le0} \\
\;\;\FN{order} & (\DC{1+}\vx\Pp) & (\DC{1+}\vy\Pp) &
    \begin{array}[t]{|c@{\;}l}
      \FN{order}\;\vx\Pp\;\vy\Pp \\
      \DC{le}\;\VV{x\Pp\!ley\Pp} & \cq
             \DC{le}\;(\DC{leS}\;\VV{x\Pp\!ley\Pp}) \\
      \DC{ge}\;\VV{y\Pp\!lex\Pp} & \cq
             \DC{ge}\;(\DC{leS}\;\VV{y\Pp\!lex\Pp}) \\
    \end{array}
\end{array}
\]

What has happened here? We started with a program which \emph{did} the
right thing but did not \emph{say} so. It should be no surprise that
when we try to make this program say what it does, we learn how to say
the right thing.

However, some of you may be wondering whether it is
worth saying the right thing, if it means spending heap on this data
structure of evidence and losing the tail-call optimisation into the
bargain. Fortunately, our $\LE$ type has the property of being
\emph{content-free}~\cite{edwin.conor.james:family.indices}: just as with
$\EQ$, for any given indices $\vx$ and $\vy$, $\vx\LE\vy$ contains at
most one value, so the evidence can be erased at run-time and the
tail-call restored. Moreover, it is no accident that $\LE$ is
content-free: our method of packing up the cases arising
ensured that $\DC{le0}$ and $\DC{leS}$ covered distinct indices, and
that the indices of any recursive proofs were determined in turn.

Just to be sure, let us check that $\LE$ is a total ordering---given
$\FN{order}$, we just need

\[\AR{
\RW{let}\;\;
\Axiom{\FN{leRefl}_{\vx}\hab\vx\LE\vx}
\qquad
  \AR{
    \FN{leRefl}_{\vx}\;\BY\RW{rec}\;\vx \\
    \;\;\begin{array}{@{}l@{\;}l}
       \FN{leRefl}_{\DC{0}} & \cq \DC{le0} \\
       \FN{leRefl}_{(\DC{1+}\:\vx\Pp)} & \cq \DC{leS}\;\FN{leRefl}_{\vx\Pp}
    \end{array}
  }
\medskip \\
\RW{let}\;\;
\Rule{\VV{xley}\hab\vx\LE\vy\quad\VV{ylez}\hab\vy\LE\vz}
     {\FN{leTrans}\;\VV{xley}\;\VV{ylez}\hab\vx\LE\vz}
\qquad
  \begin{array}[t]{@{}l@{\;}c@{\;}c@{\;}l}
    \FN{leTrans}     & \VV{xley} & \VV{ylez} & \BY\RW{rec}\;\VV{xley} \\
    \;\;\FN{leTrans} & \DC{le0}  & \VV{ylez} & \cq \DC{le0} \\
    \;\;\FN{leTrans} & (\DC{leS}\;\VV{xley}\Pp)
                     & (\DC{leS}\;\VV{ylez}\Pp) 
                     & \cq \DC{leS}\;
                       (\FN{leTrans}\;\VV{xley}\Pp\;\VV{ylez}\Pp)\\
 \end{array}
\medskip \\
\RW{let}\;\;
\Rule{\VV{xley}\hab\vx\LE\vy\quad\VV{ylex}\hab\vy\LE\vx}
     {\FN{leASym}\;\VV{xley}\;\VV{ylex}\hab\vx\EQ\vy}
\qquad
  \begin{array}[t]{@{}l@{\;}c@{\;}c@{\;}l}
    \FN{leASym}     & \VV{xley} & \VV{ylex} & \BY\RW{rec}\;\VV{xley} \\
    \;\;\FN{leASym} & \DC{le0}  & \DC{le0} & \cq \DC{refl} \\
    \;\;\FN{leASym} & (\DC{leS}\;\VV{xley}\Pp)
                    & (\DC{leS}\;\VV{ylex}\Pp) 
                    & \cq \CONG{\FN{leASym}\;\VV{xley}\Pp\;\VV{ylex}\Pp}\\
 \end{array}
}\]

As it happens, these properties of $\LE$ are not necessary in order to
implement sorting. To be sure that a list is sorted, you need at least to
have checked that each adjacent pair is in order---that it's \emph{locally
sorted}. It's not hard to see that a list can always be locally sorted
with respect to any binary relation which always holds one way around or
the other (when inserting a new element, if it fits nowhere before the
end, then it must fit at the end). Of course, knowing that it is
also a partial order enhances what you can deduce from a locally sorted
list.


\subsection{Locally Sorted Lists}

How shall we define locally sorted lists? We shall clearly have to index
lists with some sort of interval, but there are several ways we might
do it: one bound or two? open bounds or closed bounds? As with the design
of the sized $\TC{DealT}$, we should take care to ensure that the decision
leads to operations which are as cleanly defined as possible. For pedagogical
purposes, we shall sort lists rather than vectors---sizing and sorting are
independent refinements of the list structure. We do not verify the fact
that the resulting list is a permutation of the input here.

\textbf{One bound or two?}$\quad$ In order to do a `cons', we shall certainly
need to know that the head and the tail are suitably ordered, so the tail
will require a lower bound. Meanwhile, $\FN{merge}$ makes no restrictions
between the bounds of its inputs---only between the input bounds and the
output bounds. That suggests that we can get away with a lower bound for
this example. Of course, if we wanted to \emph{concatenate} sorted lists
(in an algorithm based on pivoting, say), we should need upper bounds too.

\textbf{Open or closed?}$\quad$ It is perhaps a little tricky to give a
precise lower bound for the empty list---we could make a bound `elements
lifted with $\red{\infty}$' and lift $\LE$ accordingly:

\[
  \RW{data}\;\;
  \Rule{\vb\hab\TC{Lifted}\;\Nat}
       {\TC{CList}\;\vb\hab\Type}
  \;\;\RW{where}\;\;
  \Axiom{\DC{cnil}\hab\TC{CList}\;\red{\infty}} \;\;
  \Rule{\vx\hab\Nat\quad \VV{xley}\hab\vx\LE\vy\quad
        \vxs\hab\TC{CList}\;\vy}
       {\DC{ccons}\;\vx\;\VV{xley}\;\vxs\hab\TC{CList}\;(\DC{lift}\;\vx)}
\]

The lifting is not a big issue here---we only get away without
$\red{-\infty}$ because $\Nat$ stops at $\DC{0}$. The advantage of this
definition is its precision: each list is assigned its tightest bound.
The disadvantage of this definition is also its precision---when we
are making a $\TC{CList}$, we must either specify its lower bound
(eg., $\FN{sort}$ is bounded by $\FN{min}$) and satisfy that specification,
or say `don't care' with an existential. But we do care! The recursive calls
in $\FN{merge}$ can't have any old lower bound---the lower bound of the tail
must be at least the head!

The fact that we can \emph{assign} closed bounds to sorted lists does
not necessarily make them a good choice for a type, because we also
need to \emph{prescribe} bounds. We can usually think of an open bound,
even if it is not the best one. Let us try:

\[
  \RW{data}\;\;
  \Rule{\vb\hab\Nat}
       {\TC{OList}\;\vb\hab\Type}
  \;\;\RW{where}\;\;
  \Axiom{\DC{onil}\hab\TC{OList}\;\vb} \;\;
  \Rule{\vx\hab\Nat\quad \VV{blex}\hab\vb\LE\vx\quad
        \vxs\hab\TC{OList}\;\vx}
       {\DC{ocons}\;\vx\;\VV{blex}\;\vxs\hab\TC{OList}\;\vb}
\]

Here, we have that $\DC{onil}$ trivially satisfies any prescribed bound,
whilst for $\DC{ocons}$, the head must exceed the prescribed lower bound
and bound the tail in turn. Any old sorted list can certainly be
represented as an element of $\TC{OList}\;\DC{0}$. Meanwhile, if \emph{both}
inputs to $\FN{merge}$ share a lower bound, then the output certainly
shares it too.

\[
\RW{let}\;\;\AR{
\Rule{\vxs,\vys\hab\TC{OList}\:\vb}
     {\FN{merge}\;\vxs\;\vys\hab\TC{OList}\;\vb}
\smallskip \\
\begin{array}{@{}l@{\;}c@{\;}c@{\;}l}
\FN{merge}&\vxs&\vys&\BY\RW{rec}\;\vxs \\
\;\;  \FN{merge} & \DC{onil} & \vys & \cq \vys \\
\;\;  \FN{merge} & (\DC{ocons}\;\vx\;\VV{blex}\;\vxs\Pp)
                 & \vys & \BY \RW{rec}\;\vys \\
\;\;  \;\;\FN{merge} & (\DC{ocons}\;\vx\;\VV{blex}\;\vxs\Pp) & \DC{onil} & \cq \vxs \\
\;\;  \;\;\FN{merge} & (\DC{ocons}\;\vx\;\VV{blex}\;\vxs\Pp)
                     & (\DC{ocons}\;\vy\;\VV{bley}\;\vys\Pp) \\
\multicolumn{4}{l}{\;\; \;\; \quad
    \begin{array}[t]{|c@{\;}l}
      \FN{order}\;\vx\;\vy \\
      \DC{le}\;\VV{xley} & \cq \DC{ocons}\;\vx\;\VV{blex}\;
              (\FN{merge}\;\vxs\Pp\;(\DC{ocons}\;\vy\;\VV{xley}\;\vys\Pp)) \\
      \DC{ge}\;\VV{ylex} & \cq \DC{ocons}\;\vy\;\VV{bley}\;
              (\FN{merge}\;(\DC{ocons}\;\vx\;\VV{ylex}\;\vxs\Pp)\;\vys\Pp) \\
    \end{array}
}
\end{array}
}
\]

Each input list already satisfies the bound required for the output, so
the $\DC{onil}$ cases are trivial. When we have two $\DC{ocons}$es, we
know both heads satisfy the lower bound, but whichever we pick must
bound the recursive $\FN{merge}$. Hence, we had better pick the smaller
one---the evidence we get from $\FN{order}$ is exactly what we need to
show that the list which keeps its head satisfies the newer tighter bound.

Now we can flatten a $\TC{DealT}$ into a sorted list. This gives us a
new back end which we can compose with the old $\FN{dealT}$ to get a
sort which produces sorted output:

\[\AR{
\RW{let}\;\;
\Rule{\vt\hab\TC{DealT}\;\TC{Nat}}
     {\FN{mergeT}\;\vt\hab\TC{OList}\;\DC{0}}
\qquad
\begin{array}[t]{@{}l@{\;}c@{\;}l}
\FN{mergeT} & \vt & \BY \RW{rec}\;\vt \\
\;\;\FN{mergeT} & \DC{empT} & \cq \DC{onil} \\
\;\;\FN{mergeT} & (\DC{leafT}\;\vx)
                & \cq \DC{ocons}\;\vx\;\DC{le0}\;\DC{onil} \\
\;\;\FN{mergeT} & (\DC{nodeT}\;\vp\;\vl\;\vr) & \cq
  \FN{merge}\;(\FN{mergeT}\;\vl)\;(\FN{mergeT}\;\vr)
\end{array}
\medskip\\
%
\RW{let}\;\;
\Rule{\vxs\hab\TC{List}\;\TC{Nat}}
     {\FN{sort}\;\vxs\hab\TC{OList}\;\DC{0}}
\qquad
  \FN{sort} \; \cq \FN{mergeT}\cdot\FN{dealT}
}\]

\textbf{Remark.}$\quad$ In the above definitions of $\FN{merge}$ and
$\FN{mergeT}$, all of the $\le$ proofs which we supply in the lists
we build are either by $\DC{le0}$ or by direct appeal to a hypothesis
in scope. The proofs which we uncover by case analysis are only used
in this way. It seems reasonable to consider suppressing all of them
by default from the explicit syntax of the program. Implicit hypotheses
could be kept in the context and searched whenever an implicit proof is
required, in much the way that Haskell handles the implicit dictionaries
when unpacking and packing existential types. Of course, a `manual
override' is still necessary---not all proofs are so immediate.


\subsection{Programming with Evidence}

The idea that one datatype can represent evidence about the values in
another is alien to mainstream functional programming languages, but its
absence is beginning to cause pain. A recent experiment in `dynamically
typed' generic programming---the `Scrap Your Boilerplate' library
of traversal operators by Ralf L\"ammel and Simon Peyton
Jones~\cite{laemmel.peytonjones:boilerplate}---is a case in point.
The library relies on a `type safe cast' operator, effectively comparing
types at run time by comparing their encodings as data:

\begin{alltt}
  cast :: (Typeable a, Typeable b) => a -> Maybe b
  cast x = r
         where
           r = if typeOf x == typeOf (get r)
               then Just (unsafeCoerce x)
               else Nothing

           get :: Maybe a -> a
           get x = undefined
\end{alltt}

Here, the Boolean test $\texttt{typeOf x == typeOf (get r)}$ serves
the purpose of comparing type $\texttt{a}$ with type $\texttt{b}$, but
the Haskell typechecker cannot interpret the value $\texttt{True}$ as
a reason to unify types. In the absence of evidence, the programmers
resort to coercion. \emph{We} trust them, of course, but this degree
of trust should not be necessary. This is an example
where static dependency on dynamic data is the solution, not the
problem.

We now work in a setting where we expect operations which test their
input in some way to have a dependent type which explains what the
output reveals about the input---$\FN{order}$ is a simple
example. Without some kind of movement between terms and types this is
impossible---you can exploit valid data (eg., writing a type-preserving
evaluator for a typed expression language) , but you cannot \emph{validate}
it dynamically (eg., by writing a typechecker).

Traditional dependent types achieve this movement directly---the
argument of a function can appear in the type of its result, and often
will, if the result is evidence of the argument's properties. An
alternative, adopted for DML's numerical indices, is to push the other
way with \emph{singleton types}---types of dynamic data constrained to
be equal to the corresponding static data. We would have something like
this:

\[
  \FN{order}\hab\forall\vx,\vy:\Nat\TO
    \TC{Only}\;\vx \To \TC{Only}\;\vy \To \TC{Order}\;\vx\;\vy
\]

Here every piece of data about which we seek evidence must be
abstracted twice---the type of evidence depends on the static copy,
but the testing itself is performed on the dynamic copy. In the
context of DML, this is a sensible separation---erasing the indices is
intended to yield a well typed SML program. In a broader context,
where we might need to represent properties of any kind of dynamic
data, this approach seems unlikely to scale. Effectively, we may need
to replace each type $\vT$ by the existential pairing of its static
and dynamic copies $\exists\vt:\vT\TO\TC{Only}\;\vt$ throughout our
programs, in order to have data at all the levels where it is used.

Singleton types thus provide a convenient way to retro-fit a more
sophisticated type system to an existing compiler architecture.
With just one notion of data at all levels, dependent types
provide a convenient way for programmers to exploit the potential
of working with data as evidence. Our typechecker example in
\cite{ConorJames:vfl} not only generates enough evidence about
the types being inferred and compared to feed the tagless interpreter,
it generates evidence about the program being checked---it is the
first typechecker in the literature which is statically guaranteed to
check its input!


%=======
%\subsection{Programming with Evidence}

%The idea that one datatype can represent evidence about the values in
%another is alien to mainstream functional programming languages, but its
%absence is beginning to cause pain. A recent experiment in `dynamically
%typed' generic programming---the `Scrap Your Boilerplate' library
%of traversal operators by Ralf L\"ammel and Simon Peyton
%Jones~\cite{laemmel.peytonjones:boilerplate}---is a case in point.
%The library relies on a `type safe cast' operator, effectively comparing
%types at run time by comparing their encodings as data:

%\begin{alltt}
%  cast :: (Typeable a, Typeable b) => a -> Maybe b
%  cast x = r
%         where
%           r = if typeOf x == typeOf (get r)
%               then Just (unsafeCoerce x)
%               else Nothing

%           get :: Maybe a -> a
%           get x = undefined
%\end{alltt}

%Here, the Boolean test $\texttt{typeOf x == typeOf (get r)}$ serves
%the purpose of comparing type $\texttt{a}$ with type $\texttt{b}$, but
%the Haskell typechecker cannot interpret the value $\texttt{True}$ as
%a reason to unify types. In the absence of evidence, the programmers
%resort to coercion. \emph{We} trust them, of course, but this degree
%of trust should not be necessary. This is an example
%where static dependency on dynamic data is the solution, not the
%problem.

%We now work in a setting where we expect operations which test their
%input in some way to have a dependent type which explains what the
%output reveals about the input---$\FN{order}$ is a simple
%example. Without some kind of movement between terms and types this is
%impossible---you can exploit valid data (eg., writing a type-preserving
%evaluator for a typed expression language) , but you cannot \emph{validate}
%it dynamically (eg., by writing a typechecker).

%Traditional dependent types achieve this movement directly---the
%argument of a function can appear in the type of its result, and often
%will, if the result is evidence of the argument's properties. An
%alternative, adopted for DML's numerical indices, is to push the other
%way with \emph{singleton types}---types of dynamic data constrained to
%be equal to the corresponding static data. We would have something like
%this:

%\[
%  \FN{order}\hab\forall\vx,\vy:\Nat\TO
%    \TC{Only}\;\vx \To \TC{Only}\;\vy \To \TC{Order}\;\vx\;\vy
%\]

%Here every piece of data about which we seek evidence must be
%abstracted twice---the type of evidence depends on the static copy,
%but the testing itself is performed on the dynamic copy. In the
%context of DML, this is a sensible separation---erasing the indices is
%intended to yield a well typed SML program. In a broader context,
%where we might need to represent properties of any kind of dynamic
%data, this approach seems unlikely to scale. Effectively, we may need
%to replace each type $\vT$ by the existential pairing of its static
%and dynamic copies $\exists\vt:\vT\TO\TC{Only}\;\vt$ throughout our
%programs, in order to have data at all the levels where it is used.

%Singleton types thus provide a convenient way to retro-fit a more
%sophisticated type system to an existing compiler architecture.
%With just one notion of data at all levels, dependent types
%provide a convenient way for programmers to exploit the potential
%of working with data as evidence. Our typechecker example in
%\cite{conor.james:viewfromleft} not only generates enough evidence about
%the types being inferred and compared to feed the tagless interpreter,
%it generates evidence about the program being checked---it is the
%first typechecker in the literature which is statically guaranteed to
%check its input!


\section{The Tools of the Trade}
\label{sec:tools}

Programming is a complex task which can be made easier for people to
do with the help of computers. The conventional cycle of programming
with a text editor then compiling in `batch mode' is a welcome
shortening of the feedback loop since the days of punched cards, but it
clearly under-uses the technology available today. Any typed
programming language can benefit from the capacity---but not
necessarily the compulsion---to invoke the typechecker incrementally
on incomplete subprograms whilst they are under development. The more
powerful the type system, the more pressing this need becomes---it just
gets harder to do it in your head, especially when types contain
\emph{computations}, for which computers are inherently useful.

Moreover, a type acts as a partial specification of a program, and can
thus be used to narrow the search space for correct programs, even if
only as a key for searching a library. The choice of a program's type
is inherently suggestive of the programming strategies with which that
type is naturally equipped---constructors and case analysis for
datatypes, abstraction and application for functions, and so on.  It
is a tragic waste if types play only a passive r\^ole in
programming, providing a basis for error reporting. Our technology should
enable programmers to exploit the clues which types provide.

% For example,
%Haskell's \emph{ad hoc}
%polymorphism~\cite{wadler.blott:less.ad.hoc,%
%peytonjones.jones.meijer:type.classes} allows standard components to
%be constructed automatically in a type-directed way---the programmer can
%indicate the general idea and leave the machine to do the tedious `plumbing'.

\subsection{Dependent Types Also Matter Behind the Scenes}

The key innovation of Epigram is its use of types to express
programming tasks and programming patterns. For example, the pattern
of primitive recursion on a datatype is expressed by an \emph{induction
principle}, like
\[
\AR{\forall\vP:\Nat\to\Type\;\TO \\
       \vP\;\DC{0} \To \;\;
       (\forall \vn\Pp:\Nat\TO\;\vP\;\vn\Pp\to \vP\;(\DC{1+}\;\vn\Pp)) \To \\
       \forall \vn:\Nat\TO\;\vP\;\vn}
\]
%
The conventional type of primitive recursion may be shorter, but it is less
informative.
\[
\forall\vP\TO\vP\To(\Nat\to\vP\to\vP)\To\Nat\To\vP
\]
%
This type does not connect the arguments to the tasks they serve,
whereas the induction principle is a dependent type which explains
that a recursive computation for $\vn$ needs a method for $\DC{0}$ and
a method for $(\DC{1+\;\vn\Pp})$.
Correspondingly, the system transforms a programming problem such as
\[
  \RW{let}\;\;
  \Rule{\vx,\vy\hab\Nat}
       {\vx\plus\vy\hab\Nat}
\qquad\qquad\mbox{into a `computability proof' goal}\qquad\qquad
  \orange{?}\hab\forall\vx,\vy\hab\Nat\TO\;\LBTY{\vx\plus\vy:\Nat}
\]
We read $\LBTY{\vx\plus\vy:\Nat}$ as `$\vx\plus\vy$ is computable'.
If we attack this goal with induction, we acquire subgoals like these:
\[\AR{
  \orange{?}\hab\forall\vy\hb\Nat\TO\;\LBTY{\DC{0}\plus\vy:\Nat} \\
  \orange{?}\hab
    \AR{\forall\vx\Pp\hb\Nat\TO \\
        (\forall\vy\hb\Nat\TO\;\LBTY{\vx\Pp\plus\vy:\Nat})\To \\
        \forall\vy\hb\Nat\TO\;\LBTY{(\DC{1+}\;\vx\Pp)\plus\vy:\Nat}
       }
}\]
These goals tell us exactly the `left-hand sides' for the subprograms
which the recursive strategy requires.

By using dependent types to represent programming problems and
programming patterns, we have acquired an interactive programming
environment for the price of an interactive proof system. The basic
constructs of the Epigram language give a `programming' presentation
to the basic tactics of the system. The $\BY$ construct is just
McBride's `elimination with a motive' tactic~\cite{conor:motive} which
synthesizes an appropriate `$\vP$' parameter for any induction-like
rule. The $|\VV{scrutinee}$ construct is just `cut'.
%
The details of the process by which Epigram code is \emph{elaborated}
into the underlying type theory---a variation of Luo's
UTT~\cite{luo:typetheoryforcs}---are given in
\cite{ConorJames:vfl}. Edwin Brady's compiler for UTT,
which erases extraneous information at run time, is presented
in \cite{edwin:thesis}.

The point is this: UTT is not our programming language---UTT is our
language for describing how programming works.  Epigram has no
hard-wired construct for constructor case analysis or
constructor-guarded recursion---these are just programming patterns
specified by type and supplied as standard with every datatype you
define. However, UTT types are Epigram types, so you are free to
extend our language by specifying and implementing your own
patterns. Our typechecking example in \cite{ConorJames:vfl} is a
derived case analysis principle for expressions, exposing not their
syntax, but their types or type errors. By making programming patterns
first-class citizens via dependent types, we raise the level of
abstraction available to programmers and provide interactive support for its
deployment at a single stroke.


\subsection{Programming Interactively}

The interface to Epigram is inspired by the Alf
proof editor~\cite{magnusson.nordstrom:alf}, which introduced
a type-directed structure editor for incomplete programs.
In Epigram's  concrete syntax, a \emph{shed} $\SHEDC{\texttt{raw
text}}$ may stand in for any subexpression. The elaborator is not
permitted inside a \emph{shed}, so the text it contains may be edited
freely.  There are two basic editing moves---removing the brackets to
admit the elaborator (which will process as far as any nested sheds)
and placing brackets around an elaborated subexpression to `undo' it
and return it to a shed. Correspondingly, the full spectrum of
interactivity is supported: an entire program can be written (or, more
to the point, reloaded) inside a shed, then elaborated in `batch
mode'; or a single syntactic construct can be elaborated, with sheds
for subexpressions. The advantages of structure editing are available,
but the disadvantages are not compulsory.

The interactive development of a program is a kind of dialogue.  The
system poses the problems---the left-hand sides of programs.  We
supply the solutions by filling in the right-hand sides, either by
directly giving the program's output $\TO\;\vt$, or by invoking a
programming pattern which reduces the problem to subproblems which are
then posed in turn. There is a direct mapping from sheds
$\SHEDC{\cdots}$ in source code to metavariables, $\orange{?x}$ in the
underlying proof state---when you elaborate a shed, your code triggers
a refinement of the proof state, and the resulting refinement to the
source code is read off from the types of the subgoals. Nothing is
hidden---the proof state is recoverable from the source code simply by
re-elaborating it in a single step.

Our approach to metavariables basically follows McBride's OLEG
system~\cite{conor-phd}---metavariables represent not only the missing
contents of sheds, but also all the unknowns arising from implicit
quantification. The latter are resolved, where possible, by solving
the unification constraints which arise during typechecking---we
follow Dale Miller's `mixed prefix'
approach~\cite{miller:mixed}. Epigram will not guess the type of your
program, but it will infer the bits of your program which the type
determines. The Damas-Milner approach to type
inference~\cite{damas.milner:principal} is alive and well and working
harder than ever, even though we have dispensed with the shackles on
programming which allow it to be complete.


\section{Further Work}
\label{sec:further}

We have hardly started. Exploiting
the expressivity of dependent types in a practicable way involves
a wide range of challenges in the development of the theory, the
design of language, the engineering of tools and the pragmatics of
programming. In this section, we summarize just a few of them.

\textbf{First-Class Modules.}$\quad$ No programming language can succeed
without strong support for the large-scale engineering of
systems. Dependent type systems already allow us to express record
types which pack up data structures, operations over them---and also
proofs of the properties of those operations. \emph{Manifest} record
types, which specify the values of some of their fields, can be used
to express sharing between records, and between the the inputs and
outputs of record-transforming operations~\cite{pollack:records}.
Epigram's first-class notion of programming pattern allows an abstract
datatype to offer admissible notions of pattern matching which hide
the actual data representation but are guaranteed to be faithful to
it---we have Wadler's \emph{views} for free~\cite{wadler:views}.

We now need a practical theory of subtyping to deliver a suitable form
of inheritance, and a convenient high-level syntax for working with
records. Our elaboration mechanism naturally lends itself to the approach
of \emph{coercive subtyping}, where subsumptions in source code elaborate to
explicit coercion functions---typically projections---in the underlying
theory~\cite{luo:coercive}.

\textbf{Universe Polymorphism.}$\quad$ What is the type of types, and how
do we quantify over them safely and consistently? In
\cite{ConorJames:vfl}, we follow the \emph{predicative} fragment of Luo's
Extended Calculus of Constructions~\cite{luo:ecc},
installing a cumulative hierarchy of universes
$\Type_0,\Type_1,\ldots$ each of which both inhabits and embeds in the 
next, so that $\Type_i:\Type_{i+1}$ holds and $\vT:\Type_i$ implies
$\vT:\Type_{i+1}$. As Harper and Pollack have
shown~\cite{harper.pollack:universes}, the user need never
write a universe level explicitly---the machine can maintain a graph
of relative level constraints and protest if any construction induces
a cycle.

This much is certainly safe, and it allows every type to find its own
particular level---this is called \emph{typical ambiguity}. The
trouble is that, as things stand, there is no satisfactory way to
define datatype constructors which operate at \emph{every}
level---this is called \emph{universe polymorphism}. A simple example
shows up if we have a list $\VV{Ts}$ of element types, and we want to
construct the corresponding list of list types
\[
  \FN{map}\;\TC{List}\;\VV{Ts}
\]
The types \emph{in} $\VV{Ts}$ live one level below the type \emph{of}
$\VV{Ts}$, so we need $\TC{List}$ to operate at \emph{both}
levels. There has been some very promising theoretical work in this
area~\cite{paul.zhaohui:coercions.universes,courant02tphols} but
again, a clear and convenient design has yet to emerge. In the
interim, we have adopted the cheap but inconsistent fudge of taking
$\Type:\Type$.


\textbf{Generics and Reflection.}$\quad$ Any function $\vT\hab\vU\to\Type$
naturally \emph{reflects} a sublanguage or \emph{universe} of types---those
given by $\vT\;\vu$. We can think of $\vu$ as the `name' of a type in the
universe.
If $\vU$ happens to be a datatype, we can write generic programs which work
for all the types named by a value $\vu$. Perhaps $\vU$ is the type of
regular expressions and $\vT$ computes for each regular expression the
type of its words, yielding \emph{regular expression
types}~\cite{pierce.vouillon.hosoya:regular.expression.types}.
Perhaps $\vU$ represents a collection of datatypes with a decidable equality
\[
  \FN{eq}\hab\forall\vu:\vU;\,\vx,\vy:\vT\;\vu\;\TO\;
    \vx\EQ\vy\;\blue{\vee}\;\vx\;\blue{\neq}\;\vy
\]
Peter Morris has recently implemented exactly such a generic equality
in Epigram for the universe of regular types~\cite{Morris:bctcs05}.
Dependently typed generic programming is a lively research area
\cite{alti:wcgp02,benke.dybjer.jansson:universes}, inspired by the pioneering
work on generics in
Haskell~\cite{backhouse.jansson.jeuring.meertens:generic,%
              hinze.jeuring:tidts,genHaskell}.

Alternatively, perhaps our universe $\vU$ represents a class of decidable
propositions, equipped with a decision procedure
\[
  \FN{decide}\hab\forall\vu:\vU\;\TO\;
    \vT\;\vu\;\blue{\vee}\;\vT\;\vu\To\blue{\bot}
\]
Such a procedure could be used to extend the elaborator's capacity to solve
simple proof obligations and equational constraints automatically, following
the lead of DML~\cite{DML99}, but without wiring a particular constraint
domain into the language design and the compiler.
However, to make this work conveniently, we need language support for the
declaration of universes $(\vU,\vT)$ where $\vU$ reflects a given class of
types and $\vT$ is invertible by construction, enabling the elaborator to
infer the appropriate $\vu:\vU$ when invoking a generic operation.


\textbf{Observational Type Theory.}$\quad$ The type of equality proofs
used in an intensional theory like Epigram's underlying Type Theory,
is inconvenient when working with infinite objects like function types
or lazy lists (codata). We plan to overcome this restriction without
affecting the decidability of type checking by implementing an
\emph{Observational Type Theory} based on \cite{alti:lics99,mh:phd}.

\textbf{Monadic interfaces to the real world.}$\quad$ A dependently typed
language offers the opportunity to develop the idea of monadic IO further.
Such a monadic interface comes in two guises: a static, denotational semantics
which can be used to reason about the programs and an operational semantics,
which is employed at runtime. This approach is not only relevant for IO,
based on \cite{venanzio-general} we can develop a \emph{partiality monad}
which allows us not only to implement but also reason about genuinely partial
programs like interpreters or programs on the computable reals. 

\textbf{Refactoring.}$\quad$ Epigram's
\emph{refinement} style of editing supports the process of working from
types to programs, but it does not help with the inevitable iterations
of the design cycle, as our plans evolve. Indeed, it is quite normal
to build up the index structure for our in layers, as we did for our
$\FN{sort}$ example. Our interactive editing technology should support
these process also, allowing us to experiment at pushing different
type refinements through our programs. We should be able to try out
the options which gave rise to the `open lower bound' choice for
sorted lists.

More generally, we can seek to emulate existing tools
for \emph{refactoring} evolving the design of data structures and
programs~\cite{kent-refactoring}, now in the context of a system which
supports incomplete objects. There is plenty of scope to develop tools
which really reflect the way most programmers work, iteratively improving
program \emph{attempts}---good ideas often come a bit at a time.



\section{Conclusions}
\label{sec:concl}

This much is clear: many programmers are already finding practical
uses for the approximants to dependent types which mainstream
functional languages (especially Haskell) admit, by hook or by crook.
From arrows and functional reactive
programming~\cite{hughes:afp-arrows,henrik:dynamic-optimisation},
through extensible records~\cite{oleg.ralf.keean:hlists},
database programming~\cite{bringert.hockersten:haskellDB}  and
dynamic web scripting~\cite{thiemann:programmabletypes} to code
generation~\cite{baars.swierstra:selfinspecting}, people are `faking it'
any way they can to great effect.
Each little step along the road to dependent types makes the task
a little easier, the code a little neater and the next improvement a
little closer: the arrival of GADTs in Haskell is a joy and a
relief~\cite{peyton-jones-wobbly}.

Now is the time to recognize the direction of these developments and
pursue it by design, not drift. In the long term, there is a great
deal more chaos and confusion to be feared from fumbling through
today's jungle of type class Prolog, singletons and proxies than from
a dependent type system whose core rules have been well studied and
fit on the back of an envelope. The Epigram project is our attempt to
plunder the proof systems based on Type Theory for the technology they
can offer programmers. It is also a platform for radical
experimentation without industrial inertia---an attempt to discover in
which ways dependent types might affect the assumptions upon which
mainstream functional language designs have been predicated. We are
having a lot of fun, and there is plenty of fun left for many more
researchers, but we do not expect to change the mainstream
overnight. What we can hope to do is contribute a resource of
experiments---successful or otherwise---to this design process.

More technically, what we have tried to demonstrate here is that the
distinctions term/type, dynamic/static, explicit/inferred are no longer
naturally aligned to each other in a type system which recognizes the
relationships between values. We have decoupled these dichotomies and
found a language which enables us to explore the continuum of
pragmatism and precision and find new sweet spots within it.
Of course this continuum also contains opportunities for remarkable
ugliness and convolution---one can never legislate against bad design---but
that is no reason to toss away its opportunities. Often, by
bringing out the ideas which lie behind good designs, by expressing
the things which matter, dependent types make data and programs fit better.

\bibliographystyle{fundam}
\bibliography{handy,obsdec,alti,local} 

\end{document}
